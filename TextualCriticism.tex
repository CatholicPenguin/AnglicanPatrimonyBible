\begin{onecolumn}	
\chapter{The Problem of Textual Criticism}
The reality of textual criticism has posed quite the difficulty for many churches, leading to great confusion and schisms.
\par
The \textit{Anglican Patrimony Version} is meant to be read, both in personal use and in the parish setting. Therefore, it does not make extensive use of textual footnotes and explanations. However, great care has been taken to be faithful to the original languages, as we best understand them. Therefore, textual criticism has been essential for this edition. Due to this, I thought it critical to provide not just an academic explanation of textual criticism but a theological reflection and explanation for what textual criticism \textit{means} (instead of leaving vague footnotes to confuse the reader like many recent translations have sadly done).
\section{What is Textual Criticism?}
Textual criticism is the result of three realities:
\begin{enumerate}
	\item The original autographs of the Scriptures were written by God through human authors.
	\item We do not possess those original autographs.
	\item The manuscripts---of those autographs---which we possess differ from each other in certain places.
\end{enumerate}
The goal of the text critic is to discern which variation reflects what was actually present in the original autograph.
\section{An Unsure Text?}
Some Christians desire to dismiss textual criticism categorically. If the Catholic Church is guided by the Holy Ghost, how could it lose the original text? Also, if the Church is the guardian of the Sacred Scriptures, then how could it fail in its mission so as to not know what those Scriptures are? Even worse, does the Church then need to depend on the natural sciences to perform its supernatural mission?

These are all very good concerns, and they need to be taken seriously and answered before we move forward with making critical editions of the Scriptures.
\subsection{Was the Original Text Lost?}
To the first point, the issue our current situation presents is not having lost text and needing to recover it. It is much better! Rather, we have too much text. For example, we know that, over time, Christians were reading and using their Bibles. Sometimes they wrote theological commentary in the margin. Unfortunately, the margin is also where a scribe would write a verse he initially forgot to copy in the body of the text (they didn't have erasers or White-Out). So, sometimes, the commentary would be copied into the main text by a later copyist: intermingling the inspired text of Sacred Scripture with commentary on that Scripture. This is not the only problem that can occur in transmission, but it shows that we are not digging for ``lost books'' or ``lost verses'' to add. Likewise, we are not trying to remove inspired text from the Bible. Instead, the goal of the text critic is to untangle what scribes centuries before have tangled and intermingled.

The original text is not lost. It is simply that non-inspired words---or even sometimes mis-copying or accidentally skipping a line---have been included in some manuscripts (though those same errors do not happen in others). The issue then is detangling the text based on all of the manuscripts we have.
\subsection{How Does the Church Guard the Scriptures?}
To the second point, the Word of God constitutes the Catholic Church, and the Catholic Church receives the Word of God and cherishes it. However, at its divine founding, the Church's understanding of the Sacred Scriptures was not immediate and fully comprehensive. Instead, since the Church is guided by the Holy Ghost, it discerns the Scriptures over time: both their meaning and extent.

This is seen in the canon of the Scriptures. There was a time when there was not catholic agreement on the canon of the New Testament. Yet, over time, the Holy Ghost guided the Church, and the Church listened to the Good Shepherd's voice, receiving only those books which are inspired. Similarly, the Church right now is discerning His voice regarding the Old Testament canon, where there is not catholic agreement on exactly which books belong. In the same way, there is a discernment that happens over time regarding verses. This is not failure: it is simply the divine timescale. And part of living on the divine timescale is reforming ourselves constantly unto the law of God.

Though, it must be said, it is the sad reality that the Church can perform her mission imperfectly. Accretions can develop which obscure the face of Christ, hinder the work of the Church, and even cause theological missteps. However, at the same time, the Sacred Scriptures can be ornamented with textual jewels, so to say, of good commentary. While the technical language for these interpolations is ``corruption,'' it is the editor's opinion that most of them are better called ornamentation. And while the ornamentation itself is not bad, it is necessary to be able look at the image of Christ as it is purely, without ornamentation, lest the ornaments be confused for the image. This is the work of textual criticism: discern the inspired and non-inspired words, rejecting nothing that is good but putting it in its proper place.

\subsection{Does the Church Need Natural Science?}
To the third point, the Catholic Church has always needed natural realities for her mission. That is at the essence of the Incarnation. Natural realities are never destroyed but raised up for a supernatural purpose: Grace perfects (not destroys) nature. Therefore, the natural and supernatural ought not to be put in opposition. Instead, the natural---such as natural law, natural substances, and natural skills---are embedded and incorporated into the Scriptures: into the very promises of God. For example, to be faithful to the Sacred Scriptures, knowledge of the natural language of Greek is necessary (something the Western Church sadly neglected for some time). Likewise, to give the Scriptures to others, it requires gathering paper and collecting \& copying manuscripts and giving them to others: all of which are natural skills and substances. It is at this point where the natural science of textual criticism exists at the service of the Word for the Church.
\subsection{Conclusion}
I hope this provides a helpful primer and apologetic to textual criticism. This critical edition has been prepared out of love for the Scriptures: the key treasure \& guiding light of the Church. So, while some verses may be moved into the margins or a separate section, this is not to remove and undermine the Word of God. Instead, it is to honour the Word of God by keeping the inspired text front-and-centre and the excellent notes and spiritual commentary ready on the side for Christians to appreciate in its proper context.
\end{onecolumn}