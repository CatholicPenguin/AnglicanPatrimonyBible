\documentclass[twocolumn,twoside,titlepage,10pt]{book}\usepackage[paperheight=210mm,paperwidth=148mm,inner=25.4mm,outer=20mm,top=15mm,bottom=15mm,heightrounded]{geometry}
\usepackage{fontspec}
\setmainfont[Numbers={Lowercase, Proportional}, Ligatures={Common, Rare}]{EB Garamond}
%Margins...
\usepackage{marginnote}
\usepackage[switch*]{lineno}
\usepackage{nicefrac, xfrac}
\usepackage{cjhebrew}
\usepackage[noorphans=true,indentfirst=false]{quoting}
\usepackage{ragged2e}
%Middle Line Separating Columns
\setlength{\columnseprule}{0.4pt}

%For First Letter Images
\usepackage{graphicx}
\graphicspath{ {images/vector}}
\usepackage{wrapfig}
\usepackage{epsfig}
\usepackage{type1cm, lettrine}

%Footnotes
\usepackage[hang, flushmargin,perpage]{footmisc}
\renewcommand{\thefootnote}{\alph{footnote}}
%\renewcommand{\thefootnote}{\fnsymbol{footnote}}

%Bible Verses
\newcommand{\bv}[2]{\textsuperscript{#1}{#2}}

%\newcommand{\qv}[4]{
	%\begin{quoting}
	%\marginnote{\textit{\small{#1}}}
	%{\textsuperscript{#3}}#2
	%\end{quoting}
	%#4}
%\newcommand{\qvv}[4]{
	%\begin{quoting}
	%{\textsuperscript{#3}}#2
	%\end{quoting}
	%\marginnote{\textit{\small{#1}}}
	%#4}
\interfootnotelinepenalty=10000

%For when a contemporary character has a long quotation
\newcommand{\canticle}[1]{
\par
\noindent
%\hspace{2em}#1
\begin{verse}
	#1
\end{verse}
%\begin{quotation}
%	#1
%\end{quotation}
}
\newcommand{\VN}[1]{
\textsuperscript{#1}%
}
%Old Testament Quote
\newcommand{\otQuote}[2]{
\par
\noindent
	%\begin{verse}
	%	#2\mref{#1}%
	%\end{verse}
	\begin{quoting}
	%\mref{#1}
	#2\footnote{#1}%
	\end{quoting}
}
\newcommand{\shortQ}[2]{
	``#2''\footnote{#1}%
}
%Marginal Glosses \& References
\usepackage{microtype}
\usepackage[Ragged, shape=up, size=script]{sidenotesplus}
\newcommand{\mcomm}[1]{\sidenote*{#1}}
\newcommand{\forcewhite}[1]{\textcolor{White}{#1}}
\newcommand{\mref}[1]{
%\marginpar{\small{\textit{#1}}}
\footnote{#1}
}
%Different font for editorial additions.
\newcommand{\supptext}[1]{%
\texttt{\footnotesize{#1}}%
}

\newcommand{\chaphead}[1]{
\fancyhead[LE,RO]{#1}
\section{#1}
}
\newcommand{\chapdesc}[1]{
	%\marginpar{\small{\textit{#1}}}
	%
	%\begin{center}
	%\small{\textit{#1}}
	%\end{center}
	%\par
	%\noindent
	\subsection{#1}
	\noindent
}
%Formatting for the description of sections within the chapter.
\newcommand{\chapsec}[1]{
	%\par
	%\marginpar{\\{\small{\textit{#1}}}}
	%\subsection{#1}
	\par
	\begin{center}
		\textit{#1}
	\end{center}
	\par
	\noindent
}
%Try to avoid chapters at bottom.
\usepackage[nobottomtitles*]{titlesec}
%Trying to fix marginnotes for left text fixing on right...

%Red Letter Option
\usepackage[dvipsnames]{xcolor}
%\newcommand{\redlet}[1]{\textcolor{BrickRed}{#1}}
\newcommand{\redlet}[1]{#1}
\newcommand{\god}[1]{#1}
%\newcommand{\ann}[1]{\footnote[8]{#1}}
%Hanging chapter descriptions
\usepackage{hanging}
\frenchspacing
%\usepackage{mparhack}
%Page Headers
\usepackage{fancyhdr}
\pagestyle{fancy}
\fancyhf{}
\fancyfoot[C]{\thepage}
\fancypagestyle{plain}{
	\fancyhf{}
	\fancyfoot[C]{}
	\renewcommand{\headrulewidth}{0pt}
}
%Adjust Headings
\usepackage{titlesec}
	\titleclass{\chapter}{straight}
	\titleformat{\chapter}[display]{\normalfont\bfseries\scshape\centering}{}{0pt}{\Large}
	\titlespacing*{\chapter}{0pt}{0pt}{5pt}
	\titleformat{\section}[hang]{\centering}{}{0pt}{\large}
	\titlespacing*{\section}{0pt}{2ex}{2ex}
	\titleformat{\subsection}[hang]{\centering}{}{0pt}{\small \itshape}
	\titlespacing*{\subsection}{0pt}{1ex}{1ex}
\begin{document}
	\begin{titlepage}
		\begin{center}
			\vspace*{2cm}
			\par
			{
			\textsc{
			\LARGE{The}\\
			\vspace{1ex}
			\Huge{New}\\
			\vspace{1.5ex}
			\huge{Testament}
			}\\
			\vspace{2ex}
			%\Large{Containing the Old Testament,\\
			%the Apocrypha,\\
			%and the New Testament}
			%\Large{Containing the New Testament}
			\Large{of Our Lord Jesus Christ}
			\par
			\vspace{.5cm}
			%\large{Revised according to the Original Languages}
			\large{\textit{Anglican Patrimony Version}}
			}
			\par
			\vspace{1cm}
			%{\textit{Anglican Patrimony Version}}
			\par
			%\vfill
			\includegraphics[scale=.3]{Pantokrator2.eps}
			\par
			\vspace{1cm}
			\textsc{Anno Domini 2022}
		\end{center}
	\end{titlepage}
    \begin{onecolumn}
	\chapter{Copyright}
	\copyright 2023 Apologia Anglicana, LLC. Some Rights Reserved. Creative Commons Attribution-ShareAlike 4.0 International (https://creativecommons.org/licenses/by-sa/4.0/).
	\end{onecolumn}
	\begin{onecolumn}
	\chapter{Why a New Edition?}
	Ever since the middle twentieth century, there has been a flood of new biblical translations. In recent years, it seems like the more the Bible is translated, the less it is read. With so many translations, what is the need for the \textit{Anglican Patrimony Version}? With all of these many great Bibles, none have sought to translate the Bible with the wisdom of the Anglican tradition. In this version, you will find a translation that is both critical and beautiful.
	\section{Critical \& Beautiful}
The translators of the King James Bible sought to create a translation that used the best original language manuscripts and was equally beautiful \& understandable by the average parishioner. Sadly, the latter has often been neglected, losing the beauty of the King James Bible for post-modern language. It is confusing why the Anglican churches would expect their parishioners to not understand the same language they hear in the liturgy every Sunday and every day during the Daily Office, such as ``thou'' and ``ye.'' Or, even worse, some churches which call themselves ``Anglican'' or ``Episcopal'' have abandoned this treasure of the Anglican patrimony altogether. It is out of love for this treasure that we produce the \textit{Anglican Patrimony Version}.
\section{Textual Basis \& Translation Decisions}
The \textit{Anglican Patrimony Version} is a fresh typeset of the American Standard Version, with light modifications. The ASV did a good job to conservatively revise the Authorised Version with the wisdom of textual criticism. It also sought, again sparingly, to update the language when it was truly necessary and expedient. The goal of this edition is to continue as close to the American Standard Version as possible and to change the base text sparingly, only when necessary to:
\begin{enumerate}
\item Conform it to the earliest manuscripts we have.
\item Replace words that truly cannot properly communicate the underlying meaning to the average church-going Anglican without a dictionary.
\end{enumerate}
We hope you come to value and appreciate God's Word by means of this version.
	\end{onecolumn}
	\twocolumn
	\newpage
	\setcounter{tocdepth}{0}
	\tableofcontents
	\newpage
	\begin{onecolumn}	
\chapter{The Problem of Textual Criticism}
The reality of textual criticism has posed quite the difficulty for many churches, leading to great confusion and schisms.
\par
The \textit{Anglican Patrimony Version} is meant to be read, both in personal use and in the parish setting. Therefore, it does not make extensive use of textual footnotes and explanations. However, great care has been taken to be faithful to the original languages, as we best understand them. Therefore, textual criticism has been essential for this edition. Due to this, I thought it critical to provide not just an academic explanation of textual criticism but a theological reflection and explanation for what textual criticism \textit{means} (instead of leaving vague footnotes to confuse the reader like many recent translations have sadly done).
\section{What is Textual Criticism?}
Textual criticism is the result of three realities:
\begin{enumerate}
	\item The original autographs of the Scriptures were written by God through human authors.
	\item We do not possess those original autographs.
	\item The manuscripts---of those autographs---which we possess differ from each other in certain places.
\end{enumerate}
The goal of the text critic is to discern which variation reflects what was actually present in the original autograph.
\section{An Unsure Text?}
Some Christians desire to dismiss textual criticism categorically. If the Catholic Church is guided by the Holy Ghost, how could it lose the original text? Also, if the Church is the guardian of the Sacred Scriptures, then how could it fail in its mission so as to not know what those Scriptures are? Even worse, does the Church then need to depend on the natural sciences to perform its supernatural mission?

These are all very good concerns, and they need to be taken seriously and answered before we move forward with making critical editions of the Scriptures.
\subsection{Was the Original Text Lost?}
To the first point, the issue our current situation presents is not having lost text and needing to recover it. It is much better! Rather, we have too much text. For example, we know that, over time, Christians were reading and using their Bibles. Sometimes they wrote theological commentary in the margin. Unfortunately, the margin is also where a scribe would write a verse he initially forgot to copy in the body of the text (they didn't have erasers or White-Out). So, sometimes, the commentary would be copied into the main text by a later copyist: intermingling the inspired text of Sacred Scripture with commentary on that Scripture. This is not the only problem that can occur in transmission, but it shows that we are not digging for ``lost books'' or ``lost verses'' to add. Likewise, we are not trying to remove inspired text from the Bible. Instead, the goal of the text critic is to untangle what scribes centuries before have tangled and intermingled.

The original text is not lost. It is simply that non-inspired words---or even sometimes mis-copying or accidentally skipping a line---have been included in some manuscripts (though those same errors do not happen in others). The issue then is detangling the text based on all of the manuscripts we have.
\subsection{How Does the Church Guard the Scriptures?}
To the second point, the Word of God constitutes the Catholic Church, and the Catholic Church receives the Word of God and cherishes it. However, at its divine founding, the Church's understanding of the Sacred Scriptures was not immediate and fully comprehensive. Instead, since the Church is guided by the Holy Ghost, it discerns the Scriptures over time: both their meaning and extent.

This is seen in the canon of the Scriptures. There was a time when there was not catholic agreement on the canon of the New Testament. Yet, over time, the Holy Ghost guided the Church, and the Church listened to the Good Shepherd's voice, receiving only those books which are inspired. Similarly, the Church right now is discerning His voice regarding the Old Testament canon, where there is not catholic agreement on exactly which books belong. In the same way, there is a discernment that happens over time regarding verses. This is not failure: it is simply the divine timescale. And part of living on the divine timescale is reforming ourselves constantly unto the law of God.

Though, it must be said, it is the sad reality that the Church can perform her mission imperfectly. Accretions can develop which obscure the face of Christ, hinder the work of the Church, and even cause theological missteps. However, at the same time, the Sacred Scriptures can be ornamented with textual jewels, so to say, of good commentary. While the technical language for these interpolations is ``corruption,'' it is the editor's opinion that most of them are better called ornamentation. And while the ornamentation itself is not bad, it is necessary to be able look at the image of Christ as it is purely, without ornamentation, lest the ornaments be confused for the image. This is the work of textual criticism: discern the inspired and non-inspired words, rejecting nothing that is good but putting it in its proper place.

\subsection{Does the Church Need Natural Science?}
To the third point, the Catholic Church has always needed natural realities for her mission. That is at the essence of the Incarnation. Natural realities are never destroyed but raised up for a supernatural purpose: Grace perfects (not destroys) nature. Therefore, the natural and supernatural ought not to be put in opposition. Instead, the natural---such as natural law, natural substances, and natural skills---are embedded and incorporated into the Scriptures: into the very promises of God. For example, to be faithful to the Sacred Scriptures, knowledge of the natural language of Greek is necessary (something the Western Church sadly neglected for some time). Likewise, to give the Scriptures to others, it requires gathering paper and collecting \& copying manuscripts and giving them to others: all of which are natural skills and substances. It is at this point where the natural science of textual criticism exists at the service of the Word for the Church.
\subsection{Conclusion}
I hope this provides a helpful primer and apologetic to textual criticism. This critical edition has been prepared out of love for the Scriptures: the key treasure \& guiding light of the Church. So, while some verses may be moved into the margins or a separate section, this is not to remove and undermine the Word of God. Instead, it is to honour the Word of God by keeping the inspired text front-and-centre and the excellent notes and spiritual commentary ready on the side for Christians to appreciate in its proper context.
\end{onecolumn}
	%\onecolumn
\vspace*{50ex}
\begin{center}
	\rule{15em}{.25mm}\\
	\vspace{1.5ex}
	{\Large{\scshape The Old Testament}}
	\par
	\rule{15em}{.25mm}
\end{center}
\vfill
\twocolumn
	%\chapter{The Book of the Prophet Malachi}
\fancyhead[RE,LO]{The Book of Malachi}
\chaphead{Chapter I}
\chapdesc{God's Love for Israel}
\lettrine[image=true, lines=4, findent=3pt, nindent=0pt]{T-Mal.eps}{he} burden of the word of Yahweh to Israel by Malachi.
\bv{2}{I have loved you, saith Yahweh. Yet ye say, Wherein hast thou loved us? Was not Esau Jacob's brother? saith Yahweh: yet I loved Jacob;}
\bv{3}{but Esau I hated, and made his mountains a desolation, and \supptext{gave} his heritage to the jackals of the wilderness.}
\bv{4}{Whereas Edom saith, We are beaten down, but we will return and build the waste places; thus saith Yahweh of hosts, They shall build, but I will throw down; and men shall call them The border of wickedness, and The people against whom Yahweh hath indignation for ever.}
\bv{5}{And your eyes shall see, and ye shall say, Yahweh be magnified beyond the border of Israel.}
\chapsec{The Sins of the Restoration Priests}
\bv{6}{A son honoreth his father, and a servant his master: if then I am a father, where is mine honor? and if I am a master, where is my fear? saith Yahweh of hosts unto you, O priests, that despise my name. And ye say, Wherein have we despised thy name?}
\bv{7}{Ye offer polluted bread upon mine altar. And ye say, Wherein have we polluted thee? In that ye say, The table of Yahweh is contemptible.}
\bv{8}{And when ye offer the blind for sacrifice, it is no evil! and when ye offer the lame and sick, it is no evil! Present it now unto thy governor; will he be pleased with thee? or will he accept thy person? saith Yahweh of hosts.}
\bv{9}{And now, I pray you, entreat the favor of God, that he may be gracious unto us: this hath been by your means: will he accept any of your persons? saith Yahweh of hosts.}
\par
\bv{10}{Oh that there were one among you that would shut the doors, that ye might not kindle \supptext{fire on} mine altar in vain! I have no pleasure in you, saith Yahweh of hosts, neither will I accept an offering at your hand.}
\bv{11}{For from the rising of the sun even unto the going down of the same my name \supptext{shall be} great among the Gentiles; and in every place incense \supptext{shall be} great among the Gentiles, saith Yahweh of hosts.}
\bv{12}{But ye profane it, in that ye say, The table of Yahweh is polluted, and the fruit thereof, even its food, is contemptible.}
\par
\bv{13}{Ye say also, Behold, what a weariness is it! and ye have snuffed at it, saith Yahweh of hosts; and ye have brought that which was taken by violence, and the lame, and the sick; thus ye bring the offering: should I accept this at your hand? saith Yahweh.}
\bv{14}{But cursed be the deceiver, who hath in his flock a male, and voweth, and sacrificeth unto the Lord a blemished thing; for I am a great King, saith Yahweh of hosts, and my name is terrible among the Gentiles.}
\chaphead{Chapter II}
\chapdesc{Message to the Priests Continued}
\lettrine[image=true, lines=4, findent=3pt, nindent=0pt]{A-Mal.eps}{nd} now, O ye priests, this commandment is for you.
\bv{2}{If ye will not hear, and if ye will not lay it to heart, to give glory unto my name, saith Yahweh of hosts, then will I send the curse upon you, and I will curse your blessings; yea, I have cursed them already, because ye do not lay it to heart.}
\bv{3}{Behold, I will rebuke your seed, and will spread dung upon your faces, even the dung of your feasts; and ye shall be taken away with it.}
\bv{4}{And ye shall know that I have sent this commandment unto you, that my covenant may be with Levi, saith Yahweh of hosts.}
\bv{5}{My covenant was with him of life and peace; and I gave them to him that he might fear; and he feared me, and stood in awe of my name.}
\bv{6}{The law of truth was in his mouth, and unrighteousness was not found in his lips: he walked with me in peace and uprightness, and turned many away from iniquity.}
\bv{7}{For the priest's lips should keep knowledge, and they should seek the law at his mouth; for he is the messenger of Yahweh of hosts.}
\bv{8}{But ye are turned aside out of the way; ye have caused many to stumble in the law; ye have corrupted the covenant of Levi, saith Yahweh of hosts.}
\bv{9}{Therefore have I also made you contemptible and base before all the people, according as ye have not kept my ways, but have had respect of persons in the law.}
\chapsec{Sins against Brotherhood}
\bv{10}{Have we not all one father? hath not one God created us? why do we deal treacherously every man against his brother, profaning the covenant of our fathers?}
\chapsec{Sins against Family}
\bv{11}{Judah hath dealt treacherously, and an abomination is committed in Israel and in Jerusalem; for Judah hath profaned the holiness of Yahweh which he loveth, and hath married the daughter of a foreign god.}
\bv{12}{Yahweh will cut off, to the man that doeth this, him that waketh and him that answereth, out of the tents of Jacob, and him that offereth an offering unto Yahweh of hosts.}
\bv{13}{And this again ye do: ye cover the altar of Yahweh with tears, with weeping, and with sighing, insomuch that he regardeth not the offering any more, neither receiveth it with good will at your hand.}
\bv{14}{Yet ye say, Wherefore? Because Yahweh hath been witness between thee and the wife of thy youth, against whom thou hast dealt treacherously, though she is thy companion, and the wife of thy covenant.}
\bv{15}{And did he not make one, although he had the residue of the Spirit? And wherefore one? He sought a godly seed. Therefore take heed to your spirit, and let none deal treacherously against the wife of his youth.}
\bv{16}{For I hate putting away, saith Yahweh, the God of Israel, and him that covereth his garment with violence, saith Yahweh of hosts: therefore take heed to your spirit, that ye deal not treacherously.}
\bv{17}{Ye have wearied Yahweh with your words. Yet ye say, Wherein have we wearied him? In that ye say, Every one that doeth evil is good in the sight of Yahweh, and he delighteth in them; or where is the God of justice?}
\chaphead{Chapter III}
\chapdesc{Ministries of St. John \& Jesus Foretold}
\lettrine[image=true, lines=4, findent=3pt, nindent=0pt]{B-Mal.eps}{ehold} I send my messenger, and he shall prepare the way before me: and the Lord, whom ye seek, will suddenly come to his temple; and the messenger of the covenant, whom ye desire, behold, he cometh, saith Yahweh of hosts.
\bv{2}{But who can abide the day of his coming? and who shall stand when he appeareth? for he is like a refiner's fire, and like fullers' soap:}
\bv{3}{and he will sit as a refiner and purifier of silver, and he will purify the sons of Levi, and refine them as gold and silver; and they shall offer unto Yahweh offerings in righteousness.}
\bv{4}{Then shall the offering of Judah and Jerusalem be pleasant unto Yahweh, as in the days of old, and as in ancient years.}
\bv{5}{And I will come near to you to judgement; and I will be a swift witness against the sorcerers, and against the adulterers, and against the false swearers, and against those that oppress the hireling in his wages, the widow, and the fatherless, and that turn aside the sojourner \supptext{from his right}, and fear not me, saith Yahweh of hosts.}
\bv{6}{For I, Yahweh, change not; therefore ye, O sons of Jacob, are not consumed.}
\chapsec{The People Have Robbed God}
\bv{7}{From the days of your fathers ye have turned aside from mine ordinances, and have not kept them. Return unto me, and I will return unto you, saith Yahweh of hosts. But ye say, Wherein shall we return?}
\bv{8}{Will a man rob God? yet ye rob me. But ye say, Wherein have we robbed thee? In tithes and offerings.}
\bv{9}{Ye are cursed with the curse; for ye rob me, even this whole nation.}
\bv{10}{Bring ye the whole tithe into the store-house, that there may be food in my house, and prove me now herewith, saith Yahweh of hosts, if I will not open you the windows of heaven, and pour you out a blessing, that there shall not be room enough \supptext{to receive it}.}
\bv{11}{And I will rebuke the devourer for your sakes, and he shall not destroy the fruits of your ground; neither shall your vine cast its fruit before the time in the field, saith Yahweh of hosts.}
\bv{12}{And all nations shall call you happy; for ye shall be a delightsome land, saith Yahweh of hosts.}
\bv{13}{Your words have been stout against me, saith Yahweh. Yet ye say, What have we spoken against thee?}
\bv{14}{Ye have said, It is vain to serve God; and what profit is it that we have kept his charge, and that we have walked mournfully before Yahweh of hosts?}
\bv{15}{and now we call the proud happy; yea, they that work wickedness are built up; yea, they tempt God, and escape.}
\chapsec{The Faithful Remnant}
\bv{16}{Then they that feared Yahweh spake one with another; and Yahweh hearkened, and heard, and a book of remembrance was written before him, for them that feared Yahweh, and that thought upon his name.}
\bv{17}{And they shall be mine, saith Yahweh of hosts, \supptext{even} mine own possession, in the day that I make; and I will spare them, as a man spareth his own son that serveth him.}
\bv{18}{Then shall ye return and discern between the righteous and the wicked, between him that serveth God and him that serveth him not.}
\chaphead{Chapter IV}
\chapdesc{The Day of Yahweh}
\lettrine[image=true, lines=4, findent=3pt, nindent=0pt]{F-Mal.eps}{or} behold, the day cometh, it burneth as a furnace; and all the proud, and all that work wickedness, shall be stubble; and the day that cometh shall burn them up, saith Yahweh of hosts, that it shall leave them neither root nor branch.
\chapsec{The Second Coming of Christ}
\bv{2}{But unto you that fear my name shall the sun of righteousness arise with healing in its wings; and ye shall go forth, and gambol as calves of the stall.}
\bv{3}{And ye shall tread down the wicked; for they shall be ashes under the soles of your feet in the day that I make, saith Yahweh of hosts.}
\bv{4}{Remember ye the law of Moses my servant, which I commanded unto him in Horeb for all Israel, even statutes and ordinances.}
\chapsec{Elijah to Come before the Day of Yahweh}
\bv{5}{Behold, I will send you Elijah the prophet before the great and terrible day of Yahweh come.}
\bv{6}{And he shall turn the heart of the fathers to the children, and the heart of the children to their fathers; lest I come and smite the earth with a curse.}
	%\input{./books/Apocrypha/Apocrypha.tex}
	\fancyhead[RE,LO]{}
\fancyhead[LE,RO]{}
\onecolumn
\vspace*{50ex}
\begin{center}
	\rule{15em}{.25mm}\\
	\vspace{1.5ex}
	{\Large{\scshape Prayer before Reading Sacred Scripture}}
	\par
	\rule{15em}{.25mm}
\end{center}
\vfill
\twocolumn
	\fancyhead[RE,LO]{The Prayer of Manasseh}
\chapter{The Prayer of Manasseh}
\fancyhead[LE,RO]{}
\chapdesc{When He Was Held Captive in Babylon}
\lettrine[image=true, lines=4, findent=3pt, nindent=0pt]{O.eps}{} LORD Almighty,\mcomm{that art in heaven} thou God of our fathers, of Abraham, and Isaac, and Jacob, and of their righteous seed;
\bv{2}{who hast made heaven and earth, with all the ornament\mcomm{order or array (Gen. 2:1 LXX)} thereof;}
\bv{3}{who hast bound the sea by the word of thy commandment; who hast shut up the deep, and sealed it by thy terrible and glorious name;}
\bv{4}{whom all things fear, yea, tremble before thy power;}
\bv{5}{for the majesty of thy glory cannot be borne, and the anger of thy threatening toward sinners is importable:}
\bv{6}{thy merciful promise is unmeasurable and unsearchable;}
\bv{7}{for thou art the Lord Most High, of great compassion, longsuffering and abundant in mercy, and repentest of the evils of men.}
\par
\bv{8}{Thou, O Lord, according to thy great goodness hast promised repentance and forgiveness to them that have sinned against thee: and of thine infinite mercies\mcomm{Thou hast promised that repentance shall be the way for them to return to thee.} hast appointed repentance unto sinners, that they may be saved. Thou therefore, O Lord, that art the God of the just, hast not appointed repentance to the just, to Abraham, and Isaac, and Jacob,\mcomm{He calls their sins nothing, but attributes unto them righteousness.} which have not sinned against thee; but thou hast appointed repentance unto me that am a sinner:}
\bv{9}{for I have sinned above the number of the sands of the sea.}
\par
My transgressions are multiplied, O Lord: my transgressions are multiplied, and I am not worthy to behold and see the height of heaven for the multitude of mine iniquities.
\bv{10}{I am bowed down with many iron bands, that I cannot lift up mine head by reason of my sins, neither have I any respite: for I have provoked thy wrath, and done that which is evil before thee: I did not thy will, neither kept I thy commandments: I have set up abominations, and have multiplied detestable things\mcomm{stumbling-blocks}.}
\par
\bv{11}{Now therefore I bow the knee of mine heart, beseeching thee of grace.}
\bv{12}{I have sinned, O Lord, I have sinned, and I acknowledge mine iniquities:}
\bv{13}{but, I humbly beseech thee, forgive me, O Lord, forgive me, and destroy me not with mine iniquities. Be not angry with me for ever, by reserving evil for me; neither condemn me into the lower parts of the earth. For thou, O Lord\mcomm{O God}, art the God of them that repent;}
\bv{14}{and in me thou wilt shew all thy goodness: for thou wilt save me, that am unworthy, according to thy great mercy.}
\par
\bv{15}{And I will praise thee for ever all the days of my life: for all the host of heaven doth sing thy praise, and thine is the glory for ever and ever. Amen.}
\fancyhead[RE,LO]{}
\fancyhead[LE,RO]{}
	\clearpage
	\fancyhead[RE,LO]{}
\fancyhead[LE,RO]{}
\onecolumn
\vspace*{50ex}
\begin{center}
	\rule{15em}{.25mm}\\
	\vspace{1.5ex}
	{\Large{\scshape The New Testament}}
	\par
	\rule{15em}{.25mm}
\end{center}
\vfill
\twocolumn
	\clearpage
	\chapter{The Holy Gospel of Jesus Christ according to Saint Matthew}
\fancyhead[RE,LO]{The Gospel according to Matthew}
\chaphead{Chapter I}
\chapdesc{The Genealogy of Jesus Christ}
\lettrine[image=true, lines=4, findent=3pt, nindent=0pt]{NT/Matthew/Matt1-T.ps}{he} book of the generation of Jesus Christ, the son of David, the son of Abraham.
\bv{2}{Abraham begat Isaac; and Isaac begat Jacob; and Jacob begat Judah and his brethren;}
\bv{3}{and Judah begat Perez and Zerah of Tamar; and Perez begat Hezron; and Hezron begat Ram;}
\bv{4}{and Ram begat Amminadab; and Amminadab begat Nahshon; and Nahshon begat Salmon;}
\bv{5}{and Salmon begat Boaz of Rahab; and Boaz begat Obed of Ruth; and Obed begat Jesse;}
\bv{6}{and Jesse begat David the king.}
\par
And David begat Solomon of her \supptext{that had been the wife} of Uriah;
\bv{7}{and Solomon begat Rehoboam; and Rehoboam begat Abijah; and Abijah begat Asa;}
\bv{8}{and Asa begat Jehoshaphat; and Jehoshaphat begat Joram; and Joram begat Uzziah;}
\bv{9}{and Uzziah begat Jotham; and Jotham begat Ahaz; and Ahaz begat Hezekiah;}
\bv{10}{and Hezekiah begat Manasseh; and Manasseh begat Amon; and Amon begat Josiah;}
\bv{11}{and Josiah begat Jechoniah and his brethren, at the time of the carrying away to Babylon.}
\par
\bv{12}{And after the carrying away to Babylon, Jechoniah begat Shealtiel; and Shealtiel begat Zerubbabel;}
\bv{13}{and Zerubbabel begat Abiud; and Abiud begat Eliakim; and Eliakim begat Azor;}
\bv{14}{and Azor begat Sadoc; and Sadoc begat Achim; and Achim begat Eliud;}
\bv{15}{and Eliud begat Eleazar; and Eleazar begat Matthan; and Matthan begat Jacob;}
\bv{16}{and Jacob begat Joseph the husband of Mary, of whom was born Jesus, who is called Christ.}
\par
\bv{17}{So all the generations from Abraham unto David are fourteen generations; and from David unto the carrying away to Babylon fourteen generations; and from the carrying away to Babylon unto the Christ fourteen generations.}
\chapsec{The Birth of Jesus Christ}
\bv{18}{Now the birth of Jesus Christ was on this wise: When his mother Mary had been betrothed to Joseph, before they came together she was found with child of the Holy Ghost.}
\bv{19}{And Joseph her husband, being a righteous man, and not willing to make her a public example, was minded to put her away privily.}
\bv{20}{But when he thought on these things, behold, an angel of the Lord appeared unto him in a dream, saying, ``Joseph, thou son of David, fear not to take unto thee Mary thy wife: for that which is conceived in her is of the Holy Ghost.}
\bv{21}{And she shall bring forth a son; and thou shalt call his name {\scshape Jesus}; for it is he that shall save his people from their sins.''}
\bv{22}{Now all this is come to pass, that it might be fulfilled which was spoken by the Lord through the prophet, saying,}
\otQuote{Is. 7:14}{\bv{23}{Behold, the virgin shall be with child, and shall bring forth a son, 
And they shall call his name Immanuel;}}
which is, being interpreted, ``God with us.''
\bv{24}{And Joseph arose from his sleep, and did as the angel of the Lord commanded him, and took unto him his wife;}
\bv{25}{and knew her not till she had brought forth a son: and he called his name {\scshape Jesus}.}
\chaphead{Chapter II}
\chapdesc{Visit of the Magi}
\lettrine[image=true, lines=4, findent=3pt, nindent=0pt]{NT/Matthew/Mt-Now.eps}{ow} when Jesus was born in Bethlehem of Judæa in the days of Herod the king, behold, Wise-men from the east came to Jerusalem, saying,
\bv{2}{``Where is he that is born King of the Jews? for we saw his star in the east, and are come to worship him.''}
\bv{3}{And when Herod the king heard it, he was troubled, and all Jerusalem with him.}
\bv{4}{And gathering together all the chief priests and scribes of the people, he inquired of them where the Christ should be born.}
\bv{5}{And they said unto him, ``In Bethlehem of Judæa:'' for thus it is written through the prophet,}
\otQuote{Mic. 5:2}{\bv{6}{And thou Bethlehem, land of Judah,
Art in no wise least among the princes of Judah:
For out of thee shall come forth a governor,
Who shall be shepherd of my people Israel.}}
\bv{7}{Then Herod privily called the Wise-men, and learned of them exactly what time the star appeared.}
\bv{8}{And he sent them to Bethlehem, and said, ``Go and search out exactly concerning the young child; and when ye have found \supptext{him}, bring me word, that I also may come and worship him.''}
\par
\bv{9}{And they, having heard the king, went their way; and lo, the star, which they saw in the east, went before them, till it came and stood over where the young child was.}
\bv{10}{And when they saw the star, they rejoiced with exceeding great joy.}
\bv{11}{And they came into the house and saw the young child with Mary his mother; and they fell down and worshipped him; and opening their treasures they offered unto him gifts, gold and frankincense and myrrh.}
\bv{12}{And being warned \supptext{of God} in a dream that they should not return to Herod, they departed into their own country another way.}
\chapsec{The Flight into Egypt}
\bv{13}{Now when they were departed, behold, an angel of the Lord appeareth to Joseph in a dream, saying, ``Arise and take the young child and his mother, and flee into Egypt, and be thou there until I tell thee: for Herod will seek the young child to destroy him.''}
\bv{14}{And he arose and took the young child and his mother by night, and departed into Egypt;}
\bv{15}{and was there until the death of Herod: that it might be fulfilled which was spoken by the Lord through the prophet, saying,}
\otQuote{Hos. 11:1}{Out of Egypt did I call my son.}
\chapsec{Herod's Slaughter of the Innocents}
\bv{16}{Then Herod, when he saw that he was mocked of the Wise-men, was exceeding wroth, and sent forth, and slew all the male children that were in Bethlehem, and in all the borders thereof, from two years old and under, according to the time which he had exactly learned of the Wise-men.}
\bv{17}{Then was fulfilled that which was spoken through Jeremiah the prophet, saying,}
\otQuote{Jer. 31:15}{\bv{18}{A voice was heard in Ramah,
Weeping and great mourning,
Rachel weeping for her children;
And she would not be comforted, because they are not.}}
\chapsec{The Return to Nazareth}
\bv{19}{But when Herod was dead, behold, an angel of the Lord appeareth in a dream to Joseph in Egypt, saying,}
\bv{20}{``Arise and take the young child and his mother, and go into the land of Israel: for they are dead that sought the young child's life.''}
\bv{21}{And he arose and took the young child and his mother, and came into the land of Israel.}
\bv{22}{But when he heard that Archelaus was reigning over Judæa in the room of his father Herod, he was afraid to go thither; and being warned \supptext{of God} in a dream, he withdrew into the parts of Galilee,}
\bv{23}{and came and dwelt in a city called Nazareth; that it might be fulfilled which was spoken through the prophets, that he should be called a Nazarene.}
\chaphead{Chapter III}
\chapdesc{The Ministry of St. John the Baptist}
\lettrine[image=true, lines=4, findent=3pt, nindent=0pt]{NT/Matthew/Mt-And.eps}{nd} in those days cometh John the Baptist, preaching in the wilderness of Judæa, saying,
\bv{2}{``Repent ye; for the kingdom of heaven is at hand.''}
\bv{3}{For this is he that was spoken of through Isaiah the prophet, saying,}
\otQuote{Is. 40:3}{The voice of one crying in the wilderness,
Make ye ready the way of the Lord,
Make his paths straight.}
\bv{4}{Now John himself had his raiment of camel's hair, and a leathern girdle about his loins; and his food was locusts and wild honey.}
\bv{5}{Then went out unto him Jerusalem, and all Judæa, and all the region round about the Jordan;}
\bv{6}{and they were baptised of him in the river Jordan, confessing their sins.}
\bv{7}{But when he saw many of the Pharisees and Sadducees coming to his baptism, he said unto them, ``Ye offspring of vipers, who warned you to flee from the wrath to come?}
\bv{8}{Bring forth therefore fruit worthy of repentance:}
\bv{9}{and think not to say within yourselves, `We have Abraham to our father:' for I say unto you, that God is able of these stones to raise up children unto Abraham.}
\bv{10}{And even now the axe lieth at the root of the trees: every tree therefore that bringeth not forth good fruit is hewn down, and cast into the fire.}
\bv{11}{I indeed baptise you in water unto repentance: but he that cometh after me is mightier than I, whose shoes I am not worthy to bear: he shall baptise you in the Holy Ghost and \supptext{in} fire:}
\bv{12}{whose fan is in his hand, and he will thoroughly cleanse his threshing-floor; and he will gather his wheat into the garner, but the chaff he will burn up with unquenchable fire.''}
\chapsec{Baptism of Jesus}
\bv{13}{Then cometh Jesus from Galilee to the Jordan unto John, to be baptised of him.}
\bv{14}{But John would have hindered him, saying, ``I have need to be baptised of thee, and comest thou to me?''}
\bv{15}{But Jesus answering said unto him, ``Suffer \supptext{it} now: for thus it becometh us to fulfil all righteousness.'' Then he suffereth him.}
\bv{16}{And Jesus, when he was baptised, went up straightway from the water: and lo, the heavens were opened unto him, and he saw the Spirit of God descending as a dove, and coming upon him;}
\bv{17}{and lo, a voice out of the heavens, saying, \god{``This is my beloved Son, in whom I am well pleased.''}}
\chaphead{Chapter IV}
\chapdesc{The Temptation of Jesus}
\lettrine[image=true, lines=4, findent=3pt, nindent=0pt]{T.ps}{hen} was Jesus led up of the Spirit into the wilderness to be tempted of the devil.
\bv{2}{And when he had fasted forty days and forty nights, he afterward hungered.}
\bv{3}{And the tempter came and said unto him, ``If thou art the Son of God, command that these stones become bread.''}
\bv{4}{But he answered and said, \redlet{``It is written,}}
\otQuote{Deut. 8:3}{\redlet{Man shall not live by bread alone, but by every word that proceedeth out of the mouth of God.''}}
\bv{5}{Then the devil taketh him into the holy city; and he set him on the pinnacle of the temple,}
\bv{6}{and saith unto him, ``If thou art the Son of God, cast thyself down: for it is written,}
\otQuote{Ps. 91:11}{He shall give his angels charge concerning thee:}
{and,}
\otQuote{Ps. 91:12}{On their hands they shall bear thee up,
Lest haply thou dash thy foot against a stone.''}
\bv{7}{Jesus said unto him, \redlet{``Again it is written,}}
\otQuote{Deut. 6:16}{Thou shalt not make trial of the Lord thy God.''}
\bv{8}{Again, the devil taketh him unto an exceeding high mountain, and showeth him all the kingdoms of the world, and the glory of them;}
\bv{9}{and he said unto him, ``All these things will I give thee, if thou wilt fall down and worship me.''}
\bv{10}{Then saith Jesus unto him, \redlet{``Get thee hence, Satan: for it is written,}}
\otQuote{Deut. 6:13}{\redlet{Thou shalt worship the Lord thy God, and him only shalt thou serve.''}}
\bv{11}{Then the devil leaveth him; and behold, angels came and ministered unto him.}
\chapsec{Jesus Begins his Public Ministry}
\bv{12}{Now when he heard that John was delivered up, he withdrew into Galilee;}
\bv{13}{and leaving Nazareth, he came and dwelt in Capernaum, which is by the sea, in the borders of Zebulun and Naphtali:}
\bv{14}{that it might be fulfilled which was spoken through Isaiah the prophet, saying,}
\otQuote{Is. 9:1-2}{\bv{15}{The land of Zebulun and the land of Naphtali,
Toward the sea, beyond the Jordan,
Galilee of the Gentiles,}
\bv{16}{The people that sat in darkness
Saw a great light,
And to them that sat in the region and shadow of death,
To them did light spring up.}}
\chapsec{The Call of Sts. Peter \& Andrew}
\bv{17}{From that time began Jesus to preach, and to say, \redlet{``Repent ye; for the kingdom of heaven is at hand.''}}
\bv{18}{And walking by the sea of Galilee, he saw two brethren, Simon who is called Peter, and Andrew his brother, casting a net into the sea; for they were fishers.}
\bv{19}{And he saith unto them, \redlet{``Come ye after me, and I will make you fishers of men.''}}
\bv{20}{And they straightway left the nets, and followed him.}
\chapsec{The Call of Sts. James \& John}
\bv{21}{And going on from thence he saw two other brethren, James the \supptext{son} of Zebedee, and John his brother, in the boat with Zebedee their father, mending their nets; and he called them.}
\bv{22}{And they straightway left the boat and their father, and followed him.}
\bv{23}{And Jesus went about in all Galilee, teaching in their synagogues, and preaching the gospel of the kingdom, and healing all manner of disease and all manner of sickness among the people.}
\bv{24}{And the report of him went forth into all Syria: and they brought unto him all that were sick, holden with divers diseases and torments, possessed with demons, and epileptic, and palsied; and he healed them.}
\bv{25}{And there followed him great multitudes from Galilee and Decapolis and Jerusalem and Judæa and \supptext{from} beyond the Jordan.}
\chaphead{Chapter V}
\chapdesc{The Sermon on the Mount}\mcomm{The Beatitudes}
\lettrine[image=true, lines=4, findent=3pt, nindent=0pt]{Mt-A.eps}{nd} seeing the multitudes, he went up into the mountain: and when he had sat down, his disciples came unto him:
\bv{2}{and he opened his mouth and taught them, saying,}
\bv{3}{\redlet{``Blessed are the poor in spirit: for theirs is the kingdom of heaven.}}
\bv{4}{\redlet{Blessed are they that mourn: for they shall be comforted.}}
\bv{5}{\redlet{Blessed are the meek: for they shall inherit the earth.}}
\bv{6}{\redlet{Blessed are they that hunger and thirst after righteousness: for they shall be filled.}}
\bv{7}{\redlet{Blessed are the merciful: for they shall obtain mercy.}}
\bv{8}{\redlet{Blessed are the pure in heart: for they shall see God.}}
\bv{9}{\redlet{Blessed are the peacemakers: for they shall be called sons of God.}}
\bv{10}{\redlet{Blessed are they that have been persecuted for righteousness' sake: for theirs is the kingdom of heaven.}}
\bv{11}{\redlet{Blessed are ye when \supptext{men} shall reproach you, and persecute you, and say all manner of evil against you falsely, for my sake.}}
\bv{12}{\redlet{Rejoice, and be exceeding glad: for great is your reward in heaven: for so persecuted they the prophets that were before you.}}
\chapsec{Similitudes of the Believer}
\bv{13}{\redlet{Ye are the salt of the earth: but if the salt have lost its savor, wherewith shall it be salted? it is thenceforth good for nothing, but to be cast out and trodden under foot of men.}}
\bv{14}{\redlet{Ye are the light of the world. A city set on a hill cannot be hid.}}
\bv{15}{\redlet{Neither do \supptext{men} light a lamp, and put it under the bushel, but on the stand; and it shineth unto all that are in the house.}}
\bv{16}{\redlet{Even so let your light shine before men; that they may see your good works, and glorify your Father who is in heaven.}}
\chapsec{Relation of Christ to the Law}
\bv{17}{\redlet{Think not that I came to destroy the law or the prophets: I came not to destroy, but to fulfil.}}
\bv{18}{\redlet{For verily I say unto you, Till heaven and earth pass away, one jot or one tittle shall in no wise pass away from the law, till all things be accomplished.}}
\bv{19}{\redlet{Whosoever therefore shall break one of these least commandments, and shall teach men so, shall be called least in the kingdom of heaven: but whosoever shall do and teach them, he shall be called great in the kingdom of heaven.}}
\bv{20}{\redlet{For I say unto you, that except your righteousness shall exceed \supptext{the righteousness} of the scribes and Pharisees, ye shall in no wise enter into the kingdom of heaven.}}
\chapsec{Sinful Anger}
\bv{21}{\redlet{Ye have heard that it was said to them of old time, `Thou shalt not kill; and whosoever shall kill shall be in danger of the judgement:'}\mref{Ex. 20:13}}
\bv{22}{\redlet{but I say unto you, that every one who is angry with his brother shall be in danger of the judgement; and whosoever shall say to his brother, `Raca,' shall be in danger of the council; and whosoever shall say, `Moreh,' shall be in danger of the hell of fire.}}
\par
\bv{23}{\redlet{If therefore thou art offering thy gift at the altar, and there rememberest that thy brother hath aught against thee,}}
\bv{24}{\redlet{leave there thy gift before the altar, and go thy way, first be reconciled to thy brother, and then come and offer thy gift.}}
\bv{25}{\redlet{Agree with thine adversary quickly, while thou art with him in the way; lest haply the adversary deliver thee to the judge, and the judge deliver thee to the officer, and thou be cast into prison.}}
\bv{26}{\redlet{Verily I say unto thee, Thou shalt by no means come out thence, till thou have paid the last quadrans.}\mcomm{Quadrans: $\frac{1}{64}$\textsuperscript{th} of a denarius (a day's wages)}}
\chapsec{Adultery}
\bv{27}{\redlet{Ye have heard that it was said, `Thou shalt not commit adultery:'}\mref{Ex. 20:14}}
\bv{28}{\redlet{But I say unto you, that every one that looketh on a woman to lust after her hath committed adultery with her already in his heart.}}
\bv{29}{\redlet{And if thy right eye causeth thee to stumble, pluck it out, and cast it from thee: for it is profitable for thee that one of thy members should perish, and not thy whole body be cast into hell.}}
\bv{30}{\redlet{And if thy right hand causeth thee to stumble, cut it off, and cast it from thee: for it is profitable for thee that one of thy members should perish, and not thy whole body go into hell.}}
\chapsec{Divorce}
\bv{31}{\redlet{It was said also, `Whosoever shall put away his wife, let him give her a writing of divorcement:'}\mref{Deut. 24:1}}
\bv{32}{\redlet{but I say unto you, that every one that putteth away his wife, saving for the cause of fornication, maketh her an adulteress: and whosoever shall marry her when she is put away committeth adultery.}}
\chapsec{False Swearing}
\bv{33}{\redlet{Again, ye have heard that it was said to them of old time, `Thou shalt not forswear thyself, but shalt perform unto the Lord thine oaths:'}\mref{Lev. 19:12}}
\bv{34}{\redlet{but I say unto you, Swear not at all; neither by the heaven, for it is the throne of God;}}
\bv{35}{\redlet{nor by the earth, for it is the footstool of his feet; nor by Jerusalem, for it is the city of the great King.}}
\bv{36}{\redlet{Neither shalt thou swear by thy head, for thou canst not make one hair white or black.}}
\bv{37}{\redlet{But let your speech be, `Yea, yea;' `Nay, nay:' and whatsoever is more than these is of the evil \supptext{one}.}}
\chapsec{Hate}
\bv{38}{\redlet{Ye have heard that it was said, `An eye for an eye, and a tooth for a tooth:}'}\mref{Ex. 21:24}
\bv{39}{\redlet{but I say unto you, Resist not him that is evil: but whosoever smiteth thee on thy right cheek, turn to him the other also.}}
\bv{40}{\redlet{And if any man would go to law with thee, and take away thy coat, let him have thy cloak also.}}
\bv{41}{\redlet{And whosoever shall compel thee to go one mile, go with him two.}}
\bv{42}{\redlet{Give to him that asketh thee, and from him that would borrow of thee turn not thou away.}}
\chapsec{Call to Love}
\bv{43}{\redlet{Ye have heard that it was said, `Thou shalt love thy neighbor, and hate thine enemy:'}\mref{cf. Lev. 19:18; Ps. 139:22}}
\bv{44}{\redlet{but I say unto you, Love your enemies, and pray for them that persecute you;}}
\bv{45}{\redlet{that ye may be sons of your Father who is in heaven: for he maketh his sun to rise on the evil and the good, and sendeth rain on the just and the unjust.}}
\bv{46}{\redlet{For if ye love them that love you, what reward have ye? do not even the publicans the same?}}
\bv{47}{\redlet{And if ye salute your brethren only, what do ye more \supptext{than others}? do not even the Gentiles the same?}}
\bv{48}{\redlet{Ye therefore shall be perfect, as your heavenly Father is perfect.}}
\chaphead{Chapter VI}
\chapdesc{Jesus Condemns False Religion}
\lettrine[image=true, lines=4, findent=3pt, nindent=0pt]{T.ps}{\redlet{ake}} \redlet{heed that ye do not your righteousness before men, to be seen of them: else ye have no reward with your Father who is in heaven.}
\bv{2}{\redlet{When therefore thou doest alms, sound not a trumpet before thee, as the hypocrites do in the synagogues and in the streets, that they may have glory of men. Verily I say unto you, They have received their reward.}}
\bv{3}{\redlet{But when thou doest alms, let not thy left hand know what thy right hand doeth:}}
\bv{4}{\redlet{that thine alms may be in secret: and thy Father who seeth in secret shall recompense thee.}}
\bv{5}{\redlet{And when ye pray, ye shall not be as the hypocrites: for they love to stand and pray in the synagogues and in the corners of the streets, that they may be seen of men. Verily I say unto you, They have received their reward.}}
\bv{6}{\redlet{But thou, when thou prayest, enter into thine inner chamber, and having shut thy door, pray to thy Father who is in secret, and thy Father who seeth in secret shall recompense thee.}}
\bv{7}{\redlet{And in praying use not vain repetitions, as the Gentiles do: for they think that they shall be heard for their much speaking.}}
\chapsec{The Lord's Prayer}
\bv{8}{\redlet{Be not therefore like unto them: for your Father knoweth what things ye have need of, before ye ask him.}}
\bv{9}{\redlet{After this manner therefore pray ye:}}
\canticle{\redlet{Our Father who art in heaven, Hallowed be thy name.\\
\bv{10}{Thy kingdom come. Thy will be done, as in heaven, so on earth.}\\
\bv{11}{Give us this day our daily bread.}\\
\bv{12}{And forgive us our debts, as we also have forgiven our debtors.}\\
\bv{13}{And bring us not into temptation, but deliver us from the evil \supptext{one}.}}}\mcomm{For thine is the kingdom, and the power, and the glory, forever and ever. Amen.}
\bv{14}{\redlet{For if ye forgive men their trespasses, your heavenly Father will also forgive you.}}
\bv{15}{\redlet{But if ye forgive not men their trespasses, neither will your Father forgive your trespasses.}}
\par
\bv{16}{\redlet{Moreover when ye fast, be not, as the hypocrites, of a sad countenance: for they disfigure their faces, that they may be seen of men to fast. Verily I say unto you, They have received their reward.}}
\bv{17}{\redlet{But thou, when thou fastest, anoint thy head, and wash thy face;}}
\bv{18}{\redlet{that thou be not seen of men to fast, but of thy Father who is in secret: and thy Father, who seeth in secret, shall recompense thee.}}
\chapsec{Man's True Treasures}
\bv{19}{\redlet{Lay not up for yourselves treasures upon the earth, where moth and rust consume, and where thieves break through and steal:}}
\bv{20}{\redlet{but lay up for yourselves treasures in heaven, where neither moth nor rust doth consume, and where thieves do not break through nor steal:}}
\bv{21}{\redlet{for where thy treasure is, there will thy heart be also.}}
\chapsec{The Light of Man}
\bv{22}{\redlet{The lamp of the body is the eye: if therefore thine eye be single, thy whole body shall be full of light.}}
\bv{23}{\redlet{But if thine eye be evil, thy whole body shall be full of darkness. If therefore the light that is in thee be darkness, how great is the darkness!}}
\bv{24}{\redlet{No man can serve two masters: for either he will hate the one, and love the other; or else he will hold to one, and despise the other. Ye cannot serve God and mammon.}\mcomm{Mammon: money}}
\chapsec{Exhortation against Anxiety}
\bv{25}{\redlet{Therefore I say unto you, Be not anxious for your life, what ye shall eat, or what ye shall drink; nor yet for your body, what ye shall put on. Is not the life more than the food, and the body than the raiment?}}
\bv{26}{\redlet{Behold the birds of the heaven, that they sow not, neither do they reap, nor gather into barns; and your heavenly Father feedeth them. Are not ye of much more value than they?}}
\bv{27}{\redlet{And which of you by being anxious can add one cubit unto the measure of his life?}}
\bv{28}{\redlet{And why are ye anxious concerning raiment? Consider the lilies of the field, how they grow; they toil not, neither do they spin:}}
\bv{29}{\redlet{yet I say unto you, that even Solomon in all his glory was not arrayed like one of these.}}
\bv{30}{\redlet{But if God doth so clothe the grass of the field, which to-day is, and to-morrow is cast into the oven, \supptext{shall he} not much more \supptext{clothe} you, O ye of little faith?}}
\bv{31}{\redlet{Be not therefore anxious, saying, `What shall we eat?' or, `What shall we drink?' or, `Wherewithal shall we be clothed?'}}
\bv{32}{\redlet{For after all these things do the Gentiles seek; for your heavenly Father knoweth that ye have need of all these things.}}
\bv{33}{\redlet{But seek ye first his kingdom, and his righteousness; and all these things shall be added unto you.}}
\bv{34}{\redlet{Be not therefore anxious for the morrow: for the morrow will be anxious for itself. Sufficient unto the day is the evil thereof.}}
\chaphead{Chapter VII}
\chapdesc{False Judgement Condemned}
\lettrine[image=true, lines=4, findent=3pt, nindent=0pt]{J.eps}{\redlet{udge}} \redlet{not, that ye be not judged.}
\bv{2}{\redlet{For with what judgement ye judge, ye shall be judged: and with what measure ye mete, it shall be measured unto you.}}
\bv{3}{\redlet{And why beholdest thou the mote that is in thy brother's eye, but considerest not the beam that is in thine own eye?}}
\bv{4}{\redlet{Or how wilt thou say to thy brother, Let me cast out the mote out of thine eye; and lo, the beam is in thine own eye?}}
\bv{5}{\redlet{Thou hypocrite, cast out first the beam out of thine own eye; and then shalt thou see clearly to cast out the mote out of thy brother's eye.}}
\bv{6}{\redlet{Give not that which is holy unto the dogs, neither cast your pearls before the swine, lest haply they trample them under their feet, and turn and rend you.}}
\chapsec{Encouragement to Pray}
\bv{7}{\redlet{Ask, and it shall be given you; seek, and ye shall find; knock, and it shall be opened unto you:}}
\bv{8}{\redlet{for every one that asketh receiveth; and he that seeketh findeth; and to him that knocketh it shall be opened.}}
\bv{9}{\redlet{Or what man is there of you, who, if his son shall ask him for a loaf, will give him a stone;}}
\bv{10}{\redlet{or if he shall ask for a fish, will give him a serpent?}}
\bv{11}{\redlet{If ye then, being evil, know how to give good gifts unto your children, how much more shall your Father who is in heaven give good things to them that ask him?}}
\chapsec{Summary of Righteousness}
\bv{12}{\redlet{All things therefore whatsoever ye would that men should do unto you, even so do ye also unto them: for this is the law and the prophets.}}
\chapsec{The Two Ways}
\bv{13}{\redlet{Enter ye in by the narrow gate: for wide is the gate, and broad is the way, that leadeth to destruction, and many are they that enter in thereby.}}
\bv{14}{\redlet{For narrow is the gate, and straitened the way, that leadeth unto life, and few are they that find it.}}
\chapsec{Warning against False Teachers}
\bv{15}{\redlet{Beware of false prophets, who come to you in sheep's clothing, but inwardly are ravening wolves.}}
\bv{16}{\redlet{By their fruits ye shall know them. Do \supptext{men} gather grapes of thorns, or figs of thistles?}}
\bv{17}{\redlet{Even so every good tree bringeth forth good fruit; but the corrupt tree bringeth forth evil fruit.}}
\bv{18}{\redlet{A good tree cannot bring forth evil fruit, neither can a corrupt tree bring forth good fruit.}}
\bv{19}{\redlet{Every tree that bringeth not forth good fruit is hewn down, and cast into the fire.}}
\bv{20}{\redlet{Therefore by their fruits ye shall know them.}}
\chapsec{The Danger of Faithless Religion}
\bv{21}{\redlet{Not every one that saith unto me, `Lord, Lord,' shall enter into the kingdom of heaven; but he that doeth the will of my Father who is in heaven.}}
\bv{22}{\redlet{Many will say to me in that day, `Lord, Lord, did we not prophesy by thy name, and by thy name cast out demons, and by thy name do many mighty works?'}}
\bv{23}{\redlet{And then will I profess unto them, I never knew you: depart from me, ye that work iniquity.}}
\chapsec{The Two Foundations}
\bv{24}{\redlet{Every one therefore that heareth these words of mine, and doeth them, shall be likened unto a wise man, who built his house upon the rock:}}
\bv{25}{\redlet{and the rain descended, and the floods came, and the winds blew, and beat upon that house; and it fell not: for it was founded upon the rock.}}
\bv{26}{\redlet{And every one that heareth these words of mine, and doeth them not, shall be likened unto a foolish man, who built his house upon the sand:}}
\bv{27}{\redlet{and the rain descended, and the floods came, and the winds blew, and smote upon that house; and it fell: and great was the fall thereof.''}}
\bv{28}{And it came to pass, when Jesus had finished these words, the multitudes were astonished at his teaching:}
\bv{29}{for he taught them as \supptext{one} having authority, and not as their scribes.}
\chaphead{Chapter VIII}
\chapdesc{Jesus Heals a Leper}
\lettrine[image=true, lines=4, findent=3pt, nindent=0pt]{Mt-A.eps}{nd} when he was come down from the mountain, great multitudes followed him.
\bv{2}{And behold, there came to him a leper and worshipped him, saying, ``Lord, if thou wilt, thou canst make me clean.''}
\bv{3}{And he stretched forth his hand, and touched him, saying, \redlet{``I will; be thou made clean.''} And straightway his leprosy was cleansed.}
\bv{4}{And Jesus saith unto him, \redlet{``See thou tell no man; but go, show thyself to the priest, and offer the gift that Moses commanded, for a testimony unto them.''}}
\chapsec{Jesus Heals the Centurion's Servant}
\bv{5}{And when he was entered into Capernaum, there came unto him a centurion, beseeching him,}
\bv{6}{and saying, ``Lord, my servant lieth in the house sick of the palsy, grievously tormented.''}
\bv{7}{And he saith unto him, \redlet{``I will come and heal him.''}}
\bv{8}{And the centurion answered and said, ``Lord, I am not worthy that thou shouldest come under my roof; but only say the word, and my servant shall be healed.}
\bv{9}{For I also am a man under authority, having under myself soldiers: and I say to this one, `Go,' and he goeth; and to another, `Come,' and he cometh; and to my servant, `Do this,' and he doeth it.''}
\bv{10}{And when Jesus heard it, he marvelled, and said to them that followed, \redlet{``Verily I say unto you, I have not found so great faith, no, not in Israel.}}
\bv{11}{\redlet{And I say unto you, that many shall come from the east and the west, and shall sit down with Abraham, and Isaac, and Jacob, in the kingdom of heaven:}}
\bv{12}{\redlet{but the sons of the kingdom shall be cast forth into the outer darkness: there shall be the weeping and the gnashing of teeth.''}}
\bv{13}{And Jesus said unto the centurion, \redlet{``Go thy way; as thou hast believed, \supptext{so} be it done unto thee.''} And the servant was healed in that hour.}
\chapsec{Jesus Heals St. Peter's Mother-in-law}
\bv{14}{And when Jesus was come into Peter's house, he saw his wife's mother lying sick of a fever.}
\bv{15}{And he touched her hand, and the fever left her; and she arose, and ministered unto him.}
\bv{16}{And when even was come, they brought unto him many possessed with demons: and he cast out the spirits with a word, and healed all that were sick:}
\bv{17}{that it might be fulfilled which was spoken through Isaiah the prophet, saying,}
\otQuote{Is. 53:4}{Himself took our infirmities, and bare our diseases.}
\bv{18}{Now when Jesus saw great multitudes about him, he gave commandment to depart unto the other side.}
\chapsec{Professed Disciples Tested}
\bv{19}{And there came a scribe, and said unto him, ``Teacher, I will follow thee whithersoever thou goest.''}
\bv{20}{And Jesus saith unto him, \redlet{``The foxes have holes, and the birds of the heaven \supptext{have} nests; but the Son of man hath not where to lay his head.''}}
\bv{21}{And another of the disciples said unto him, ``Lord, suffer me first to go and bury my father.''}
\bv{22}{But Jesus saith unto him, \redlet{``Follow me; and leave the dead to bury their own dead.''}}
\chapsec{Jesus Stills the Waves}
\bv{23}{And when he was entered into a boat, his disciples followed him.}
\bv{24}{And behold, there arose a great tempest in the sea, insomuch that the boat was covered with the waves: but he was asleep.}
\bv{25}{And they came to him, and awoke him, saying, ``Save, Lord; we perish.''}
\bv{26}{And he saith unto them, \redlet{``Why are ye fearful, O ye of little faith?''} Then he arose, and rebuked the winds and the sea; and there was a great calm.}
\bv{27}{And the men marvelled, saying, ``What manner of man is this, that even the winds and the sea obey him?''}
\chapsec{Jesus Casts out Demons}
\bv{28}{And when he was come to the other side into the country of the Gadarenes, there met him two possessed with demons, coming forth out of the tombs, exceeding fierce, so that no man could pass by that way.}
\bv{29}{And behold, they cried out, saying, ``What have we to do with thee, thou Son of God? art thou come hither to torment us before the time?''}
\bv{30}{Now there was afar off from them a herd of many swine feeding.}
\bv{31}{And the demons besought him, saying, If thou cast us out, send us away into the herd of swine.}
\bv{32}{And he said unto them, \redlet{``Go.''} And they came out, and went into the swine: and behold, the whole herd rushed down the steep into the sea, and perished in the waters.}
\bv{33}{And they that fed them fled, and went away into the city, and told everything, and what was befallen to them that were possessed with demons.}
\bv{34}{And behold, all the city came out to meet Jesus: and when they saw him, they besought \supptext{him} that he would depart from their borders.}
\chaphead{Chapter IX}
\chapdesc{Jesus Heals the Paralytic}
\lettrine[image=true, lines=4, findent=3pt, nindent=0pt]{Mt-A.eps}{nd} he entered into a boat, and crossed over, and came into his own city.
\bv{2}{And behold, they brought to him a paralytic, lying on a bed: and Jesus seeing their faith said unto the paralytic, \redlet{``Son, be of good cheer; thy sins are forgiven.''}}
\bv{3}{And behold, certain of the scribes said within themselves, ``This man blasphemeth.''}
\bv{4}{And Jesus knowing their thoughts said, \redlet{``Wherefore think ye evil in your hearts?}}
\bv{5}{\redlet{For which is easier, to say, `Thy sins are forgiven;' or to say, `Arise, and walk?'}}
\bv{6}{\redlet{But that ye may know that the Son of man hath authority on earth to forgive sins''} (then saith he to the paralytic), \redlet{``Arise, and take up thy bed, and go unto thy house.''}}
\bv{7}{And he arose, and departed to his house.}
\bv{8}{But when the multitudes saw it, they were afraid, and glorified God, who had given such authority unto men.}
\chapsec{The Call of St. Matthew}
\bv{9}{And as Jesus passed by from thence, he saw a man, called Matthew, sitting at the place of toll: and he saith unto him, \redlet{``Follow me.''} And he arose, and followed him.}
\chapsec{Jesus Answers the Pharisees}
\bv{10}{And it came to pass, as he sat at meat in the house, behold, many publicans and sinners came and sat down with Jesus and his disciples.}
\bv{11}{And when the Pharisees saw it, they said unto his disciples, ``Why eateth your Teacher with the publicans and sinners?''}
\bv{12}{But when he heard it, he said, \redlet{``They that are whole have no need of a physician, but they that are sick.}}
\bv{13}{\redlet{But go ye and learn what \supptext{this} meaneth,}}
\otQuote{Hos. 6:6}{\redlet{`I desire mercy, and not sacrifice:'}}
\redlet{for I came not to call the righteous, but sinners.''}
\bv{14}{Then come to him the disciples of John, saying, ``Why do we and the Pharisees fast oft, but thy disciples fast not?''}
\bv{15}{And Jesus said unto them, \redlet{``Can the sons of the bridechamber mourn, as long as the bridegroom is with them? but the days will come, when the bridegroom shall be taken away from them, and then will they fast.}}
\chapsec{Parables of the Garment \& Bottles}
\bv{16}{\redlet{And no man putteth a piece of undressed cloth upon an old garment; for that which should fill it up taketh from the garment, and a worse rent is made.}}
\bv{17}{\redlet{Neither do \supptext{men} put new wine into old wine-skins: else the skins burst, and the wine is spilled, and the skins perish: but they put new wine into fresh wine-skins, and both are preserved.''}}
\chapsec{Jesus Heals the Woman \& Jairus' Daughter}
\bv{18}{While he spake these things unto them, behold, there came a ruler, and worshipped him, saying, ``My daughter is even now dead: but come and lay thy hand upon her, and she shall live.''}
\bv{19}{And Jesus arose, and followed him, and \supptext{so did} his disciples.}
\bv{20}{And behold, a woman, who had an issue of blood twelve years, came behind him, and touched the border of his garment:}
\bv{21}{for she said within herself, ``If I do but touch his garment, I shall be made whole.''}
\bv{22}{But Jesus turning and seeing her said, \redlet{``Daughter, be of good cheer; thy faith hath made thee whole.''} And the woman was made whole from that hour.}
\bv{23}{And when Jesus came into the ruler's house, and saw the flute-players, and the crowd making a tumult,}
\bv{24}{he said, \redlet{``Give place: for the damsel is not dead, but sleepeth.''} And they laughed him to scorn.}
\bv{25}{But when the crowd was put forth, he entered in, and took her by the hand; and the damsel arose.}
\bv{26}{And the fame hereof went forth into all that land.}
\chapsec{Two Blind Men Healed}
\bv{27}{And as Jesus passed by from thence, two blind men followed him, crying out, and saying, ``Have mercy on us, thou son of David.''}
\bv{28}{And when he was come into the house, the blind men came to him: and Jesus saith unto them, \redlet{``Believe ye that I am able to do this?''} They say unto him, ``Yea, Lord.''}
\bv{29}{Then touched he their eyes, saying, \redlet{``According to your faith be it done unto you.''}}
\bv{30}{And their eyes were opened. And Jesus strictly charged them, saying, \redlet{``See that no man know it.''}}
\bv{31}{But they went forth, and spread abroad his fame in all that land.}
\chapsec{A Demon Cast Out}
\bv{32}{And as they went forth, behold, there was brought to him a dumb man possessed with a demon.}
\bv{33}{And when the demon was cast out, the dumb man spake: and the multitudes marvelled, saying, ``It was never so seen in Israel.''}
\bv{34}{But the Pharisees said, ``By the prince of the demons casteth he out demons.''}
\chapsec{Jesus Preaches \& Heals}
\bv{35}{And Jesus went about all the cities and the villages, teaching in their synagogues, and preaching the gospel of the kingdom, and healing all manner of disease and all manner of sickness.}
\bv{36}{But when he saw the multitudes, he was moved with compassion for them, because they were distressed and scattered, as sheep not having a shepherd.}
\bv{37}{Then saith he unto his disciples, \redlet{``The harvest indeed is plenteous, but the laborers are few.}}
\bv{38}{\redlet{Pray ye therefore the Lord of the harvest, that he send forth laborers into his harvest.''}}
\chaphead{Chapter X}
\chapdesc{The Twelve Instructed \& Sent}
\lettrine[image=true, lines=4, findent=3pt, nindent=0pt]{Mt-A.eps}{nd} he called unto him his twelve disciples, and gave them authority over unclean spirits, to cast them out, and to heal all manner of disease and all manner of sickness.
\bv{2}{Now the names of the twelve apostles are these: The first, Simon, who is called Peter, and Andrew his brother; James the \supptext{son} of Zebedee, and John his brother;}
\bv{3}{Philip, and Bartholomew; Thomas, and Matthew the publican; James the \supptext{son} of Alphæus, and Thaddæus;}
\bv{4}{Simon the Cananæan, and Judas Iscariot, who also betrayed him.}
\par
\bv{5}{These twelve Jesus sent forth, and charged them, saying, \redlet{``Go not into \supptext{any} way of the Gentiles, and enter not into any city of the Samaritans:}}
\bv{6}{\redlet{but go rather to the lost sheep of the house of Israel.}}
\bv{7}{\redlet{And as ye go, preach, saying, `The kingdom of heaven is at hand.'}}
\bv{8}{\redlet{Heal the sick, raise the dead, cleanse the lepers, cast out demons: freely ye received, freely give.}}
\par
\bv{9}{\redlet{Get you no gold, nor silver, nor brass in your purses;}}
\bv{10}{\redlet{no wallet for \supptext{your} journey, neither two coats, nor shoes, nor staff: for the laborer is worthy of his food.}}
\bv{11}{\redlet{And into whatsoever city or village ye shall enter, search out who in it is worthy; and there abide till ye go forth.}}
\bv{12}{\redlet{And as ye enter into the house, salute it.}}
\par
\bv{13}{\redlet{And if the house be worthy, let your peace come upon it: but if it be not worthy, let your peace return to you.}}
\bv{14}{\redlet{And whosoever shall not receive you, nor hear your words, as ye go forth out of that house or that city, shake off the dust of your feet.}}
\par
\bv{15}{\redlet{Verily I say unto you, It shall be more tolerable for the land of Sodom and Gomorrah in the day of judgement, than for that city.}}
\chapsec{Persecution Foretold}
\bv{16}{\redlet{Behold, I send you forth as sheep in the midst of wolves: be ye therefore wise as serpents, and harmless as doves.}}
\bv{17}{\redlet{But beware of men: for they will deliver you up to councils, and in their synagogues they will scourge you;}}
\bv{18}{\redlet{yea and before governors and kings shall ye be brought for my sake, for a testimony to them and to the Gentiles.}}
\bv{19}{\redlet{But when they deliver you up, be not anxious how or what ye shall speak: for it shall be given you in that hour what ye shall speak.}}
\bv{20}{\redlet{For it is not ye that speak, but the Spirit of your Father that speaketh in you.}}
\par
\bv{21}{\redlet{And brother shall deliver up brother to death, and the father his child: and children shall rise up against parents, and cause them to be put to death.}}
\bv{22}{\redlet{And ye shall be hated of all men for my name's sake: but he that endureth to the end, the same shall be saved.}}
\bv{23}{\redlet{But when they persecute you in this city, flee into the next: for verily I say unto you, Ye shall not have gone through the cities of Israel, till the Son of man be come.}}
\par
\bv{24}{\redlet{A disciple is not above his teacher, nor a servant above his lord.}}
\bv{25}{\redlet{It is enough for the disciple that he be as his teacher, and the servant as his lord. If they have called the master of the house Beelzebub, how much more them of his household!}}
\bv{26}{\redlet{Fear them not therefore: for there is nothing covered, that shall not be revealed; and hid, that shall not be known.}}
\bv{27}{\redlet{What I tell you in the darkness, speak ye in the light; and what ye hear in the ear, proclaim upon the house-tops.}}
\bv{28}{\redlet{And be not afraid of them that kill the body, but are not able to kill the soul: but rather fear him who is able to destroy both soul and body in hell.}}
\chapsec{The Father's Love for his Creation}
\bv{29}{\redlet{Are not two sparrows sold for a penny? and not one of them shall fall on the ground without your Father:}}
\bv{30}{\redlet{but the very hairs of your head are all numbered.}}
\bv{31}{\redlet{Fear not therefore: ye are of more value than many sparrows.}}
\bv{32}{\redlet{Every one therefore who shall confess me before men, him will I also confess before my Father who is in heaven.}}
\bv{33}{\redlet{But whosoever shall deny me before men, him will I also deny before my Father who is in heaven.}}
\chapsec{The Peace of Christ}
\bv{34}{\redlet{Think not that I came to send peace on the earth: I came not to send peace, but a sword.}}
\bv{35}{\redlet{For I came to set a man at variance against his father, and the daughter against her mother, and the daughter in law against her mother in law:}}
\bv{36}{\redlet{and a man's foes \supptext{shall be} they of his own household.}}
\bv{37}{\redlet{He that loveth father or mother more than me is not worthy of me; and he that loveth son or daughter more than me is not worthy of me.}}
\bv{38}{\redlet{And he that doth not take his cross and follow after me, is not worthy of me.}}
\bv{39}{\redlet{He that findeth his life shall lose it; and he that loseth his life for my sake shall find it.}}
\par
\bv{40}{\redlet{He that receiveth you receiveth me, and he that receiveth me receiveth him that sent me.}}
\bv{41}{\redlet{He that receiveth a prophet in the name of a prophet shall receive a prophet's reward: and he that receiveth a righteous man in the name of a righteous man shall receive a righteous man's reward.}}
\bv{42}{\redlet{And whosoever shall give to drink unto one of these little ones a cup of cold water only, in the name of a disciple, verily I say unto you he shall in no wise lose his reward.''}}
\chaphead{Chapter XI}
\chapdesc{St. John the Baptist Sends his Disciples to Jesus}
\lettrine[image=true, lines=4, findent=3pt, nindent=0pt]{Mt-A.eps}{nd} it came to pass when Jesus had finished commanding his twelve disciples, he departed thence to teach and preach in their cities.
\bv{2}{Now when John heard in the prison the works of the Christ, he sent by his disciples}
\bv{3}{and said unto him, ``Art thou he that cometh, or look we for another?''}
\bv{4}{And Jesus answered and said unto them, \redlet{``Go and tell John the things which ye hear and see:}}
\bv{5}{\redlet{the blind receive their sight, and the lame walk, the lepers are cleansed, and the deaf hear, and the dead are raised up, and the poor have good tidings preached to them.}}
\bv{6}{\redlet{And blessed is he, whosoever shall find no occasion of stumbling in me.''}}
\bv{7}{And as these went their way, Jesus began to say unto the multitudes concerning John, \redlet{``What went ye out into the wilderness to behold? a reed shaken with the wind?}}
\bv{8}{\redlet{But what went ye out to see? a man clothed in soft \supptext{raiment}? Behold, they that wear soft \supptext{raiment} are in kings' houses.}}
\bv{9}{\redlet{But wherefore went ye out? to see a prophet? Yea, I say unto you, and much more than a prophet.}}
\par
\bv{10}{\redlet{This is he, of whom it is written,}}
\otQuote{Mal. 3:1}{\redlet{`Behold, I send my messenger before thy face,
Who shall prepare thy way before thee.'}}
\bv{11}{\redlet{Verily I say unto you, Among them that are born of women there hath not arisen a greater than John the Baptist: yet he that is but little in the kingdom of heaven is greater than he.}}
\bv{12}{\redlet{And from the days of John the Baptist until now the kingdom of heaven suffereth violence, and men of violence take it by force.}}
\bv{13}{\redlet{For all the prophets and the law prophesied until John.}}
\bv{14}{\redlet{And if ye are willing to receive \supptext{it}, this is Elijah, that is to come.}}
\bv{15}{\redlet{He that hath ears to hear, let him hear.}}
\chapsec{The Hypocrisy of the Faithless}
\bv{16}{\redlet{But whereunto shall I liken this generation? It is like unto children sitting in the marketplaces, who call unto their fellows}}
\bv{17}{\redlet{and say, `We piped unto you, and ye did not dance; we wailed, and ye did not mourn.'}}
\bv{18}{\redlet{For John came neither eating nor drinking, and they say, `He hath a demon.'}}
\bv{19}{\redlet{The Son of man came eating and drinking, and they say, `Behold, a gluttonous man and a winebibber, a friend of publicans and sinners!' And wisdom is justified by her works.''}}
\chapsec{Jesus Rejected \& Predicts Judgement}
\bv{20}{Then began he to upbraid the cities wherein most of his mighty works were done, because they repented not.}
\bv{21}{\redlet{``Woe unto thee, Chorazin! woe unto thee, Bethsaida! for if the mighty works had been done in Tyre and Sidon which were done in you, they would have repented long ago in sackcloth and ashes.}}
\bv{22}{\redlet{But I say unto you, it shall be more tolerable for Tyre and Sidon in the day of judgement, than for you.}}
\bv{23}{\redlet{And thou, Capernaum, shalt thou be exalted unto heaven? thou shalt go down unto Hades: for if the mighty works had been done in Sodom which were done in thee, it would have remained until this day.}}
\bv{24}{\redlet{But I say unto you that it shall be more tolerable for the land of Sodom in the day of judgement, than for thee.''}}
\bv{25}{At that season Jesus answered and said, \redlet{``I thank thee, O Father, Lord of heaven and earth, that thou didst hide these things from the wise and understanding, and didst reveal them unto babes:}}
\bv{26}{\redlet{yea, Father, for so it was well-pleasing in thy sight.}}
\bv{27}{\redlet{All things have been delivered unto me of my Father: and no one knoweth the Son, save the Father; neither doth any know the Father, save the Son, and he to whomsoever the Son willeth to reveal \supptext{him}.}}
\chapsec{Comfortable Words}
\bv{28}{\redlet{Come unto me, all ye that labor and are heavy laden, and I will give you rest.}}
\bv{29}{\redlet{Take my yoke upon you, and learn of me; for I am meek and lowly in heart: and ye shall find rest unto your souls.}}
\bv{30}{\redlet{For my yoke is easy, and my burden is light.''}}
\chaphead{Chapter XII}
\chapdesc{Jesus: Lord of the Sabbath}
\lettrine[image=true, lines=4, findent=3pt, nindent=0pt]{Mt-A.eps}{t} that season Jesus went on the sabbath day through the grainfields; and his disciples were hungry and began to pluck ears and to eat.
\bv{2}{But the Pharisees, when they saw it, said unto him, ``Behold, thy disciples do that which it is not lawful to do upon the sabbath.''}
\bv{3}{But he said unto them, \redlet{``Have ye not read what David did, when he was hungry, and they that were with him;}}
\bv{4}{\redlet{how he entered into the house of God, and ate the showbread, which it was not lawful for him to eat, neither for them that were with him, but only for the priests?}}
\bv{5}{\redlet{Or have ye not read in the law, that on the sabbath day the priests in the temple profane the sabbath, and are guiltless?}}
\bv{6}{\redlet{But I say unto you, that one greater than the temple is here.}}
\par
\bv{7}{\redlet{But if ye had known what this meaneth,}}
\otQuote{Hos. 6:6}{\redlet{`I desire mercy, and not sacrifice,'}}
\redlet{ye would not have condemned the guiltless.}
\bv{8}{\redlet{For the Son of man is lord of the sabbath.''}}
\chapsec{The Healing of the Withered Hand}
\bv{9}{And he departed thence, and went into their synagogue:}
\bv{10}{and behold, a man having a withered hand. And they asked him, saying, ``Is it lawful to heal on the sabbath day?'' that they might accuse him.}
\bv{11}{And he said unto them, \redlet{``What man shall there be of you, that shall have one sheep, and if this fall into a pit on the sabbath day, will he not lay hold on it, and lift it out?}}
\bv{12}{\redlet{How much then is a man of more value than a sheep! Wherefore it is lawful to do good on the sabbath day.''}}
\bv{13}{Then saith he to the man, \redlet{``Stretch forth thy hand.''} And he stretched it forth; and it was restored whole, as the other.}
\bv{14}{But the Pharisees went out, and took counsel against him, how they might destroy him.}
\bv{15}{And Jesus perceiving \supptext{it} withdrew from thence: and many followed him; and he healed them all,}
\bv{16}{and charged them that they should not make him known:}
\bv{17}{that it might be fulfilled which was spoken through Isaiah the prophet, saying,}
\otQuote{Is. 42:1-3}{\bv{18}{Behold, my servant whom I have chosen;
My beloved in whom my soul is well pleased:
I will put my Spirit upon him,
And he shall declare judgement to the Gentiles.}
\bv{19}{He shall not strive, nor cry aloud;
Neither shall any one hear his voice in the streets.}
\bv{20}{A bruised reed shall he not break,
And smoking flax shall he not quench,
Till he send forth judgement unto victory.}
\bv{21}{And in his name shall the Gentiles hope.}}
\chapsec{A Demoniac Healed}
\bv{22}{Then was brought unto him one possessed with a demon, blind and dumb: and he healed him, insomuch that the dumb man spake and saw.}
\bv{23}{And all the multitudes were amazed, and said, ``Can this be the son of David?''}
\bv{24}{But when the Pharisees heard it, they said, ``This man doth not cast out demons, but by Beelzebub the prince of the demons.''}
\bv{25}{And knowing their thoughts he said unto them, \redlet{``Every kingdom divided against itself is brought to desolation; and every city or house divided against itself shall not stand:}}
\bv{26}{\redlet{and if Satan casteth out Satan, he is divided against himself; how then shall his kingdom stand?}}
\bv{27}{\redlet{And if I by Beelzebub cast out demons, by whom do your sons cast them out? therefore shall they be your judges.}}
\bv{28}{\redlet{But if I by the Spirit of God cast out demons, then is the kingdom of God come upon you.}}
\chapsec{Parable of the Strong Man}
\bv{29}{\redlet{Or how can one enter into the house of the strong \supptext{man}, and spoil his goods, except he first bind the strong \supptext{man}? and then he will spoil his house.}}
\bv{30}{\redlet{He that is not with me is against me; and he that gathereth not with me scattereth.}}
\chapsec{Blasphemy against the Holy Ghost}
\bv{31}{\redlet{Therefore I say unto you, Every sin and blasphemy shall be forgiven unto men; but the blasphemy against the Spirit shall not be forgiven.}}
\bv{32}{\redlet{And whosoever shall speak a word against the Son of man, it shall be forgiven him; but whosoever shall speak against the Holy Ghost, it shall not be forgiven him, neither in this world, nor in that which is to come.}}
\chapsec{Good Fruit \& Idle Words}
\bv{33}{\redlet{Either make the tree good, and its fruit good; or make the tree corrupt, and its fruit corrupt: for the tree is known by its fruit.}}
\bv{34}{\redlet{Ye offspring of vipers, how can ye, being evil, speak good things? for out of the abundance of the heart the mouth speaketh.}}
\bv{35}{\redlet{The good man out of his good treasure bringeth forth good things: and the evil man out of his evil treasure bringeth forth evil things.}}
\bv{36}{\redlet{And I say unto you, that every idle word that men shall speak, they shall give account thereof in the day of judgement.}}
\bv{37}{\redlet{For by thy words thou shalt be justified, and by thy words thou shalt be condemned.''}}
\chapsec{The Sign of the Prophet Jonah}
\bv{38}{Then certain of the scribes and Pharisees answered him, saying, ``Teacher, we would see a sign from thee.''}
\bv{39}{But he answered and said unto them, \redlet{``An evil and adulterous generation seeketh after a sign; and there shall no sign be given to it but the sign of Jonah the prophet:}}
\bv{40}{\redlet{for as Jonah was three days and three nights in the belly of the whale; so shall the Son of man be three days and three nights in the heart of the earth.}}
\bv{41}{\redlet{The men of Nineveh shall stand up in the judgement with this generation, and shall condemn it: for they repented at the preaching of Jonah; and behold, a greater than Jonah is here.}}
\bv{42}{\redlet{The queen of the south shall rise up in the judgement with this generation, and shall condemn it: for she came from the ends of the earth to hear the wisdom of Solomon; and behold, a greater than Solomon is here.}}
\chapsec{Stages of Sanctification}
\bv{43}{\redlet{But the unclean spirit, when he is gone out of the man, passeth through waterless places, seeking rest, and findeth it not.}}
\bv{44}{\redlet{Then he saith, `I will return into my house whence I came out;' and when he is come, he findeth it empty, swept, and garnished.}}
\bv{45}{\redlet{Then goeth he, and taketh with himself seven other spirits more evil than himself, and they enter in and dwell there: and the last state of that man becometh worse than the first. Even so shall it be also unto this evil generation.''}}
\chapsec{Jesus' Mother \& Brethren}
\bv{46}{While he was yet speaking to the multitudes, behold, his mother and his brethren stood without, seeking to speak to him.}
\bv{47}{And one said unto him, ``Behold, thy mother and thy brethren stand without, seeking to speak to thee.''}
\bv{48}{But he answered and said unto him that told him, \redlet{``Who is my mother? and who are my brethren?''}}
\bv{49}{And he stretched forth his hand towards his disciples, and said, \redlet{``Behold, my mother and my brethren!}}
\bv{50}{\redlet{For whosoever shall do the will of my Father who is in heaven, he is my brother, and sister, and mother.''}}
\chaphead{Chapter XIII}
\chapdesc{The Parable of the Sower}
\lettrine[image=true, lines=4, findent=3pt, nindent=0pt]{O.eps}{n} that day went Jesus out of the house, and sat by the sea side.
\bv{2}{And there were gathered unto him great multitudes, so that he entered into a boat, and sat; and all the multitude stood on the beach.}
\bv{3}{And he spake to them many things in parables, saying, \redlet{``Behold, the sower went forth to sow;}}
\bv{4}{\redlet{and as he sowed, some \supptext{seeds} fell by the way side, and the birds came and devoured them:}}
\bv{5}{\redlet{and others fell upon the rocky places, where they had not much earth: and straightway they sprang up, because they had no deepness of earth:}}
\bv{6}{\redlet{and when the sun was risen, they were scorched; and because they had no root, they withered away.}}
\bv{7}{\redlet{And others fell upon the thorns; and the thorns grew up and choked them:}}
\bv{8}{\redlet{and others fell upon the good ground, and yielded fruit, some a hundredfold, some sixty, some thirty.}}
\bv{9}{\redlet{He that hath ears, let him hear.'}}
\chapsec{The Purpose of Parables}
\bv{10}{And the disciples came, and said unto him, ``Why speakest thou unto them in parables?''}
\bv{11}{And he answered and said unto them, \redlet{``Unto you it is given to know the mysteries of the kingdom of heaven, but to them it is not given.}}
\bv{12}{\redlet{For whosoever hath, to him shall be given, and he shall have abundance: but whosoever hath not, from him shall be taken away even that which he hath.}}
\bv{13}{\redlet{Therefore speak I to them in parables; because seeing they see not, and hearing they hear not, neither do they understand.}}
\bv{14}{\redlet{And unto them is fulfilled the prophecy of Isaiah, which saith,}}
\otQuote{Is. 6:9-10}{\redlet{By hearing ye shall hear, and shall in no wise understand;
And seeing ye shall see, and shall in no wise perceive:}
\bv{15}{\redlet{For this people's heart is waxed gross,
And their eyes they have closed;
Lest haply they should perceive with their eyes,
And hear with their ears,
And understand with their heart,
And should turn again,
And I should heal them.}}}
\bv{16}{\redlet{But blessed are your eyes, for they see; and your ears, for they hear.}}
\bv{17}{\redlet{For verily I say unto you, that many prophets and righteous men desired to see the things which ye see, and saw them not; and to hear the things which ye hear, and heard them not.}}
\chapsec{Explanation of the Parable of the Sower}
\bv{18}{\redlet{Hear then ye the parable of the sower.}}
\bv{19}{\redlet{When any one heareth the word of the kingdom, and understandeth it not, \supptext{then} cometh the evil \supptext{one}, and snatcheth away that which hath been sown in his heart. This is he that was sown by the way side.}}
\bv{20}{\redlet{And he that was sown upon the rocky places, this is he that heareth the word, and straightway with joy receiveth it;}}
\bv{21}{\redlet{yet hath he not root in himself, but endureth for a while; and when tribulation or persecution ariseth because of the word, straightway he stumbleth.}}
\bv{22}{\redlet{And he that was sown among the thorns, this is he that heareth the word; and the care of the world, and the deceitfulness of riches, choke the word, and he becometh unfruitful.}}
\bv{23}{\redlet{And he that was sown upon the good ground, this is he that heareth the word, and understandeth it; who verily beareth fruit, and bringeth forth, some a hundredfold, some sixty, some thirty.''}}
\chapsec{The Tares among the Wheat}
\bv{24}{Another parable set he before them, saying, \redlet{``The kingdom of heaven is likened unto a man that sowed good seed in his field:}}
\bv{25}{\redlet{but while men slept, his enemy came and sowed tares also among the wheat, and went away.}}
\bv{26}{\redlet{But when the blade sprang up and brought forth fruit, then appeared the tares also.}}
\bv{27}{\redlet{And the servants of the householder came and said unto him, `Sir, didst thou not sow good seed in thy field? whence then hath it tares?'}}
\bv{28}{\redlet{And he said unto them, `An enemy hath done this.' And the servants say unto him, `Wilt thou then that we go and gather them up?'}}
\bv{29}{\redlet{But he saith, `Nay; lest haply while ye gather up the tares, ye root up the wheat with them.}}
\bv{30}{\redlet{Let both grow together until the harvest: and in the time of the harvest I will say to the reapers, `Gather up first the tares, and bind them in bundles to burn them; but gather the wheat into my barn.'{'}{''}}}
\chapsec{The Grain of Mustard Seed}
\bv{31}{Another parable set he before them, saying, \redlet{``The kingdom of heaven is like unto a grain of mustard seed, which a man took, and sowed in his field:}}
\bv{32}{\redlet{which indeed is less than all seeds; but when it is grown, it is greater than the herbs, and becometh a tree, so that the birds of the heaven come and lodge in the branches thereof.''}}
\chapsec{The Leaven}
\bv{33}{Another parable spake he unto them; \redlet{``The kingdom of heaven is like unto leaven, which a woman took, and hid in three measures of meal, till it was all leavened.''}}
\bv{34}{All these things spake Jesus in parables unto the multitudes; and without a parable spake he nothing unto them:}
\bv{35}{that it might be fulfilled which was spoken through the prophet, saying,}
\otQuote{Ps. 78:2}{I will open my mouth in parables;
I will utter things hidden from the foundation of the world.}
\chapsec{Explanation of the Parables of the Tares}
\bv{36}{Then he left the multitudes, and went into the house: and his disciples came unto him, saying, ``Explain unto us the parable of the tares of the field.''}
\bv{37}{And he answered and said, \redlet{``He that soweth the good seed is the Son of man;}}
\bv{38}{\redlet{and the field is the world; and the good seed, these are the sons of the kingdom; and the tares are the sons of the evil \supptext{one};}}
\bv{39}{\redlet{and the enemy that sowed them is the devil: and the harvest is the end of the world; and the reapers are angels.}}
\bv{40}{\redlet{As therefore the tares are gathered up and burned with fire; so shall it be in the end of the world.}}
\bv{41}{\redlet{The Son of man shall send forth his angels, and they shall gather out of his kingdom all things that cause stumbling, and them that do iniquity,}}
\bv{42}{\redlet{and shall cast them into the furnace of fire: there shall be the weeping and the gnashing of teeth.}}
\bv{43}{\redlet{Then shall the righteous shine forth as the sun in the kingdom of their Father. He that hath ears, let him hear.}}
\chapsec{The Hid Treasure}
\bv{44}{\redlet{The kingdom of heaven is like unto a treasure hidden in the field; which a man found, and hid; and in his joy he goeth and selleth all that he hath, and buyeth that field.}}
\chapsec{The Pearl of Great Price}
\bv{45}{\redlet{Again, the kingdom of heaven is like unto a man that is a merchant seeking goodly pearls:}}
\bv{46}{\redlet{and having found one pearl of great price, he went and sold all that he had, and bought it.}}
\chapsec{The Net}
\bv{47}{\redlet{Again, the kingdom of heaven is like unto a net, that was cast into the sea, and gathered of every kind:}}
\bv{48}{\redlet{which, when it was filled, they drew up on the beach; and they sat down, and gathered the good into vessels, but the bad they cast away.}}
\bv{49}{\redlet{So shall it be in the end of the world: the angels shall come forth, and sever the wicked from among the righteous,}}
\bv{50}{\redlet{and shall cast them into the furnace of fire: there shall be the weeping and the gnashing of teeth.}}
\bv{51}{\redlet{Have ye understood all these things?''} They say unto him, ``Yea.''}
\chapsec{Parable of the Householder}
\bv{52}{And he said unto them, \redlet{``Therefore every scribe who hath been made a disciple to the kingdom of heaven is like unto a man that is a householder, who bringeth forth out of his treasure things new and old.''}}
\chapsec{Jesus Returns to Nazareth}
\bv{53}{And it came to pass, when Jesus had finished these parables, he departed thence.}
\bv{54}{And coming into his own country he taught them in their synagogue, insomuch that they were astonished, and said, ``Whence hath this man this wisdom, and these mighty works?}
\bv{55}{Is not this the carpenter's son? is not his mother called Mary? and his brethren, James, and Joseph, and Simon, and Judas?}
\bv{56}{And his sisters, are they not all with us? Whence then hath this man all these things?''}
\bv{57}{And they were offended in him. But Jesus said unto them, \redlet{``A prophet is not without honor, save in his own country, and in his own house.''}}
\bv{58}{And he did not many mighty works there because of their unbelief.}
\chaphead{Chapter XIV}
\chapdesc{Herod's Troubled Conscience}
\lettrine[image=true, lines=4, findent=3pt, nindent=0pt]{Mt-A.eps}{t} that season Herod the tetrarch heard the report concerning Jesus,
\bv{2}{and said unto his servants, ``This is John the Baptist; he is risen from the dead; and therefore do these powers work in him.''}
\bv{3}{For Herod had laid hold on John, and bound him, and put him in prison for the sake of Herodias, his brother Philip's wife.}
\bv{4}{For John said unto him, ``It is not lawful for thee to have her.''}
\bv{5}{And when he would have put him to death, he feared the multitude, because they counted him as a prophet.}
\chapsec{The Murder of St. John the Baptist}
\bv{6}{But when Herod's birthday came, the daughter of Herodias danced in the midst, and pleased Herod.}
\bv{7}{Whereupon he promised with an oath to give her whatsoever she should ask.}
\bv{8}{And she, being put forward by her mother, saith, ``Give me here on a platter the head of John the Baptist.''}
\bv{9}{And the king was grieved; but for the sake of his oaths, and of them that sat at meat with him, he commanded it to be given;}
\bv{10}{and he sent and beheaded John in the prison.}
\bv{11}{And his head was brought on a platter, and given to the damsel: and she brought it to her mother.}
\bv{12}{And his disciples came, and took up the corpse, and buried him; and they went and told Jesus.}
\bv{13}{Now when Jesus heard \supptext{it}, he withdrew from thence in a boat, to a desert place apart: and when the multitudes heard \supptext{thereof}, they followed him on foot from the cities.}
\bv{14}{And he came forth, and saw a great multitude, and he had compassion on them, and healed their sick.}
\chapsec{The Feeding of the 5,000}
\bv{15}{And when even was come, the disciples came to him, saying, ``The place is desert, and the time is already past; send the multitudes away, that they may go into the villages, and buy themselves food.''}
\bv{16}{But Jesus said unto them, \redlet{``They have no need to go away; give ye them to eat.''}}
\bv{17}{And they say unto him, ``We have here but five loaves, and two fishes.''}
\bv{18}{And he said, \redlet{``Bring them hither to me.''}}
\bv{19}{And he commanded the multitudes to sit down on the grass; and he took the five loaves, and the two fishes, and looking up to heaven, he blessed, and brake and gave the loaves to the disciples, and the disciples to the multitudes.}
\bv{20}{And they all ate, and were filled: and they took up that which remained over of the broken pieces, twelve baskets full.}
\bv{21}{And they that did eat were about five thousand men, besides women and children.}
\chapsec{Jesus Walks on Water}
\bv{22}{And straightway he constrained the disciples to enter into the boat, and to go before him unto the other side, till he should send the multitudes away.}
\bv{23}{And after he had sent the multitudes away, he went up into the mountain apart to pray: and when even was come, he was there alone.}
\bv{24}{But the boat was now in the midst of the sea, distressed by the waves; for the wind was contrary.}
\bv{25}{And in the fourth watch of the night he came unto them, walking upon the sea.}
\bv{26}{And when the disciples saw him walking on the sea, they were troubled, saying, ``It is a ghost;'' and they cried out for fear.}
\bv{27}{But straightway Jesus spake unto them, saying, \redlet{``Be of good cheer; it is I; be not afraid.''}}
\bv{28}{And Peter answered him and said, ``Lord, if it be thou, bid me come unto thee upon the waters.''}
\bv{29}{And he said, \redlet{``Come.''} And Peter went down from the boat, and walked upon the waters to come to Jesus.}
\bv{30}{But when he saw the wind, he was afraid; and beginning to sink, he cried out, saying, ``Lord, save me.''}
\bv{31}{And immediately Jesus stretched forth his hand, and took hold of him, and saith unto him, \redlet{``O thou of little faith, wherefore didst thou doubt?''}}
\bv{32}{And when they were gone up into the boat, the wind ceased.}
\par
\bv{33}{And they that were in the boat worshipped him, saying, ``Of a truth thou art the Son of God.''}
\bv{34}{And when they had crossed over, they came to the land, unto Gennesaret.}
\bv{35}{And when the men of that place knew him, they sent into all that region round about, and brought unto him all that were sick;}
\bv{36}{and they besought him that they might only touch the border of his garment: and as many as touched were made whole.}
\chaphead{Chapter XV}
\chapdesc{Jesus Rebukes the Scribes \& Pharisees}
\lettrine[image=true, lines=4, findent=3pt, nindent=0pt]{T.ps}{hen} there come to Jesus from Jerusalem Pharisees and scribes, saying,
\bv{2}{``Why do thy disciples transgress the tradition of the elders? for they wash not their hands when they eat bread.''}
\bv{3}{And he answered and said unto them, \redlet{``Why do ye also transgress the commandment of God because of your tradition?}}
\bv{4}{\redlet{For God said, `Honor thy father and thy mother:' and, `He that speaketh evil of father or mother, let him die the death.'}}
\bv{5}{\redlet{But ye say, `Whosoever shall say to his father or his mother, `That wherewith thou mightest have been profited by me is given \supptext{to God};'}}
\bv{6}{\redlet{he shall not honor his father.' And ye have made void the word of God because of your tradition.}}
\bv{7}{\redlet{Ye hypocrites, well did Isaiah prophesy of you, saying,}}
\otQuote{Is. 29:13}{\bv{8}{\redlet{This people honoreth me with their lips;
But their heart is far from me.}}
\bv{9}{\redlet{But in vain do they worship me,
Teaching \supptext{as their} doctrines the precepts of men.''}}}
\bv{10}{And he called to him the multitude, and said unto them, \redlet{``Hear, and understand:}}
\bv{11}{\redlet{Not that which entereth into the mouth defileth the man; but that which proceedeth out of the mouth, this defileth the man.''}}
\par
\bv{12}{Then came the disciples, and said unto him, ``Knowest thou that the Pharisees were offended, when they heard this saying?''}
\bv{13}{But he answered and said, \redlet{``Every plant which my heavenly Father planted not, shall be rooted up.}}
\bv{14}{\redlet{Let them alone: they are blind guides. And if the blind guide the blind, both shall fall into a pit.''}}
\par
\bv{15}{And Peter answered and said unto him, ``Declare unto us the parable.''}
\bv{16}{And he said, \redlet{``Are ye also even yet without understanding?}}
\bv{17}{\redlet{Perceive ye not, that whatsoever goeth into the mouth passeth into the belly, and is cast out into the draught?}}
\bv{18}{\redlet{But the things which proceed out of the mouth come forth out of the heart; and they defile the man.}}
\bv{19}{\redlet{For out of the heart come forth evil thoughts, murders, adulteries, fornications, thefts, false witness, railings:}}
\bv{20}{\redlet{these are the things which defile the man; but to eat with unwashen hands defileth not the man.''}}
\bv{21}{And Jesus went out thence, and withdrew into the parts of Tyre and Sidon.}
\chapsec{Humbling of the Canaanite Woman}
\bv{22}{And behold, a Canaanite woman came out from those borders, and cried, saying, ``Have mercy on me, O Lord, thou son of David; my daughter is grievously vexed with a demon.''}
\bv{23}{But he answered her not a word. And his disciples came and besought him, saying, ``Send her away; for she crieth after us.''}
\bv{24}{But he answered and said, \redlet{``I was not sent but unto the lost sheep of the house of Israel.''}}
\bv{25}{But she came and worshipped him, saying, ``Lord, help me.''}
\bv{26}{And he answered and said, \redlet{``It is not meet to take the children's bread and cast it to the dogs.''}}
\bv{27}{But she said, ``Yea, Lord: for even the dogs eat of the crumbs which fall from their masters' table.''}
\bv{28}{Then Jesus answered and said unto her, \redlet{``O woman, great is thy faith: be it done unto thee even as thou wilt.''} And her daughter was healed from that hour.}
\chapsec{Jesus Heals Multitudes}
\bv{29}{And Jesus departed thence, and came nigh unto the sea of Galilee; and he went up into the mountain, and sat there.}
\bv{30}{And there came unto him great multitudes, having with them the lame, blind, dumb, maimed, and many others, and they cast them down at his feet; and he healed them:}
\bv{31}{insomuch that the multitude wondered, when they saw the dumb speaking, the maimed whole, and the lame walking, and the blind seeing: and they glorified the God of Israel.}
\chapsec{The Feeding of the 4,000}
\bv{32}{And Jesus called unto him his disciples, and said, \redlet{``I have compassion on the multitude, because they continue with me now three days and have nothing to eat: and I would not send them away fasting, lest haply they faint on the way.''}}
\bv{33}{And the disciples say unto him, ``Whence should we have so many loaves in a desert place as to fill so great a multitude?''}
\bv{34}{And Jesus said unto them, \redlet{``How many loaves have ye?''} And they said, ``Seven, and a few small fishes.''}
\bv{35}{And he commanded the multitude to sit down on the ground;}
\bv{36}{and he took the seven loaves and the fishes; and he gave thanks and brake, and gave to the disciples, and the disciples to the multitudes.}
\bv{37}{And they all ate, and were filled: and they took up that which remained over of the broken pieces, seven baskets full.}
\bv{38}{And they that did eat were four thousand men, besides women and children.}
\bv{39}{And he sent away the multitudes, and entered into the boat, and came into the borders of Magadan.}
\chaphead{Chapter XVI}
\chapdesc{Jesus Rebukes the Blind Pharisees}
\lettrine[image=true, lines=4, findent=3pt, nindent=0pt]{Mt-A.eps}{nd} the Pharisees and Sadducees came, and trying him asked him to show them a sign from heaven.
\bv{2}{But he answered and said unto them, \redlet{``When it is evening, ye say, `\supptext{It will be} fair weather: for the heaven is red.'}}
\bv{3}{\redlet{And in the morning, `\supptext{It will be} foul weather to-day: for the heaven is red and lowering.' Ye know how to discern the face of the heaven; but ye cannot \supptext{discern} the signs of the times.}}
\bv{4}{\redlet{An evil and adulterous generation seeketh after a sign; and there shall no sign be given unto it, but the sign of Jonah.'' And he left them, and departed.}}
\chapsec{Explanation of the Leaven}
\bv{5}{And the disciples came to the other side and forgot to take bread.}
\bv{6}{And Jesus said unto them, \redlet{``Take heed and beware of the leaven of the Pharisees and Sadducees.''}}
\bv{7}{And they reasoned among themselves, saying, ``We took no bread.''}
\bv{8}{And Jesus perceiving it said, \redlet{``O ye of little faith, why reason ye among yourselves, because ye have no bread?}}
\bv{9}{\redlet{Do ye not yet perceive, neither remember the five loaves of the five thousand, and how many baskets ye took up?}}
\bv{10}{\redlet{Neither the seven loaves of the four thousand, and how many baskets ye took up?}}
\bv{11}{\redlet{How is it that ye do not perceive that I spake not to you concerning bread? But beware of the leaven of the Pharisees and Sadducees.''}}
\bv{12}{Then understood they that he bade them not beware of the leaven of bread, but of the teaching of the Pharisees and Sadducees.}
\chapsec{St. Peter's Confession}
\bv{13}{Now when Jesus came into the parts of Cæsarea Philippi, he asked his disciples, saying, \redlet{``Who do men say that the Son of man is?''}}
\bv{14}{And they said, ``Some \supptext{say} John the Baptist; some, Elijah; and others, Jeremiah, or one of the prophets.''}
\bv{15}{He saith unto them, \redlet{``But who say ye that I am?''}}
\bv{16}{And Simon Peter answered and said, ``Thou art the Christ, the Son of the living God.''}
\bv{17}{And Jesus answered and said unto him, \redlet{``Blessed art thou, Simon Bar-Jonah: for flesh and blood hath not revealed it unto thee, but my Father who is in heaven.}}
\bv{18}{\redlet{And I also say unto thee, that thou art Peter, and upon this rock I will build my church; and the gates of Hades shall not prevail against it.}\mcomm{``shall not prevail against it,'' that is, the church.}}
\chapsec{Promise to Give the Keys to St. Peter}
\bv{19}{\redlet{I will give unto thee the keys of the kingdom of heaven: and whatsoever thou shalt bind on earth shall be bound in heaven; and whatsoever thou shalt loose on earth shall be loosed in heaven.''}}
\bv{20}{Then charged he the disciples that they should tell no man that he was the Christ.}
\chapsec{Jesus Foretells his Death \& Resurrection}
\bv{21}{From that time began Jesus to show unto his disciples, that he must go unto Jerusalem, and suffer many things of the elders and chief priests and scribes, and be killed, and the third day be raised up.}
\bv{22}{And Peter took him, and began to rebuke him, saying, ``Be it far from thee, Lord: this shall never be unto thee.''}
\bv{23}{But he turned, and said unto Peter, \redlet{``Get thee behind me, Satan: thou art a stumbling-block unto me: for thou mindest not the things of God, but the things of men.''}}
\bv{24}{Then said Jesus unto his disciples, \redlet{``If any man would come after me, let him deny himself, and take up his cross, and follow me.}}
\bv{25}{\redlet{For whosoever would save his life shall lose it: and whosoever shall lose his life for my sake shall find it.}}
\bv{26}{\redlet{For what shall a man be profited, if he shall gain the whole world, and forfeit his life? or what shall a man give in exchange for his life?}}
\bv{27}{\redlet{For the Son of man shall come in the glory of his Father with his angels; and then shall he render unto every man according to his deeds.}}
\chapdesc{The Promise of the Transfiguration}
\bv{28}{\redlet{Verily I say unto you, There are some of them that stand here, who shall in no wise taste of death, till they see the Son of man coming in his kingdom.''}}
\chaphead{Chapter XVII}
\chapdesc{The Transfiguration}
\lettrine[image=true, lines=4, findent=3pt, nindent=0pt]{Mt-A.eps}{nd} after six days Jesus taketh with him Peter, and James, and John his brother, and bringeth them up into a high mountain apart:
\bv{2}{and he was transfigured before them; and his face did shine as the sun, and his garments became white as the light.}
\bv{3}{And behold, there appeared unto them Moses and Elijah talking with him.}
\bv{4}{And Peter answered, and said unto Jesus, ``Lord, it is good for us to be here: if thou wilt, I will make here three tabernacles; one for thee, and one for Moses, and one for Elijah.''}
\bv{5}{While he was yet speaking, behold, a bright cloud overshadowed them: and behold, a voice out of the cloud, saying, \god{``This is my beloved Son, in whom I am well pleased; hear ye him.''}}
\bv{6}{And when the disciples heard it, they fell on their face, and were sore afraid.}
\bv{7}{And Jesus came and touched them and said, \redlet{``Arise, and be not afraid.''}}
\bv{8}{And lifting up their eyes, they saw no one, save Jesus only.}
\par
\bv{9}{And as they were coming down from the mountain, Jesus commanded them, saying, \redlet{``Tell the vision to no man, until the Son of man be risen from the dead.''}}
\bv{10}{And his disciples asked him, saying, ``Why then say the scribes that Elijah must first come?''}
\bv{11}{And he answered and said, \redlet{``Elijah indeed cometh, and shall restore all things:}}
\bv{12}{\redlet{but I say unto you, that Elijah is come already, and they knew him not, but did unto him whatsoever they would. Even so shall the Son of man also suffer of them.''}}
\bv{13}{Then understood the disciples that he spake unto them of John the Baptist.}
\chapsec{The Powerless Disciples}
\bv{14}{And when they were come to the multitude, there came to him a man, kneeling to him, and saying,}
\bv{15}{``Lord, have mercy on my son: for he is epileptic, and suffereth grievously; for oft-times he falleth into the fire, and oft-times into the water.}
\bv{16}{And I brought him to thy disciples, and they could not cure him.''}
\bv{17}{And Jesus answered and said, \redlet{``O faithless and perverse generation, how long shall I be with you? how long shall I bear with you? bring him hither to me.''}}
\bv{18}{And Jesus rebuked him; and the demon went out of him: and the boy was cured from that hour.}
\bv{19}{Then came the disciples to Jesus apart, and said, ``Why could not we cast it out?''}
\bv{20}{And he saith unto them, \redlet{``Because of your little faith: for verily I say unto you, If ye have faith as a grain of mustard seed, ye shall say unto this mountain, Remove hence to yonder place; and it shall remove; and nothing shall be impossible unto you.''}\mcomm{But this kind goeth not out save by prayer and fasting.}}
\chapsec{Jesus again Foretells his Death \& Resurrection}
\bv{22}{And while they abode in Galilee, Jesus said unto them, \redlet{``The Son of man shall be delivered up into the hands of men;}}
\bv{23}{\redlet{and they shall kill him, and the third day he shall be raised up.''} And they were exceeding sorry.}
\chapsec{The Miracle of the Tribute Money}
\bv{24}{And when they were come to Capernaum, they that received the half-shekel came to Peter, and said, ``Doth not your teacher pay the half-shekel?''}
\bv{25}{He saith, ``Yea.'' And when he came into the house, Jesus spake first to him, saying, \redlet{``What thinkest thou, Simon? the kings of the earth, from whom do they receive toll or tribute? from their sons, or from strangers?''}}
\bv{26}{And when he said, ``From strangers,'' Jesus said unto him, \redlet{``Therefore the sons are free.}}
\bv{27}{\redlet{But, lest we cause them to stumble, go thou to the sea, and cast a hook, and take up the fish that first cometh up; and when thou hast opened his mouth, thou shalt find a shekel: that take, and give unto them for me and thee.''}}
\chaphead{Chapter XVIII}
\chapdesc{The Sermon on the Child}
\lettrine[image=true, lines=4, findent=3pt, nindent=0pt]{Mt-I.eps}{n} that hour came the disciples unto Jesus, saying, ``Who then is greatest in the kingdom of heaven?''
\bv{2}{And he called to him a little child, and set him in the midst of them,}
\bv{3}{and said, \redlet{``Verily I say unto you, Except ye turn, and become as little children, ye shall in no wise enter into the kingdom of heaven.}}
\bv{4}{\redlet{Whosoever therefore shall humble himself as this little child, the same is the greatest in the kingdom of heaven.}}
\bv{5}{\redlet{And whoso shall receive one such little child in my name receiveth me:}}
\bv{6}{\redlet{but whoso shall cause one of these little ones that believe on me to stumble, it is profitable for him that a great millstone should be hanged about his neck, and \supptext{that} he should be sunk in the depth of the sea.}}
\par
\bv{7}{\redlet{Woe unto the world because of occasions of stumbling! for it must needs be that the occasions come; but woe to that man through whom the occasion cometh!}}
\bv{8}{\redlet{And if thy hand or thy foot causeth thee to stumble, cut it off, and cast it from thee: it is good for thee to enter into life maimed or halt, rather than having two hands or two feet to be cast into the eternal fire.}}
\bv{9}{\redlet{And if thine eye causeth thee to stumble, pluck it out, and cast it from thee: it is good for thee to enter into life with one eye, rather than having two eyes to be cast into the hell of fire.}}
\bv{10}{\redlet{See that ye despise not one of these little ones: for I say unto you, that in heaven their angels do always behold the face of my Father who is in heaven.}\mcomm{For the son of man came to save that which was lost.}}
\chapsec{The Lost Sheep}
\bv{12}{\redlet{How think ye? if any man have a hundred sheep, and one of them be gone astray, doth he not leave the ninety and nine, and go unto the mountains, and seek that which goeth astray?}}
\bv{13}{\redlet{And if so be that he find it, verily I say unto you, he rejoiceth over it more than over the ninety and nine which have not gone astray.}}
\bv{14}{\redlet{Even so it is not the will of your Father who is in heaven, that one of these little ones should perish.}}
\chapsec{Discipline in the Church}
\bv{15}{\redlet{And if thy brother sin against thee, go, show him his fault between thee and him alone: if he hear thee, thou hast gained thy brother.}}
\bv{16}{\redlet{But if he hear \supptext{thee} not, take with thee one or two more, that at the mouth of two witnesses or three every word may be established.}}
\bv{17}{\redlet{And if he refuse to hear them, tell it unto the church: and if he refuse to hear the church also, let him be unto thee as the Gentile and the publican.}}
\chapsec{The Keys Given to the Apostles}
\bv{18}{\redlet{Verily I say unto you, What things soever ye bind on earth shall be bound in heaven; and what things soever ye loose on earth shall be loosed in heaven.}}
\bv{19}{\redlet{Again I say unto you, that if two of you agree on earth as touching anything that they ask, it shall be done for them of my Father who is in heaven.}}
\bv{20}{\redlet{For where two or three are gathered together in my name, there am I in the midst of them.''}}
\chapsec{The Law of Forgiveness}
\bv{21}{Then came Peter and said to him, ``Lord, how oft shall my brother sin against me, and I forgive him? until seven times?''}
\bv{22}{Jesus saith unto him, \redlet{``I say not unto thee, Until seven times; but, Until seventy times seven.}}
\chapsec{The Parable of the Ungrateful Servant}
\bv{23}{\redlet{Therefore is the kingdom of heaven likened unto a certain king, who would make a reckoning with his servants.}}
\bv{24}{\redlet{And when he had begun to reckon, one was brought unto him, that owed him ten thousand talents.}}
\bv{25}{\redlet{But forasmuch as he had not \supptext{wherewith} to pay, his lord commanded him to be sold, and his wife, and children, and all that he had, and payment to be made.}}
\bv{26}{\redlet{The servant therefore fell down and worshipped him, saying, `Lord, have patience with me, and I will pay thee all.'}}
\bv{27}{\redlet{And the lord of that servant, being moved with compassion, released him, and forgave him the debt.}}
\bv{28}{\redlet{But that servant went out, and found one of his fellow-servants, who owed him a hundred shillings: and he laid hold on him, and took \supptext{him} by the throat, saying, `Pay what thou owest.'}}
\bv{29}{\redlet{So his fellow-servant fell down and besought him, saying, `Have patience with me, and I will pay thee.'}}
\bv{30}{\redlet{And he would not: but went and cast him into prison, till he should pay that which was due.}}
\par
\bv{31}{\redlet{So when his fellow-servants saw what was done, they were exceeding sorry, and came and told unto their lord all that was done.}}
\bv{32}{\redlet{Then his lord called him unto him, and saith to him, `Thou wicked servant, I forgave thee all that debt, because thou besoughtest me:}}
\bv{33}{\redlet{shouldest not thou also have had mercy on thy fellow-servant, even as I had mercy on thee?'}}
\bv{34}{\redlet{And his lord was wroth, and delivered him to the tormentors, till he should pay all that was due.}}
\bv{35}{\redlet{So shall also my heavenly Father do unto you, if ye forgive not every one his brother from your hearts.''}}
\chaphead{Chapter XIX}
\chapdesc{Jesus again in Judaea}
\lettrine[image=true, lines=4, findent=3pt, nindent=0pt]{Mt-A.eps}{nd} it came to pass when Jesus had finished these words, he departed from Galilee, and came into the borders of Judæa beyond the Jordan;
\bv{2}{and great multitudes followed him; and he healed them there.}
\chapsec{Teaching on Divorce}
\bv{3}{And there came unto him Pharisees, trying him, and saying, ``Is it lawful \supptext{for a man} to put away his wife for every cause?''}
\bv{4}{And he answered and said, \redlet{``Have ye not read, that he who made \supptext{them} from the beginning made them male and female,}}
\bv{5}{\redlet{and said, `For this cause shall a man leave his father and mother, and shall cleave to his wife; and the two shall become one flesh?'\mref{Gen. 2:24}}}
\bv{6}{\redlet{So that they are no more two, but one flesh. What therefore God hath joined together, let not man put asunder.''}}
\bv{7}{They say unto him, ``Why then did Moses command to give a bill of divorcement, and to put \supptext{her} away?''}
\bv{8}{He saith unto them, \redlet{``Moses for your hardness of heart suffered you to put away your wives: but from the beginning it hath not been so.}}
\bv{9}{\redlet{And I say unto you, Whosoever shall put away his wife, except for fornication, and shall marry another, committeth adultery: and he that marrieth her when she is put away committeth adultery.''}}
\chapsec{The Vocation of Celibacy}
\bv{10}{The disciples say unto him, ``If the case of the man is so with his wife, it is not expedient to marry.''}
\bv{11}{But he said unto them, \redlet{``Not all men can receive this saying, but they to whom it is given.}}
\bv{12}{\redlet{For there are eunuchs, that were so born from their mother's womb: and there are eunuchs, that were made eunuchs by men: and there are eunuchs, that made themselves eunuchs for the kingdom of heaven's sake. He that is able to receive it, let him receive it.''}}
\chapsec{Jesus Receives the Children}
\bv{13}{Then were there brought unto him little children, that he should lay his hands on them, and pray: and the disciples rebuked them.}
\bv{14}{But Jesus said, \redlet{``Suffer the little children, and forbid them not, to come unto me: for to such belongeth the kingdom of heaven.''}}
\bv{15}{And he laid his hands on them, and departed thence.}
\chapsec{The Rich Young Ruler}
\bv{16}{And behold, one came to him and said, ``Teacher, what good thing shall I do, that I may have eternal life?''}
\bv{17}{And he said unto him, \redlet{``Why askest thou me concerning that which is good? One there is who is good: but if thou wouldest enter into life, keep the commandments.''}}
\bv{18}{He saith unto him, ``Which?'' And Jesus said, \redlet{``Thou shalt not kill, Thou shalt not commit adultery, Thou shalt not steal, Thou shalt not bear false witness,}}
\bv{19}{\redlet{Honor thy father and thy mother; and, Thou shalt love thy neighbor as thyself.''}}
\bv{20}{The young man saith unto him, ``All these things have I observed: what lack I yet?''}
\bv{21}{Jesus said unto him, \redlet{``If thou wouldest be perfect, go, sell that which thou hast, and give to the poor, and thou shalt have treasure in heaven: and come, follow me.''}}
\bv{22}{But when the young man heard the saying, he went away sorrowful; for he was one that had great possessions.}
\bv{23}{And Jesus said unto his disciples, \redlet{``Verily I say unto you, It is hard for a rich man to enter into the kingdom of heaven.}}
\bv{24}{\redlet{And again I say unto you, It is easier for a camel to go through a needle's eye, than for a rich man to enter into the kingdom of God.''}}
\bv{25}{And when the disciples heard it, they were astonished exceedingly, saying, ``Who then can be saved?''}
\bv{26}{And Jesus looking upon \supptext{them} said to them, \redlet{``With men this is impossible; but with God all things are possible.''}}
\chapsec{The Apostles' Place in the Kingdom}
\bv{27}{Then answered Peter and said unto him, ``Lo, we have left all, and followed thee; what then shall we have?''}
\bv{28}{And Jesus said unto them, \redlet{``Verily I say unto you, that ye who have followed me, in the regeneration when the Son of man shall sit on the throne of his glory, ye also shall sit upon twelve thrones, judging the twelve tribes of Israel.}}
\bv{29}{\redlet{And every one that hath left houses, or brethren, or sisters, or father, or mother, or children, or lands, for my name's sake, shall receive a hundredfold, and shall inherit eternal life.}}
\bv{30}{\redlet{But many shall be last \supptext{that are} first; and first \supptext{that are} last.''}}
\chaphead{Chapter XX}
\chapdesc{Parable of the Labourers in the Vineyard}
\lettrine[image=true, lines=4, findent=3pt, nindent=0pt]{Lk-F.eps}{\redlet{or}} \redlet{the kingdom of heaven is like unto a man that was a householder, who went out early in the morning to hire laborers into his vineyard.}
\bv{2}{\redlet{And when he had agreed with the laborers for a shilling a day, he sent them into his vineyard.}}
\bv{3}{\redlet{And he went out about the third hour, and saw others standing in the marketplace idle;}}
\bv{4}{\redlet{and to them he said, `Go ye also into the vineyard, and whatsoever is right I will give you.' And they went their way.}}
\bv{5}{\redlet{Again he went out about the sixth and the ninth hour, and did likewise.}}
\bv{6}{\redlet{And about the eleventh \supptext{hour} he went out, and found others standing; and he saith unto them, `Why stand ye here all the day idle?'}}
\bv{7}{\redlet{They say unto him, `Because no man hath hired us.' He saith unto them, `Go ye also into the vineyard.'}}
\par
\bv{8}{\redlet{And when even was come, the lord of the vineyard saith unto his steward, `Call the laborers, and pay them their hire, beginning from the last unto the first.'}}
\bv{9}{\redlet{And when they came that \supptext{were hired} about the eleventh hour, they received every man a denarius.}}
\bv{10}{\redlet{And when the first came, they supposed that they would receive more; and they likewise received every man a denarius.}}
\bv{11}{\redlet{And when they received it, they murmured against the householder,}}
\bv{12}{\redlet{saying, `These last have spent \supptext{but} one hour, and thou hast made them equal unto us, who have borne the burden of the day and the scorching heat.'}}
\bv{13}{\redlet{But he answered and said to one of them, `Friend, I do thee no wrong: didst not thou agree with me for a denarius?}}
\bv{14}{\redlet{Take up that which is thine, and go thy way; it is my will to give unto this last, even as unto thee.}}
\bv{15}{\redlet{Is it not lawful for me to do what I will with mine own? or is thine eye evil, because I am good?'}}
\bv{16}{\redlet{So the last shall be first, and the first last.''}}
\par
\bv{17}{And as Jesus was going up to Jerusalem, he took the twelve disciples apart, and on the way he said unto them,}
\bv{18}{\redlet{``Behold, we go up to Jerusalem; and the Son of man shall be delivered unto the chief priests and scribes; and they shall condemn him to death,}}
\bv{19}{\redlet{and shall deliver him unto the Gentiles to mock, and to scourge, and to crucify: and the third day he shall be raised up.''}}
\chapsec{Sts. James \& John with their Mother}
\bv{20}{Then came to him the mother of the sons of Zebedee with her sons, worshipping \supptext{him}, and asking a certain thing of him.}
\bv{21}{And he said unto her, \redlet{``What wouldest thou?''} She saith unto him, ``Command that these my two sons may sit, one on thy right hand, and one on thy left hand, in thy kingdom.''}
\bv{22}{But Jesus answered and said, \redlet{``Ye know not what ye ask. Are ye able to drink the cup that I am about to drink?''} They say unto him, ``We are able.''}
\bv{23}{He saith unto them, \redlet{``My cup indeed ye shall drink: but to sit on my right hand, and on \supptext{my} left hand, is not mine to give; but \supptext{it is for them} for whom it hath been prepared of my Father.''}}
\par
\bv{24}{And when the ten heard it, they were moved with indignation concerning the two brethren.}
\bv{25}{But Jesus called them unto him, and said, \redlet{``Ye know that the rulers of the Gentiles lord it over them, and their great ones exercise authority over them.}}
\bv{26}{\redlet{Not so shall it be among you: but whosoever would become great among you shall be your minister;}}
\bv{27}{\redlet{and whosoever would be first among you shall be your servant:}}
\bv{28}{\redlet{even as the Son of man came not to be ministered unto, but to minister, and to give his life a ransom for many.''}}
\chapsec{The Healing of the Two Blind Men}
\bv{29}{And as they went out from Jericho, a great multitude followed him.}
\bv{30}{And behold, two blind men sitting by the way side, when they heard that Jesus was passing by, cried out, saying, ``Lord, have mercy on us, thou son of David.''}
\bv{31}{And the multitude rebuked them, that they should hold their peace: but they cried out the more, saying, ``Lord, have mercy on us, thou son of David.''}
\bv{32}{And Jesus stood still, and called them, and said, \redlet{``What will ye that I should do unto you?''}}
\bv{33}{They say unto him, ``Lord, that our eyes may be opened.''}
\bv{34}{And Jesus, being moved with compassion, touched their eyes; and straightway they received their sight, and followed him.}
\chaphead{Chapter XXI}
\chapdesc{Preparation for the Entrance into Jerusalem}
\lettrine[image=true, lines=4, findent=3pt, nindent=0pt]{Mt-A.eps}{nd} when they drew nigh unto Jerusalem, and came unto Bethphage, unto the mount of Olives, then Jesus sent two disciples,
\bv{2}{saying unto them, \redlet{``Go into the village that is over against you, and straightway ye shall find an ass tied, and a colt with her: loose \supptext{them}, and bring \supptext{them} unto me.}}
\bv{3}{\redlet{And if any one say aught unto you, ye shall say, `The Lord hath need of them; and straightway he will send them.'{''}}}
\bv{4}{Now this is come to pass, that it might be fulfilled which was spoken through the prophet, saying,}
\otQuote{Zech. 9:9}{\bv{5}{Tell ye the daughter of Zion,
Behold, thy King cometh unto thee,
Meek, and riding upon an ass,
And upon a colt the foal of an ass.}}
\bv{6}{And the disciples went, and did even as Jesus appointed them,}
\bv{7}{and brought the ass, and the colt, and put on them their garments; and he sat thereon.}
\chapsec{Entrance into Jerusalem}
\bv{8}{And the most part of the multitude spread their garments in the way; and others cut branches from the trees, and spread them in the way.}
\bv{9}{And the multitudes that went before him, and that followed, cried, saying, ``Hosanna to the son of David: Blessed \supptext{is} he that cometh in the name of the Lord; Hosanna in the highest.''}
\bv{10}{And when he was come into Jerusalem, all the city was stirred, saying, ``Who is this?''}
\bv{11}{And the multitudes said, ``This is the prophet, Jesus, from Nazareth of Galilee.''}
\chapsec{Jesus' Second Purification of the Temple}
\bv{12}{And Jesus entered into the temple of God, and cast out all them that sold and bought in the temple, and overthrew the tables of the money-changers, and the seats of them that sold the doves;}
\bv{13}{and he saith unto them, \redlet{``It is written, `My house shall be called a house of prayer: but ye make it a den of robbers.'{''}}}
\bv{14}{And the blind and the lame came to him in the temple; and he healed them.}
\bv{15}{But when the chief priests and the scribes saw the wonderful things that he did, and the children that were crying in the temple and saying, ``Hosanna to the son of David;'' they were moved with indignation,}
\bv{16}{and said unto him, ``Hearest thou what these are saying?'' And Jesus saith unto them, \redlet{``Yea: did ye never read, `Out of the mouth of babes and sucklings thou hast perfected praise?'{''}}}
\bv{17}{And he left them, and went forth out of the city to Bethany, and lodged there.}
\chapsec{The Barren Fig Tree Cursed}
\bv{18}{Now in the morning as he returned to the city, he hungered.}
\bv{19}{And seeing a fig tree by the way side, he came to it, and found nothing thereon, but leaves only; and he saith unto it, \redlet{``Let there be no fruit from thee henceforward for ever.''} And immediately the fig tree withered away.}
\bv{20}{And when the disciples saw it, they marvelled, saying, ``How did the fig tree immediately wither away?''}
\bv{21}{And Jesus answered and said unto them, \redlet{``Verily I say unto you, If ye have faith, and doubt not, ye shall not only do what is done to the fig tree, but even if ye shall say unto this mountain, `Be thou taken up and cast into the sea,' it shall be done.}}
\bv{22}{\redlet{And all things, whatsoever ye shall ask in prayer, believing, ye shall receive.''}}
\chapsec{Jesus' Authority Questioned}
\bv{23}{And when he was come into the temple, the chief priests and the elders of the people came unto him as he was teaching, and said, ``By what authority doest thou these things? and who gave thee this authority?''}
\bv{24}{And Jesus answered and said unto them, \redlet{``I also will ask you one question, which if ye tell me, I likewise will tell you by what authority I do these things.}}
\bv{25}{\redlet{The baptism of John, whence was it? from heaven or from men?''} And they reasoned with themselves, saying, ``If we shall say, `From heaven;' he will say unto us, `Why then did ye not believe him?'}
\bv{26}{But if we shall say, `From men;' we fear the multitude; for all hold John as a prophet.''}
\bv{27}{And they answered Jesus, and said, ``We know not.'' He also said unto them, \redlet{``Neither tell I you by what authority I do these things.}}
\chapsec{Parable of the Two Sons}
\bv{28}{\redlet{But what think ye? A man had two sons; and he came to the first, and said, `Son, go work to-day in the vineyard.'}}
\bv{29}{\redlet{And he answered and said, `I will not:' but afterward he repented himself, and went.}}
\bv{30}{\redlet{And he came to the second, and said likewise. And he answered and said, `I \supptext{go}, sir:' and went not.}}
\bv{31}{\redlet{Which of the two did the will of his father?''} They say, ``The first.'' Jesus saith unto them, \redlet{``Verily I say unto you, that the publicans and the harlots go into the kingdom of God before you.}}
\bv{32}{\redlet{For John came unto you in the way of righteousness, and ye believed him not; but the publicans and the harlots believed him: and ye, when ye saw it, did not even repent yourselves afterward, that ye might believe him.}}
\chapsec{Parable of the Householder}
\bv{33}{\redlet{Hear another parable: There was a man that was a householder, who planted a vineyard, and set a hedge about it, and digged a winepress in it, and built a tower, and let it out to husbandmen, and went into another country.}}
\bv{34}{\redlet{And when the season of the fruits drew near, he sent his servants to the husbandmen, to receive his fruits.}}
\bv{35}{\redlet{And the husbandmen took his servants, and beat one, and killed another, and stoned another.}}
\bv{36}{\redlet{Again, he sent other servants more than the first: and they did unto them in like manner.}}
\bv{37}{\redlet{But afterward he sent unto them his son, saying, `They will reverence my son.'}}
\bv{38}{\redlet{But the husbandmen, when they saw the son, said among themselves, `This is the heir; come, let us kill him, and take his inheritance.'}}
\bv{39}{\redlet{And they took him, and cast him forth out of the vineyard, and killed him.}}
\bv{40}{\redlet{When therefore the lord of the vineyard shall come, what will he do unto those husbandmen?''}}
\bv{41}{They say unto him, ``He will miserably destroy those miserable men, and will let out the vineyard unto other husbandmen, who shall render him the fruits in their seasons.''}
\bv{42}{Jesus saith unto them, \redlet{``Did ye never read in the scriptures,}}
\otQuote{Ps. 118:22-3}{\redlet{The stone which the builders rejected,
The same was made the head of the corner;
This was from the Lord,
And it is marvellous in our eyes?}}
\bv{43}{\redlet{Therefore say I unto you, The kingdom of God shall be taken away from you, and shall be given to a nation bringing forth the fruits thereof.}}
\bv{44}{\redlet{And he that falleth on this stone shall be broken to pieces: but on whomsoever it shall fall, it will scatter him as dust.''}}
\bv{45}{And when the chief priests and the Pharisees heard his parables, they perceived that he spake of them.}
\bv{46}{And when they sought to lay hold on him, they feared the multitudes, because they took him for a prophet.}
\chaphead{Chapter XXII}
\chapdesc{Parable of the Marriage Feast}
\lettrine[image=true, lines=4, findent=3pt, nindent=0pt]{Mt-A.eps}{nd} Jesus answered and spake again in parables unto them, saying,
\bv{2}{\redlet{``The kingdom of heaven is likened unto a certain king, who made a marriage feast for his son,}}
\bv{3}{\redlet{and sent forth his servants to call them that were bidden to the marriage feast: and they would not come.}}
\bv{4}{\redlet{Again he sent forth other servants, saying, `Tell them that are bidden, `Behold, I have made ready my dinner; my oxen and my fatlings are killed, and all things are ready: come to the marriage feast.'{'}}}
\bv{5}{\redlet{But they made light of it, and went their ways, one to his own farm, another to his merchandise;}}
\bv{6}{\redlet{and the rest laid hold on his servants, and treated them shamefully, and killed them.}}
\bv{7}{\redlet{But the king was wroth; and he sent his armies, and destroyed those murderers, and burned their city.}}
\bv{8}{\redlet{Then saith he to his servants, `The wedding is ready, but they that were bidden were not worthy.}}
\bv{9}{\redlet{Go ye therefore unto the partings of the highways, and as many as ye shall find, bid to the marriage feast.'}}
\par
\bv{10}{\redlet{And those servants went out into the highways, and gathered together all as many as they found, both bad and good: and the wedding was filled with guests.}}
\bv{11}{\redlet{But when the king came in to behold the guests, he saw there a man who had not on a wedding-garment:}}
\bv{12}{\redlet{and he saith unto him, `Friend, how camest thou in hither not having a wedding-garment?' And he was speechless.}}
\bv{13}{\redlet{Then the king said to the servants, `Bind him hand and foot, and cast him out into the outer darkness; there shall be the weeping and the gnashing of teeth.'}}
\bv{14}{\redlet{For many are called, but few chosen.''}}
\chapsec{Jesus Answers the Herodians}
\bv{15}{Then went the Pharisees, and took counsel how they might ensnare him in \supptext{his} talk.}
\bv{16}{And they send to him their disciples, with the Herodians, saying, ``Teacher, we know that thou art true, and teachest the way of God in truth, and carest not for any one: for thou regardest not the person of men.}
\bv{17}{Tell us therefore, What thinkest thou? Is it lawful to give tribute unto Cæsar, or not?''}
\bv{18}{But Jesus perceived their wickedness, and said, \redlet{``Why make ye trial of me, ye hypocrites?}}
\bv{19}{\redlet{Show me the tribute money.''} And they brought unto him a denarius.}
\bv{20}{And he saith unto them, \redlet{``Whose is this image and superscription?''}}
\bv{21}{They say unto him, ``Cæsar's.'' Then saith he unto them, \redlet{``Render therefore unto Cæsar the things that are Cæsar's; and unto God the things that are God's.''}}
\bv{22}{And when they heard it, they marvelled, and left him, and went away.}
\chapsec{Jesus Answers the Sadducees}
\bv{23}{On that day there came to him Sadducees, they that say that there is no resurrection: and they asked him,}
\bv{24}{saying, ``Teacher, Moses said, `If a man die, having no children, his brother shall marry his wife, and raise up seed unto his brother.'}
\bv{25}{Now there were with us seven brethren: and the first married and deceased, and having no seed left his wife unto his brother;}
\bv{26}{in like manner the second also, and the third, unto the seventh.}
\bv{27}{And after them all, the woman died.}
\bv{28}{In the resurrection therefore whose wife shall she be of the seven? for they all had her.''}
\bv{29}{But Jesus answered and said unto them, \redlet{``Ye do err, not knowing the scriptures, nor the power of God.}}
\bv{30}{\redlet{For in the resurrection they neither marry, nor are given in marriage, but are as angels in heaven.}}
\bv{31}{\redlet{But as touching the resurrection of the dead, have ye not read that which was spoken unto you by God, saying,}}
\bv{32}{\redlet{`I am the God of Abraham, and the God of Isaac, and the God of Jacob?' God is not \supptext{the God} of the dead, but of the living.''}}
\bv{33}{And when the multitudes heard it, they were astonished at his teaching.}
\chapsec{Jesus Answers the Pharisees}
\bv{34}{But the Pharisees, when they heard that he had put the Sadducees to silence, gathered themselves together.}
\bv{35}{And one of them, a lawyer, asked him a question, trying him:}
\bv{36}{``Teacher, which is the great commandment in the law?''}
\bv{37}{And he said unto him, \redlet{``Thou shalt love the Lord thy God with all thy heart, and with all thy soul, and with all thy mind.}}
\bv{38}{\redlet{This is the great and first commandment.}}
\bv{39}{\redlet{And a second like \supptext{unto it} is this, `Thou shalt love thy neighbor as thyself.'}}
\bv{40}{\redlet{On these two commandments the whole law hangeth, and the prophets.''}}
\chapsec{Jesus Questions the Pharisees}
\bv{41}{Now while the Pharisees were gathered together, Jesus asked them a question,}
\bv{42}{saying, \redlet{``What think ye of the Christ? whose son is he?''} They say unto him, ``\supptext{The son} of David.''}
\bv{43}{He saith unto them, \redlet{``How then doth David in the Spirit call him Lord, saying,}}
\otQuote{Ps. 110:1}{\redlet{\bv{44}{The Lord said unto my Lord,
Sit thou on my right hand,
Till I put thine enemies underneath thy feet?}}}
\bv{45}{\redlet{If David then calleth him Lord, how is he his son?''}}
\bv{46}{And no one was able to answer him a word, neither durst any man from that day forth ask him any more questions.}
\chaphead{Chapter XXIII}
\chapdesc{The Marks of a Pharisee}
\lettrine[image=true, lines=4, findent=3pt, nindent=0pt]{T.ps}{hen} spake Jesus to the multitudes and to his disciples,
\bv{2}{saying, \redlet{``The scribes and the Pharisees sit on Moses' seat:}}
\bv{3}{\redlet{all things therefore whatsoever they bid you, \supptext{these} do and observe: but do not ye after their works; for they say, and do not.}}
\bv{4}{\redlet{Yea, they bind heavy burdens and grievous to be borne, and lay them on men's shoulders; but they themselves will not move them with their finger.}}
\bv{5}{\redlet{But all their works they do to be seen of men: for they make broad their phylacteries, and enlarge the borders \supptext{of their garments},}}
\bv{6}{\redlet{and love the chief place at feasts, and the chief seats in the synagogues,}}
\bv{7}{\redlet{and the salutations in the marketplaces, and to be called of men, Rabbi.}}
\bv{8}{\redlet{But be not ye called Rabbi: for one is your teacher, and all ye are brethren.}}
\bv{9}{\redlet{And call no man your father on the earth: for one is your Father, \supptext{even} he who is in heaven.}}
\bv{10}{\redlet{Neither be ye called masters: for one is your master, \supptext{even} the Christ.}}
\bv{11}{\redlet{But he that is greatest among you shall be your servant.}}
\bv{12}{\redlet{And whosoever shall exalt himself shall be humbled; and whosoever shall humble himself shall be exalted.}}
\chapsec{Jesus Denounces Woe upon the Pharisees}
\bv{13}{\redlet{But woe unto you, scribes and Pharisees, hypocrites! because ye shut the kingdom of heaven against men: for ye enter not in yourselves, neither suffer ye them that are entering in to enter.\mcomm{Woe unto you scribes and Pharisees, hypocrites! for you devour widows' houses, even while for a pretence ye make long prayes: therefore ye shall receive greater condemnation.}}}
\bv{15}{\redlet{Woe unto you, scribes and Pharisees, hypocrites! for ye compass sea and land to make one proselyte; and when he is become so, ye make him twofold more a son of hell than yourselves.}}
\bv{16}{\redlet{Woe unto you, ye blind guides, that say, Whosoever shall swear by the temple, it is nothing; but whosoever shall swear by the gold of the temple, he is a debtor.}}
\par
\bv{17}{\redlet{Ye fools and blind: for which is greater, the gold, or the temple that hath sanctified the gold?}}
\bv{18}{\redlet{And, Whosoever shall swear by the altar, it is nothing; but whosoever shall swear by the gift that is upon it, he is a debtor.}}
\bv{19}{\redlet{Ye blind: for which is greater, the gift, or the altar that sanctifieth the gift?}}
\bv{20}{\redlet{He therefore that sweareth by the altar, sweareth by it, and by all things thereon.}}
\bv{21}{\redlet{And he that sweareth by the temple, sweareth by it, and by him that dwelleth therein.}}
\bv{22}{\redlet{And he that sweareth by the heaven, sweareth by the throne of God, and by him that sitteth thereon.}}
\par
\bv{23}{\redlet{Woe unto you, scribes and Pharisees, hypocrites! for ye tithe mint and anise and cummin, and have left undone the weightier matters of the law, justice, and mercy, and faith: but these ye ought to have done, and not to have left the other undone.}}
\bv{24}{\redlet{Ye blind guides, that strain out the gnat, and swallow the camel!}}
\bv{25}{\redlet{Woe unto you, scribes and Pharisees, hypocrites! for ye cleanse the outside of the cup and of the platter, but within they are full from extortion and excess.}}
\bv{26}{\redlet{Thou blind Pharisee, cleanse first the inside of the cup and of the platter, that the outside thereof may become clean also.}}
\par
\bv{27}{\redlet{Woe unto you, scribes and Pharisees, hypocrites! for ye are like unto whited sepulchres, which outwardly appear beautiful, but inwardly are full of dead men's bones, and of all uncleanness.}}
\bv{28}{\redlet{Even so ye also outwardly appear righteous unto men, but inwardly ye are full of hypocrisy and iniquity.}}
\bv{29}{\redlet{Woe unto you, scribes and Pharisees, hypocrites! for ye build the sepulchres of the prophets, and garnish the tombs of the righteous,}}
\bv{30}{\redlet{and say, `If we had been in the days of our fathers, we should not have been partakers with them in the blood of the prophets.'}}
\bv{31}{\redlet{Wherefore ye witness to yourselves, that ye are sons of them that slew the prophets.}}
\bv{32}{\redlet{Fill ye up then the measure of your fathers.}}
\par
\bv{33}{\redlet{Ye serpents, ye offspring of vipers, how shall ye escape the judgement of hell?}}
\bv{34}{\redlet{Therefore, behold, I send unto you prophets, and wise men, and scribes: some of them shall ye kill and crucify; and some of them shall ye scourge in your synagogues, and persecute from city to city:}}
\bv{35}{\redlet{that upon you may come all the righteous blood shed on the earth, from the blood of Abel the righteous unto the blood of Zechariah son of Barachiah, whom ye slew between the sanctuary and the altar.}}
\bv{36}{\redlet{Verily I say unto you, All these things shall come upon this generation.}}
\chapsec{The Lament over Jerusalem}
\bv{37}{\redlet{O Jerusalem, Jerusalem, that killeth the prophets, and stoneth them that are sent unto her! how often would I have gathered thy children together, even as a hen gathereth her chickens under her wings, and ye would not!}}
\bv{38}{\redlet{Behold, your house is left unto you desolate.}}
\bv{39}{\redlet{For I say unto you, Ye shall not see me henceforth, till ye shall say, `Blessed \supptext{is} he that cometh in the name of the Lord.'{''}}}
\chaphead{Chapter XXIV}
\chapdesc{The Olivet Discourse}
\lettrine[image=true, lines=4, findent=3pt, nindent=0pt]{Mt-A.eps}{nd} Jesus went out from the temple, and was going on his way; and his disciples came to him to show him the buildings of the temple.
\bv{2}{But he answered and said unto them, \redlet{``See ye not all these things? verily I say unto you, There shall not be left here one stone upon another, that shall not be thrown down.''}}
\par
\bv{3}{And as he sat on the mount of Olives, the disciples came unto him privately, saying, ``Tell us, when shall these things be? and what \supptext{shall be} the sign of thy coming, and of the end of the world?''}
\par
\bv{4}{And Jesus answered and said unto them, \redlet{``Take heed that no man lead you astray.}}
\bv{5}{\redlet{For many shall come in my name, saying, `I am the Christ;' and shall lead many astray.}}
\bv{6}{\redlet{And ye shall hear of wars and rumors of wars; see that ye be not troubled: for \supptext{these things} must needs come to pass; but the end is not yet.}}
\bv{7}{\redlet{For nation shall rise against nation, and kingdom against kingdom; and there shall be famines and earthquakes in divers places.}}
\bv{8}{\redlet{But all these things are the beginning of travail.}}
\bv{9}{\redlet{Then shall they deliver you up unto tribulation, and shall kill you: and ye shall be hated of all the nations for my name's sake.}}
\bv{10}{\redlet{And then shall many stumble, and shall deliver up one another, and shall hate one another.}}
\bv{11}{\redlet{And many false prophets shall arise, and shall lead many astray.}}
\bv{12}{\redlet{And because iniquity shall be multiplied, the love of the many shall wax cold.}}
\bv{13}{\redlet{But he that endureth to the end, the same shall be saved.}}
\bv{14}{\redlet{And this gospel of the kingdom shall be preached in the whole world for a testimony unto all the nations; and then shall the end come.}}
\chapsec{The Great Tribulation}
\bv{15}{\redlet{When therefore ye see the abomination of desolation, which was spoken of through Daniel the prophet, standing in the holy place (let him that readeth understand),}}
\bv{16}{\redlet{then let them that are in Judæa flee unto the mountains:}}
\bv{17}{\redlet{let him that is on the housetop not go down to take out the things that are in his house:}}
\bv{18}{\redlet{and let him that is in the field not return back to take his cloak.}}
\bv{19}{\redlet{But woe unto them that are with child and to them that give suck in those days!}}
\bv{20}{\redlet{And pray ye that your flight be not in the winter, neither on a sabbath:}}
\bv{21}{\redlet{for then shall be great tribulation, such as hath not been from the beginning of the world until now, no, nor ever shall be.}}
\bv{22}{\redlet{And except those days had been shortened, no flesh would have been saved: but for the elect's sake those days shall be shortened.}}
\par
\bv{23}{\redlet{Then if any man shall say unto you, `Lo, here is the Christ,' or, `Here;' believe \supptext{it} not.}}
\bv{24}{\redlet{For there shall arise false Christs, and false prophets, and shall show great signs and wonders; so as to lead astray, if possible, even the elect.}}
\bv{25}{\redlet{Behold, I have told you beforehand.}}
\bv{26}{\redlet{If therefore they shall say unto you, `Behold, he is in the wilderness;' go not forth: `Behold, he is in the inner chambers;' believe \supptext{it} not.}}
\chapsec{The Return of the King in Glory}
\bv{27}{\redlet{For as the lightning cometh forth from the east, and is seen even unto the west; so shall be the coming of the Son of man.}}
\bv{28}{\redlet{Wheresoever the carcase is, there will the eagles be gathered together.}}
\bv{29}{\redlet{But immediately after the tribulation of those days the sun shall be darkened, and the moon shall not give her light, and the stars shall fall from heaven, and the powers of the heavens shall be shaken:}}
\bv{30}{\redlet{and then shall appear the sign of the Son of man in heaven: and then shall all the tribes of the earth mourn, and they shall see the Son of man coming on the clouds of heaven with power and great glory.}}
\bv{31}{\redlet{And he shall send forth his angels with a great sound of a trumpet, and they shall gather together his elect from the four winds, from one end of heaven to the other.}}
\chapsec{Parable of the Fig Tree}
\bv{32}{\redlet{Now from the fig tree learn her parable: when her branch is now become tender, and putteth forth its leaves, ye know that the summer is nigh;}}
\bv{33}{\redlet{even so ye also, when ye see all these things, know ye that he is nigh, \supptext{even} at the doors.}}
\bv{34}{\redlet{Verily I say unto you, This generation shall not pass away, till all these things be accomplished.}}
\bv{35}{\redlet{Heaven and earth shall pass away, but my words shall not pass away.}}
\bv{36}{\redlet{But of that day and hour knoweth no one, not even the angels of heaven, neither the Son, but the Father only.}}
\par
\bv{37}{\redlet{And as \supptext{were} the days of Noah, so shall be the coming of the Son of man.}}
\bv{38}{\redlet{For as in those days which were before the flood they were eating and drinking, marrying and giving in marriage, until the day that Noah entered into the ark,}}
\bv{39}{\redlet{and they knew not until the flood came, and took them all away; so shall be the coming of the Son of man.}}
\bv{40}{\redlet{Then shall two men be in the field; one is taken, and one is left:}}
\bv{41}{\redlet{two women \supptext{shall be} grinding at the mill; one is taken, and one is left.}}
\bv{42}{\redlet{Watch therefore: for ye know not on what day your Lord cometh.}}
\bv{43}{\redlet{But know this, that if the master of the house had known in what watch the thief was coming, he would have watched, and would not have suffered his house to be broken through.}}
\bv{44}{\redlet{Therefore be ye also ready; for in an hour that ye think not the Son of man cometh.}}
\par
\bv{45}{\redlet{Who then is the faithful and wise servant, whom his lord hath set over his household, to give them their food in due season?}}
\bv{46}{\redlet{Blessed is that servant, whom his lord when he cometh shall find so doing.}}
\bv{47}{\redlet{Verily I say unto you, that he will set him over all that he hath.}}
\bv{48}{\redlet{But if that evil servant shall say in his heart, `My lord tarrieth;'}}
\bv{49}{\redlet{and shall begin to beat his fellow-servants, and shall eat and drink with the drunken;}}
\bv{50}{\redlet{the lord of that servant shall come in a day when he expecteth not, and in an hour when he knoweth not,}}
\bv{51}{\redlet{and shall cut him asunder, and appoint his portion with the hypocrites: there shall be the weeping and the gnashing of teeth.}}
\chaphead{Chapter XXV}
\chapdesc{Parable of the Foolish Virgins}
\lettrine[image=true, lines=4, findent=3pt, nindent=0pt]{T.ps}{\redlet{hen}} \redlet{shall the kingdom of heaven be likened unto ten virgins, who took their lamps, and went forth to meet the bridegroom.}
\bv{2}{\redlet{And five of them were foolish, and five were wise.}}
\bv{3}{\redlet{For the foolish, when they took their lamps, took no oil with them:}}
\bv{4}{\redlet{but the wise took oil in their vessels with their lamps.}}
\bv{5}{\redlet{Now while the bridegroom tarried, they all slumbered and slept.}}
\bv{6}{\redlet{But at midnight there is a cry, `Behold, the bridegroom! Come ye forth to meet him.'}}
\bv{7}{\redlet{Then all those virgins arose, and trimmed their lamps.}}
\bv{8}{\redlet{And the foolish said unto the wise, `Give us of your oil; for our lamps are going out.'}}
\bv{9}{\redlet{But the wise answered, saying, `Peradventure there will not be enough for us and you: go ye rather to them that sell, and buy for yourselves.'}}
\bv{10}{\redlet{And while they went away to buy, the bridegroom came; and they that were ready went in with him to the marriage feast: and the door was shut.}}
\bv{11}{\redlet{Afterward came also the other virgins, saying, `Lord, Lord, open to us.'}}
\bv{12}{\redlet{But he answered and said, `Verily I say unto you, I know you not.'}}
\bv{13}{\redlet{Watch therefore, for ye know not the day nor the hour.}}\
\chapsec{The Parable of the Servants}
\bv{14}{\redlet{For \supptext{it is} as \supptext{when} a man, going into another country, called his own servants, and delivered unto them his goods.}}
\bv{15}{\redlet{And unto one he gave five talents,\mcomm{Talent: Unit of Weight or Measurement} to another two, to another one; to each according to his several ability; and he went on his journey.}}
\bv{16}{\redlet{Straightway he that received the five talents went and traded with them, and made other five talents.}}
\bv{17}{\redlet{In like manner he also that \supptext{received} the two gained other two.}}
\bv{18}{\redlet{But he that received the one went away and digged in the earth, and hid his lord's money.}}
\bv{19}{\redlet{Now after a long time the lord of those servants cometh, and maketh a reckoning with them.}}
\par
\bv{20}{\redlet{And he that received the five talents came and brought other five talents, saying, `Lord, thou deliveredst unto me five talents: lo, I have gained other five talents.'}}
\bv{21}{\redlet{His lord said unto him, `Well done, good and faithful servant: thou hast been faithful over a few things, I will set thee over many things; enter thou into the joy of thy lord.'}}
\bv{22}{\redlet{And he also that \supptext{received} the two talents came and said, `Lord, thou deliveredst unto me two talents: lo, I have gained other two talents.'}}
\bv{23}{\redlet{His lord said unto him, `Well done, good and faithful servant: thou hast been faithful over a few things, I will set thee over many things; enter thou into the joy of thy lord.'}}
\bv{24}{\redlet{And he also that had received the one talent came and said, `Lord, I knew thee that thou art a hard man, reaping where thou didst not sow, and gathering where thou didst not scatter;}}
\bv{25}{\redlet{and I was afraid, and went away and hid thy talent in the earth: lo, thou hast thine own.'}}
\bv{26}{\redlet{But his lord answered and said unto him, `Thou wicked and slothful servant, thou knewest that I reap where I sowed not, and gather where I did not scatter;}}
\bv{27}{\redlet{thou oughtest therefore to have put my money to the bankers, and at my coming I should have received back mine own with interest.}}
\bv{28}{\redlet{Take ye away therefore the talent from him, and give it unto him that hath the ten talents.'}}
\bv{29}{\redlet{For unto every one that hath shall be given, and he shall have abundance: but from him that hath not, even that which he hath shall be taken away.}}
\bv{30}{\redlet{And cast ye out the unprofitable servant into the outer darkness: there shall be the weeping and the gnashing of teeth.}}
\chapsec{Final Judgement}
\bv{31}{\redlet{But when the Son of man shall come in his glory, and all the angels with him, then shall he sit on the throne of his glory:}}
\bv{32}{\redlet{and before him shall be gathered all the nations: and he shall separate them one from another, as the shepherd separateth the sheep from the goats;}}
\bv{33}{\redlet{and he shall set the sheep on his right hand, but the goats on the left.}}
\bv{34}{\redlet{Then shall the King say unto them on his right hand, `Come, ye blessed of my Father, inherit the kingdom prepared for you from the foundation of the world:}}
\bv{35}{\redlet{for I was hungry, and ye gave me to eat; I was thirsty, and ye gave me drink; I was a stranger, and ye took me in;}}
\bv{36}{\redlet{naked, and ye clothed me; I was sick, and ye visited me; I was in prison, and ye came unto me.'}}
\bv{37}{\redlet{Then shall the righteous answer him, saying, `Lord, when saw we thee hungry, and fed thee? or athirst, and gave thee drink?}}
\bv{38}{\redlet{And when saw we thee a stranger, and took thee in? or naked, and clothed thee?}}
\bv{39}{\redlet{And when saw we thee sick, or in prison, and came unto thee?'}}
\bv{40}{\redlet{And the King shall answer and say unto them, `Verily I say unto you, Inasmuch as ye did it unto one of these my brethren, \supptext{even} these least, ye did it unto me.'}}
\par
\bv{41}{\redlet{Then shall he say also unto them on the left hand, `Depart from me, ye cursed, into the eternal fire which is prepared for the devil and his angels:}}
\bv{42}{\redlet{for I was hungry, and ye did not give me to eat; I was thirsty, and ye gave me no drink;}}
\bv{43}{\redlet{I was a stranger, and ye took me not in; naked, and ye clothed me not; sick, and in prison, and ye visited me not.'}}
\bv{44}{\redlet{Then shall they also answer, saying, `Lord, when saw we thee hungry, or athirst, or a stranger, or naked, or sick, or in prison, and did not minister unto thee?'}}
\bv{45}{\redlet{Then shall he answer them, saying, `Verily I say unto you, Inasmuch as ye did it not unto one of these least, ye did it not unto me.'}}
\bv{46}{\redlet{And these shall go away into eternal punishment: but the righteous into eternal life.''}}
\chaphead{Chapter XXVI}
\chapdesc{The Jews Conspire to Put Jesus to Death}
\lettrine[image=true, lines=4, findent=3pt, nindent=0pt]{Mt-A.eps}{nd} it came to pass, when Jesus had finished all these words, he said unto his disciples,
\bv{2}{\redlet{``Ye know that after two days the passover cometh, and the Son of man is delivered up to be crucified.''}}
\bv{3}{Then were gathered together the chief priests, and the elders of the people, unto the court of the high priest, who was called Caiaphas;}
\bv{4}{and they took counsel together that they might take Jesus by subtlety, and kill him.}
\bv{5}{But they said, ``Not during the feast, lest a tumult arise among the people.''}
\chapsec{Mary of Bethany Anoints Jesus}
\bv{6}{Now when Jesus was in Bethany, in the house of Simon the leper,}
\bv{7}{there came unto him a woman having an alabaster cruse of exceeding precious ointment, and she poured it upon his head, as he sat at meat.}
\bv{8}{But when the disciples saw it, they had indignation, saying, ``To what purpose is this waste?}
\bv{9}{For this \supptext{ointment} might have been sold for much, and given to the poor.''}
\bv{10}{But Jesus perceiving it said unto them, \redlet{	``Why trouble ye the woman? for she hath wrought a good work upon me.}}
\bv{11}{\redlet{For ye have the poor always with you; but me ye have not always.}}
\bv{12}{\redlet{For in that she poured this ointment upon my body, she did it to prepare me for burial.}}
\bv{13}{\redlet{Verily I say unto you, Wheresoever this gospel shall be preached in the whole world, that also which this woman hath done shall be spoken of for a memorial of her.''}}
\chapsec{Judas Iscariot Sells the Lord}
\bv{14}{Then one of the twelve, who was called Judas Iscariot, went unto the chief priests,}
\bv{15}{and said, ``What are ye willing to give me, and I will deliver him unto you?'' And they weighed unto him thirty pieces of silver.}
\bv{16}{And from that time he sought opportunity to deliver him \supptext{unto them}.}
\chapsec{Preparation of the Passover}
\bv{17}{Now on the first \supptext{day} of unleavened bread the disciples came to Jesus, saying, \redlet{``Where wilt thou that we make ready for thee to eat the passover?''}}
\bv{18}{And he said, \redlet{``Go into the city to such a man, and say unto him, `The Teacher saith, `My time is at hand; I keep the passover at thy house with my disciples.'{'}{''}}}
\bv{19}{And the disciples did as Jesus appointed them; and they made ready the passover.}
\chapsec{The Last Passover}
\bv{20}{Now when even was come, he was sitting at meat with the twelve disciples;}
\bv{21}{and as they were eating, he said, \redlet{``Verily I say unto you, that one of you shall betray me.''}}
\bv{22}{And they were exceeding sorrowful, and began to say unto him every one, ``Is it I, Lord?''}
\bv{23}{And he answered and said, \redlet{``He that dipped his hand with me in the dish, the same shall betray me.}}
\bv{24}{\redlet{The Son of man goeth, even as it is written of him: but woe unto that man through whom the Son of man is betrayed! good were it for that man if he had not been born.''}}
\bv{25}{And Judas, who betrayed him, answered and said, ``Is it I, Rabbi?'' He saith unto him, \redlet{``Thou hast said.''}}
\chapsec{Institution of the Eucharist}
\bv{26}{And as they were eating, Jesus took bread, and blessed, and brake it; and he gave to the disciples, and said, \redlet{``Take, eat; this is my body.''}}
\bv{27}{And he took a cup, and gave thanks, and gave to them, saying, \redlet{``Drink ye all of it;}}
\bv{28}{\redlet{for this is my blood of the covenant, which is poured out for many unto remission of sins.}}
\bv{29}{\redlet{But I say unto you, I shall not drink henceforth of this fruit of the vine, until that day when I drink it new with you in my Father's kingdom.''}}
\chapsec{Jesus Foretells St. Peter's Denial}
\bv{30}{And when they had sung a hymn, they went out into the mount of Olives.}
\bv{31}{Then saith Jesus unto them, \redlet{``All ye shall be offended in me this night: for it is written,}}
\otQuote{Zech. 13:7}{\redlet{I will smite the shepherd, and the sheep of the flock shall be scattered abroad.}}
\bv{32}{\redlet{But after I am raised up, I will go before you into Galilee.''}}
\bv{33}{But Peter answered and said unto him, ``If all shall be offended in thee, I will never be offended.''}
\bv{34}{Jesus said unto him, \redlet{``Verily I say unto thee, that this night, before the cock crow, thou shalt deny me thrice.''}}
\bv{35}{Peter saith unto him, ``Even if I must die with thee, \supptext{yet} will I not deny thee.'' Likewise also said all the disciples.}
\chapsec{The Agony in the Garden}
\bv{36}{Then cometh Jesus with them unto a place called Gethsemane, and saith unto his disciples, \redlet{``Sit ye here, while I go yonder and pray.''}}
\bv{37}{And he took with him Peter and the two sons of Zebedee, and began to be sorrowful and sore troubled.}
\bv{38}{Then saith he unto them, \redlet{``My soul is exceeding sorrowful, even unto death: abide ye here, and watch with me.''}}
\par
\bv{39}{And he went forward a little, and fell on his face, and prayed, saying, \redlet{``My Father, if it be possible, let this cup pass away from me: nevertheless, not as I will, but as thou wilt.''}}
\par
\bv{40}{And he cometh unto the disciples, and findeth them sleeping, and saith unto Peter, \redlet{``What, could ye not watch with me one hour?}}
\bv{41}{\redlet{Watch and pray, that ye enter not into temptation: the spirit indeed is willing, but the flesh is weak.''}}
\par
\bv{42}{Again a second time he went away, and prayed, saying, \redlet{``My Father, if this cannot pass away, except I drink it, thy will be done.''}}
\bv{43}{And he came again and found them sleeping, for their eyes were heavy.}
\par
\bv{44}{And he left them again, and went away, and prayed a third time, saying again the same words.}
\bv{45}{Then cometh he to the disciples, and saith unto them, \redlet{``Sleep on now, and take your rest: behold, the hour is at hand, and the Son of man is betrayed into the hands of sinners.}}
\bv{46}{\redlet{Arise, let us be going: behold, he is at hand that betrayeth me.''}}
\chapsec{The Betrayal \& Arrest of Jesus}
\bv{47}{And while he yet spake, lo, Judas, one of the twelve, came, and with him a great multitude with swords and staves, from the chief priests and elders of the people.}
\bv{48}{Now he that betrayed him gave them a sign, saying, ``Whomsoever I shall kiss, that is he: take him.''}
\bv{49}{And straightway he came to Jesus, and said, ``Hail, Rabbi;'' and kissed him.}
\bv{50}{And Jesus said unto him, \redlet{``Friend, \supptext{do} that for which thou art come.''} Then they came and laid hands on Jesus, and took him.}
\bv{51}{And behold, one of them that were with Jesus stretched out his hand, and drew his sword, and smote the servant of the high priest, and struck off his ear.}
\bv{52}{Then saith Jesus unto him, \redlet{``Put up again thy sword into its place: for all they that take the sword shall perish with the sword.}}
\bv{53}{\redlet{Or thinkest thou that I cannot beseech my Father, and he shall even now send me more than twelve legions of angels?}}
\bv{54}{\redlet{How then should the scriptures be fulfilled, that thus it must be?''}}
\bv{55}{In that hour said Jesus to the multitudes, \redlet{``Are ye come out as against a robber with swords and staves to seize me? I sat daily in the temple teaching, and ye took me not.}}
\bv{56}{\redlet{But all this is come to pass, that the scriptures of the prophets might be fulfilled.''} Then all the disciples left him, and fled.}
\chapsec{Jesus Brought before Caiaphas}
\bv{57}{And they that had taken Jesus led him away to \supptext{the house of} Caiaphas the high priest, where the scribes and the elders were gathered together.}
\bv{58}{But Peter followed him afar off, unto the court of the high priest, and entered in, and sat with the officers, to see the end.}
\bv{59}{Now the chief priests and the whole council sought false witness against Jesus, that they might put him to death;}
\bv{60}{and they found it not, though many false witnesses came. But afterward came two,}
\bv{61}{and said, ``This man said, `I am able to destroy the temple of God, and to build it in three days.'{''}}
\bv{62}{And the high priest stood up, and said unto him, ``Answerest thou nothing? what is it which these witness against thee?''}
\bv{63}{But Jesus held his peace. And the high priest said unto him, ``I adjure thee by the living God, that thou tell us whether thou art the Christ, the Son of God.''}
\bv{64}{Jesus saith unto him, \redlet{``Thou hast said: nevertheless I say unto you, Henceforth ye shall see the Son of man sitting at the right hand of Power, and coming on the clouds of heaven.''}}
\par
\bv{65}{Then the high priest rent his garments, saying, ``He hath spoken blasphemy: what further need have we of witnesses? behold, now ye have heard the blasphemy:}
\bv{66}{what think ye?'' They answered and said, ``He is worthy of death.''}
\bv{67}{Then did they spit in his face and buffet him: and some smote him with the palms of their hands,}
\bv{68}{saying, ``Prophesy unto us, thou Christ: who is he that struck thee?''}
\chapsec{St. Peter Denies the Lord}
\bv{69}{Now Peter was sitting without in the court: and a maid came unto him, saying, ``Thou also wast with Jesus the Galilæan.''}
\bv{70}{But he denied before them all, saying, ``I know not what thou sayest.''}
\bv{71}{And when he was gone out into the porch, another \supptext{maid} saw him, and saith unto them that were there, ``This man also was with Jesus of Nazareth.''}
\bv{72}{And again he denied with an oath, ``I know not the man.''}
\bv{73}{And after a little while they that stood by came and said to Peter, ``Of a truth thou also art \supptext{one} of them; for thy speech maketh thee known.''}
\bv{74}{Then began he to curse and to swear, ``I know not the man.'' And straightway the cock crew.}
\bv{75}{And Peter remembered the word which Jesus had said, \redlet{``Before the cock crow, thou shalt deny me thrice.''} And he went out, and wept bitterly.}
\chaphead{Chapter XXVII}
\chapdesc{The Sanhedrin Deliver Jesus to Pilate}
\lettrine[image=true, lines=4, findent=3pt, nindent=0pt]{Mt-N.eps}{ow} when morning was come, all the chief priests and the elders of the people took counsel against Jesus to put him to death:
\bv{2}{and they bound him, and led him away, and delivered him up to Pilate the governor.}
\chapsec{Judas Iscariot's Proud Repentance}
\bv{3}{Then Judas, who betrayed him, when he saw that he was condemned, repented himself, and brought back the thirty pieces of silver to the chief priests and elders,}
\bv{4}{saying, ``I have sinned in that I betrayed innocent blood.'' But they said, ``What is that to us? see thou \supptext{to it}.''}
\bv{5}{And he cast down the pieces of silver into the sanctuary, and departed; and he went away and hanged himself.}
\bv{6}{And the chief priests took the pieces of silver, and said, ``It is not lawful to put them into the treasury, since it is the price of blood.''}
\bv{7}{And they took counsel, and bought with them the potter's field, to bury strangers in.}
\bv{8}{Wherefore that field was called, ``The field of blood,'' unto this day.}
\bv{9}{Then was fulfilled that which was spoken through Jeremiah the prophet, saying,}
\otQuote{Zech. 11:13}{And they took the thirty pieces of silver, the price of him that was priced, whom \supptext{certain} of the children of Israel did price; \bv{10}{and they gave them for the potter's field, as the Lord appointed me.}}
\chapsec{Jesus Interrogated by Pilate}
\bv{11}{Now Jesus stood before the governor: and the governor asked him, saying, ``Art thou the King of the Jews?'' And Jesus said unto him, \redlet{``Thou sayest.''}}
\bv{12}{And when he was accused by the chief priests and elders, he answered nothing.}
\bv{13}{Then saith Pilate unto him, ``Hearest thou not how many things they witness against thee?''}
\bv{14}{And he gave him no answer, not even to one word: insomuch that the governor marvelled greatly.}
\chapsec{Jesus or Barabbas?}
\bv{15}{Now at the feast the governor was wont to release unto the multitude one prisoner, whom they would.}
\bv{16}{And they had then a notable prisoner, called Barabbas.}
\bv{17}{When therefore they were gathered together, Pilate said unto them, ``Whom will ye that I release unto you? Barabbas, or Jesus who is called Christ?''}
\bv{18}{For he knew that for envy they had delivered him up.}
\bv{19}{And while he was sitting on the judgement-seat, his wife sent unto him, saying, ``Have thou nothing to do with that righteous man; for I have suffered many things this day in a dream because of him.''}
\bv{20}{Now the chief priests and the elders persuaded the multitudes that they should ask for Barabbas, and destroy Jesus.}
\chapsec{Jewish Apostacy}
\bv{21}{But the governor answered and said unto them, ``Which of the two will ye that I release unto you?'' And they said, ``Barabbas.''}
\bv{22}{Pilate saith unto them, ``What then shall I do unto Jesus who is called Christ?'' They all say, ``Let him be crucified.''}
\bv{23}{And he said, ``Why, what evil hath he done?'' But they cried out exceedingly, saying, ``Let him be crucified.''}
\bv{24}{So when Pilate saw that he prevailed nothing, but rather that a tumult was arising, he took water, and washed his hands before the multitude, saying, ``I am innocent of the blood of this righteous man; see ye \supptext{to it}.''}
\bv{25}{And all the people answered and said, ``His blood \supptext{be} on us, and on our children.''}
\chapsec{Barabbas Released}
\bv{26}{Then released he unto them Barabbas; but Jesus he scourged and delivered to be crucified.}
\chapsec{Crowning with Thorns}
\bv{27}{Then the soldiers of the governor took Jesus into the Prætorium, and gathered unto him the whole band.}
\bv{28}{And they stripped him, and put on him a scarlet robe.}
\bv{29}{And they platted a crown of thorns and put it upon his head, and a reed in his right hand; and they kneeled down before him, and mocked him, saying, ``Hail, King of the Jews!''}
\bv{30}{And they spat upon him, and took the reed and smote him on the head.}
\bv{31}{And when they had mocked him, they took off from him the robe, and put on him his garments, and led him away to crucify him.}
\bv{32}{And as they came out, they found a man of Cyrene, Simon by name: him they compelled to go \supptext{with them}, that he might bear his cross.}
\chapsec{The Crucifixion}
\bv{33}{And when they were come unto a place called Golgotha, that is to say, ``The place of a skull,''}
\bv{34}{they gave him wine to drink mingled with gall: and when he had tasted it, he would not drink.}
\par
\bv{35}{And when they had crucified him, they parted his garments among them, casting lots;}
\bv{36}{and they sat and watched him there.}
\bv{37}{And they set up over his head his accusation written, {\scshape This is Jesus the King of the Jews.}}
\bv{38}{Then are there crucified with him two robbers, one on the right hand and one on the left.}
\bv{39}{And they that passed by railed on him, wagging their heads,}
\bv{40}{and saying, ``Thou that destroyest the temple, and buildest it in three days, save thyself: if thou art the Son of God, come down from the cross.''}
\bv{41}{In like manner also the chief priests mocking \supptext{him}, with the scribes and elders, said,}
\bv{42}{``He saved others; himself he cannot save. He is the King of Israel; let him now come down from the cross, and we will believe on him.}
\bv{43}{He trusteth on God; let him deliver him now, if he desireth him: for he said, `I am the Son of God.'{''}}
\bv{44}{And the robbers also that were crucified with him cast upon him the same reproach.}
\chapsec{The Death of Jesus Christ}
\bv{45}{Now from the sixth hour there was darkness over all the land until the ninth hour.}
\bv{46}{And about the ninth hour Jesus cried with a loud voice, saying, \redlet{``Eli, Eli, lama sabachthani?''}\mref{Ps. 22} that is, \redlet{``My God, my God, why hast thou forsaken me?''}}
\bv{47}{And some of them that stood there, when they heard it, said, ``This man calleth Elijah.''}
\bv{48}{And straightway one of them ran, and took a sponge, and filled it with vinegar, and put it on a reed, and gave him to drink.}
\bv{49}{And the rest said, ``Let be; let us see whether Elijah cometh to save him.''}
\bv{50}{And Jesus cried again with a loud voice, and yielded up the ghost.}
\chapsec{The Fulfilment of the Law}
\bv{51}{And behold, the veil of the temple was rent in two from the top to the bottom; and the earth did quake; and the rocks were rent;}
\bv{52}{and the tombs were opened; and many bodies of the saints that had fallen asleep were raised;}
\bv{53}{and coming forth out of the tombs after his resurrection they entered into the holy city and appeared unto many.}
\bv{54}{Now the centurion, and they that were with him watching Jesus, when they saw the earthquake, and the things that were done, feared exceedingly, saying, ``Truly this was the Son of God.''}
\bv{55}{And many women were there beholding from afar, who had followed Jesus from Galilee, ministering unto him:}
\bv{56}{among whom was Mary Magdalene, and Mary the mother of James and Joses, and the mother of the sons of Zebedee.}
\chapsec{Christ's Burial}
\bv{57}{And when even was come, there came a rich man from Arimathæa, named Joseph, who also himself was Jesus' disciple:}
\bv{58}{this man went to Pilate, and asked for the body of Jesus. Then Pilate commanded it to be given up.}
\bv{59}{And Joseph took the body, and wrapped it in a clean linen cloth,}
\bv{60}{and laid it in his own new tomb, which he had hewn out in the rock: and he rolled a great stone to the door of the tomb, and departed.}
\bv{61}{And Mary Magdalene was there, and the other Mary, sitting over against the sepulchre.}
\chapsec{The Sepulchre Sealed \& Guarded}
\bv{62}{Now on the morrow, which is \supptext{the day} after the Preparation, the chief priests and the Pharisees were gathered together unto Pilate,}
\bv{63}{saying, ``Sir, we remember that that deceiver said while he was yet alive, `After three days I rise again.'}
\bv{64}{Command therefore that the sepulchre be made sure until the third day, lest haply his disciples come and steal him away, and say unto the people, `He is risen from the dead:' and the last error will be worse than the first.''}
\bv{65}{Pilate said unto them, ``Ye have a guard: go, make it \supptext{as} sure as ye can.''}
\bv{66}{So they went, and made the sepulchre sure, sealing the stone, the guard being with them.}
\chaphead{Chapter XXVIII}
\chapdesc{The Resurrection of Jesus Christ}
\lettrine[image=true, lines=4, findent=3pt, nindent=0pt]{Mt-N.eps}{ow} late on the sabbath day, as it began to dawn toward the first \supptext{day} of the week, came Mary Magdalene and the other Mary to see the sepulchre.
\bv{2}{And behold, there was a great earthquake; for an angel of the Lord descended from heaven, and came and rolled away the stone, and sat upon it.}
\bv{3}{His appearance was as lightning, and his raiment white as snow:}
\bv{4}{and for fear of him the watchers did quake, and became as dead men.}
\bv{5}{And the angel answered and said unto the women, ``Fear not ye; for I know that ye seek Jesus, who hath been crucified.}
\bv{6}{He is not here; for he is risen, even as he said. Come, see the place where the Lord lay.}
\bv{7}{And go quickly, and tell his disciples, `He is risen from the dead;' and lo, he goeth before you into Galilee; there shall ye see him: lo, I have told you.''}
\bv{8}{And they departed quickly from the tomb with fear and great joy, and ran to bring his disciples word.}
\bv{9}{And behold, Jesus met them, saying, \redlet{``All hail.''} And they came and took hold of his feet, and worshipped him.}
\bv{10}{Then saith Jesus unto them, \redlet{``Fear not: go tell my brethren that they depart into Galilee, and there shall they see me.''}}
\bv{11}{Now while they were going, behold, some of the guard came into the city, and told unto the chief priests all the things that were come to pass.}
\bv{12}{And when they were assembled with the elders, and had taken counsel, they gave much money unto the soldiers,}
\bv{13}{saying, ``Say ye, `His disciples came by night, and stole him away while we slept.'}
\bv{14}{And if this come to the governor's ears, we will persuade him, and rid you of care.''}
\bv{15}{So they took the money, and did as they were taught: and this saying was spread abroad among the Jews, \supptext{and continueth} until this day.}
\bv{16}{But the eleven disciples went into Galilee, unto the mountain where Jesus had appointed them.}
\bv{17}{And when they saw him, they worshipped \supptext{him}; but some doubted.}
\bv{18}{And Jesus came to them and spake unto them, saying, \redlet{``All authority hath been given unto me in heaven and on earth.}}
\bv{19}{\redlet{Go ye therefore, and make disciples of all the nations, baptising them into the name of the Father and of the Son and of the Holy Ghost:}}
\bv{20}{\redlet{teaching them to observe all things whatsoever I commanded you: and lo, I am with you always, even unto the end of the world.''}}
\begin{center}
	{\scshape [Here Endeth the Gospel of Matthew]}
\end{center}
	\clearpage
	\chapter{The Holy Gospel of Jesus Christ according to Saint Mark}
\fancyhead[RE,LO]{The Gospel according to Mark}
\chaphead{Chapter I}
\chapdesc{The Ministry of St. John the Baptist}
\lettrine[image=true, lines=4, findent=3pt, nindent=0pt]{NT/Mark/Mk1-T.eps}{he} beginning of the gospel of Jesus Christ, the Son of God.
\bv{2}{Even as it is written in Isaiah the prophet,}
\otQuote{Malachi 3:1; Isaiah 40:3}{``Behold, I send my messenger before thy face, Who shall prepare thy way; The voice of one crying in the wilderness, Make ye ready the way of the Lord, Make his paths straight;''}
\bv{4}{John came, who baptised in the wilderness and preached the baptism of repentance unto remission of sins.}
\bv{5}{And there went out unto him all the country of Judæa, and all they of Jerusalem; and they were baptised of him in the river Jordan, confessing their sins.}
\bv{6}{And John was clothed with camel's hair, and \supptext{had} a leathern girdle about his loins, and did eat locusts and wild honey.}
\bv{7}{And he preached, saying, ``There cometh after me he that is mightier than I, the latchet of whose shoes I am not worthy to stoop down and unloose.}
\bv{8}{I baptised you in water; but he shall baptise you in the Holy Ghost.''}
\chapsec{The Baptism of Jesus}
\bv{9}{And it came to pass in those days, that Jesus came from Nazareth of Galilee, and was baptised of John in the Jordan.}
\bv{10}{And straightway coming up out of the water, he saw the heavens rent asunder, and the Spirit as a dove descending upon him:}
\bv{11}{and a voice came out of the heavens, ``Thou art my beloved Son, in thee I am well pleased.''}
\chapsec{The Temptation of Jesus}
\bv{12}{And straightway the Spirit driveth him forth into the wilderness.}
\bv{13}{And he was in the wilderness forty days tempted of Satan; and he was with the wild beasts; and the angels ministered unto him.}
\chapsec{The First Galilean Ministry}
\bv{14}{Now after John was delivered up, Jesus came into Galilee, preaching the gospel of God,}
\bv{15}{and saying, \redlet{``The time is fulfilled, and the kingdom of God is at hand: repent ye, and believe in the gospel.''}}
\chapsec{The Call of Sts. Peter \& Andrew}
\bv{16}{And passing along by the sea of Galilee, he saw Simon and Andrew the brother of Simon casting a net in the sea; for they were fishers.}
\bv{17}{And Jesus said unto them, \redlet{``Come ye after me, and I will make you to become fishers of men.''}}
\bv{18}{And straightway they left the nets, and followed him.}
\bv{19}{And going on a little further, he saw James the \supptext{son} of Zebedee, and John his brother, who also were in the boat mending the nets.}
\bv{20}{And straightway he called them: and they left their father Zebedee in the boat with the hired servants, and went after him.}
\chapsec{Jesus Casts out Demons in Capernaum}
\bv{21}{And they go into Capernaum; and straightway on the sabbath day he entered into the synagogue and taught.}
\bv{22}{And they were astonished at his teaching: for he taught them as having authority, and not as the scribes.}
\bv{23}{And straightway there was in their synagogue a man with an unclean spirit; and he cried out,}
\bv{24}{saying, ``What have we to do with thee, Jesus thou Nazarene? art thou come to destroy us? I know thee who thou art, the Holy One of God.''}
\bv{25}{And Jesus rebuked him, saying, \redlet{``Hold thy peace, and come out of him.''}}
\bv{26}{And the unclean spirit, tearing him and crying with a loud voice, came out of him.}
\bv{27}{And they were all amazed, insomuch that they questioned among themselves, saying, ``What is this? a new teaching! with authority he commandeth even the unclean spirits, and they obey him.''}
\bv{28}{And the report of him went out straightway everywhere into all the region of Galilee round about.}
\chapsec{Simon's Mother-in-Law Healed of a Fever}
\bv{29}{And straightway, when they were come out of the synagogue, they came into the house of Simon and Andrew, with James and John.}
\bv{30}{Now Simon's wife's mother lay sick of a fever; and straightway they tell him of her:}
\bv{31}{and he came and took her by the hand, and raised her up; and the fever left her, and she ministered unto them.}
\chapsec{Demons Rebuked \& Many Healed}
\bv{32}{And at even, when the sun did set, they brought unto him all that were sick, and them that were possessed with demons.}
\bv{33}{And all the city was gathered together at the door.}
\bv{34}{And he healed many that were sick with divers diseases, and cast out many demons; and he suffered not the demons to speak, because they knew him.}
\chapsec{Jesus Prays \& Preaches in Galilee}
\bv{35}{And in the morning, a great while before day, he rose up and went out, and departed into a desert place, and there prayed.}
\bv{36}{And Simon and they that were with him followed after him;}
\bv{37}{and they found him, and say unto him, ``All are seeking thee.''}
\bv{38}{And he saith unto them, \redlet{``Let us go elsewhere into the next towns, that I may preach there also; for to this end came I forth.''}}
\bv{39}{And he went into their synagogues throughout all Galilee, preaching and casting out demons.}
\chapsec{Jesus Heals a Leper}
\bv{40}{And there cometh to him a leper, beseeching him, and kneeling down to him, and saying unto him, ``If thou wilt, thou canst make me clean.''}
\bv{41}{And being moved with compassion, he stretched forth his hand, and touched him, and saith unto him, \redlet{``I will; be thou made clean.''}}
\bv{42}{And straightway the leprosy departed from him, and he was made clean.}
\bv{43}{And he strictly charged him, and straightway sent him out,}
\bv{44}{and saith unto him, \redlet{``See thou say nothing to any man: but go show thyself to the priest, and offer for thy cleansing the things which Moses commanded, for a testimony unto them.''}}
\bv{45}{But he went out, and began to publish it much, and to spread abroad the matter, insomuch that Jesus could no more openly enter into a city, but was without in desert places: and they came to him from every quarter.}
\chaphead{Chapter II}
\chapdesc{Jesus Heals a Paralytic}
\lettrine[image=true, lines=4, findent=3pt, nindent=0pt]{NT/Mark/Mk-And.eps}{nd} when he entered again into Capernaum after some days, it was noised that he was in the house.
\bv{2}{And many were gathered together, so that there was no longer room \supptext{for them}, no, not even about the door: and he spake the word unto them.}
\bv{3}{And they come, bringing unto him a paralytic, borne of four.}
\bv{4}{And when they could not come nigh unto him for the crowd, they uncovered the roof where he was: and when they had broken it up, they let down the bed whereon the paralytic lay.}
\bv{5}{And Jesus seeing their faith saith unto the paralytic, \redlet{``Son, thy sins are forgiven.''}}
\bv{6}{But there were certain of the scribes sitting there, and reasoning in their hearts,}
\bv{7}{``Why doth this man thus speak? he blasphemeth: who can forgive sins but one, \supptext{even} God?''}
\bv{8}{And straightway Jesus, perceiving in his spirit that they so reasoned within themselves, saith unto them, \redlet{``Why reason ye these things in your hearts?}}
\bv{9}{\redlet{Which is easier, to say to the paralytic, Thy sins are forgiven; or to say, Arise, and take up thy bed, and walk?}}
\bv{10}{\redlet{But that ye may know that the Son of man hath authority on earth to forgive sins''} (he saith to the paralytic),}
\bv{11}{\redlet{``I say unto thee, Arise, take up thy bed, and go unto thy house.''}}
\bv{12}{And he arose, and straightway took up the bed, and went forth before them all; insomuch that they were all amazed, and glorified God, saying, ``We never saw it on this fashion.''}
\chapsec{Christ Calls St. Matthew (Levi)}
\bv{13}{And he went forth again by the sea side; and all the multitude resorted unto him, and he taught them.}
\bv{14}{And as he passed by, he saw Levi the \supptext{son} of Alphæus sitting at the place of toll, and he saith unto him, Follow me. And he arose and followed him.}
\bv{15}{And it came to pass, that he was sitting at meat in his house, and many publicans and sinners sat down with Jesus and his disciples: for there were many, and they followed him.}
\bv{16}{And the scribes of the Pharisees, when they saw that he was eating with the sinners and publicans, said unto his disciples, ``\supptext{How is it} that he eateth and drinketh with publicans and sinners?''}
\bv{17}{And when Jesus heard it, he saith unto them, \redlet{``They that are whole have no need of a physician, but they that are sick: I came not to call the righteous, but sinners.''}\mref{cf. Prayer of Manasseh 8}}
\chapsec{A Teaching on Fasting}
\bv{18}{And John's disciples and the Pharisees were fasting: and they come and say unto him, ``Why do John's disciples and the disciples of the Pharisees fast, but thy disciples fast not?''}
\bv{19}{And Jesus said unto them, \redlet{``Can the sons of the bridechamber fast, while the bridegroom is with them? as long as they have the bridegroom with them, they cannot fast.}}
\bv{20}{\redlet{But the days will come, when the bridegroom shall be taken away from them, and then will they fast in that day.}}
\chapsec{Parables of the Cloths \& Bottles}
\bv{21}{\redlet{No man seweth a piece of undressed cloth on an old garment: else that which should fill it up taketh from it, the new from the old, and a worse rent is made.}}
\bv{22}{\redlet{And no man putteth new wine into old wine-skins; else the wine will burst the skins, and the wine perisheth, and the skins: but \supptext{they put} new wine into fresh wine-skins.''}}
\chapsec{Jesus Lord of the Sabbath}
\bv{23}{And it came to pass, that he was going on the sabbath day through the grainfields; and his disciples began, as they went, to pluck the ears.}
\bv{24}{And the Pharisees said unto him, ``Behold, why do they on the sabbath day that which is not lawful?''}
\bv{25}{And he said unto them, \redlet{``Did ye never read what David did, when he had need, and was hungry, he, and they that were with him?}}
\bv{26}{\redlet{How he entered into the house of God when Abiathar was high priest, and ate the showbread, which it is not lawful to eat save for the priests, and gave also to them that were with him?''}}
\bv{27}{And he said unto them, \redlet{``The sabbath was made for man, and not man for the sabbath:}}
\bv{28}{\redlet{so that the Son of man is lord even of the sabbath.''}}
\chaphead{Chapter III}
\chapdesc{Jesus Heals the Withered Hand on the Sabbath}
\lettrine[image=true, lines=4, findent=3pt, nindent=0pt]{NT/Mark/Mk-And.eps}{nd} he entered again into the synagogue; and there was a man there who had his hand withered.
\bv{2}{And they watched him, whether he would heal him on the sabbath day; that they might accuse him.}
\bv{3}{And he saith unto the man that had his hand withered, \redlet{``Stand forth.''}}
\bv{4}{And he saith unto them, \redlet{``Is it lawful on the sabbath day to do good, or to do harm? to save a life, or to kill?''} But they held their peace.}
\bv{5}{And when he had looked round about on them with anger, being grieved at the hardening of their heart, he saith unto the man, \redlet{``Stretch forth thy hand.''} And he stretched it forth; and his hand was restored.}
\bv{6}{And the Pharisees went out, and straightway with the Herodians took counsel against him, how they might destroy him.}
\chapsec{A Great Crowd Follows Jesus}
\bv{7}{And Jesus with his disciples withdrew to the sea: and a great multitude from Galilee followed; and from Judæa,}
\bv{8}{and from Jerusalem, and from Idumæa, and beyond the Jordan, and about Tyre and Sidon, a great multitude, hearing what great things he did, came unto him.}
\bv{9}{And he spake to his disciples, that a little boat should wait on him because of the crowd, lest they should throng him:}
\bv{10}{for he had healed many; insomuch that as many as had plagues pressed upon him that they might touch him.}
\bv{11}{And the unclean spirits, whensoever they beheld him, fell down before him, and cried, saying, ``Thou art the Son of God.''}
\bv{12}{And he charged them much that they should not make him known.}
\chapsec{The Twelve Apostles}
\bv{13}{And he goeth up into the mountain, and calleth unto him whom he himself would; and they went unto him.}
\bv{14}{And he appointed twelve, that they might be with him, and that he might send them forth to preach,}
\bv{15}{and to have authority to cast out demons:}
\bv{16}{and Simon he surnamed Peter;}
\bv{17}{and James the \supptext{son} of Zebedee, and John the brother of James; and them he surnamed Boanerges, which is, Sons of thunder:}
\bv{18}{and Andrew, and Philip, and Bartholomew, and Matthew, and Thomas, and James the \supptext{son} of Alphæus, and Thaddæus, and Simon the Cananæan,}
\bv{19}{and Judas Iscariot, who also betrayed him. And he cometh into a house.}
\bv{20}{And the multitude cometh together again, so that they could not so much as eat bread.}
\bv{21}{And when his friends heard it, they went out to lay hold on him: for they said, ``He is beside himself.''}
\chapsec{Blasphemy Against the Holy Ghost}
\bv{22}{And the scribes that came down from Jerusalem said, ``He hath Beelzebub, and, By the prince of the demons casteth he out the demons.''}
\bv{23}{And he called them unto him, and said unto them in parables, \redlet{``How can Satan cast out Satan?}}
\bv{24}{\redlet{And if a kingdom be divided against itself, that kingdom cannot stand.}}
\bv{25}{\redlet{And if a house be divided against itself, that house will not be able to stand.}}
\bv{26}{\redlet{And if Satan hath risen up against himself, and is divided, he cannot stand, but hath an end.}}
\bv{27}{\redlet{But no one can enter into the house of the strong \supptext{man}, and spoil his goods, except he first bind the strong \supptext{man}; and then he will spoil his house.}}
\bv{28}{\redlet{Verily I say unto you, All their sins shall be forgiven unto the sons of men, and their blasphemies wherewith soever they shall blaspheme:}}
\bv{29}{\redlet{but whosoever shall blaspheme against the Holy Ghost hath never forgiveness, but is guilty of an eternal sin:''}}
\bv{30}{because they said, ``He hath an unclean spirit.''}
\chapsec{Jesus’ Mother and Brothers}
\bv{31}{And there come his mother and his brethren; and, standing without, they sent unto him, calling him.}
\bv{32}{And a multitude was sitting about him; and they say unto him, ``Behold, thy mother and thy brethren without seek for thee.''}
\bv{33}{And he answereth them, and saith, \redlet{``Who is my mother and my brethren?''}}
\bv{34}{And looking round on them that sat round about him, he saith, \redlet{``Behold, my mother and my brethren!}}
\bv{35}{\redlet{For whosoever shall do the will of God, the same is my brother, and sister, and mother.''}}
\chaphead{Chapter IV}
\chapdesc{The Parable of the Sower}
\lettrine[image=true, lines=4, findent=3pt, nindent=0pt]{NT/Mark/Mk-And.eps}{nd} again he began to teach by the sea side. And there is gathered unto him a very great multitude, so that he entered into a boat, and sat in the sea; and all the multitude were by the sea on the land.
\bv{2}{And he taught them many things in parables, and said unto them in his teaching,}
\bv{3}{\redlet{``Hearken: Behold, the sower went forth to sow:}}
\bv{4}{\redlet{and it came to pass, as he sowed, some \supptext{seed} fell by the way side, and the birds came and devoured it.}}
\bv{5}{\redlet{And other fell on the rocky \supptext{ground}, where it had not much earth; and straightway it sprang up, because it had no deepness of earth:}}
\bv{6}{\redlet{and when the sun was risen, it was scorched; and because it had no root, it withered away.}}
\bv{7}{\redlet{And other fell among the thorns, and the thorns grew up, and choked it, and it yielded no fruit.}}
\bv{8}{\redlet{And others fell into the good ground, and yielded fruit, growing up and increasing; and brought forth, thirtyfold, and sixtyfold, and a hundredfold.''}}
\bv{9}{And he said, \redlet{``Who hath ears to hear, let him hear.''}}
\chapsec{The Purpose of the Parables}
\bv{10}{And when he was alone, they that were about him with the twelve asked of him the parables.}
\bv{11}{And he said unto them, \redlet{``Unto you is given the mystery of the kingdom of God: but unto them that are without, all things are done in parables:}}
\bv{12}{\redlet{that seeing they may see, and not perceive; and hearing they may hear, and not understand; lest haply they should turn again, and it should be forgiven them.''}}
\bv{13}{And he saith unto them, \redlet{``Know ye not this parable? and how shall ye know all the parables?}}
\bv{14}{\redlet{The sower soweth the word.}}
\bv{15}{\redlet{And these are they by the way side, where the word is sown; and when they have heard, straightway cometh Satan, and taketh away the word which hath been sown in them.}}
\bv{16}{\redlet{And these in like manner are they that are sown upon the rocky \supptext{places}, who, when they have heard the word, straightway receive it with joy;}}
\bv{17}{\redlet{and they have no root in themselves, but endure for a while; then, when tribulation or persecution ariseth because of the word, straightway they stumble.}}
\bv{18}{\redlet{And others are they that are sown among the thorns; these are they that have heard the word,}}
\bv{19}{\redlet{and the cares of the world, and the deceitfulness of riches, and the lusts of other things entering in, choke the word, and it becometh unfruitful.}}
\bv{20}{\redlet{And those are they that were sown upon the good ground; such as hear the word, and accept it, and bear fruit, thirtyfold, and sixtyfold, and a hundredfold.''}}
\chapsec{A Lamp Under a Basket}
\bv{21}{And he said unto them, \redlet{``Is the lamp brought to be put under the bushel, or under the bed, \supptext{and} not to be put on the stand?}}
\bv{22}{\redlet{For there is nothing hid, save that it should be manifested; neither was \supptext{anything} made secret, but that it should come to light.}}
\bv{23}{\redlet{If any man hath ears to hear, let him hear.''}}
\bv{24}{And he said unto them, \redlet{``Take heed what ye hear: with what measure ye mete it shall be measured unto you; and more shall be given unto you.}}
\bv{25}{\redlet{For he that hath, to him shall be given: and he that hath not, from him shall be taken away even that which he hath.''}}
\chapsec{The Parable of the Seed Growing}
\bv{26}{And he said, \redlet{``So is the kingdom of God, as if a man should cast seed upon the earth;}}
\bv{27}{\redlet{and should sleep and rise night and day, and the seed should spring up and grow, he knoweth not how.}}
\bv{28}{\redlet{The earth beareth fruit of herself; first the blade, then the ear, then the full grain in the ear.}}
\bv{29}{\redlet{But when the fruit is ripe, straightway he putteth forth the sickle, because the harvest is come.''}}
\chapsec{The Parable of the Mustard Seed}
\bv{30}{And he said, \redlet{``How shall we liken the kingdom of God? or in what parable shall we set it forth?}}
\bv{31}{\redlet{It is like a grain of mustard seed, which, when it is sown upon the earth, though it be less than all the seeds that are upon the earth,}}
\bv{32}{\redlet{yet when it is sown, groweth up, and becometh greater than all the herbs, and putteth out great branches; so that the birds of the heaven can lodge under the shadow thereof.''}}
\bv{33}{And with many such parables spake he the word unto them, as they were able to hear it;}
\bv{34}{and without a parable spake he not unto them: but privately to his own disciples he expounded all things.}
\chapsec{Jesus Calms a Storm}
\bv{35}{And on that day, when even was come, he saith unto them, \redlet{``Let us go over unto the other side.''}}
\bv{36}{And leaving the multitude, they take him with them, even as he was, in the boat. And other boats were with him.}
\bv{37}{And there ariseth a great storm of wind, and the waves beat into the boat, insomuch that the boat was now filling.}
\bv{38}{And he himself was in the stern, asleep on the cushion: and they awake him, and say unto him, ``Teacher, carest thou not that we perish?''}
\bv{39}{And he awoke, and rebuked the wind, and said unto the sea, \redlet{``Peace, be still.''} And the wind ceased, and there was a great calm.}
\bv{40}{And he said unto them, \redlet{``Why are ye fearful? have ye not yet faith?''}}
\bv{41}{And they feared exceedingly, and said one to another, ``Who then is this, that even the wind and the sea obey him?''}
\chaphead{Chapter V}
\chapdesc{Jesus Heals a Man with a Demon}
\lettrine[image=true, lines=4, findent=3pt, nindent=0pt]{NT/Mark/Mk-And.eps}{nd} they came to the other side of the sea, into the country of the Gerasenes.
\bv{2}{And when he was come out of the boat, straightway there met him out of the tombs a man with an unclean spirit,}
\bv{3}{who had his dwelling in the tombs: and no man could any more bind him, no, not with a chain;}
\bv{4}{because that he had been often bound with fetters and chains, and the chains had been rent asunder by him, and the fetters broken in pieces: and no man had strength to tame him.}
\bv{5}{And always, night and day, in the tombs and in the mountains, he was crying out, and cutting himself with stones.}
\bv{6}{And when he saw Jesus from afar, he ran and worshipped him;}
\bv{7}{and crying out with a loud voice, he saith, ``What have I to do with thee, Jesus, thou Son of the Most High God? I adjure thee by God, torment me not.''}
\bv{8}{For he said unto him, \redlet{``Come forth, thou unclean spirit, out of the man.''}}
\bv{9}{And he asked him, \redlet{``What is thy name?''} And he saith unto him, ``My name is Legion; for we are many.''}
\bv{10}{And he besought him much that he would not send them away out of the country.}
\bv{11}{Now there was there on the mountain side a great herd of swine feeding.}
\bv{12}{And they besought him, saying, ``Send us into the swine, that we may enter into them.''}
\bv{13}{And he gave them leave. And the unclean spirits came out, and entered into the swine: and the herd rushed down the steep into the sea, \supptext{in number} about two thousand; and they were drowned in the sea.}
\par
\bv{14}{And they that fed them fled, and told it in the city, and in the country. And they came to see what it was that had come to pass.}
\bv{15}{And they come to Jesus, and behold him that was possessed with demons sitting, clothed and in his right mind, \supptext{even} him that had the legion: and they were afraid.}
\bv{16}{And they that saw it declared unto them how it befell him that was possessed with demons, and concerning the swine.}
\bv{17}{And they began to beseech him to depart from their borders.}
\bv{18}{And as he was entering into the boat, he that had been possessed with demons besought him that he might be with him.}
\bv{19}{And he suffered him not, but saith unto him, \redlet{``Go to thy house unto thy friends, and tell them how great things the Lord hath done for thee, and \supptext{how} he had mercy on thee.''}}
\bv{20}{And he went his way, and began to publish in Decapolis how great things Jesus had done for him: and all men marvelled.}
\chapsec{Jesus Heals a Woman and Jairus’ Daughter}
\bv{21}{And when Jesus had crossed over again in the boat unto the other side, a great multitude was gathered unto him; and he was by the sea.}
\bv{22}{And there cometh one of the rulers of the synagogue, Jaïrus by name; and seeing him, he falleth at his feet,}
\bv{23}{and beseecheth him much, saying, ``My little daughter is at the point of death: \supptext{I pray thee}, that thou come and lay thy hands on her, that she may be made whole, and live.''}
\bv{24}{And he went with him; and a great multitude followed him, and they thronged him.}
\par
\bv{25}{And a woman, who had an issue of blood twelve years,}
\bv{26}{and had suffered many things of many physicians, and had spent all that she had, and was nothing bettered, but rather grew worse,}
\bv{27}{having heard the things concerning Jesus, came in the crowd behind, and touched his garment.}
\bv{28}{For she said, ``If I touch but his garments, I shall be made whole.''}
\bv{29}{And straightway the fountain of her blood was dried up; and she felt in her body that she was healed of her plague.}
\bv{30}{And straightway Jesus, perceiving in himself that the power \supptext{proceeding} from him had gone forth, turned him about in the crowd, and said, \redlet{``Who touched my garments?''}}
\bv{31}{And his disciples said unto him, ``Thou seest the multitude thronging thee, and sayest thou, `Who touched me?'{''}}
\bv{32}{And he looked round about to see her that had done this thing.}
\bv{33}{But the woman fearing and trembling, knowing what had been done to her, came and fell down before him, and told him all the truth.}
\bv{34}{And he said unto her, \redlet{``Daughter, thy faith hath made thee whole; go in peace, and be whole of thy plague.''}}
\par
\bv{35}{While he yet spake, they come from the ruler of the synagogue's \supptext{house}, saying, ``Thy daughter is dead: why troublest thou the Teacher any further?''}
\bv{36}{But Jesus, not heeding the word spoken, saith unto the ruler of the synagogue, \redlet{``Fear not, only believe.''}}
\bv{37}{And he suffered no man to follow with him, save Peter, and James, and John the brother of James.}
\bv{38}{And they come to the house of the ruler of the synagogue; and he beholdeth a tumult, and \supptext{many} weeping and wailing greatly.}
\bv{39}{And when he was entered in, he saith unto them, \redlet{``Why make ye a tumult, and weep? the child is not dead, but sleepeth.''}}
\bv{40}{And they laughed him to scorn. But he, having put them all forth, taketh the father of the child and her mother and them that were with him, and goeth in where the child was.}
\bv{41}{And taking the child by the hand, he saith unto her, \redlet{``Talitha cumi;''} which is, being interpreted, \redlet{``Damsel, I say unto thee, Arise.''}}
\bv{42}{And straightway the damsel rose up, and walked; for she was twelve years old. And they were amazed straightway with a great amazement.}
\bv{43}{And he charged them much that no man should know this: and he commanded that \supptext{something} should be given her to eat.}
\chaphead{Chapter VI}
\chapdesc{Jesus Rejected at Nazareth}
\lettrine[image=true, lines=4, findent=3pt, nindent=0pt]{NT/Mark/Mk-And.eps}{nd} he went out from thence; and he cometh into his own country; and his disciples follow him.
\bv{2}{And when the sabbath was come, he began to teach in the synagogue: and many hearing him were astonished, saying, ``Whence hath this man these things?'' and, ``What is the wisdom that is given unto this man,'' and ``\supptext{what mean} such mighty works wrought by his hands?}
\bv{3}{Is not this the carpenter, the son of Mary, and brother of James, and Joses, and Judas, and Simon? and are not his sisters here with us?'' And they were offended in him.}
\bv{4}{And Jesus said unto them, \redlet{``A prophet is not without honor, save in his own country, and among his own kin, and in his own house.''}}
\bv{5}{And he could there do no mighty work, save that he laid his hands upon a few sick folk, and healed them.}
\bv{6}{And he marvelled because of their unbelief. And he went round about the villages teaching.}
\chapsec{Jesus Sends Out the Twelve Apostles}
\bv{7}{And he calleth unto him the twelve, and began to send them forth by two and two; and he gave them authority over the unclean spirits;}
\bv{8}{and he charged them that they should take nothing for \supptext{their} journey, save a staff only; no bread, no wallet, no money in their purse;}
\bv{9}{but \supptext{to go} shod with sandals: and, \supptext{said he}, put not on two coats.}
\bv{10}{And he said unto them, \redlet{``Wheresoever ye enter into a house, there abide till ye depart thence.}}
\bv{11}{\redlet{And whatsoever place shall not receive you, and they hear you not, as ye go forth thence, shake off the dust that is under your feet for a testimony unto them.''}}\mcomm{Verily I say unto you, It shall be more tolerable for Sodom and Gomorrha in the day of judgement, than for that city.}
\bv{12}{And they went out, and preached that \supptext{men} should repent.}
\bv{13}{And they cast out many demons, and anointed with oil many that were sick, and healed them.}
\chapsec{The Death of St. John the Baptist}
\bv{14}{And king Herod heard \supptext{thereof}; for his name had become known: and he said, ``John the baptiser is risen from the dead, and therefore do these powers work in him.''}
\bv{15}{But others said, ``It is Elijah.'' And others said, ``\supptext{It is} a prophet, \supptext{even} as one of the prophets.''}
\bv{16}{But Herod, when he heard \supptext{thereof}, said, ``John, whom I beheaded, he is risen.''}
\bv{17}{For Herod himself had sent forth and laid hold upon John, and bound him in prison for the sake of Herodias, his brother Philip's wife; for he had married her.}
\bv{18}{For John said unto Herod, ``It is not lawful for thee to have thy brother's wife.''}
\bv{19}{And Herodias set herself against him, and desired to kill him; and she could not;}
\bv{20}{for Herod feared John, knowing that he was a righteous and holy man, and kept him safe. And when he heard him, he was much perplexed; and he heard him gladly.}
\bv{21}{And when a convenient day was come, that Herod on his birthday made a supper to his lords, and the high captains, and the chief men of Galilee;}
\bv{22}{and when the daughter of Herodias herself came in and danced, she pleased Herod and them that sat at meat with him; and the king said unto the damsel, ``Ask of me whatsoever thou wilt, and I will give it thee.''}
\bv{23}{And he sware unto her, ``Whatsoever thou shalt ask of me, I will give it thee, unto the half of my kingdom.''}
\bv{24}{And she went out, and said unto her mother, ``What shall I ask?'' And she said, ``The head of John the baptiser.''}
\bv{25}{And she came in straightway with haste unto the king, and asked, saying, ``I will that thou forthwith give me on a platter the head of John the Baptist.''}
\bv{26}{And the king was exceeding sorry; but for the sake of his oaths, and of them that sat at meat, he would not reject her.}
\bv{27}{And straightway the king sent forth a soldier of his guard, and commanded to bring his head: and he went and beheaded him in the prison,}
\bv{28}{and brought his head on a platter, and gave it to the damsel; and the damsel gave it to her mother.}
\bv{29}{And when his disciples heard \supptext{thereof}, they came and took up his corpse, and laid it in a tomb.}
\chapsec{Jesus Feeds the Five Thousand}
\bv{30}{And the apostles gather themselves together unto Jesus; and they told him all things, whatsoever they had done, and whatsoever they had taught.}
\bv{31}{And he saith unto them, \redlet{``Come ye yourselves apart into a desert place, and rest a while.''} For there were many coming and going, and they had no leisure so much as to eat.}
\bv{32}{And they went away in the boat to a desert place apart.}
\bv{33}{And \supptext{the people} saw them going, and many knew \supptext{them}, and they ran together there on foot from all the cities, and outwent them.}
\bv{34}{And he came forth and saw a great multitude, and he had compassion on them, because they were as sheep not having a shepherd: and he began to teach them many things.}
\bv{35}{And when the day was now far spent, his disciples came unto him, and said, ``The place is desert, and the day is now far spent;}
\bv{36}{send them away, that they may go into the country and villages round about, and buy themselves somewhat to eat.''}
\bv{37}{But he answered and said unto them, \redlet{``Give ye them to eat.''} And they say unto him, ``Shall we go and buy two hundred denarii's\mcomm{Denarius: a day's wages.} worth of bread, and give them to eat?''}
\bv{38}{And he saith unto them, \redlet{``How many loaves have ye? go \supptext{and} see.''} And when they knew, they say, ``Five, and two fishes.''}
\bv{39}{And he commanded them that all should sit down by companies upon the green grass.}
\bv{40}{And they sat down in ranks, by hundreds, and by fifties.}
\bv{41}{And he took the five loaves and the two fishes, and looking up to heaven, he blessed, and brake the loaves; and he gave to the disciples to set before them; and the two fishes divided he among them all.}
\bv{42}{And they all ate, and were filled.}
\bv{43}{And they took up broken pieces, twelve basketfuls, and also of the fishes.}
\bv{44}{And they that ate the loaves were five thousand men.}
\chapsec{Jesus Walks on the Water}
\bv{45}{And straightway he constrained his disciples to enter into the boat, and to go before \supptext{him} unto the other side to Bethsaida, while he himself sendeth the multitude away.}
\bv{46}{And after he had taken leave of them, he departed into the mountain to pray.}
\bv{47}{And when even was come, the boat was in the midst of the sea, and he alone on the land.}
\bv{48}{And seeing them distressed in rowing, for the wind was contrary unto them, about the fourth watch of the night he cometh unto them, walking on the sea; and he would have passed by them:}
\bv{49}{but they, when they saw him walking on the sea, supposed that it was a ghost, and cried out;}
\bv{50}{for they all saw him, and were troubled. But he straightway spake with them, and saith unto them, \redlet{``Be of good cheer: it is I; be not afraid.''}}
\bv{51}{And he went up unto them into the boat; and the wind ceased: and they were sore amazed in themselves;}
\bv{52}{for they understood not concerning the loaves, but their heart was hardened.}
\chapsec{Jesus Heals the Sick in Gennesaret}
\bv{53}{And when they had crossed over, they came to the land unto Gennesaret, and moored to the shore.}
\bv{54}{And when they were come out of the boat, straightway \supptext{the people} knew him,}
\bv{55}{and ran round about that whole region, and began to carry about on their beds those that were sick, where they heard he was.}
\bv{56}{And wheresoever he entered, into villages, or into cities, or into the country, they laid the sick in the marketplaces, and besought him that they might touch if it were but the border of his garment: and as many as touched him were made whole.}
\chaphead{Chapter VII}
\chapdesc{Traditions and Commandments}
\lettrine[image=true, lines=4, findent=3pt, nindent=0pt]{NT/Mark/Mk-And.eps}{nd} there are gathered together unto him the Pharisees, and certain of the scribes, who had come from Jerusalem,
\bv{2}{and had seen that some of his disciples ate their bread with defiled, that is, unwashen, hands.}
\bv{3}{(For the Pharisees, and all the Jews, except they wash their hands diligently, eat not, holding the tradition of the elders;}
\bv{4}{and \supptext{when they come} from the marketplace, except they bathe themselves, they eat not; and many other things there are, which they have received to hold, washings of cups, and pots, and brasen vessels.)}
\bv{5}{And the Pharisees and the scribes ask him, ``Why walk not thy disciples according to the tradition of the elders, but eat their bread with defiled hands?''}
\bv{6}{And he said unto them, \redlet{``Well did Isaiah prophesy of you hypocrites, as it is written,}}
\otQuote{Isaiah 29:13 LXX}{\redlet{This people honoreth me with their lips, But their heart is far from me. \bv{7}{But in vain do they worship me, Teaching \supptext{as their} doctrines the precepts of men.}}}
\bv{8}{\redlet{Ye leave the commandment of God, and hold fast the tradition of men.''}}\mcomm{the washing of pitchers and cups, and you do many other such things.}
\bv{9}{And he said unto them, \redlet{``Full well do ye reject the commandment of God, that ye may keep your tradition.}}
\bv{10}{\redlet{For Moses said, `Honor thy father and thy mother;' and, `He that speaketh evil of father or mother, let him die the death:'}}
\bv{11}{\redlet{but ye say, `If a man shall say to his father or his mother, That wherewith thou mightest have been profited by me is Corban, that is to say, Given \supptext{to God};}}
\bv{12}{\redlet{ye no longer suffer him to do aught for his father or his mother;'}}
\bv{13}{\redlet{making void the word of God by your tradition, which ye have delivered: and many such like things ye do.''}}
\chapsec{What Defiles a Person}
\bv{14}{And he called to him the multitude again, and said unto them, \redlet{``Hear me all of you, and understand:}}
\bv{15}{\redlet{there is nothing from without the man, that going into him can defile him; but the things which proceed out of the man are those that defile the man.''}}\mcomm{If any man hath ears to hear, let him hear.}
\bv{17}{And when he was entered into the house from the multitude, his disciples asked of him the parable.}
\bv{18}{And he saith unto them, \redlet{``Are ye so without understanding also? Perceive ye not, that whatsoever from without goeth into the man, \supptext{it} cannot defile him;}}
\bv{19}{\redlet{because it goeth not into his heart, but into his belly, and goeth out into the draught?''} \supptext{This he said}, making all meats clean.}
\bv{20}{And he said, \redlet{``That which proceedeth out of the man, that defileth the man.}}
\bv{21}{\redlet{For from within, out of the heart of men, evil thoughts proceed, fornications, thefts, murders, adulteries,}}
\bv{22}{\redlet{covetings, wickednesses, deceit, lasciviousness, an evil eye, railing, pride, foolishness:}}
\bv{23}{\redlet{all these evil things proceed from within, and defile the man.''}}
\chapsec{The Syrophoenician Woman's Faith}
\bv{24}{And from thence he arose, and went away into the borders of Tyre and Sidon. And he entered into a house, and would have no man know it; and he could not be hid.}
\bv{25}{But straightway a woman, whose little daughter had an unclean spirit, having heard of him, came and fell down at his feet.}
\bv{26}{Now the woman was a Greek, a Syrophoenician by race. And she besought him that he would cast forth the demon out of her daughter.}
\bv{27}{And he said unto her, \redlet{``Let the children first be filled: for it is not meet to take the children's bread and cast it to the dogs.''}}
\bv{28}{But she answered and saith unto him, ``Yea, Lord; even the dogs under the table eat of the children's crumbs.''}\mcomm{The proud woman is brought to humility by Our Lord.}
\bv{29}{And he said unto her, \redlet{``For this saying go thy way; the demon is gone out of thy daughter.''}}
\bv{30}{And she went away unto her house, and found the child laid upon the bed, and the demon gone out.}
\chapsec{Jesus Heals a Deaf Man}
\bv{31}{And again he went out from the borders of Tyre, and came through Sidon unto the sea of Galilee, through the midst of the borders of Decapolis.}
\bv{32}{And they bring unto him one that was deaf, and had an impediment in his speech; and they beseech him to lay his hand upon him.}
\bv{33}{And he took him aside from the multitude privately, and put his fingers into his ears, and he spat, and touched his tongue;}
\bv{34}{and looking up to heaven, he sighed, and saith unto him, \redlet{``Ephphatha,''} that is, \redlet{``Be opened.''}}
\bv{35}{And his ears were opened, and the bond of his tongue was loosed, and he spake plain.}
\bv{36}{And he charged them that they should tell no man: but the more he charged them, so much the more a great deal they published it.}
\bv{37}{And they were beyond measure astonished, saying, ``He hath done all things well; he maketh even the deaf to hear, and the dumb to speak.''}
\chaphead{Chapter VIII}
\chapdesc{Jesus Feeds the Four Thousand}
\lettrine[image=true, lines=4, findent=3pt, nindent=0pt]{NT/Mark/Mk-In.eps}{n} those days, when there was again a great multitude, and they had nothing to eat, he called unto him his disciples, and saith unto them,
\bv{2}{\redlet{``I have compassion on the multitude, because they continue with me now three days, and have nothing to eat:}}
\bv{3}{\redlet{and if I send them away fasting to their home, they will faint on the way; and some of them are come from far.''}}
\bv{4}{And his disciples answered him, ``Whence shall one be able to fill these men with bread here in a desert place?''}
\bv{5}{And he asked them, \redlet{``How many loaves have ye?''} And they said, ``Seven.''}
\bv{6}{And he commandeth the multitude to sit down on the ground: and he took the seven loaves, and having given thanks, he brake, and gave to his disciples, to set before them; and they set them before the multitude.}
\bv{7}{And they had a few small fishes: and having blessed them, he commanded to set these also before them.}
\bv{8}{And they ate, and were filled: and they took up, of broken pieces that remained over, seven baskets.}
\bv{9}{And they were about four thousand: and he sent them away.}
\bv{10}{And straightway he entered into the boat with his disciples, and came into the parts of Dalmanutha.}
\chapsec{The Pharisees Demand a Sign}
\bv{11}{And the Pharisees came forth, and began to question with him, seeking of him a sign from heaven, trying him.}
\bv{12}{And he sighed deeply in his spirit, and saith, \redlet{``Why doth this generation seek a sign? verily I say unto you, There shall no sign be given unto this generation.''}}
\bv{13}{And he left them, and again entering into \supptext{the boat} departed to the other side.}
\chapsec{The Leaven of the Pharisees and Herod}
\bv{14}{And they forgot to take bread; and they had not in the boat with them more than one loaf.}
\bv{15}{And he charged them, saying, \redlet{``Take heed, beware of the leaven of the Pharisees and the leaven of Herod.''}}
\bv{16}{And they reasoned one with another, saying, `We have no bread.'}
\bv{17}{And Jesus perceiving it saith unto them, \redlet{``Why reason ye, because ye have no bread? do ye not yet perceive, neither understand? have ye your heart hardened?}}
\bv{18}{\redlet{Having eyes, see ye not? and having ears, hear ye not? and do ye not remember?}}
\bv{19}{\redlet{When I brake the five loaves among the five thousand, how many baskets full of broken pieces took ye up?''} They say unto him, ``Twelve.''}
\bv{20}{\redlet{``And when the seven among the four thousand, how many basketfuls of broken pieces took ye up?''} And they say unto him, ``Seven.''}
\bv{21}{And he said unto them, \redlet{``Do ye not yet understand?''}}
\chapsec{Jesus Heals a Blind Man at Bethsaida}
\bv{22}{And they come unto Bethsaida. And they bring to him a blind man, and beseech him to touch him.}
\bv{23}{And he took hold of the blind man by the hand, and brought him out of the village; and when he had spit on his eyes, and laid his hands upon him, he asked him, \redlet{``Seest thou aught?''}}
\bv{24}{And he looked up, and said, ``I see men; for I behold \supptext{them} as trees, walking.''}
\bv{25}{Then again he laid his hands upon his eyes; and he looked stedfastly, and was restored, and saw all things clearly.}
\bv{26}{And he sent him away to his home, saying, \redlet{``Do not even enter into the village.''}}
\chapsec{Peter Confesses Jesus as the Christ}
\bv{27}{And Jesus went forth, and his disciples, into the villages of Cæsarea Philippi: and on the way he asked his disciples, saying unto them, \redlet{``Who do men say that I am?''}}
\bv{28}{And they told him, saying, ``John the Baptist; and others, Elijah; but others, One of the prophets.''}
\bv{29}{And he asked them, \redlet{``But who say ye that I am?''} Peter answereth and saith unto him, ``Thou art the Christ.''}
\bv{30}{And he charged them that they should tell no man of him.}
\chapsec{Jesus Foretells His Death and Resurrection}
\bv{31}{And he began to teach them, that the Son of man must suffer many things, and be rejected by the elders, and the chief priests, and the scribes, and be killed, and after three days rise again.}
\bv{32}{And he spake the saying openly. And Peter took him, and began to rebuke him.}
\bv{33}{But he turning about, and seeing his disciples, rebuked Peter, and saith, \redlet{``Get thee behind me, Satan; for thou mindest not the things of God, but the things of men.''}}
\bv{34}{And he called unto him the multitude with his disciples, and said unto them, \redlet{``If any man would come after me, let him deny himself, and take up his cross, and follow me.}}
\bv{35}{\redlet{For whosoever would save his life shall lose it; and whosoever shall lose his life for my sake and the gospel's shall save it.}}
\bv{36}{\redlet{For what doth it profit a man, to gain the whole world, and forfeit his life?}}
\bv{37}{\redlet{For what should a man give in exchange for his life?}}
\bv{38}{\redlet{For whosoever shall be ashamed of me and of my words in this adulterous and sinful generation, the Son of man also shall be ashamed of him, when he cometh in the glory of his Father with the holy angels.''}}
\chaphead{Chapter IX}
\chapdesc{The Transfiguration}
\lettrine[image=true, lines=4, findent=3pt, nindent=0pt]{NT/Mark/Mk-And.eps}{nd} he said unto them, \redlet{``Verily I say unto you, There are some here of them that stand \supptext{by}, who shall in no wise taste of death, till they see the kingdom of God come with power.''}
\bv{2}{And after six days Jesus taketh with him Peter, and James, and John, and bringeth them up into a high mountain apart by themselves: and he was transfigured before them;}
\bv{3}{and his garments became glistering, exceeding white, so as no fuller on earth can whiten them.}
\bv{4}{And there appeared unto them Elijah with Moses: and they were talking with Jesus.}
\bv{5}{And Peter answereth and saith to Jesus, ``Rabbi, it is good for us to be here: and let us make three tabernacles; one for thee, and one for Moses, and one for Elijah.''}
\bv{6}{For he knew not what to answer; for they became sore afraid.}
\bv{7}{And there came a cloud overshadowing them: and there came a voice out of the cloud, \god{This is my beloved Son: hear ye him.}}
\bv{8}{And suddenly looking round about, they saw no one any more, save Jesus only with themselves.}
\bv{9}{And as they were coming down from the mountain, he charged them that they should tell no man what things they had seen, save when the Son of man should have risen again from the dead.}
\bv{10}{And they kept the saying, questioning among themselves what the rising again from the dead should mean.}
\chapsec{Instruction of the Disciples concerning Elijah}
\bv{11}{And they asked him, saying, ``\supptext{How is it} that the scribes say that Elijah must first come?''}
\bv{12}{And he said unto them, \redlet{``Elijah indeed cometh first, and restoreth all things: and how is it written of the Son of man, that he should suffer many things and be set at nought?}}
\bv{13}{\redlet{But I say unto you, that Elijah is come, and they have also done unto him whatsoever they would, even as it is written of him.''}}
\chapsec{Jesus Rebukes a Deaf \& Dumb Spirit}
\bv{14}{And when they came to the disciples, they saw a great multitude about them, and scribes questioning with them.}
\bv{15}{And straightway all the multitude, when they saw him, were greatly amazed, and running to him saluted him.}
\bv{16}{And he asked them, \redlet{``What question ye with them?''}}
\bv{17}{And one of the multitude answered him, ``Teacher, I brought unto thee my son, who hath a dumb spirit;}
\bv{18}{and wheresoever it taketh him, it dasheth him down: and he foameth, and grindeth his teeth, and pineth away: and I spake to thy disciples that they should cast it out; and they were not able.''}
\bv{19}{And he answereth them and saith, \redlet{``O faithless generation, how long shall I be with you? how long shall I bear with you? bring him unto me.''}}
\bv{20}{And they brought him unto him: and when he saw him, straightway the spirit tare him grievously; and he fell on the ground, and wallowed foaming.}
\bv{21}{And he asked his father, \redlet{``How long time is it since this hath come unto him?''} And he said, ``From a child.}
\bv{22}{And oft-times it hath cast him both into the fire and into the waters, to destroy him: but if thou canst do anything, have compassion on us, and help us.''}
\bv{23}{And Jesus said unto him, \redlet{``If thou canst! All things are possible to him that believeth.''}}
\bv{24}{Straightway the father of the child cried out, and said, ``I believe; help thou mine unbelief.''}
\bv{25}{And when Jesus saw that a multitude came running together, he rebuked the unclean spirit, saying unto him, \redlet{``Thou dumb and deaf spirit, I command thee, come out of him, and enter no more into him.''}}
\bv{26}{And having cried out, and torn him much, he came out: and \supptext{the boy} became as one dead; insomuch that the more part said, ``He is dead.''}
\bv{27}{But Jesus took him by the hand, and raised him up; and he arose.}
\bv{28}{And when he was come into the house, his disciples asked him privately, ``\supptext{How is it} that we could not cast it out?''}
\bv{29}{And he said unto them, \redlet{``This kind can come out by nothing, save by prayer.''}}
\chapsec{Jesus Fortells his Death \& Resurrection}
\bv{30}{And they went forth from thence, and passed through Galilee; and he would not that any man should know it.}
\bv{31}{For he taught his disciples, and said unto them, \redlet{``The Son of man is delivered up into the hands of men, and they shall kill him; and when he is killed, after three days he shall rise again.''}}
\bv{32}{But they understood not the saying, and were afraid to ask him.}
\chapsec{Jesus Exhorts his Disciples to Humility}
\bv{33}{And they came to Capernaum: and when he was in the house he asked them, \redlet{``What were ye reasoning on the way?}}
\bv{34}{But they held their peace: for they had disputed one with another on the way, who \supptext{was} the greatest.}
\bv{35}{And he sat down, and called the twelve; and he saith unto them, \redlet{``If any man would be first, he shall be last of all, and servant of all.''}}
\bv{36}{And he took a little child, and set him in the midst of them: and taking him in his arms, he said unto them,}
\bv{37}{\redlet{``Whosoever shall receive one of such little children in my name, receiveth me: and whosoever receiveth me, receiveth not me, but him that sent me.''}}
\chapsec{The Rebuke of Sectarianism}
\bv{38}{John said unto him, ``Teacher, we saw one casting out demons in thy name;\mcomm{and he followeth not us:} and we forbade him, because he followed not us.''}
\bv{39}{But Jesus said, \redlet{``Forbid him not: for there is no man who shall do a mighty work in my name, and be able quickly to speak evil of me.}}
\bv{40}{\redlet{For he that is not against us is for us.}}
\bv{41}{\redlet{For whosoever shall give you a cup of water to drink, because ye are Christ's, verily I say unto you, he shall in no wise lose his reward.}}
\chapsec{Jesus' Solemn Warning of Hell}
\bv{42}{\redlet{And whosoever shall cause one of these little ones that believe on me to stumble, it were better for him if a great millstone were hanged about his neck, and he were cast into the sea.}}
\bv{43}{\redlet{And if thy hand cause thee to stumble, cut it off: it is good for thee to enter into life maimed, rather than having thy two hands to go into hell, into the unquenchable fire.}}\mcomm{Where their worm dieth not, and the fire is not quenched.}
\bv{45}{\redlet{And if thy foot cause thee to stumble, cut it off: it is good for thee to enter into life halt, rather than having thy two feet to be cast into hell.}}\mcomm{Where their worm dieth not, and the fire is not quenched.}
\bv{47}{\redlet{And if thine eye cause thee to stumble, cast it out: it is good for thee to enter into the kingdom of God with one eye, rather than having two eyes to be cast into hell;}}
\bv{48}{\redlet{where their worm dieth not, and the fire is not quenched.}}
\bv{49}{\redlet{For every one shall be salted with fire.}}\mcomm{and every sacrifice shall be salted with salt.}
\bv{50}{\redlet{Salt is good: but if the salt have lost its saltness, wherewith will ye season it? Have salt in yourselves, and be at peace one with another.''}}
\chaphead{Chapter X}
\chapdesc{Jesus' Law of Divorce}
\lettrine[image=true, lines=4, findent=3pt, nindent=0pt]{NT/Mark/Mk-And.eps}{nd} he arose from thence, and cometh into the borders of Judæa and beyond the Jordan: and multitudes come together unto him again; and, as he was wont, he taught them again.
\bv{2}{And there came unto him Pharisees, and asked him, ``Is it lawful for a man to put away \supptext{his} wife?'' trying him.}
\bv{3}{And he answered and said unto them, \redlet{``What did Moses command you?''}}
\bv{4}{And they said, ``Moses suffered to write a bill of divorcement, and to put her away.''}
\bv{5}{But Jesus said unto them, \redlet{``For your hardness of heart he wrote you this commandment.}}
\bv{6}{\redlet{But from the beginning of the creation, Male and female made he them.}}
\bv{7}{\redlet{For this cause shall a man leave his father and mother, and shall cleave to his wife;}}
\bv{8}{\redlet{and the two shall become one flesh: so that they are no more two, but one flesh.}}
\bv{9}{\redlet{What therefore God hath joined together, let not man put asunder.''}}
\bv{10}{And in the house the disciples asked him again of this matter.}
\bv{11}{And he saith unto them, \redlet{``Whosoever shall put away his wife, and marry another, committeth adultery against her:}}
\bv{12}{\redlet{and if she herself shall put away her husband, and marry another, she committeth adultery.''}}
\chapsec{Jesus Blesses Little Children}
\bv{13}{And they were bringing unto him little children, that he should touch them: and the disciples rebuked them.}
\bv{14}{But when Jesus saw it, he was moved with indignation, and said unto them, \redlet{``Suffer the little children to come unto me; forbid them not: for to such belongeth the kingdom of God.}}
\bv{15}{\redlet{Verily I say unto you, Whosoever shall not receive the kingdom of God as a little child, he shall in no wise enter therein.''}}
\bv{16}{And he took them in his arms, and blessed them, laying his hands upon them.}
\chapsec{The Rich Young Ruler}
\bv{17}{And as he was going forth into the way, there ran one to him, and kneeled to him, and asked him, ``Good Teacher, what shall I do that I may inherit eternal life?''}
\bv{18}{And Jesus said unto him, \redlet{Why callest thou me good? none is good save one, \supptext{even} God.}}
\bv{19}{\redlet{Thou knowest the commandments, ``Do not kill, Do not commit adultery, Do not steal, Do not bear false witness, Do not defraud, Honor thy father and mother.''}}
\bv{20}{And he said unto him, ``Teacher, all these things have I observed from my youth.''}
\bv{21}{And Jesus looking upon him loved him, and said unto him, \redlet{``One thing thou lackest: go, sell whatsoever thou hast, and give to the poor, and thou shalt have treasure in heaven: and come, follow me.''}}
\bv{22}{But his countenance fell at the saying, and he went away sorrowful: for he was one that had great possessions.}
\chapsec{The Warning against Riches}
\bv{23}{And Jesus looked round about, and saith unto his disciples, \redlet{``How hardly shall they that have riches enter into the kingdom of God!''}}
\bv{24}{And the disciples were amazed at his words. But Jesus answereth again, and saith unto them, \redlet{``Children, how hard is it for them that trust in riches to enter into the kingdom of God!}}
\bv{25}{\redlet{It is easier for a camel to go through a needle's eye, than for a rich man to enter into the kingdom of God.''}}
\bv{26}{And they were astonished exceedingly, saying unto him, ``Then who can be saved?''}
\bv{27}{Jesus looking upon them saith, \redlet{``With men it is impossible, but not with God: for all things are possible with God.''}}
\bv{28}{Peter began to say unto him, ``Lo, we have left all, and have followed thee.''}
\bv{29}{Jesus said, \redlet{``Verily I say unto you, There is no man that hath left house, or brethren, or sisters, or mother, or father, or children, or lands, for my sake, and for the gospel's sake,}}
\bv{30}{\redlet{but he shall receive a hundredfold now in this time, houses, and brethren, and sisters, and mothers, and children, and lands, with persecutions; and in the world to come eternal life.}}
\bv{31}{\redlet{But many \supptext{that are} first shall be last; and the last first.''}}
\chapsec{Jesus again Foretells his Death \& Resurrection}
\bv{32}{And they were on the way, going up to Jerusalem; and Jesus was going before them: and they were amazed; and they that followed were afraid. And he took again the twelve, and began to tell them the things that were to happen unto him,}
\bv{33}{\supptext{saying}, \redlet{``Behold, we go up to Jerusalem; and the Son of man shall be delivered unto the chief priests and the scribes; and they shall condemn him to death, and shall deliver him unto the Gentiles:}}
\bv{34}{\redlet{and they shall mock him, and shall spit upon him, and shall scourge him, and shall kill him; and after three days he shall rise again.''}}
\chapsec{The Desire of Sts. James \& John}
\bv{35}{And there come near unto him James and John, the sons of Zebedee, saying unto him, ``Teacher, we would that thou shouldest do for us whatsoever we shall ask of thee.''}
\bv{36}{And he said unto them, \redlet{``What would ye that I should do for you?''}}
\bv{37}{And they said unto him, ``Grant unto us that we may sit, one on thy right hand, and one on \supptext{thy} left hand, in thy glory.''}
\bv{38}{But Jesus said unto them, \redlet{``Ye know not what ye ask. Are ye able to drink the cup that I drink? or to be baptised with the baptism that I am baptised with?''}}
\bv{39}{And they said unto him, ``We are able.'' And Jesus said unto them, \redlet{``The cup that I drink ye shall drink; and with the baptism that I am baptised withal shall ye be baptised:}}
\bv{40}{\redlet{but to sit on my right hand or on \supptext{my} left hand is not mine to give; but \supptext{it is for them} for whom it hath been prepared.''}}
\bv{41}{And when the ten heard it, they began to be moved with indignation concerning James and John.}
\bv{42}{And Jesus called them to him, and saith unto them, \redlet{``Ye know that they who are accounted to rule over the Gentiles lord it over them; and their great ones exercise authority over them.}}
\bv{43}{\redlet{But it is not so among you: but whosoever would become great among you, shall be your minister;}}
\bv{44}{\redlet{and whosoever would be first among you, shall be servant of all.}}
\bv{45}{\redlet{For the Son of man also came not to be ministered unto, but to minister, and to give his life a ransom for many.''}}
\chapsec{Bartimæus Receives his Sight}
\bv{46}{And they come to Jericho: and as he went out from Jericho, with his disciples and a great multitude, the son of Timæus, Bartimæus, a blind beggar, was sitting by the way side.}
\bv{47}{And when he heard that it was Jesus the Nazarene, he began to cry out, and say, ``Jesus, thou son of David, have mercy on me.''}
\bv{48}{And many rebuked him, that he should hold his peace: but he cried out the more a great deal, ``Thou son of David, have mercy on me.''}
\bv{49}{And Jesus stood still, and said, \redlet{``Call ye him.''} And they call the blind man, saying unto him, ``Be of good cheer: rise, he calleth thee.''}
\bv{50}{And he, casting away his garment, sprang up, and came to Jesus.}
\bv{51}{And Jesus answered him, and said, \redlet{``What wilt thou that I should do unto thee?''} And the blind man said unto him, ``Rabboni, that I may receive my sight.''}
\bv{52}{And Jesus said unto him, \redlet{``Go thy way; thy faith hath made thee whole.''} And straightway he received his sight, and followed him in the way.}
\chaphead{Chapter XI}
\chapdesc{The Official Presentation of Jesus as King}
\lettrine[image=true, lines=4, findent=3pt, nindent=0pt]{NT/Mark/Mk-And.eps}{nd} when they draw nigh unto Jerusalem, unto Bethphage and Bethany, at the mount of Olives, he sendeth two of his disciples,
\bv{2}{and saith unto them, \redlet{``Go your way into the village that is over against you: and straightway as ye enter into it, ye shall find a colt tied, whereon no man ever yet sat; loose him, and bring him.}}
\bv{3}{\redlet{And if any one say unto you, `Why do ye this?' say ye, `The Lord hath need of him; and straightway he will send him back hither.'{''}}}
\bv{4}{And they went away, and found a colt tied at the door without in the open street; and they loose him.}
\bv{5}{And certain of them that stood there said unto them, ``What do ye, loosing the colt?''}
\bv{6}{And they said unto them even as Jesus had said: and they let them go.}
\bv{7}{And they bring the colt unto Jesus, and cast on him their garments; and he sat upon him.}
\bv{8}{And many spread their garments upon the way; and others branches, which they had cut from the fields.}
\bv{9}{And they that went before, and they that followed, cried, ``Hosanna; Blessed \supptext{is} he that cometh in the name of the Lord:}
\bv{10}{Blessed \supptext{is} the kingdom that cometh, \supptext{the kingdom} of our father David: Hosanna in the highest.''}
\bv{11}{And he entered into Jerusalem, into the temple; and when he had looked round about upon all things, it being now eventide, he went out unto Bethany with the twelve.}
\chapsec{The Barren Fig Tree}
\bv{12}{And on the morrow, when they were come out from Bethany, he hungered.}
\bv{13}{And seeing a fig tree afar off having leaves, he came, if haply he might find anything thereon: and when he came to it, he found nothing but leaves; for it was not the season of figs.}
\bv{14}{And he answered and said unto it, \redlet{``No man eat fruit from thee henceforward for ever.''} And his disciples heard it.}
\chapsec{Jesus Purifies the Temple}
\bv{15}{And they come to Jerusalem: and he entered into the temple, and began to cast out them that sold and them that bought in the temple, and overthrew the tables of the money-changers, and the seats of them that sold the doves;}
\bv{16}{and he would not suffer that any man should carry a vessel through the temple.}
\bv{17}{And he taught, and said unto them, \redlet{``Is it not written, `My house shall be called a house of prayer for all the nations?'\mref{Is. 56:7} but ye have made it a den of robbers.}}
\bv{18}{And the chief priests and the scribes heard it, and sought how they might destroy him: for they feared him, for all the multitude was astonished at his teaching.}
\bv{19}{And every evening he went forth out of the city.}
\bv{20}{And as they passed by in the morning, they saw the fig tree withered away from the roots.}
\bv{21}{And Peter calling to remembrance saith unto him, ``Rabbi, behold, the fig tree which thou cursedst is withered away.''}
\chapsec{The Prayer of Faith}
\bv{22}{And Jesus answering saith unto them, \redlet{``Have faith in God.}}
\bv{23}{\redlet{Verily I say unto you, Whosoever shall say unto this mountain, `Be thou taken up and cast into the sea;' and shall not doubt in his heart, but shall believe that what he saith cometh to pass; he shall have it.}}
\bv{24}{\redlet{Therefore I say unto you, All things whatsoever ye pray and ask for, believe that ye receive them, and ye shall have them.}}
\bv{25}{\redlet{And whensoever ye stand praying, forgive, if ye have aught against any one; that your Father also who is in heaven may forgive you your trespasses.}}\mcomm{But if ye do not forgive, neither will your Father who is in heaven forgive your trespasses.}
\chapsec{Jesus' Authority Questioned}
\bv{27}{And they come again to Jerusalem: and as he was walking in the temple, there come to him the chief priests, and the scribes, and the elders;}
\bv{28}{and they said unto him, ``By what authority doest thou these things? or who gave thee this authority to do these things?''}
\bv{29}{And Jesus said unto them, \redlet{``I will ask of you one question, and answer me, and I will tell you by what authority I do these things.}}
\bv{30}{\redlet{The baptism of John, was it from heaven, or from men? answer me.''}}
\bv{31}{And they reasoned with themselves, saying, ``If we shall say, `From heaven;' he will say, `Why then did ye not believe him?'}
\bv{32}{But should we say, `From men'{''}---they feared the people: for all verily held John to be a prophet.}
\bv{33}{And they answered Jesus and say, ``We know not.'' And Jesus saith unto them, \redlet{``Neither tell I you by what authority I do these things.''}}
\chaphead{Chapter XII}
\chapdesc{Parable of the Householder}
\lettrine[image=true, lines=4, findent=3pt, nindent=0pt]{NT/Mark/Mk-And.eps}{nd} he began to speak unto them in parables. \redlet{``A man planted a vineyard, and set a hedge about it, and digged a pit for the winepress, and built a tower, and let it out to husbandmen, and went into another country.}
\bv{2}{\redlet{And at the season he sent to the husbandmen a servant, that he might receive from the husbandmen of the fruits of the vineyard.}}
\bv{3}{\redlet{And they took him, and beat him, and sent him away empty.}}
\bv{4}{\redlet{And again he sent unto them another servant; and him they wounded in the head, and handled shamefully.}}
\bv{5}{\redlet{And he sent another; and him they killed: and many others; beating some, and killing some.}}
\bv{6}{\redlet{He had yet one, a beloved son: he sent him last unto them, saying, `They will reverence my son.'}}
\bv{7}{\redlet{But those husbandmen said among themselves, `This is the heir; come, let us kill him, and the inheritance shall be ours.'}}
\bv{8}{\redlet{And they took him, and killed him, and cast him forth out of the vineyard.}}
\bv{9}{\redlet{What therefore will the lord of the vineyard do? he will come and destroy the husbandmen, and will give the vineyard unto others.}}
\bv{10}{\redlet{Have ye not read even this scripture:}}
\otQuote{Ps. 118:22-3}{\redlet{The stone which the builders rejected, The same was made the head of the corner; {\textsuperscript{11}}This was from the Lord, And it is marvelous in our eyes?''}}
\bv{12}{And they sought to lay hold on him; and they feared the multitude; for they perceived that he spake the parable against them: and they left him, and went away.}
\chapsec{The Question of Tribute}
\bv{13}{And they send unto him certain of the Pharisees and of the Herodians, that they might catch him in talk.}
\bv{14}{And when they were come, they say unto him, ``Teacher, we know that thou art true, and carest not for any one; for thou regardest not the person of men, but of a truth teachest the way of God: Is it lawful to give tribute unto Cæsar, or not?}
\bv{15}{Shall we give, or shall we not give?'' But he, knowing their hypocrisy, said unto them, \redlet{``Why make ye trial of me? bring me a denarius, that I may see it.''}}
\bv{16}{And they brought it. And he saith unto them, \redlet{``Whose is this image and superscription?''} And they said unto him, ``Cæsar's.''}
\bv{17}{And Jesus said unto them, \redlet{``Render unto Cæsar the things that are Cæsar's, and unto God the things that are God's.''} And they marvelled greatly at him.}
\chapsec{Jesus Answers the Sadducees}
\bv{18}{And there come unto him Sadducees, who say that there is no resurrection; and they asked him, saying,}
\bv{19}{``Teacher, Moses wrote unto us, If a man's brother die, and leave a wife behind him, and leave no child, that his brother should take his wife, and raise up seed unto his brother.}
\bv{20}{There were seven brethren: and the first took a wife, and dying left no seed;}
\bv{21}{and the second took her, and died, leaving no seed behind him; and the third likewise:}
\bv{22}{and the seven left no seed. Last of all the woman also died.}
\bv{23}{In the resurrection whose wife shall she be of them? for the seven had her to wife.''}
\bv{24}{Jesus said unto them, \redlet{``Is it not for this cause that ye err, that ye know not the scriptures, nor the power of God?}}
\bv{25}{\redlet{For when they shall rise from the dead, they neither marry, nor are given in marriage; but are as angels in heaven.}}
\bv{26}{\redlet{But as touching the dead, that they are raised; have ye not read in the book of Moses, in \supptext{the place concerning} the Bush, how God spake unto him, saying,}}
\otQuote{}{\redlet{I \supptext{am} the God of Abraham, and the God of Isaac, and the God of Jacob?}}
\bv{27}{\redlet{He is not the God of the dead, but of the living: ye do greatly err.''}}
\chapsec{The Great Commandments}
\bv{28}{And one of the scribes came, and heard them questioning together, and knowing that he had answered them well, asked him, ``What commandment is the first of all?''}
\bv{29}{Jesus answered, \redlet{``The first is,}}
\otQuote{Deut. 6:4-5}{\redlet{Hear, O Israel; The Lord our God, the Lord is one:
\bv{30}{and thou shalt love the Lord thy God with all thy heart, and with all thy soul, and with all thy mind, and with all thy strength.}}}
\bv{31}{\redlet{The second is this,}}
\otQuote{Lev. 19:18}{\redlet{Thou shalt love thy neighbor as thyself.}}
\redlet{There is none other commandment greater than these.}
\bv{32}{And the scribe said unto him, ``Of a truth, Teacher, thou hast well said that he is one; and there is none other but he:}
\bv{33}{and to love him with all the heart, and with all the understanding, and with all the strength, and to love his neighbor as himself, is much more than all whole burnt-offerings and sacrifices.''}
\bv{34}{And when Jesus saw that he answered discreetly, he said unto him, \redlet{``Thou art not far from the kingdom of God.''} And no man after that durst ask him any question.}
\chapsec{Jesus Questions the Pharisees}
\bv{35}{And Jesus answered and said, as he taught in the temple, \redlet{``How say the scribes that the Christ is the son of David?}}
\bv{36}{\redlet{David himself said in the Holy Ghost,}}
\otQuote{Ps. 110:1}{\redlet{The Lord said unto my Lord, Sit thou on my right hand, Till I make thine enemies the footstool of thy feet.}}
\bv{37}{\redlet{David himself calleth him Lord; and whence is he his son?''} And the common people heard him gladly.}
\bv{38}{And in his teaching he said, \redlet{``Beware of the scribes, who desire to walk in long robes, and \supptext{to have} salutations in the marketplaces,}}
\bv{39}{\redlet{and chief seats in the synagogues, and chief places at feasts:}}
\bv{40}{\redlet{they that devour widows' houses, and for a pretence make long prayers; these shall receive greater condemnation.''}}
\chapsec{Jesus \& the Widow's Lepta}
\bv{41}{And he sat down over against the treasury, and beheld how the multitude cast money into the treasury: and many that were rich cast in much.}
\bv{42}{And there came a poor widow, and she cast in two lepta, which make a quadrans.\mcomm{Lepton: $\frac{1}{128}$\textsuperscript{th} of a denarius (see Mark 6:37)\\Quadrans: $\frac{1}{64}$\textsuperscript{th} of a denarius}}
\bv{43}{And he called unto him his disciples, and said unto them, \redlet{``Verily I say unto you, This poor widow cast in more than all they that are casting into the treasury:}}
\bv{44}{\redlet{for they all did cast in of their superfluity; but she of her want did cast in all that she had, \supptext{even} all her living.''}}
\chaphead{Chapter XIII}
\chapdesc{The Olivet Discourse}
\lettrine[image=true, lines=4, findent=3pt, nindent=0pt]{NT/Mark/Mk-And.eps}{nd} as he went forth out of the temple, one of his disciples saith unto him, Teacher, behold, what manner of stones and what manner of buildings!
\bv{2}{And Jesus said unto him, \redlet{``Seest thou these great buildings? there shall not be left here one stone upon another, which shall not be thrown down.''}}
\bv{3}{And as he sat on the mount of Olives over against the temple, Peter and James and John and Andrew asked him privately,}
\bv{4}{``Tell us, when shall these things be? and what \supptext{shall be} the sign when these things are all about to be accomplished?''}
\chapsec{The Course of this Age}
\bv{5}{And Jesus began to say unto them, \redlet{Take heed that no man lead you astray.}}
\bv{6}{\redlet{Many shall come in my name, saying, `I am \supptext{he};' and shall lead many astray.}}
\bv{7}{\redlet{And when ye shall hear of wars and rumors of wars, be not troubled: \supptext{these things} must needs come to pass; but the end is not yet.}}
\bv{8}{\redlet{For nation shall rise against nation, and kingdom against kingdom; there shall be earthquakes in divers places; there shall be famines: these things are the beginning of travail.}}
\bv{9}{\redlet{But take ye heed to yourselves: for they shall deliver you up to councils; and in synagogues shall ye be beaten; and before governors and kings shall ye stand for my sake, for a testimony unto them.}}
\bv{10}{\redlet{And the gospel must first be preached unto all the nations.}}
\bv{11}{\redlet{And when they lead you \supptext{to judgement}, and deliver you up, be not anxious beforehand what ye shall speak: but whatsoever shall be given you in that hour, that speak ye; for it is not ye that speak, but the Holy Ghost.}}
\bv{12}{\redlet{And brother shall deliver up brother to death, and the father his child; and children shall rise up against parents, and cause them to be put to death.}}
\bv{13}{\redlet{And ye shall be hated of all men for my name's sake: but he that endureth to the end, the same shall be saved.}}
\chapsec{The Great Tribulation}
\bv{14}{\redlet{But when ye see the abomination of desolation}\mcomm{spoken of by Daniel the prophet [Dan. 9:17]} \redlet{standing where he ought not (let him that readeth understand), then let them that are in Judæa flee unto the mountains:}}
\bv{15}{\redlet{and let him that is on the housetop not go down, nor enter in, to take anything out of his house:}}
\bv{16}{\redlet{and let him that is in the field not return back to take his cloak.}}
\bv{17}{\redlet{But woe unto them that are with child and to them that give suck in those days!}}
\bv{18}{\redlet{And pray ye that it be not in the winter.}}
\bv{19}{\redlet{For those days shall be tribulation, such as there hath not been the like from the beginning of the creation which God created until now, and never shall be.}}
\bv{20}{\redlet{And except the Lord had shortened the days, no flesh would have been saved; but for the elect's sake, whom he chose, he shortened the days.}}
\bv{21}{\redlet{And then if any man shall say unto you, `Lo, here is the Christ;' or, `Lo, there;' believe \supptext{it} not:}}
\bv{22}{\redlet{for there shall arise false Christs and false prophets, and shall show signs and wonders, that they may lead astray, if possible, the elect.}}
\bv{23}{\redlet{But take ye heed: behold, I have told you all things beforehand.}}
\chapsec{The Lord's Return in Glory}
\bv{24}{\redlet{But in those days, after that tribulation, the sun shall be darkened, and the moon shall not give her light,}}
\bv{25}{\redlet{and the stars shall be falling from heaven, and the powers that are in the heavens shall be shaken.}}
\bv{26}{\redlet{And then shall they see the Son of man coming in clouds with great power and glory.}}
\bv{27}{\redlet{And then shall he send forth the angels, and shall gather together his elect from the four winds, from the uttermost part of the earth to the uttermost part of heaven.}}
\chapsec{Parable of the Fig Tree}
\bv{28}{\redlet{Now from the fig tree learn her parable: when her branch is now become tender, and putteth forth its leaves, ye know that the summer is nigh;}}
\bv{29}{\redlet{even so ye also, when ye see these things coming to pass, know ye that he is nigh, \supptext{even} at the doors.}}
\bv{30}{\redlet{Verily I say unto you, This generation shall not pass away, until all these things be accomplished.}}
\bv{31}{\redlet{Heaven and earth shall pass away: but my words shall not pass away.}}
\bv{32}{\redlet{But of that day or that hour knoweth no one, not even the angels in heaven, neither the Son, but the Father.}}
\bv{33}{\redlet{Take ye heed, watch and pray: for ye know not when the time is.}}
\chapsec{Watchfulness in View of the Lord's Return}
\bv{34}{\redlet{\supptext{It is} as \supptext{when} a man, sojourning in another country, having left his house, and given authority to his servants, to each one his work, commanded also the porter to watch.}}
\bv{35}{\redlet{Watch therefore: for ye know not when the lord of the house cometh, whether at even, or at midnight, or at cockcrowing, or in the morning;}}
\bv{36}{\redlet{lest coming suddenly he find you sleeping.}}
\bv{37}{\redlet{And what I say unto you I say unto all, Watch.}}
\chaphead{Chapter XIV}
\chapdesc{The Plot to Put Jesus to Death}
\lettrine[image=true, lines=4, findent=3pt, nindent=0pt]{NT/Mark/Mk-Now.eps}{ow} after two days was \supptext{the feast of} the passover and the unleavened bread: and the chief priests and the scribes sought how they might take him with subtlety, and kill him:
\bv{2}{for they said, ``Not during the feast, lest haply there shall be a tumult of the people.''}
\chapsec{Jesus Anointed by Mary of Bethany}
\bv{3}{And while he was in Bethany in the house of Simon the leper, as he sat at meat, there came a woman having an alabaster cruse of ointment of pure nard very costly; \supptext{and} she brake the cruse, and poured it over his head.}
\bv{4}{But there were some that had indignation among themselves, \supptext{saying}, ``To what purpose hath this waste of the ointment been made?}
\bv{5}{For this ointment might have been sold for above three hundred denarii, and given to the poor.'' And they murmured against her.}
\bv{6}{But Jesus said, \redlet{``Let her alone; why trouble ye her? she hath wrought a good work on me.}}
\bv{7}{\redlet{For ye have the poor always with you, and whensoever ye will ye can do them good: but me ye have not always.}}
\bv{8}{\redlet{She hath done what she could; she hath anointed my body beforehand for the burying.}}
\bv{9}{\redlet{And verily I say unto you, Wheresoever the gospel shall be preached throughout the whole world, that also which this woman hath done shall be spoken of for a memorial of her.''}}
\chapsec{Judas Covenants to Betray Jesus}
\bv{10}{And Judas Iscariot, he that was one of the twelve, went away unto the chief priests, that he might deliver him unto them.}
\bv{11}{And they, when they heard it, were glad, and promised to give him money. And he sought how he might conveniently deliver him \supptext{unto them}.}
\chapsec{The Preparation of the Passover}
\bv{12}{And on the first day of unleavened bread, when they sacrificed the passover, his disciples say unto him, ``Where wilt thou that we go and make ready that thou mayest eat the passover?''}
\bv{13}{And he sendeth two of his disciples, and saith unto them, \redlet{``Go into the city, and there shall meet you a man bearing a pitcher of water: follow him;}}
\bv{14}{\redlet{and wheresoever he shall enter in, say to the master of the house, `The Teacher saith, `Where is my guest-chamber, where I shall eat the passover with my disciples?'{'}}}
\bv{15}{\redlet{And he will himself show you a large upper room furnished \supptext{and} ready: and there make ready for us.''}}
\bv{16}{And the disciples went forth, and came into the city, and found as he had said unto them: and they made ready the passover.}
\chapsec{The Last Passover}
\bv{17}{And when it was evening he cometh with the twelve.}
\bv{18}{And as they sat and were eating, Jesus said, \redlet{``Verily I say unto you, One of you shall betray me, \supptext{even} he that eateth with me.''}}
\bv{19}{They began to be sorrowful, and to say unto him one by one, ``Is it I?''}
\bv{20}{And he said unto them, \redlet{``\supptext{It is} one of the twelve, he that dippeth with me in the dish.}}
\bv{21}{\redlet{For the Son of man goeth, even as it is written of him: but woe unto that man through whom the Son of man is betrayed! good were it for that man if he had not been born.''}}
\chapsec{Institution of the Eucharist}
\bv{22}{And as they were eating, he took bread, and when he had blessed, he brake it, and gave to them, and said, \redlet{``Take ye: this is my body.''}}
\bv{23}{And he took a cup, and when he had given thanks, he gave to them: and they all drank of it.}
\bv{24}{And he said unto them, \redlet{``This is my blood of the covenant,}\mcomm{the new covenant} \redlet{which is poured out for many.}}
\bv{25}{\redlet{Verily I say unto you, I shall no more drink of the fruit of the vine, until that day when I drink it new in the kingdom of God.''}}
\chapsec{St. Peter's Denial Foretold}
\bv{26}{And when they had sung a hymn, they went out unto the mount of Olives.}
\bv{27}{And Jesus saith unto them, \redlet{``All ye shall be offended:}}\mcomm{because of me tonight} \redlet{for it is written, `I will smite the shepherd, and the sheep shall be scattered abroad.'\mref{Zech. 13:7}}
\bv{28}{\redlet{Howbeit, after I am raised up, I will go before you into Galilee.''}}
\bv{29}{But Peter said unto him, ``Although all shall be offended, yet will not I.''}
\bv{30}{And Jesus saith unto him, \redlet{``Verily I say unto thee, that thou to-day, \supptext{even} this night, before the cock crow twice, shalt deny me thrice.''}}
\bv{31}{But he spake exceeding vehemently, ``If I must die with thee, I will not deny thee.'' And in like manner also said they all.}
\chapsec{The Agony in the Garden}
\bv{32}{And they come unto a place which was named Gethsemane: and he saith unto his disciples, \redlet{``Sit ye here, while I pray.''}}
\bv{33}{And he taketh with him Peter and James and John, and began to be greatly amazed, and sore troubled.}
\bv{34}{And he saith unto them, \redlet{``My soul is exceeding sorrowful even unto death: abide ye here, and watch.''}}
\chapsec{The First Prayer}
\bv{35}{And he went forward a little, and fell on the ground, and prayed that, if it were possible, the hour might pass away from him.}
\bv{36}{And he said, \redlet{``Abba, Father, all things are possible unto thee; remove this cup from me: howbeit not what I will, but what thou wilt.''}}
\bv{37}{And he cometh, and findeth them sleeping, and saith unto Peter, \redlet{``Simon, sleepest thou? couldest thou not watch one hour?}}
\bv{38}{\redlet{Watch and pray, that ye enter not into temptation: the spirit indeed is willing, but the flesh is weak.''}}
\chapsec{The Second Prayer}
\bv{39}{And again he went away, and prayed, saying the same words.}
\bv{40}{And again he came, and found them sleeping, for their eyes were very heavy; and they knew not what to answer him.}
\chapsec{The Third Prayer}
\bv{41}{And he cometh the third time, and saith unto them, \redlet{``Sleep on now, and take your rest: it is enough; the hour is come; behold, the Son of man is betrayed into the hands of sinners.}}
\bv{42}{\redlet{Arise, let us be going: behold, he that betrayeth me is at hand.''}}
\chapsec{The Betrayal \& Arrest of Jesus}
\bv{43}{And straightway, while he yet spake, cometh Judas, one of the twelve, and with him a multitude with swords and staves, from the chief priests and the scribes and the elders.}
\bv{44}{Now he that betrayed him had given them a token, saying, ``Whomsoever I shall kiss, that is he; take him, and lead him away safely.''}
\bv{45}{And when he was come, straightway he came to him, and saith, ``Rabbi;'' and kissed him.}
\bv{46}{And they laid hands on him, and took him.}
\chapsec{St. Peter Smites the Servant}
\bv{47}{But a certain one of them that stood by drew his sword, and smote the servant of the high priest, and struck off his ear.}
\bv{48}{And Jesus answered and said unto them, \redlet{``Are ye come out, as against a robber, with swords and staves to seize me?}}
\bv{49}{\redlet{I was daily with you in the temple teaching, and ye took me not: but \supptext{this is done} that the scriptures might be fulfilled.''}}
\bv{50}{And they all left him, and fled.}
\bv{51}{And a certain young man followed with him, having a linen cloth cast about him, over \supptext{his} naked \supptext{body}: and they lay hold on him;}
\bv{52}{but he left the linen cloth, and fled naked.}
\chapsec{Jesus is Brought before the Sanhedrin}
\bv{53}{And they led Jesus away to the high priest: and there come together with him all the chief priests and the elders and the scribes.}
\bv{54}{And Peter had followed him afar off, even within, into the court of the high priest; and he was sitting with the officers, and warming himself in the light \supptext{of the fire}.}
\bv{55}{Now the chief priests and the whole council sought witness against Jesus to put him to death; and found it not.}
\bv{56}{For many bare false witness against him, and their witness agreed not together.}
\bv{57}{And there stood up certain, and bare false witness against him, saying,}
\bv{58}{``We heard him say, I will destroy this temple that is made with hands, and in three days I will build another made without hands.''}
\bv{59}{And not even so did their witness agree together.}
\bv{60}{And the high priest stood up in the midst, and asked Jesus, saying, ``Answerest thou nothing? what is it which these witness against thee?''}
\bv{61}{But he held his peace, and answered nothing. Again the high priest asked him, and saith unto him, ``Art thou the Christ, the Son of the Blessed?''}
\bv{62}{And Jesus said, \redlet{``I am: and ye shall see the Son of man sitting at the right hand of Power, and coming with the clouds of heaven.''}}
\bv{63}{And the high priest rent his clothes, and saith, ``What further need have we of witnesses?}
\bv{64}{Ye have heard the blasphemy: what think ye?'' And they all condemned him to be worthy of death.}
\bv{65}{And some began to spit on him, and to cover his face, and to buffet him, and to say unto him, ``Prophesy:'' and the officers received him with blows of their hands.}
\chapsec{St. Peter Denies his Lord}
\bv{66}{And as Peter was beneath in the court, there cometh one of the maids of the high priest;}
\bv{67}{and seeing Peter warming himself, she looked upon him, and saith, ``Thou also wast with the Nazarene, \supptext{even} Jesus.''}
\bv{68}{But he denied, saying, ``I neither know, nor understand what thou sayest:'' and he went out into the porch; and the cock crew.}
\bv{69}{And the maid saw him, and began again to say to them that stood by, ``This is \supptext{one} of them.''}
\bv{70}{But he again denied it. And after a little while again they that stood by said to Peter, ``Of a truth thou art \supptext{one} of them; for thou art a Galilæan.''\mcomm{and thy speech agreeth thereto}}
\bv{71}{But he began to curse, and to swear, ``I know not this man of whom ye speak.''}
\bv{72}{And straightway the second time the cock crew. And Peter called to mind the word, how that Jesus said unto him, \redlet{``Before the cock crow twice, thou shalt deny me thrice.''} And when he thought thereon, he wept.}
\chaphead{Chapter XV}
\chapdesc{Jesus Sent before Pilate}
\lettrine[image=true, lines=4, findent=3pt, nindent=0pt]{NT/Mark/Mk-And.eps}{nd} straightway in the morning the chief priests with the elders and scribes, and the whole council, held a consultation, and bound Jesus, and carried him away, and delivered him up to Pilate.
\bv{2}{And Pilate asked him, ``Art thou the King of the Jews?'' And he answering saith unto him, \redlet{``Thou sayest.''}}
\bv{3}{And the chief priests accused him of many things.\mcomm{but he answered nothing.}}
\bv{4}{And Pilate again asked him, saying, ``Answerest thou nothing? behold how many things they accuse thee of.''}
\bv{5}{But Jesus no more answered anything; insomuch that Pilate marvelled.}
\bv{6}{Now at the feast he used to release unto them one prisoner, whom they asked of him.}
\chapsec{Not Jesus but Barabbas}
\bv{7}{And there was one called Barabbas, \supptext{lying} bound with them that had made insurrection, men who in the insurrection had committed murder.}
\bv{8}{And the multitude went up and began to ask him \supptext{to do} as he was wont to do unto them.}
\bv{9}{And Pilate answered them, saying, ``Will ye that I release unto you the King of the Jews?''}
\bv{10}{For he perceived that for envy the chief priests had delivered him up.}
\bv{11}{But the chief priests stirred up the multitude, that he should rather release Barabbas unto them.}
\bv{12}{And Pilate again answered and said unto them, ``What then shall I do unto him whom ye call the King of the Jews?''}
\bv{13}{And they cried out again, ``Crucify him.''}
\bv{14}{And Pilate said unto them, ``Why, what evil hath he done?'' But they cried out exceedingly, ``Crucify him.''}
\bv{15}{And Pilate, wishing to content the multitude, released unto them Barabbas, and delivered Jesus, when he had scourged him, to be crucified.}
\chapsec{Jesus Crowned with Thorns}
\bv{16}{And the soldiers led him away within the court, which is the Prætorium; and they call together the whole band.}
\bv{17}{And they clothe him with purple, and platting a crown of thorns, they put it on him;}
\bv{18}{and they began to salute him, ``Hail, King of the Jews!''}
\bv{19}{And they smote his head with a reed, and spat upon him, and bowing their knees worshipped him.}
\bv{20}{And when they had mocked him, they took off from him the purple, and put on him his garments. And they lead him out to crucify him.}
\bv{21}{And they compel one passing by, Simon of Cyrene, coming from the country, the father of Alexander and Rufus, to go \supptext{with them}, that he might bear his cross.}
\bv{22}{And they bring him unto the place Golgotha, which is, being interpreted, The place of a skull.}
\bv{23}{And they offered him wine mingled with myrrh: but he received it not.}
\chapsec{Jesus Crucified}
\bv{24}{And they crucify him, and part his garments among them, casting lots upon them, what each should take.}
\bv{25}{And it was the third hour, and they crucified him.}
\bv{26}{And the superscription of his accusation was written over, ``THE KING OF THE JEWS.''}
\bv{27}{And with him they crucify two robbers; one on his right hand, and one on his left.\mcomm{And the scripture was fulfilled, which saith, And he was reckoned with transgressors. [Is. 53:12]}}
\bv{29}{And they that passed by railed on him, wagging their heads, and saying, ``Ha! thou that destroyest the temple, and buildest it in three days,}
\bv{30}{save thyself, and come down from the cross.''}
\bv{31}{In like manner also the chief priests mocking \supptext{him} among themselves with the scribes said, ``He saved others; himself he cannot save.}
\bv{32}{Let the Christ, the King of Israel, now come down from the cross, that we may see and believe.'' And they that were crucified with him reproached him.}
\bv{33}{And when the sixth hour was come, there was darkness over the whole land until the ninth hour.}
\chapsec{Jesus Christ Gives up the Ghost}
\bv{34}{And at the ninth hour Jesus cried with a loud voice, \redlet{``Eloi, Eloi, lama sabachthani?''} which is, being interpreted, \redlet{``My God, my God, why hast thou forsaken me?''}\mref{cf. Ps. 22:1}}
\bv{35}{And some of them that stood by, when they heard it, said, ``Behold, he calleth Elijah.''}
\bv{36}{And one ran, and filling a sponge full of vinegar, put it on a reed, and gave him to drink, saying, ``Let be; let us see whether Elijah cometh to take him down.''}
\bv{37}{And Jesus uttered a loud voice, and gave up the ghost.}
\bv{38}{And the veil of the temple was rent in two from the top to the bottom.}
\bv{39}{And when the centurion, who stood by over against him, saw that he so gave up the ghost, he said, ``Truly this man was the Son of God.''}
\bv{40}{And there were also women beholding from afar: among whom \supptext{were} both Mary Magdalene, and Mary the mother of James the less and of Joses, and Salome;}
\bv{41}{who, when he was in Galilee, followed him, and ministered unto him; and many other women that came up with him unto Jerusalem.}
\chapsec{The Entombment}
\bv{42}{And when even was now come, because it was the Preparation, that is, the day before the sabbath,}
\bv{43}{there came Joseph of Arimathæa, a councillor of honorable estate, who also himself was looking for the kingdom of God; and he boldly went in unto Pilate, and asked for the body of Jesus.}
\bv{44}{And Pilate marvelled if he were already dead: and calling unto him the centurion, he asked him whether he had been any while dead.}
\bv{45}{And when he learned it of the centurion, he granted the corpse to Joseph.}
\bv{46}{And he bought a linen cloth, and taking him down, wound him in the linen cloth, and laid him in a tomb which had been hewn out of a rock; and he rolled a stone against the door of the tomb.}
\bv{47}{And Mary Magdalene and Mary the \supptext{mother} of Joses beheld where he was laid.}
\chaphead{Chapter XVI}
\chapdesc{The Resurrection of Jesus Christ}
\lettrine[image=true, lines=4, findent=3pt, nindent=0pt]{NT/Mark/Mk-And.eps}{nd} when the sabbath was past, Mary Magdalene, and Mary the \supptext{mother} of James, and Salome, bought spices, that they might come and anoint him.
\bv{2}{And very early on the first day of the week, they come to the tomb when the sun was risen.}
\bv{3}{And they were saying among themselves, ``Who shall roll us away the stone from the door of the tomb?''}
\bv{4}{and looking up, they see that the stone is rolled back: for it was exceeding great.}
\bv{5}{And entering into the tomb, they saw a young man sitting on the right side, arrayed in a white robe; and they were amazed.}
\bv{6}{And he saith unto them, ``Be not amazed: ye seek Jesus, the Nazarene, who hath been crucified: he is risen; he is not here: behold, the place where they laid him!}
\bv{7}{But go, tell his disciples and Peter, He goeth before you into Galilee: there shall ye see him, as he said unto you.''}
\bv{8}{And they went out, and fled from the tomb; for trembling and astonishment had come upon them: and they said nothing to any one; for they were afraid.}
\begin{center}
	{\scshape [Here Endeth the Gospel of Mark]}
\end{center}
\clearpage
\chapter{Traditional Additions to Mark}
\chapdesc{Due to the short and abrupt ending of St. Mark's Gospel, these different endings have been added over time.}
\textit{The Shorter Ending}\\
They told all that had been commanded them briefly to those around Peter. After that, Jesus himself sent them out, from east to west, with the sacred and imperishable proclamation of eternal salvation.
\par
\noindent
\textit{The ``Freer Logion'' Ending}\\
And they made excuse, saying, ``This age of lawlessness and unbelief is under Satan, who by the unclean spirits does not allow men power to comprehend the truth of God. For this reason reveal thy righteousness now,'' they said to Christ.
\par
And Christ replied to them, ``The limit of the years of the power of Satan has been fulfilled, but other terrible things are near at hand. And I was delivered unto death on behalf of those who sinned, in order that they may return to the truth and sin no more, to the end that they may inherit the spiritual and incorruptible glory of righteousness (which glory is) in heaven. But go ye into all the world.''\par
\noindent
\textit{The Longer Ending}\\
{Now when he was risen early on the first day of the week, he appeared first to Mary Magdalene, from whom he had cast out seven demons.}
{She went and told them that had been with him, as they mourned and wept.}
{And they, when they heard that he was alive, and had been seen of her, disbelieved.}
{And after these things he was manifested in another form unto two of them, as they walked, on their way into the country.}
{And they went away and told it unto the rest: neither believed they them.}
\par
{And afterward he was manifested unto the eleven themselves as they sat at meat; and he upbraided them with their unbelief and hardness of heart, because they believed not them that had seen him after he was risen.}
{And he said unto them, ``Go ye into all the world, and preach the gospel to the whole creation.}
{He that believeth and is baptised shall be saved; but he that disbelieveth shall be condemned.}
{And these signs shall accompany them that believe: in my name shall they cast out demons; they shall speak with new tongues;}
{they shall take up serpents, and if they drink any deadly thing, it shall in no wise hurt them; they shall lay hands on the sick, and they shall recover.''}
\par
{So then the Lord Jesus, after he had spoken unto them, was received up into heaven, and sat down at the right hand of God.}
{And they went forth, and preached everywhere, the Lord working with them, and confirming the word by the signs that followed. Amen.}
	\clearpage
	\input{./books/Gospel/Luke.tex}
	\clearpage
	\chapter{The Holy Gospel of Jesus Christ according to Saint John}
\fancyhead[RE,LO]{The Gospel according to John}
\chaphead{Chapter I}
\chapdesc{The Deity of Jesus Christ}
\lettrine[image=true, lines=4, findent=3pt, nindent=0pt]{NT/John/Jn1-I.eps}{n} the beginning was the Word, and the Word was with God, and the Word was God.
\bv{2}{The same was in the beginning with God.}
\chapsec{His Pre-Incarnation Work}
\bv{3}{All things were made through him; and without him was not anything made that hath been made.}
\bv{4}{In him was life; and the life was the light of men.}
\bv{5}{And the light shineth in the darkness; and the darkness apprehended it not.}
\chapsec{Ministry of St. John the Baptist}
\bv{6}{There came a man, sent from God, whose name was John.}
\bv{7}{The same came for witness, that he might bear witness of the light, that all might believe through him.}
\bv{8}{He was not the light, but \supptext{came} that he might bear witness of the light.}
\chapsec{Jesus Christ the True Light}
\bv{9}{There was the true light, \supptext{even the light} which lighteth every man, coming into the world.}
\bv{10}{He was in the world, and the world was made through him, and the world knew him not.}
\chapsec{Sons \& Unbelievers}
\bv{11}{He came unto his own, and they that were his own received him not.}
\bv{12}{But as many as received him, to them gave he the right to become children of God, \supptext{even} to them that believe on his name:}
\bv{13}{who were born, not of blood, nor of the will of the flesh, nor of the will of man, but of God.}
\chapsec{The Incarnation}
\bv{14}{And the Word became flesh, and dwelt among us (and we beheld his glory, glory as of the only begotten from the Father), full of grace and truth.}
\chapsec{The Witness of St. John the Baptist}
\bv{15}{John beareth witness of him, and crieth, saying, ``This was he of whom I said, `He that cometh after me is become before me: for he was before me.' ''}
\bv{16}{For of his fulness we all received, and grace for grace.}
\bv{17}{For the law was given through Moses; grace and truth came through Jesus Christ.}
\bv{18}{No man hath seen God at any time; the only begotten God, who is in the bosom of the Father, he hath declared \supptext{him}.}
\par
\bv{19}{And this is the witness of John, when the Jews sent unto him from Jerusalem priests and Levites to ask him, ``Who art thou?''}
\bv{20}{And he confessed, and denied not; and he confessed, ``I am not the Christ.''}
\bv{21}{And they asked him, ``What then? Art thou Elijah?'' And he saith, ``I am not.'' ``Art thou the prophet?'' And he answered, ``No.''}
\bv{22}{They said therefore unto him, ``Who art thou? that we may give an answer to them that sent us. What sayest thou of thyself?''}
\bv{23}{He said, ``I am the voice of one crying in the wilderness, `Make straight the way of the Lord,' as said Isaiah the prophet.''\mref{Is. 40:3}}
\bv{24}{And they had been sent from the Pharisees.}
\bv{25}{And they asked him, and said unto him, ``Why then baptisest thou, if thou art not the Christ, neither Elijah, neither the prophet?''}
\bv{26}{John answered them, saying, ``I baptise in water: in the midst of you standeth one whom ye know not,}
\bv{27}{\supptext{even} he that cometh after me, the latchet of whose shoe I am not worthy to unloose.''}
\bv{28}{These things were done in Bethany beyond the Jordan, where John was baptising.}
\par
\bv{29}{On the morrow he seeth Jesus coming unto him, and saith, ``Behold, the Lamb of God, that taketh away the sin of the world!}
\bv{30}{This is he of whom I said, `After me cometh a man who is become before me: for he was before me.'}
\bv{31}{And I knew him not; but that he should be made manifest to Israel, for this cause came I baptising in water.''}
\bv{32}{And John bare witness, saying, ``I have beheld the Spirit descending as a dove out of heaven; and it abode upon him.}
\bv{33}{And I knew him not: but he that sent me to baptise in water, he said unto me, `Upon whomsoever thou shalt see the Spirit descending, and abiding upon him, the same is he that baptiseth in the Holy Ghost.'}
\bv{34}{And I have seen, and have borne witness that this is the Son of God.''}
\par
\bv{35}{Again on the morrow John was standing, and two of his disciples;}
\bv{36}{and he looked upon Jesus as he walked, and saith, ``Behold, the Lamb of God!''}
\bv{37}{And the two disciples heard him speak, and they followed Jesus.}
\bv{38}{And Jesus turned, and beheld them following, and saith unto them, \redlet{``What seek ye?''} And they said unto him, ``Rabbi'' (which is to say, being interpreted, ``Teacher''), ``where abidest thou?''}
\bv{39}{He saith unto them, \redlet{	``Come, and ye shall see.''} They came therefore and saw where he abode; and they abode with him that day: it was about the tenth hour.}
\bv{40}{One of the two that heard John \supptext{speak}, and followed him, was Andrew, Simon Peter's brother.}
\bv{41}{He findeth first his own brother Simon, and saith unto him, ``We have found the Messiah'' (which is, being interpreted, ``Christ'').}
\bv{42}{He brought him unto Jesus. Jesus looked upon him, and said, \redlet{``Thou art Simon the son of John: thou shalt be called Cephas''} (which is by interpretation, ``Peter'').}
\bv{43}{On the morrow he was minded to go forth into Galilee, and he findeth Philip: and Jesus saith unto him, \redlet{``Follow me.''}}
\bv{44}{Now Philip was from Bethsaida, of the city of Andrew and Peter.}
\bv{45}{Philip findeth Nathanael, and saith unto him, ``We have found him, of whom Moses in the law, and the prophets, wrote, Jesus of Nazareth, the son of Joseph.''}
\bv{46}{And Nathanael said unto him, ``Can any good thing come out of Nazareth?'' Philip saith unto him, ``Come and see.''}
\bv{47}{Jesus saw Nathanael coming to him, and saith of him, \redlet{``Behold, an Israelite indeed, in whom is no guile!''}}
\bv{48}{Nathanael saith unto him, ``Whence knowest thou me?'' Jesus answered and said unto him, \redlet{``Before Philip called thee, when thou wast under the fig tree, I saw thee.''}}
\bv{49}{Nathanael answered him, ``Rabbi, thou art the Son of God; thou art King of Israel.''}
\bv{50}{Jesus answered and said unto him, \redlet{``Because I said unto thee, I saw thee underneath the fig tree, believest thou? thou shalt see greater things than these.''}}
\bv{51}{And he saith unto him, \redlet{``Verily, verily, I say unto you, Ye shall see the heaven opened, and the angels of God ascending and descending upon the Son of man.''}}
\chaphead{Chapter II}
\chapdesc{The Marriage at Cana}
\lettrine[image=true, lines=4, findent=3pt, nindent=0pt]{NT/John/Jn-And.eps}{nd} the third day there was a marriage in Cana of Galilee; and the mother of Jesus was there:
\bv{2}{and Jesus also was bidden, and his disciples, to the marriage.}
\bv{3}{And when the wine failed, the mother of Jesus saith unto him, ``They have no wine.''}
\bv{4}{And Jesus saith unto her, \redlet{``Woman, what have I to do with thee? mine hour is not yet come.''}}
\bv{5}{His mother saith unto the servants, ``Whatsoever he saith unto you, do it.''}
\bv{6}{Now there were six waterpots of stone set there after the Jews' manner of purifying, containing two or three firkins apiece.}
\bv{7}{Jesus saith unto them, \redlet{``Fill the waterpots with water.''} And they filled them up to the brim.}
\bv{8}{And he saith unto them, \redlet{``Draw out now, and bear unto the ruler of the feast.''} And they bare it.}
\bv{9}{And when the ruler of the feast tasted the water now become wine, and knew not whence it was (but the servants that had drawn the water knew), the ruler of the feast calleth the bridegroom,}
\bv{10}{and saith unto him, ``Every man setteth on first the good wine; and when \supptext{men} have drunk freely, \supptext{then} that which is worse: thou hast kept the good wine until now.''}
\bv{11}{This beginning of his signs did Jesus in Cana of Galilee, and manifested his glory; and his disciples believed on him.}
\par
\bv{12}{After this he went down to Capernaum, he, and his mother, and \supptext{his} brethren, and his disciples; and there they abode not many days.}
\bv{13}{And the passover of the Jews was at hand, and Jesus went up to Jerusalem.}
\bv{14}{And he found in the temple those that sold oxen and sheep and doves, and the changers of money sitting:}
\bv{15}{and he made a scourge of cords, and cast all out of the temple, both the sheep and the oxen; and he poured out the changers' money, and overthrew their tables;}
\bv{16}{and to them that sold the doves he said, \redlet{``Take these things hence; make not my Father's house a house of merchandise.''}}
\bv{17}{His disciples remembered that it was written, \otQuote{Ps.69:9}{Zeal for thy house shall eat me up.}}
\bv{18}{The Jews therefore answered and said unto him, ``What sign showest thou unto us, seeing that thou doest these things?''}
\bv{19}{Jesus answered and said unto them, \redlet{``Destroy this temple, and in three days I will raise it up.''}}
\bv{20}{The Jews therefore said, ``Forty and six years was this temple in building, and wilt thou raise it up in three days?''}
\bv{21}{But he spake of the temple of his body.}
\bv{22}{When therefore he was raised from the dead, his disciples remembered that he spake this; and they believed the scripture, and the word which Jesus had said.}
\bv{23}{Now when he was in Jerusalem at the passover, during the feast, many believed on his name, beholding his signs which he did.}
\bv{24}{But Jesus did not trust himself unto them, for that he knew all men,}
\bv{25}{and because he needed not that any one should bear witness concerning man; for he himself knew what was in man.}
\chaphead{Chapter III}
\chapdesc{Jesus \& Nicodemus}
\lettrine[image=true, lines=4, findent=3pt, nindent=0pt]{NT/John/Jn-Now.eps}{ow} there was a man of the Pharisees, named Nicodemus, a ruler of the Jews:
\bv{2}{the same came unto him by night, and said to him, ``Rabbi, we know that thou art a teacher come from God; for no one can do these signs that thou doest, except God be with him.''}
\bv{3}{Jesus answered and said unto him, \redlet{``Verily, verily, I say unto thee, Except one be born anew, he cannot see the kingdom of God.''}}
\bv{4}{Nicodemus saith unto him, ``How can a man be born when he is old? can he enter a second time into his mother's womb, and be born?''}
\bv{5}{Jesus answered, \redlet{``Verily, verily, I say unto thee, Except one be born of water and the Spirit, he cannot enter into the kingdom of God.}}
\bv{6}{\redlet{That which is born of the flesh is flesh; and that which is born of the Spirit is spirit.}}
\bv{7}{\redlet{Marvel not that I said unto thee, `Ye must be born anew.'}}
\bv{8}{\redlet{The wind bloweth where it will, and thou hearest the voice thereof, but knowest not whence it cometh, and whither it goeth: so is every one that is born of the Spirit.''}}
\par
\bv{9}{Nicodemus answered and said unto him, ``How can these things be?''}
\bv{10}{Jesus answered and said unto him, \redlet{``Art thou the teacher of Israel, and understandest not these things?}}
\bv{11}{\redlet{Verily, verily, I say unto thee, We speak that which we know, and bear witness of that which we have seen; and ye receive not our witness.}}
\bv{12}{\redlet{If I told you earthly things and ye believe not, how shall ye believe if I tell you heavenly things?}}
\bv{13}{\redlet{And no one hath ascended into heaven, but he that descended out of heaven, \supptext{even} the Son of man, who is in heaven.}}
\bv{14}{\redlet{And as Moses lifted up the serpent in the wilderness, even so must the Son of man be lifted up;}}
\bv{15}{\redlet{that whosoever believeth may in him have eternal life.}}
\par
\bv{16}{\redlet{For God so loved the world, that he gave his only begotten Son, that whosoever believeth on him should not perish, but have eternal life.}}
\bv{17}{\redlet{For God sent not the Son into the world to judge the world; but that the world should be saved through him.}}
\bv{18}{\redlet{He that believeth on him is not judged: he that believeth not hath been judged already, because he hath not believed on the name of the only begotten Son of God.}}
\bv{19}{\redlet{And this is the judgement, that the light is come into the world, and men loved the darkness rather than the light; for their works were evil.}}
\bv{20}{\redlet{For every one that doeth evil hateth the light, and cometh not to the light, lest his works should be reproved.}}
\bv{21}{\redlet{But he that doeth the truth cometh to the light, that his works may be made manifest, that they have been wrought in God.''}}
\par
\bv{22}{After these things came Jesus and his disciples into the land of Judæa; and there he tarried with them, and baptised.}
\bv{23}{And John also was baptising in Ænon near to Salim, because there was much water there: and they came, and were baptised.}
\bv{24}{For John was not yet cast into prison.}
\bv{25}{There arose therefore a questioning on the part of John's disciples with a Jew about purifying.}
\bv{26}{And they came unto John, and said to him, ``Rabbi, he that was with thee beyond the Jordan, to whom thou hast borne witness, behold, the same baptiseth, and all men come to him.''}
\bv{27}{John answered and said, ``A man can receive nothing, except it have been given him from heaven.}
\bv{28}{Ye yourselves bear me witness, that I said, `I am not the Christ,' but, that I am sent before him.}
\bv{29}{He that hath the bride is the bridegroom: but the friend of the bridegroom, that standeth and heareth him, rejoiceth greatly because of the bridegroom's voice: this my joy therefore is made full.}
\bv{30}{He must increase, but I must decrease.}
\chapsec{Declaration about Jesus Christ}
\bv{31}{He that cometh from above is above all: he that is of the earth is of the earth, and of the earth he speaketh: he that cometh from heaven is above all.}
\bv{32}{What he hath seen and heard, of that he beareth witness; and no man receiveth his witness.}
\bv{33}{He that hath received his witness hath set his seal to \supptext{this}, that God is true.}
\bv{34}{For he whom God hath sent speaketh the words of God: for he giveth not the Spirit by measure.}
\bv{35}{The Father loveth the Son, and hath given all things into his hand.}
\bv{36}{He that believeth on the Son hath eternal life; but he that obeyeth not the Son shall not see life, but the wrath of God abideth on him.''}
\chaphead{Chapter IV}
\chapdesc{Jesus Departs into Galilee}
\lettrine[image=true, lines=4, findent=3pt, nindent=0pt]{NT/John/Jn4-W.eps}{hen} therefore the Lord knew that the Pharisees had heard that Jesus was making and baptising more disciples than John
\bv{2}{(although Jesus himself baptised not, but his disciples),}
\bv{3}{he left Judæa, and departed again into Galilee.}
\bv{4}{And he must needs pass through Samaria.}
\bv{5}{So he cometh to a city of Samaria, called Sychar, near to the parcel of ground that Jacob gave to his son Joseph:}
\chapsec{Jesus \& the Samaritan Woman}
\bv{6}{and Jacob's well was there. Jesus therefore, being wearied with his journey, sat thus by the well. It was about the sixth hour.}
\bv{7}{There cometh a woman of Samaria to draw water: Jesus saith unto her, \redlet{``Give me to drink.''}}
\bv{8}{For his disciples were gone away into the city to buy food.}
\bv{9}{The Samaritan woman therefore saith unto him, ``How is it that thou, being a Jew, askest drink of me, who am a Samaritan woman?'' (For Jews have no dealings with Samaritans.)}
\bv{10}{Jesus answered and said unto her, \redlet{``If thou knewest the gift of God, and who it is that saith to thee, `Give me to drink;' thou wouldest have asked of him, and he would have given thee living water.''}}
\bv{11}{The woman saith unto him, ``Sir, thou hast nothing to draw with, and the well is deep: whence then hast thou that living water?}
\bv{12}{Art thou greater than our father Jacob, who gave us the well, and drank thereof himself, and his sons, and his cattle?''}
\bv{13}{Jesus answered and said unto her, \redlet{``Every one that drinketh of this water shall thirst again:}}
\chapsec{The Indwelling Spirit}
\bv{14}{\redlet{but whosoever drinketh of the water that I shall give him shall never thirst; but the water that I shall give him shall become in him a well of water springing up unto eternal life.''}}
\bv{15}{The woman saith unto him, ``Sir, give me this water, that I thirst not, neither come all the way hither to draw.''}
\par
\bv{16}{Jesus saith unto her, \redlet{``Go, call thy husband, and come hither.''}}
\bv{17}{The woman answered and said unto him, ``I have no husband.'' Jesus saith unto her, \redlet{``Thou saidst well, I have no husband:}}
\bv{18}{\redlet{for thou hast had five husbands; and he whom thou now hast is not thy husband: this hast thou said truly.''}}
\par
\bv{19}{The woman saith unto him, ``Sir, I perceive that thou art a prophet.}
\bv{20}{Our fathers worshipped in this mountain; and ye say, that in Jerusalem is the place where men ought to worship.''}
\bv{21}{Jesus saith unto her, \redlet{``Woman, believe me, the hour cometh, when neither in this mountain, nor in Jerusalem, shall ye worship the Father.}}
\bv{22}{\redlet{Ye worship that which ye know not: we worship that which we know; for salvation is from the Jews.}}
\bv{23}{\redlet{But the hour cometh, and now is, when the true worshippers shall worship the Father in spirit and truth: for such doth the Father seek to be his worshippers.}}
\bv{24}{\redlet{God is a Spirit: and they that worship him must worship in spirit and truth.''}}
\bv{25}{The woman saith unto him, ``I know that Messiah cometh (he that is called Christ): when he is come, he will declare unto us all things.''}
\bv{26}{Jesus saith unto her, \redlet{``I that speak unto thee am \supptext{he}.''}}
\par
\bv{27}{And upon this came his disciples; and they marvelled that he was speaking with a woman; yet no man said, ``What seekest thou?'' or, ``Why speakest thou with her?''}
\par
\bv{28}{So the woman left her waterpot, and went away into the city, and saith to the people,}
\bv{29}{``Come, see a man, who told me all things that \supptext{ever} I did: can this be the Christ?''}
\bv{30}{They went out of the city, and were coming to him.}
\par
\bv{31}{In the mean while the disciples prayed him, saying, ``Rabbi, eat.''}
\bv{32}{But he said unto them, \redlet{``I have meat to eat that ye know not.''}}
\bv{33}{The disciples therefore said one to another, ``Hath any man brought him \supptext{aught} to eat?''}
\bv{34}{Jesus saith unto them, \redlet{``My meat is to do the will of him that sent me, and to accomplish his work.}}
\bv{35}{\redlet{Say not ye, `There are yet four months, and \supptext{then} cometh the harvest?' behold, I say unto you, Lift up your eyes, and look on the fields, that they are white already unto harvest.}}
\bv{36}{\redlet{He that reapeth receiveth wages, and gathereth fruit unto life eternal; that he that soweth and he that reapeth may rejoice together.}}
\bv{37}{\redlet{For herein is the saying true, `One soweth, and another reapeth.'}}
\bv{38}{\redlet{I sent you to reap that whereon ye have not labored: others have labored, and ye are entered into their labor.''}}
\par
\bv{39}{And from that city many of the Samaritans believed on him because of the word of the woman, who testified, ``He told me all things that \supptext{ever} I did.''}
\chapsec{Jesus \& the Samaritans}
\bv{40}{So when the Samaritans came unto him, they besought him to abide with them: and he abode there two days.}
\bv{41}{And many more believed because of his word;}
\bv{42}{and they said to the woman, ``Now we believe, not because of thy speaking: for we have heard for ourselves, and know that this is indeed the Saviour of the world.''}
\bv{43}{And after the two days he went forth from thence into Galilee.}
\bv{44}{For Jesus himself testified, that a prophet hath no honor in his own country.}
\bv{45}{So when he came into Galilee, the Galilæans received him, having seen all the things that he did in Jerusalem at the feast: for they also went unto the feast.}
\chapsec{The Nobleman's Son Healed}
\bv{46}{He came therefore again unto Cana of Galilee, where he made the water wine. And there was a certain nobleman, whose son was sick at Capernaum.}
\bv{47}{When he heard that Jesus was come out of Judæa into Galilee, he went unto him, and besought \supptext{him} that he would come down, and heal his son; for he was at the point of death.}
\bv{48}{Jesus therefore said unto him, \redlet{``Except ye see signs and wonders, ye will in no wise believe.''}}
\bv{49}{The nobleman saith unto him, ``Sir, come down ere my child die.''}
\bv{50}{Jesus saith unto him, \redlet{``Go thy way; thy son liveth.''} The man believed the word that Jesus spake unto him, and he went his way.}
\bv{51}{And as he was now going down, his servants met him, saying, that his son lived.}
\bv{52}{So he inquired of them the hour when he began to amend. They said therefore unto him, ``Yesterday at the seventh hour the fever left him.''}
\bv{53}{So the father knew that \supptext{it was} at that hour in which Jesus said unto him, ``Thy son liveth:'' and himself believed, and his whole house.}
\bv{54}{This is again the second sign that Jesus did, having come out of Judæa into Galilee.}
\chaphead{Chapter V}
\chapdesc{The Pool of Bethesda \& Healing}
\lettrine[image=true, lines=4, findent=3pt, nindent=0pt]{NT/John/Jn-After.eps}{fter} these things there was a feast of the Jews; and Jesus went up to Jerusalem.
\bv{2}{Now there is in Jerusalem by the sheep \supptext{gate} a pool, which is called in Hebrew Bethesda, having five porches.}
\bv{3}{In these lay a multitude of them that were sick, blind, halt, withered.\mcomm{waiting for the moving of the water: for an angel of the Lord went down at certain seasons into the pool, and troubled the water: whosoever then first after the troubling of the water stepped in was made whole, with whatsoever disease he was holden.}}
\bv{5}{And a certain man was there, who had been thirty and eight years in his infirmity.}
\bv{6}{When Jesus saw him lying, and knew that he had been now a long time \supptext{in that case}, he saith unto him, \redlet{``Wouldest thou be made whole?''}}
\bv{7}{The sick man answered him, ``Sir, I have no man, when the water is troubled, to put me into the pool: but while I am coming, another steppeth down before me.''}
\bv{8}{Jesus saith unto him, \redlet{``Arise, take up thy bed, and walk.''}}
\bv{9}{And straightway the man was made whole, and took up his bed and walked. Now it was the sabbath on that day.}
\bv{10}{So the Jews said unto him that was cured, ``It is the sabbath, and it is not lawful for thee to take up thy bed.''}
\bv{11}{But he answered them, ``He that made me whole, the same said unto me, `Take up thy bed, and walk.' ''}
\bv{12}{They asked him, ``Who is the man that said unto thee, `Take up \supptext{thy bed}, and walk?' ''}
\bv{13}{But he that was healed knew not who it was; for Jesus had conveyed himself away, a multitude being in the place.}
\bv{14}{Afterward Jesus findeth him in the temple, and said unto him, \redlet{``Behold, thou art made whole: sin no more, lest a worse thing befall thee.''}}
\bv{15}{The man went away, and told the Jews that it was Jesus who had made him whole.}
\bv{16}{And for this cause the Jews persecuted Jesus, because he did these things on the sabbath.}
\par
\bv{17}{But Jesus answered them, \redlet{``My Father worketh even until now, and I work.''}}
\bv{18}{For this cause therefore the Jews sought the more to kill him, because he not only brake the sabbath, but also called God his own Father, making himself equal with God.}
\bv{19}{Jesus therefore answered and said unto them, \redlet{``Verily, verily, I say unto you, The Son can do nothing of himself, but what he seeth the Father doing: for what things soever he doeth, these the Son also doeth in like manner.}}
\bv{20}{\redlet{For the Father loveth the Son, and showeth him all things that himself doeth: and greater works than these will he show him, that ye may marvel.}}
\bv{21}{\redlet{For as the Father raiseth the dead and giveth them life, even so the Son also giveth life to whom he will.}}
\bv{22}{\redlet{For neither doth the Father judge any man, but he hath given all judgement unto the Son;}}
\bv{23}{\redlet{that all may honor the Son, even as they honor the Father. He that honoreth not the Son honoreth not the Father that sent him.}}
\bv{24}{\redlet{Verily, verily, I say unto you, He that heareth my word, and believeth him that sent me, hath eternal life, and cometh not into judgement, but hath passed out of death into life.}}
\par
\bv{25}{\redlet{Verily, verily, I say unto you, The hour cometh, and now is, when the dead shall hear the voice of the Son of God; and they that hear shall live.}}
\bv{26}{\redlet{For as the Father hath life in himself, even so gave he to the Son also to have life in himself:}}
\bv{27}{\redlet{and he gave him authority to execute judgement, because he is a son of man.}}
\chapsec{The Two Resurrections}
\bv{28}{\redlet{Marvel not at this: for the hour cometh, in which all that are in the tombs shall hear his voice,}}
\bv{29}{\redlet{and shall come forth; they that have done good, unto the resurrection of life; and they that have done evil, unto the resurrection of judgement.}}
\bv{30}{\redlet{I can of myself do nothing: as I hear, I judge: and my judgement is righteous; because I seek not mine own will, but the will of him that sent me.}}
\bv{31}{\redlet{If I bear witness of myself, my witness is not true.}}
\bv{32}{\redlet{It is another that beareth witness of me; and I know that the witness which he witnesseth of me is true.}}
\chapsec{The Witness of St. John the Baptist}
\bv{33}{\redlet{Ye have sent unto John, and he hath borne witness unto the truth.}}
\bv{34}{\redlet{But the witness which I receive is not from man: howbeit I say these things, that ye may be saved.}}
\bv{35}{\redlet{He was the lamp that burneth and shineth; and ye were willing to rejoice for a season in his light.}}
\chapsec{The Witness of Works}
\bv{36}{\redlet{But the witness which I have is greater than \supptext{that of} John; for the works which the Father hath given me to accomplish, the very works that I do, bear witness of me, that the Father hath sent me.}}
\chapsec{The Witness of the Father}
\bv{37}{\redlet{And the Father that sent me, he hath borne witness of me. Ye have neither heard his voice at any time, nor seen his form.}}
\bv{38}{\redlet{And ye have not his word abiding in you: for whom he sent, him ye believe not.}}
\chapsec{The Witness of the Scriptures}
\bv{39}{\redlet{Ye search the scriptures, because ye think that in them ye have eternal life; and these are they which bear witness of me;}}
\bv{40}{\redlet{and ye will not come to me, that ye may have life.}}
\bv{41}{\redlet{I receive not glory from men.}}
\bv{42}{\redlet{But I know you, that ye have not the love of God in yourselves.}}
\bv{43}{\redlet{I am come in my Father's name, and ye receive me not: if another shall come in his own name, him ye will receive.}}
\bv{44}{\redlet{How can ye believe, who receive glory one of another, and the glory that \supptext{cometh} from the only God ye seek not?}}
\bv{45}{\redlet{Think not that I will accuse you to the Father: there is one that accuseth you, \supptext{even} Moses, on whom ye have set your hope.}}
\bv{46}{\redlet{For if ye believed Moses, ye would believe me; for he wrote of me.}}
\bv{47}{\redlet{But if ye believe not his writings, how shall ye believe my words?''}}
\chaphead{Chapter VI}
\chapdesc{Feeding the Five Thousand}
\lettrine[image=true, lines=4, findent=3pt, nindent=0pt]{NT/John/Jn-After.eps}{fter} these things Jesus went away to the other side of the sea of Galilee, which is \supptext{the sea} of Tiberias.
\bv{2}{And a great multitude followed him, because they beheld the signs which he did on them that were sick.}
\bv{3}{And Jesus went up into the mountain, and there he sat with his disciples.}
\bv{4}{Now the passover, the feast of the Jews, was at hand.}
\bv{5}{Jesus therefore lifting up his eyes, and seeing that a great multitude cometh unto him, saith unto Philip, \redlet{``Whence are we to buy bread, that these may eat?''}}
\bv{6}{And this he said to prove him: for he himself knew what he would do.}
\bv{7}{Philip answered him, ``Two hundred denarii's\mcomm{One denarius was given for a day's wages.} worth of bread is not sufficient for them, that every one may take a little.''}
\bv{8}{One of his disciples, Andrew, Simon Peter's brother, saith unto him,}
\bv{9}{``There is a lad here, who hath five barley loaves, and two fishes: but what are these among so many?''}
\bv{10}{Jesus said, \redlet{``Make the people sit down.''} Now there was much grass in the place. So the men sat down, in number about five thousand.}
\bv{11}{Jesus therefore took the loaves; and having given thanks, he distributed to them that were set down; likewise also of the fishes as much as they would.}
\bv{12}{And when they were filled, he saith unto his disciples, \redlet{``Gather up the broken pieces which remain over, that nothing be lost.''}}
\bv{13}{So they gathered them up, and filled twelve baskets with broken pieces from the five barley loaves, which remained over unto them that had eaten.}
\bv{14}{When therefore the people saw the sign which he did, they said, ``This is of a truth the prophet that cometh into the world.''}
\chapsec{Jesus Walks upon the Sea}
\bv{15}{Jesus therefore perceiving that they were about to come and take him by force, to make him king, withdrew again into the mountain himself alone.}
\bv{16}{And when evening came, his disciples went down unto the sea;}
\bv{17}{and they entered into a boat, and were going over the sea unto Capernaum. And it was now dark, and Jesus had not yet come to them.}
\bv{18}{And the sea was rising by reason of a great wind that blew.}
\bv{19}{When therefore they had rowed about five and twenty or thirty furlongs, they behold Jesus walking on the sea, and drawing nigh unto the boat: and they were afraid.}
\bv{20}{But he saith unto them, \redlet{``It is I; be not afraid.''}}
\bv{21}{They were willing therefore to receive him into the boat: and straightway the boat was at the land whither they were going.}
\chapsec{The Bread of Life Discourse}
\bv{22}{On the morrow the multitude that stood on the other side of the sea saw that there was no other boat there, save one, and that Jesus entered not with his disciples into the boat, but \supptext{that} his disciples went away alone}
\bv{23}{(howbeit there came boats from Tiberias nigh unto the place where they ate the bread after the Lord had given thanks):}
\bv{24}{when the multitude therefore saw that Jesus was not there, neither his disciples, they themselves got into the boats, and came to Capernaum, seeking Jesus.}
\bv{25}{And when they found him on the other side of the sea, they said unto him, ``Rabbi, when camest thou hither?''}
\bv{26}{Jesus answered them and said, \redlet{``Verily, verily, I say unto you, Ye seek me, not because ye saw signs, but because ye ate of the loaves, and were filled.}}
\bv{27}{\redlet{Work not for the food which perisheth, but for the food which abideth unto eternal life, which the Son of man shall give unto you: for him the Father, \supptext{even} God, hath sealed.}}
\bv{28}{They said therefore unto him, ``What must we do, that we may work the works of God?''}
\bv{29}{Jesus answered and said unto them, \redlet{``This is the work of God, that ye believe on him whom he hath sent.''}}
\bv{30}{They said therefore unto him, ``What then doest thou for a sign, that we may see, and believe thee? what workest thou?}
\par
\bv{31}{Our fathers ate the manna in the wilderness; as it is written, `He gave them bread out of heaven to eat.'\mref{Neh. 9:15}''}
\bv{32}{Jesus therefore said unto them, \redlet{``Verily, verily, I say unto you, It was not Moses that gave you the bread out of heaven; but my Father giveth you the true bread out of heaven.}}
\bv{33}{\redlet{For the bread of God is that which cometh down out of heaven, and giveth life unto the world.''}}
\bv{34}{They said therefore unto him, ``Lord, evermore give us this bread.''}
\bv{35}{Jesus said unto them, \redlet{``I am the bread of life: he that cometh to me shall not hunger, and he that believeth on me shall never thirst.}}
\bv{36}{\redlet{But I said unto you, that ye have seen me, and yet believe not.}}
\bv{37}{\redlet{All that which the Father giveth me shall come unto me; and him that cometh to me I will in no wise cast out.}}
\bv{38}{\redlet{For I am come down from heaven, not to do mine own will, but the will of him that sent me.}}
\bv{39}{\redlet{And this is the will of him that sent me, that of all that which he hath given me I should lose nothing, but should raise it up at the last day.}}
\bv{40}{\redlet{For this is the will of my Father, that every one that beholdeth the Son, and believeth on him, should have eternal life; and I will raise him up at the last day.''}}
\par
\bv{41}{The Jews therefore murmured concerning him, because he said, ``I am the bread which came down out of heaven.''}
\bv{42}{And they said, ``Is not this Jesus, the son of Joseph, whose father and mother we know? how doth he now say, I am come down out of heaven?''}
\bv{43}{Jesus answered and said unto them, \redlet{``Murmur not among yourselves.}}
\bv{44}{\redlet{No man can come to me, except the Father that sent me draw him: and I will raise him up in the last day.}}
\bv{45}{\redlet{It is written in the prophets, \otQuote{Is. 54:13}{And they shall all be taught of God.} Every one that hath heard from the Father, and hath learned, cometh unto me.}}
\bv{46}{\redlet{Not that any man hath seen the Father, save he that is from God, he hath seen the Father.}}
\bv{47}{\redlet{Verily, verily, I say unto you, He that believeth hath eternal life.}}
\bv{48}{\redlet{I am the bread of life.}}
\bv{49}{\redlet{Your fathers ate the manna in the wilderness, and they died.}}
\bv{50}{\redlet{This is the bread which cometh down out of heaven, that a man may eat thereof, and not die.}}
\bv{51}{\redlet{I am the living bread which came down out of heaven: if any man eat of this bread, he shall live for ever: yea and the bread which I will give is my flesh, for the life of the world.''}}
\par
\bv{52}{The Jews therefore strove one with another, saying, ``How can this man give us his flesh to eat?''}
\bv{53}{Jesus therefore said unto them, \redlet{``Verily, verily, I say unto you, Except ye eat the flesh of the Son of man and drink his blood, ye have not life in yourselves.}}
\bv{54}{\redlet{He that feedeth on my flesh and drinketh my blood hath eternal life; and I will raise him up at the last day.}}
\bv{55}{\redlet{For my flesh is meat indeed, and my blood is drink indeed.}}
\bv{56}{\redlet{He that feedeth on my flesh and drinketh my blood abideth in me, and I in him.}}
\bv{57}{\redlet{As the living Father sent me, and I live because of the Father; so he that feedeth on me, he also shall live because of me.}}
\bv{58}{\redlet{This is the bread which came down out of heaven: not as the fathers ate, and died; he that feedeth on this bread shall live for ever.''}}
\bv{59}{These things said he in the synagogue, as he taught in Capernaum.}
\chapsec{Discipleship Tested by Doctrine}
\bv{60}{Many therefore of his disciples, when they heard \supptext{this}, said, ``This is a hard saying; who can hear it?''}
\bv{61}{But Jesus knowing in himself that his disciples murmured at this, said unto them, \redlet{``Doth this cause you to stumble?}}
\bv{62}{\redlet{\supptext{What} then if ye should behold the Son of man ascending where he was before?}}
\bv{63}{\redlet{It is the spirit that giveth life; the flesh profiteth nothing: the words that I have spoken unto you are spirit, and are life.}}
\bv{64}{\redlet{But there are some of you that believe not.''} For Jesus knew from the beginning who they were that believed not, and who it was that should betray him.}
\bv{65}{And he said, \redlet{``For this cause have I said unto you, that no man can come unto me, except it be given unto him of the Father.''}}
\bv{66}{Upon this many of his disciples went back, and walked no more with him.}
\chapsec{Peter's Confession of Faith}
\bv{67}{Jesus said therefore unto the twelve, \redlet{``Would ye also go away?''}}
\bv{68}{Simon Peter answered him, ``Lord, to whom shall we go? thou hast the words of eternal life.}
\bv{69}{And we have believed and know that thou art the Holy One of God.''}
\par
\bv{70}{Jesus answered them, \redlet{``Did not I choose you the twelve, and one of you is a devil?''}}
\bv{71}{Now he spake of Judas \supptext{the son} of Simon Iscariot, for he it was that should betray him, \supptext{being} one of the twelve.}
\chaphead{Chapter VII}
\chapdesc{Jesus Urged to Go to Judæa}
\lettrine[image=true, lines=4, findent=3pt, nindent=0pt]{NT/John/Jn-And.eps}{nd} after these things Jesus walked in Galilee: for he would not walk in Judæa, because the Jews sought to kill him.
\bv{2}{Now the feast of the Jews, the feast of tabernacles, was at hand.}
\bv{3}{His brethren therefore said unto him, ``Depart hence, and go into Judæa, that thy disciples also may behold thy works which thou doest.}
\bv{4}{For no man doeth anything in secret, and himself seeketh to be known openly. If thou doest these things, manifest thyself to the world.''}
\bv{5}{For even his brethren did not believe on him.}
\bv{6}{Jesus therefore saith unto them, \redlet{``My time is not yet come; but your time is always ready.}}
\bv{7}{\redlet{The world cannot hate you; but me it hateth, because I testify of it, that its works are evil.}}
\bv{8}{\redlet{Go ye up unto the feast: I go not up unto this feast; because my time is not yet fulfilled.''}}
\bv{9}{And having said these things unto them, he abode \supptext{still} in Galilee.}
\chapsec{Final Departure from Galilee}
\bv{10}{But when his brethren were gone up unto the feast, then went he also up, not publicly, but as it were in secret.}
\bv{11}{The Jews therefore sought him at the feast, and said, ``Where is he?''}
\bv{12}{And there was much murmuring among the multitudes concerning him: some said, ``He is a good man;'' others said, ``Not so, but he leadeth the multitude astray.''}
\bv{13}{Yet no man spake openly of him for fear of the Jews.}
\chapsec{Jesus at the Feast of Tabernacles}
\bv{14}{But when it was now the midst of the feast Jesus went up into the temple, and taught.}
\bv{15}{The Jews therefore marvelled, saying, ``How knoweth this man letters, having never learned?''}
\bv{16}{Jesus therefore answered them, and said, \redlet{``My teaching is not mine, but his that sent me.}}
\bv{17}{\redlet{If any man willeth to do his will, he shall know of the teaching, whether it is of God, or \supptext{whether} I speak from myself.}}
\bv{18}{\redlet{He that speaketh from himself seeketh his own glory: but he that seeketh the glory of him that sent him, the same is true, and no unrighteousness is in him.}}
\bv{19}{\redlet{Did not Moses give you the law, and \supptext{yet} none of you doeth the law? Why seek ye to kill me?''}}
\bv{20}{The multitude answered, ``Thou hast a demon: who seeketh to kill thee?''}
\bv{21}{Jesus answered and said unto them, \redlet{``I did one work, and ye all marvel because thereof.}}
\bv{22}{\redlet{Moses hath given you circumcision (not that it is of Moses, but of the fathers); and on the sabbath ye circumcise a man.}}
\bv{23}{\redlet{If a man receiveth circumcision on the sabbath, that the law of Moses may not be broken; are ye wroth with me, because I made a man every whit whole on the sabbath?}}
\bv{24}{\redlet{Judge not according to appearance, but judge righteous judgement.''}}
\par
\bv{25}{Some therefore of them of Jerusalem said, ``Is not this he whom they seek to kill?}
\bv{26}{And lo, he speaketh openly, and they say nothing unto him. Can it be that the rulers indeed know that this is the Christ?}
\bv{27}{Howbeit we know this man whence he is: but when the Christ cometh, no one knoweth whence he is.''}
\bv{28}{Jesus therefore cried in the temple, teaching and saying, \redlet{``Ye both know me, and know whence I am; and I am not come of myself, but he that sent me is true, whom ye know not.}}
\bv{29}{\redlet{I know him; because I am from him, and he sent me.''}}
\bv{30}{They sought therefore to take him: and no man laid his hand on him, because his hour was not yet come.}
\bv{31}{But of the multitude many believed on him; and they said, ``When the Christ shall come, will he do more signs than those which this man hath done?''}
\bv{32}{The Pharisees heard the multitude murmuring these things concerning him; and the chief priests and the Pharisees sent officers to take him.}
\bv{33}{Jesus therefore said, \redlet{``Yet a little while am I with you, and I go unto him that sent me.}}
\bv{34}{\redlet{Ye shall seek me, and shall not find me: and where I am, ye cannot come.''}}
\bv{35}{The Jews therefore said among themselves, ``Whither will this man go that we shall not find him? will he go unto the Dispersion among the Greeks, and teach the Greeks?}
\bv{36}{What is this word that he said, `Ye shall seek me, and shall not find me; and where I am, ye cannot come?' ''}
\chapsec{The Great Prophecy of the Holy Ghost}
\bv{37}{Now on the last day, the great \supptext{day} of the feast, Jesus stood and cried, saying, \redlet{``If any man thirst, let him come unto me and drink.}}
\bv{38}{\redlet{He that believeth on me, as the scripture hath said, \otQuote{Prov. 18:4}{`from within him shall flow rivers of living water.' ''}}}
\bv{39}{But this spake he of the Spirit, which they that believed on him were to receive: for the Spirit was not yet \supptext{given}; because Jesus was not yet glorified.}
\chapsec{The People Divided in Opinion}
\bv{40}{\supptext{Some} of the multitude therefore, when they heard these words, said, ``This is of a truth the prophet.''}
\bv{41}{Others said, ``This is the Christ.'' But some said, ``What, doth the Christ come out of Galilee?}
\bv{42}{Hath not the scripture said that the Christ cometh of the seed of David, and from Bethlehem, the village where David was?''}
\bv{43}{So there arose a division in the multitude because of him.}
\bv{44}{And some of them would have taken him; but no man laid hands on him.}
\bv{45}{The officers therefore came to the chief priests and Pharisees; and they said unto them, ``Why did ye not bring him?''}
\bv{46}{The officers answered, ``Never man so spake.''}
\bv{47}{The Pharisees therefore answered them, ``Are ye also led astray?}
\bv{48}{Hath any of the rulers believed on him, or of the Pharisees?}
\bv{49}{But this multitude that knoweth not the law are accursed.''}
\bv{50}{Nicodemus saith unto them (he that came to him before, being one of them),}
\bv{51}{``Doth our law judge a man, except it first hear from himself and know what he doeth?''}
\bv{52}{They answered and said unto him, ``Art thou also of Galilee? Search, and see that out of Galilee ariseth no prophet.''}
\chaphead{Chapter VIII}
\chapdesc{Jesus is the Light of the World}
\lettrine[image=true, lines=4, findent=3pt, nindent=0pt]{NT/John/Jn-Again.eps}{gain} therefore Jesus spake unto them, saying, \redlet{``I am the light of the world: he that followeth me shall not walk in the darkness, but shall have the light of life.''}
\bv{13}{The Pharisees therefore said unto him, ``Thou bearest witness of thyself; thy witness is not true.''}
\bv{14}{Jesus answered and said unto them, \redlet{``Even if I bear witness of myself, my witness is true; for I know whence I came, and whither I go; but ye know not whence I come, or whither I go.}}
\bv{15}{\redlet{Ye judge after the flesh; I judge no man.}}
\bv{16}{\redlet{Yea and if I judge, my judgement is true; for I am not alone, but I and the Father that sent me.}}
\bv{17}{\redlet{Yea and in your law it is written, that the witness of two men is true.}}
\bv{18}{\redlet{I am he that beareth witness of myself, and the Father that sent me beareth witness of me.''}}
\bv{19}{They said therefore unto him, ``Where is thy Father?'' Jesus answered, \redlet{``Ye know neither me, nor my Father: if ye knew me, ye would know my Father also.''}}
\bv{20}{These words spake he in the treasury, as he taught in the temple: and no man took him; because his hour was not yet come.}
\par
\bv{21}{He said therefore again unto them, \redlet{``I go away, and ye shall seek me, and shall die in your sin: whither I go, ye cannot come.''}}
\bv{22}{The Jews therefore said, ``Will he kill himself, that he saith, `Whither I go, ye cannot come?' ''}
\par
\bv{23}{And he said unto them, \redlet{``Ye are from beneath; I am from above: ye are of this world; I am not of this world.}}
\bv{24}{\redlet{I said therefore unto you, that ye shall die in your sins: for except ye believe that I am \supptext{he}, ye shall die in your sins.}}
\bv{25}{They said therefore unto him, ``Who art thou?'' Jesus said unto them, \redlet{``Even that which I have also spoken unto you from the beginning.}}
\bv{26}{\redlet{I have many things to speak and to judge concerning you: howbeit he that sent me is true; and the things which I heard from him, these speak I unto the world.''}}
\bv{27}{They perceived not that he spake to them of the Father.}
\bv{28}{Jesus therefore said, \redlet{``When ye have lifted up the Son of man, then shall ye know that I am \supptext{he}, and \supptext{that} I do nothing of myself, but as the Father taught me, I speak these things.}}
\bv{29}{\redlet{And he that sent me is with me; he hath not left me alone; for I do always the things that are pleasing to him.''}}
\bv{30}{As he spake these things, many believed on him.}
\chapsec{The True Seed of Abraham}
\bv{31}{Jesus therefore said to those Jews that had believed him, \redlet{``If ye abide in my word, \supptext{then} are ye truly my disciples;}}
\bv{32}{\redlet{and ye shall know the truth, and the truth shall make you free.''}}
\bv{33}{They answered unto him, ``We are Abraham's seed, and have never yet been in bondage to any man: how sayest thou, `Ye shall be made free?' ''}
\bv{34}{Jesus answered them, \redlet{``Verily, verily, I say unto you, Every one that committeth sin is the bondservant of sin.}}
\bv{35}{\redlet{And the bondservant abideth not in the house for ever: the son abideth for ever.}}
\bv{36}{\redlet{If therefore the Son shall make you free, ye shall be free indeed.}}
\bv{37}{\redlet{I know that ye are Abraham's seed; yet ye seek to kill me, because my word hath not free course in you.}}
\bv{38}{\redlet{I speak the things which I have seen with \supptext{my} Father: and ye also do the things which ye heard from \supptext{your} father.''}\mref{cf. Wis. 2:13-24}}
\bv{39}{They answered and said unto him, ``Our father is Abraham.'' Jesus saith unto them, \redlet{``If ye were Abraham's children, ye would do the works of Abraham.}}
\bv{40}{\redlet{But now ye seek to kill me, a man that hath told you the truth, which I heard from God: this did not Abraham.}}
\bv{41}{\redlet{Ye do the works of your father.''} They said unto him, ``We were not born of fornication; we have one Father, \supptext{even} God.''}
\bv{42}{Jesus said unto them, \redlet{``If God were your Father, ye would love me: for I came forth and am come from God; for neither have I come of myself, but he sent me.}}
\bv{43}{\redlet{Why do ye not understand my speech? \supptext{Even} because ye cannot hear my word.}}
\bv{44}{\redlet{Ye are of \supptext{your} father the devil, and the lusts of your father it is your will to do. He was a murderer from the beginning, and standeth not in the truth, because there is no truth in him. When he speaketh a lie, he speaketh of his own: for he is a liar, and the father thereof.\mref{cf. Wis. 2:24}}}
\bv{45}{\redlet{But because I say the truth, ye believe me not.}}
\bv{46}{\redlet{Which of you convicteth me of sin? If I say truth, why do ye not believe me?}}
\bv{47}{\redlet{He that is of God heareth the words of God: for this cause ye hear \supptext{them} not, because ye are not of God.''}}
\bv{48}{The Jews answered and said unto him, ``Say we not well that thou art a Samaritan, and hast a demon?''}
\bv{49}{Jesus answered, \redlet{``I have not a demon; but I honor my Father, and ye dishonor me.}}
\bv{50}{\redlet{But I seek not mine own glory: there is one that seeketh and judgeth.}}
\chapsec{The Pre-Existence of Jesus Christ}
\bv{51}{\redlet{Verily, verily, I say unto you, If a man keep my word, he shall never see death.''}}
\bv{52}{The Jews said unto him, ``Now we know that thou hast a demon. Abraham died, and the prophets; and thou sayest, `If a man keep my word, he shall never taste of death.'}
\bv{53}{Art thou greater than our father Abraham, who died? and the prophets died: whom makest thou thyself?''}
\bv{54}{Jesus answered, \redlet{``If I glorify myself, my glory is nothing: it is my Father that glorifieth me; of whom ye say, that he is your God;}}
\bv{55}{\redlet{and ye have not known him: but I know him; and if I should say, I know him not, I shall be like unto you, a liar: but I know him, and keep his word.}}
\bv{56}{\redlet{Your father Abraham rejoiced to see my day; and he saw it, and was glad.''}}
\bv{57}{The Jews therefore said unto him, ``Thou art not yet fifty years old, and hast thou seen Abraham?''}
\bv{58}{Jesus said unto them, \redlet{``Verily, verily, I say unto you, Before Abraham was, {\scshape I am}.''}}
\bv{59}{They took up stones therefore to cast at him: but Jesus hid himself, and went out of the temple.}
\chaphead{Chapter IX}
\chapdesc{The Man Born Blind is Healed}
\lettrine[image=true, lines=4, findent=3pt, nindent=0pt]{NT/John/Jn-And.eps}{nd} as he passed by, he saw a man blind from his birth.
\bv{2}{And his disciples asked him, saying, ``Rabbi, who sinned, this man, or his parents, that he should be born blind?''}
\bv{3}{Jesus answered, \redlet{``Neither did this man sin, nor his parents: but that the works of God should be made manifest in him.}}
\bv{4}{\redlet{We must work the works of him that sent me, while it is day: the night cometh, when no man can work.}}
\bv{5}{\redlet{When I am in the world, I am the light of the world.''}}
\bv{6}{When he had thus spoken, he spat on the ground, and made clay of the spittle, and anointed his eyes with the clay,}
\bv{7}{and said unto him, \redlet{``Go, wash in the pool of Siloam''} (which is by interpretation, ``Sent''). He went away therefore, and washed, and came seeing.}
\bv{8}{The neighbors therefore, and they that saw him aforetime, that he was a beggar, said, ``Is not this he that sat and begged?''}
\bv{9}{Others said, ``It is he:'' others said, ``No, but he is like him.'' He said, ``I am \supptext{he}.''}
\par
\bv{10}{They said therefore unto him, ``How then were thine eyes opened?''}
\bv{11}{He answered, ``The man that is called Jesus made clay, and anointed mine eyes, and said unto me, `Go to Siloam, and wash:' so I went away and washed, and I received sight.''}
\bv{12}{And they said unto him, ``Where is he?'' He saith, ``I know not.''}
\bv{13}{They bring to the Pharisees him that aforetime was blind.}
\bv{14}{Now it was the sabbath on the day when Jesus made the clay, and opened his eyes.}
\par
\bv{15}{Again therefore the Pharisees also asked him how he received his sight. And he said unto them, ``He put clay upon mine eyes, and I washed, and I see.''}
\bv{16}{Some therefore of the Pharisees said, ``This man is not from God, because he keepeth not the sabbath.'' But others said, ``How can a man that is a sinner do such signs?'' And there was a division among them.}
\bv{17}{They say therefore unto the blind man again, ``What sayest thou of him, in that he opened thine eyes?'' And he said, ``He is a prophet.''}
\bv{18}{The Jews therefore did not believe concerning him, that he had been blind, and had received his sight, until they called the parents of him that had received his sight,}
\bv{19}{and asked them, saying, ``Is this your son, who ye say was born blind? how then doth he now see?''}
\bv{20}{His parents answered and said, ``We know that this is our son, and that he was born blind:}
\bv{21}{but how he now seeth, we know not; or who opened his eyes, we know not: ask him; he is of age; he shall speak for himself.''}
\bv{22}{These things said his parents, because they feared the Jews: for the Jews had agreed already, that if any man should confess him \supptext{to be} Christ, he should be put out of the synagogue.}
\bv{23}{Therefore said his parents, ``He is of age; ask him.''}
\par
\bv{24}{So they called a second time the man that was blind, and said unto him, ``Give glory to God: we know that this man is a sinner.''}
\bv{25}{He therefore answered, ``Whether he is a sinner, I know not: one thing I know, that, whereas I was blind, now I see.''}
\bv{26}{They said therefore unto him, ``What did he to thee? how opened he thine eyes?''}
\bv{27}{He answered them, ``I told you even now, and ye did not hear; wherefore would ye hear it again? would ye also become his disciples?''}
\bv{28}{And they reviled him, and said, ``Thou art his disciple; but we are disciples of Moses.}
\bv{29}{We know that God hath spoken unto Moses: but as for this man, we know not whence he is.''}
\bv{30}{The man answered and said unto them, ``Why, herein is the marvel, that ye know not whence he is, and \supptext{yet} he opened mine eyes.}
\bv{31}{We know that God heareth not sinners: but if any man be a worshipper of God, and do his will, him he heareth.}
\bv{32}{Since the world began it was never heard that any one opened the eyes of a man born blind.}
\bv{33}{If this man were not from God, he could do nothing.''}
\bv{34}{They answered and said unto him, ``Thou wast altogether born in sins, and dost thou teach us?'' And they cast him out.}
\par
\bv{35}{Jesus heard that they had cast him out; and finding him, he said, \redlet{``Dost thou believe on the Son of God?''}}
\bv{36}{He answered and said, ``And who is he, Lord, that I may believe on him?''}
\bv{37}{Jesus said unto him, \redlet{``Thou hast both seen him, and he it is that speaketh with thee.''}}
\bv{38}{And he said, ``Lord, I believe.'' And he worshipped him.}
\bv{39}{And Jesus said, \redlet{``For judgement came I into this world, that they that see not may see; and that they that see may become blind.''}}
\bv{40}{Those of the Pharisees who were with him heard these things, and said unto him, ``Are we also blind?''}
\bv{41}{Jesus said unto them, \redlet{``If ye were blind, ye would have no sin: but now ye say, `We see:' your sin remaineth.}}
\chaphead{Chapter X}
\chapdesc{The Good Shepherd Discourse}
\lettrine[image=true, lines=4, findent=3pt, nindent=0pt]{NT/John/Jn10-V.eps}{\redlet{erily}}, \redlet{verily, I say unto you, He that entereth not by the door into the fold of the sheep, but climbeth up some other way, the same is a thief and a robber.}
\bv{2}{\redlet{But he that entereth in by the door is the shepherd of the sheep.}}
\bv{3}{\redlet{To him the porter openeth; and the sheep hear his voice: and he calleth his own sheep by name, and leadeth them out.}}
\bv{4}{\redlet{When he hath put forth all his own, he goeth before them, and the sheep follow him: for they know his voice.}}
\bv{5}{\redlet{And a stranger will they not follow, but will flee from him: for they know not the voice of strangers.''}}
\bv{6}{This parable spake Jesus unto them: but they understood not what things they were which he spake unto them.}
\par
\bv{7}{Jesus therefore said unto them again, \redlet{``Verily, verily, I say unto you, I am the door of the sheep.}}
\bv{8}{\redlet{All that came before me are thieves and robbers: but the sheep did not hear them.}}
\bv{9}{\redlet{I am the door; by me if any man enter in, he shall be saved, and shall go in and go out, and shall find pasture.}}
\bv{10}{\redlet{The thief cometh not, but that he may steal, and kill, and destroy: I came that they may have life, and may have \supptext{it} abundantly.}}
\bv{11}{\redlet{I am the good shepherd: the good shepherd layeth down his life for the sheep.}}
\bv{12}{\redlet{He that is a hireling, and not a shepherd, whose own the sheep are not, beholdeth the wolf coming, and leaveth the sheep, and fleeth, and the wolf snatcheth them, and scattereth \supptext{them}:}}
\bv{13}{\redlet{\supptext{he fleeth} because he is a hireling, and careth not for the sheep.}}
\bv{14}{\redlet{I am the good shepherd; and I know mine own, and mine own know me,}}
\bv{15}{\redlet{even as the Father knoweth me, and I know the Father; and I lay down my life for the sheep.}}
\bv{16}{\redlet{And other sheep I have, which are not of this fold: them also I must bring, and they shall hear my voice; and they shall become one flock, one shepherd.}}
\bv{17}{\redlet{Therefore doth the Father love me, because I lay down my life, that I may take it again.}}
\bv{18}{\redlet{No one taketh it away from me, but I lay it down of myself. I have power to lay it down, and I have power to take it again. This commandment received I from my Father.''}}
\par
\bv{19}{There arose a division again among the Jews because of these words.}
\bv{20}{And many of them said, ``He hath a demon, and is mad; why hear ye him?''}
\bv{21}{Others said, ``These are not the sayings of one possessed with a demon. Can a demon open the eyes of the blind?''}
\chapsec{Jesus Asserts His Deity}
\bv{22}{And it was the feast of the dedication at Jerusalem:}
\bv{23}{it was winter; and Jesus was walking in the temple in Solomon's porch.}
\bv{24}{The Jews therefore came round about him, and said unto him, ``How long dost thou hold us in suspense? If thou art the Christ, tell us plainly.''}
\bv{25}{Jesus answered them, \redlet{``I told you, and ye believe not: the works that I do in my Father's name, these bear witness of me.}}
\bv{26}{\redlet{But ye believe not, because ye are not of my sheep.}}
\bv{27}{\redlet{My sheep hear my voice, and I know them, and they follow me:}}
\bv{28}{\redlet{and I give unto them eternal life; and they shall never perish, and no one shall snatch them out of my hand.}}
\bv{29}{\redlet{My Father, who hath given \supptext{them} unto me, is greater than all; and no one is able to snatch \supptext{them} out of the Father's hand.}}
\bv{30}{\redlet{I and the Father are one.''}}
\par
\bv{31}{The Jews took up stones again to stone him.}
\bv{32}{Jesus answered them, \redlet{``Many good works have I showed you from the Father; for which of those works do ye stone me?''}}
\bv{33}{The Jews answered him, ``For a good work we stone thee not, but for blasphemy; and because that thou, being a man, makest thyself God.''}
\bv{34}{Jesus answered them, \redlet{``Is it not written in your law, \otQuote{Ps. 82:6}{`I said, Ye are gods?'}}}
\bv{35}{\redlet{If he called them gods, unto whom the word of God came (and the scripture cannot be broken),}}
\bv{36}{\redlet{say ye of him, whom the Father sanctified and sent into the world, Thou blasphemest; because I said, `I am \supptext{the} Son of God?'}}
\bv{37}{\redlet{If I do not the works of my Father, believe me not.}}
\bv{38}{\redlet{But if I do them, though ye believe not me, believe the works: that ye may know and understand that the Father is in me, and I in the Father.''}}
\bv{39}{They sought again to take him: and he went forth out of their hand.}
\chapsec{Jesus Goes to Where He Was Baptised}
\bv{40}{And he went away again beyond the Jordan into the place where John was at the first baptising; and there he abode.}
\bv{41}{And many came unto him; and they said, ``John indeed did no sign: but all things whatsoever John spake of this man were true.''}
\bv{42}{And many believed on him there.}
\chaphead{Chapter XI}
\chapdesc{The Raising of Lazarus}
\lettrine[image=true, lines=4, findent=3pt, nindent=0pt]{NT/John/Jn-Now.eps}{ow} a certain man was sick, Lazarus of Bethany, of the village of Mary and her sister Martha.
\bv{2}{And it was that Mary who anointed the Lord with ointment, and wiped his feet with her hair, whose brother Lazarus was sick.}
\bv{3}{The sisters therefore sent unto him, saying, ``Lord, behold, he whom thou lovest is sick.''}
\bv{4}{But when Jesus heard it, he said, \redlet{``This sickness is not unto death, but for the glory of God, that the Son of God may be glorified thereby.''}}
\bv{5}{Now Jesus loved Martha, and her sister, and Lazarus.}
\bv{6}{When therefore he heard that he was sick, he abode at that time two days in the place where he was.}
\par
\bv{7}{Then after this he saith to the disciples, \redlet{``Let us go into Judæa again.''}}
\bv{8}{The disciples say unto him, ``Rabbi, the Jews were but now seeking to stone thee; and goest thou thither again?''}
\bv{9}{Jesus answered, \redlet{``Are there not twelve hours in the day? If a man walk in the day, he stumbleth not, because he seeth the light of this world.}}
\bv{10}{\redlet{But if a man walk in the night, he stumbleth, because the light is not in him.''}}
\bv{11}{These things spake he: and after this he saith unto them, \redlet{``Our friend Lazarus is fallen asleep; but I go, that I may awake him out of sleep.''}}
\bv{12}{The disciples therefore said unto him, ``Lord, if he is fallen asleep, he will recover.''}
\bv{13}{Now Jesus had spoken of his death: but they thought that he spake of taking rest in sleep.}
\bv{14}{Then Jesus therefore said unto them plainly, \redlet{``Lazarus is dead.}}
\bv{15}{\redlet{And I am glad for your sakes that I was not there, to the intent ye may believe; nevertheless let us go unto him.''}}
\bv{16}{Thomas therefore, who is called Didymus, said unto his fellow-disciples, ``Let us also go, that we may die with him.''}
\par
\bv{17}{So when Jesus came, he found that he had been in the tomb four days already.}
\bv{18}{Now Bethany was nigh unto Jerusalem, about fifteen furlongs off;}
\bv{19}{and many of the Jews had come to Martha and Mary, to console them concerning their brother.}
\bv{20}{Martha therefore, when she heard that Jesus was coming, went and met him: but Mary still sat in the house.}
\bv{21}{Martha therefore said unto Jesus, ``Lord, if thou hadst been here, my brother had not died.}
\bv{22}{And even now I know that, whatsoever thou shalt ask of God, God will give thee.''}
\bv{23}{Jesus saith unto her, \redlet{``Thy brother shall rise again.''}}
\bv{24}{Martha saith unto him, ``I know that he shall rise again in the resurrection at the last day.''}
\bv{25}{Jesus said unto her, \redlet{``I am the resurrection, and the life: he that believeth on me, though he die, yet shall he live;}}
\bv{26}{\redlet{and whosoever liveth and believeth on me shall never die. Believest thou this?''}}
\bv{27}{She saith unto him, ``Yea, Lord: I have believed that thou art the Christ, the Son of God, \supptext{even} he that cometh into the world.''}
\par
\bv{28}{And when she had said this, she went away, and called Mary her sister secretly, saying, ``The Teacher is here, and calleth thee.''}
\bv{29}{And she, when she heard it, arose quickly, and went unto him.}
\bv{30}{(Now Jesus was not yet come into the village, but was still in the place where Martha met him.)}
\bv{31}{The Jews then who were with her in the house, and were consoling her, when they saw Mary, that she rose up quickly and went out, followed her, supposing that she was going unto the tomb to weep there.}
\bv{32}{Mary therefore, when she came where Jesus was, and saw him, fell down at his feet, saying unto him, ``Lord, if thou hadst been here, my brother had not died.''}
\bv{33}{When Jesus therefore saw her weeping, and the Jews \supptext{also} weeping who came with her, he groaned in the spirit, and was troubled,}
\bv{34}{and said, \redlet{``Where have ye laid him?''} They say unto him, ``Lord, come and see.''}
\bv{35}{Jesus wept.}
\par
\bv{36}{The Jews therefore said, ``Behold how he loved him!''}
\bv{37}{But some of them said, ``Could not this man, who opened the eyes of him that was blind, have caused that this man also should not die?''}
\bv{38}{Jesus therefore again groaning in himself cometh to the tomb. Now it was a cave, and a stone lay against it.}
\bv{39}{Jesus saith, \redlet{``Take ye away the stone.''} Martha, the sister of him that was dead, saith unto him, ``Lord, by this time the body decayeth; for he hath been \supptext{dead} four days.''}
\bv{40}{Jesus saith unto her, \redlet{``Said I not unto thee, that, if thou believedst, thou shouldest see the glory of God?''}}
\bv{41}{So they took away the stone. And Jesus lifted up his eyes, and said, \redlet{``Father, I thank thee that thou heardest me.}}
\bv{42}{\redlet{And I knew that thou hearest me always: but because of the multitude that standeth around I said it, that they may believe that thou didst send me.''}}
\bv{43}{And when he had thus spoken, he cried with a loud voice, \redlet{``Lazarus, come forth.''}}
\bv{44}{He that was dead came forth, bound hand and foot with grave-clothes; and his face was bound about with a napkin. Jesus saith unto them, \redlet{``Loose him, and let him go.''}}
\chapsec{The Friends of Mary Converted}
\bv{45}{Many therefore of the Jews, who came to Mary and beheld that which he did, believed on him.}
\bv{46}{But some of them went away to the Pharisees, and told them the things which Jesus had done.}
\chapsec{The Pharisees Plot Jesus' Death}
\bv{47}{The chief priests therefore and the Pharisees gathered a council, and said, ``What do we? for this man doeth many signs.}
\bv{48}{If we let him thus alone, all men will believe on him: and the Romans will come and take away both our place and our nation.''}
\bv{49}{But a certain one of them, Caiaphas, being high priest that year, said unto them, ``Ye know nothing at all,}
\bv{50}{nor do ye take account that it is expedient for you that one man should die for the people, and that the whole nation perish not.''}
\bv{51}{Now this he said not of himself: but being high priest that year, he prophesied that Jesus should die for the nation;}
\bv{52}{and not for the nation only, but that he might also gather together into one the children of God that are scattered abroad.}
\bv{53}{So from that day forth they took counsel that they might put him to death.}
\bv{54}{Jesus therefore walked no more openly among the Jews, but departed thence into the country near to the wilderness, into a city called Ephraim; and there he tarried with the disciples.}
\par
\bv{55}{Now the passover of the Jews was at hand: and many went up to Jerusalem out of the country before the passover, to purify themselves.}
\bv{56}{They sought therefore for Jesus, and spake one with another, as they stood in the temple, ``What think ye? That he will not come to the feast?''}
\bv{57}{Now the chief priests and the Pharisees had given commandment, that, if any man knew where he was, he should show it, that they might take him.}
\chaphead{Chapter XII}
\chapdesc{The Supper at Bethany}
\lettrine[image=true, lines=4, findent=3pt, nindent=0pt]{NT/John/Jn12-J.eps}{esus} therefore six days before the passover came to Bethany, where Lazarus was, whom Jesus raised from the dead.
\bv{2}{So they made him a supper there: and Martha served; but Lazarus was one of them that sat at meat with him.}
\bv{3}{Mary therefore took a pound of ointment of pure nard, very precious, and anointed the feet of Jesus, and wiped his feet with her hair: and the house was filled with the odor of the ointment.}
\bv{4}{But Judas Iscariot, one of his disciples, that should betray him, saith,}
\bv{5}{``Why was not this ointment sold for three hundred shillings, and given to the poor?''}
\bv{6}{Now this he said, not because he cared for the poor; but because he was a thief, and having the bag took away what was put therein.}
\bv{7}{Jesus therefore said, \redlet{``Suffer her to keep it against the day of my burying.}}
\bv{8}{\redlet{For the poor ye have always with you; but me ye have not always.''}}
\bv{9}{The common people therefore of the Jews learned that he was there: and they came, not for Jesus' sake only, but that they might see Lazarus also, whom he had raised from the dead.}
\bv{10}{But the chief priests took counsel that they might put Lazarus also to death;}
\bv{11}{because that by reason of him many of the Jews went away, and believed on Jesus.}
\chapsec{The Triumphal Entry}
\bv{12}{On the morrow a great multitude that had come to the feast, when they heard that Jesus was coming to Jerusalem,}
\bv{13}{took the branches of the palm trees, and went forth to meet him, and cried out, ``Hosanna: Blessed \supptext{is} he that cometh in the name of the Lord, even the King of Israel.''}
\bv{14}{And Jesus, having found a young ass, sat thereon; as it is written,}
\otQuote{Zech. 9:9}{\bv{15}{Fear not, daughter of Zion: behold, thy King cometh, sitting on an ass's colt.}}
\bv{16}{These things understood not his disciples at the first: but when Jesus was glorified, then remembered they that these things were written of him, and that they had done these things unto him.}
\bv{17}{The multitude therefore that was with him when he called Lazarus out of the tomb, and raised him from the dead, bare witness.}
\bv{18}{For this cause also the multitude went and met him, for that they heard that he had done this sign.}
\bv{19}{The Pharisees therefore said among themselves, ``Behold how ye prevail nothing; lo, the world is gone after him.''}
\chapsec{Certain Greeks Would See Jesus}
\bv{20}{Now there were certain Greeks among those that went up to worship at the feast:}
\bv{21}{these therefore came to Philip, who was of Bethsaida of Galilee, and asked him, saying, ``Sir, we would see Jesus.''}
\bv{22}{Philip cometh and telleth Andrew: Andrew cometh, and Philip, and they tell Jesus.}
\chapsec{Jesus' Answer}
\bv{23}{And Jesus answereth them, saying, \redlet{``The hour is come, that the Son of man should be glorified.}}
\bv{24}{\redlet{Verily, verily, I say unto you, Except a grain of wheat fall into the earth and die, it abideth by itself alone; but if it die, it beareth much fruit.}}
\bv{25}{\redlet{He that loveth his life loseth it; and he that hateth his life in this world shall keep it unto life eternal.}}
\bv{26}{\redlet{If any man serve me, let him follow me; and where I am, there shall also my servant be: if any man serve me, him will the Father honor.}}
\bv{27}{\redlet{Now is my soul troubled; and what shall I say? Father, save me from this hour. But for this cause came I unto this hour.}}
\bv{28}{\redlet{Father, glorify thy name.''} There came therefore a voice out of heaven, \supptext{saying}, \god{``I have both glorified it, and will glorify it again.''}}
\bv{29}{The multitude therefore, that stood by, and heard it, said that it had thundered: others said, ``An angel hath spoken to him.''}
\bv{30}{Jesus answered and said, \redlet{``This voice hath not come for my sake, but for your sakes.}}
\bv{31}{\redlet{Now is the judgement of this world: now shall the prince of this world be cast out.}}
\bv{32}{\redlet{And I, if I be lifted up from the earth, will draw all men unto myself.''}}
\bv{33}{But this he said, signifying by what manner of death he should die.}
\bv{34}{The multitude therefore answered him, ``We have heard out of the law that the Christ abideth for ever: and how sayest thou, `The Son of man must be lifted up?' who is this Son of man?''}
\bv{35}{Jesus therefore said unto them, \redlet{``Yet a little while is the light among you. Walk while ye have the light, that darkness overtake you not: and he that walketh in the darkness knoweth not whither he goeth.}}
\bv{36}{\redlet{While ye have the light, believe on the light, that ye may become sons of light.''} These things spake Jesus, and he departed and hid himself from them.}
\par
\bv{37}{But though he had done so many signs before them, yet they believed not on him:}
\bv{38}{that the word of Isaiah the prophet might be fulfilled, which he spake,}
\otQuote{Is. 53:1}{Lord, who hath believed our report? And to whom hath the arm of the Lord been revealed?}
\bv{39}{For this cause they could not believe, for that Isaiah said again,}
\otQuote{Is. 6:10}{\bv{40}{He hath blinded their eyes, and he hardened their heart; Lest they should see with their eyes, and perceive with their heart, And should turn, And I should heal them.}}
\bv{41}{These things said Isaiah, because he saw his glory; and he spake of him.}
\bv{42}{Nevertheless even of the rulers many believed on him; but because of the Pharisees they did not confess \supptext{it}, lest they should be put out of the synagogue:}
\bv{43}{for they loved the glory \supptext{that is} of men more than the glory \supptext{that is} of God.}
\par
\bv{44}{And Jesus cried and said, \redlet{``He that believeth on me, believeth not on me, but on him that sent me.}}
\bv{45}{\redlet{And he that beholdeth me beholdeth him that sent me.}}
\bv{46}{\redlet{I am come a light into the world, that whosoever believeth on me may not abide in the darkness.}}
\bv{47}{\redlet{And if any man hear my sayings, and keep them not, I judge him not: for I came not to judge the world, but to save the world.}}
\bv{48}{\redlet{He that rejecteth me, and receiveth not my sayings, hath one that judgeth him: the word that I spake, the same shall judge him in the last day.}}
\bv{49}{\redlet{For I spake not from myself; but the Father that sent me, he hath given me a commandment, what I should say, and what I should speak.}}
\bv{50}{\redlet{And I know that his commandment is life eternal; the things therefore which I speak, even as the Father hath said unto me, so I speak.''}}
\chaphead{Chapter XIII}
\chapdesc{The Last Passover}
\lettrine[image=true, lines=4, findent=3pt, nindent=0pt]{NT/John/Jn-Now.eps}{ow} before the feast of the passover, Jesus knowing that his hour was come that he should depart out of this world unto the Father, having loved his own that were in the world, he loved them unto the end.
\chapsec{Jesus Washes the Disciples' Feet}
\bv{2}{And during supper, the devil having already put into the heart of Judas Iscariot, Simon's \supptext{son}, to betray him,}
\bv{3}{\supptext{Jesus}, knowing that the Father had given all things into his hands, and that he came forth from God, and goeth unto God,}
\bv{4}{riseth from supper, and layeth aside his garments; and he took a towel, and girded himself.}
\bv{5}{Then he poureth water into the basin, and began to wash the disciples' feet, and to wipe them with the towel wherewith he was girded.}
\bv{6}{So he cometh to Simon Peter. He saith unto him, ``Lord, dost thou wash my feet?''}
\bv{7}{Jesus answered and said unto him, \redlet{``What I do thou knowest not now; but thou shalt understand hereafter.''}}
\bv{8}{Peter saith unto him, ``Thou shalt never wash my feet.'' Jesus answered him, \redlet{``If I wash thee not, thou hast no part with me.''}}
\bv{9}{Simon Peter saith unto him, ``Lord, not my feet only, but also my hands and my head.''}
\bv{10}{Jesus saith to him, \redlet{``He that is bathed needeth not save to wash his feet, but is clean every whit: and ye are clean, but not all.''}}
\bv{11}{For he knew him that should betray him; therefore said he, \redlet{``Ye are not all clean.''}}
\par
\bv{12}{So when he had washed their feet, and taken his garments, and sat down again, he said unto them, \redlet{``Know ye what I have done to you?}}
\bv{13}{\redlet{Ye call me, Teacher, and, Lord: and ye say well; for so I am.}}
\bv{14}{\redlet{If I then, the Lord and the Teacher, have washed your feet, ye also ought to wash one another's feet.}}
\bv{15}{\redlet{For I have given you an example, that ye also should do as I have done to you.}}
\bv{16}{\redlet{Verily, verily, I say unto you, A servant is not greater than his lord; neither one that is sent greater than he that sent him.}}
\bv{17}{\redlet{If ye know these things, blessed are ye if ye do them.}}
\bv{18}{\redlet{I speak not of you all: I know whom I have chosen: but that the scripture may be fulfilled, He that eateth my bread lifted up his heel against me.}}
\bv{19}{\redlet{From henceforth I tell you before it come to pass, that, when it is come to pass, ye may believe that I am \supptext{he}.}}
\bv{20}{\redlet{Verily, verily, I say unto you, He that receiveth whomsoever I send receiveth me; and he that receiveth me receiveth him that sent me.''}}
\chapsec{Jesus Foretells His Betrayal}
\bv{21}{When Jesus had thus said, he was troubled in the spirit, and testified, and said, \redlet{``Verily, verily, I say unto you, that one of you shall betray me.''}}
\bv{22}{The disciples looked one on another, doubting of whom he spake.}
\bv{23}{There was at the table reclining in Jesus' bosom one of his disciples, whom Jesus loved.}
\bv{24}{Simon Peter therefore beckoneth to him, and saith unto him, ``Tell \supptext{us} who it is of whom he speaketh.''}
\bv{25}{He leaning back, as he was, on Jesus' breast saith unto him, ``Lord, who is it?''}
\bv{26}{Jesus therefore answereth, \redlet{``He it is, for whom I shall dip the sop, and give it him.''} So when he had dipped the sop, he taketh and giveth it to Judas, \supptext{the son} of Simon Iscariot.}
\bv{27}{And after the sop, then entered Satan into him. Jesus therefore saith unto him, \redlet{``What thou doest, do quickly.''}}
\bv{28}{Now no man at the table knew for what intent he spake this unto him.}
\bv{29}{For some thought, because Judas had the bag, that Jesus said unto him, \redlet{``Buy what things we have need of for the feast;''} or, that he should give something to the poor.}
\bv{30}{He then having received the sop went out straightway: and it was night.}
\par
\bv{31}{When therefore he was gone out, Jesus saith, \redlet{``Now is the Son of man glorified, and God is glorified in him;}}
\bv{32}{\redlet{and God shall glorify him in himself, and straightway shall he glorify him.}}
\bv{33}{\redlet{Little children, yet a little while I am with you. Ye shall seek me: and as I said unto the Jews, `Whither I go, ye cannot come;' so now I say unto you.}}
\bv{34}{\redlet{A new commandment I give unto you, that ye love one another; even as I have loved you, that ye also love one another.}}
\bv{35}{\redlet{By this shall all men know that ye are my disciples, if ye have love one to another.''}}
\chapsec{Jesus Foretells Peter's Denial}
\bv{36}{Simon Peter saith unto him, ``Lord, whither goest thou?'' Jesus answered, \redlet{``Whither I go, thou canst not follow me now; but thou shalt follow afterwards.''}}
\bv{37}{Peter saith unto him, ``Lord, why cannot I follow thee even now? I will lay down my life for thee.''}
\bv{38}{Jesus answereth, \redlet{``Wilt thou lay down thy life for me? Verily, verily, I say unto thee, The cock shall not crow, till thou hast denied me thrice.}}
\chaphead{Chapter XIV}
\chapdesc{Jesus Foretells His Coming for His Own}
\lettrine[image=true, lines=4, findent=3pt, nindent=0pt]{NT/John/Jn14-L.eps}{\redlet{et}} \redlet{not your heart be troubled: believe in God, believe also in me.}
\bv{2}{\redlet{In my Father's house are many mansions; if it were not so, I would have told you; for I go to prepare a place for you.}}
\bv{3}{\redlet{And if I go and prepare a place for you, I come again, and will receive you unto myself; that where I am, \supptext{there} ye may be also.}}
\bv{4}{\redlet{And whither I go, ye know the way.''}}
\bv{5}{Thomas saith unto him, ``Lord, we know not whither thou goest; how know we the way?''}
\bv{6}{Jesus saith unto him, \redlet{``I am the way, and the truth, and the life: no one cometh unto the Father, but by me.'}}
\chapsec{Jesus \& the Father Are One}
\bv{7}{\redlet{If ye had known me, ye would have known my Father also: from henceforth ye know him, and have seen him.''}}
\bv{8}{Philip saith unto him, ``Lord, show us the Father, and it sufficeth us.''}
\bv{9}{Jesus saith unto him, \redlet{``Have I been so long time with you, and dost thou not know me, Philip? he that hath seen me hath seen the Father; how sayest thou, `Show us the Father?'}}
\bv{10}{\redlet{Believest thou not that I am in the Father, and the Father in me? the words that I say unto you I speak not from myself: but the Father abiding in me doeth his works.}}
\bv{11}{\redlet{Believe me that I am in the Father, and the Father in me: or else believe me for the very works' sake.}}
\bv{12}{\redlet{Verily, verily, I say unto you, He that believeth on me, the works that I do shall he do also; and greater \supptext{works} than these shall he do; because I go unto the Father.}}
\chapsec{The New Promise in Prayer}
\bv{13}{\redlet{And whatsoever ye shall ask in my name, that will I do, that the Father may be glorified in the Son.}}
\bv{14}{\redlet{If ye shall ask anything in my name, that will I do.}}
\bv{15}{\redlet{If ye love me, ye will keep my commandments.}}
\chapsec{The Promise of the Spirit}
\bv{16}{\redlet{And I will pray the Father, and he shall give you another Comforter, that he may be with you for ever,}}
\bv{17}{\redlet{\supptext{even} the Spirit of truth: whom the world cannot receive; for it beholdeth him not, neither knoweth him: ye know him; for he abideth with you, and shall be in you.}}
\bv{18}{\redlet{I will not leave you desolate: I come unto you.}}
\bv{19}{\redlet{Yet a little while, and the world beholdeth me no more; but ye behold me: because I live, ye shall live also.}}
\bv{20}{\redlet{In that day ye shall know that I am in my Father, and ye in me, and I in you.}}
\bv{21}{\redlet{He that hath my commandments, and keepeth them, he it is that loveth me: and he that loveth me shall be loved of my Father, and I will love him, and will manifest myself unto him.''}}
\par
\bv{22}{Judas (not Iscariot) saith unto him, ``Lord, what is come to pass that thou wilt manifest thyself unto us, and not unto the world?''}
\bv{23}{Jesus answered and said unto him, \redlet{``If a man love me, he will keep my word: and my Father will love him, and we will come unto him, and make our abode with him.}}
\bv{24}{\redlet{He that loveth me not keepeth not my words: and the word which ye hear is not mine, but the Father's who sent me.}}
\bv{25}{\redlet{These things have I spoken unto you, while \supptext{yet} abiding with you.}}
\bv{26}{\redlet{But the Comforter, \supptext{even} the Holy Ghost, whom the Father will send in my name, he shall teach you all things, and bring to your remembrance all that I said unto you.}}
\chapsec{The Bequest of Peace}
\bv{27}{\redlet{Peace I leave with you; my peace I give unto you: not as the world giveth, give I unto you. Let not your heart be troubled, neither let it be fearful.}}
\bv{28}{\redlet{Ye heard how I said to you, I go away, and I come unto you. If ye loved me, ye would have rejoiced, because I go unto the Father: for the Father is greater than I.}}
\bv{29}{\redlet{And now I have told you before it come to pass, that, when it is come to pass, ye may believe.}}
\bv{30}{\redlet{I will no more speak much with you, for the prince of the world cometh: and he hath nothing in me;}}
\bv{31}{\redlet{but that the world may know that I love the Father, and as the Father gave me commandment, even so I do. Arise, let us go hence.}}
\chaphead{Chapter XV}
\chapdesc{The Vine \& Branches}
\lettrine[image=true, lines=4, findent=3pt, nindent=0pt]{NT/John/Jn-I.eps}{} \redlet{am the true vine, and my Father is the husbandman.}
\bv{2}{\redlet{Every branch in me that beareth not fruit, he taketh it away: and every \supptext{branch} that beareth fruit, he cleanseth it, that it may bear more fruit.}}
\bv{3}{\redlet{Already ye are clean because of the word which I have spoken unto you.}}
\bv{4}{\redlet{Abide in me, and I in you. As the branch cannot bear fruit of itself, except it abide in the vine; so neither can ye, except ye abide in me.}}
\bv{5}{\redlet{I am the vine, ye are the branches: He that abideth in me, and I in him, the same beareth much fruit: for apart from me ye can do nothing.}}
\par
\bv{6}{\redlet{If a man abide not in me, he is cast forth as a branch, and is withered; and they gather them, and cast them into the fire, and they are burned.}}
\bv{7}{\redlet{If ye abide in me, and my words abide in you, ask whatsoever ye will, and it shall be done unto you.}}
\bv{8}{\redlet{Herein is my Father glorified, that ye bear much fruit; and \supptext{so} shall ye be my disciples.}}
\bv{9}{\redlet{Even as the Father hath loved me, I also have loved you: abide ye in my love.}}
\bv{10}{\redlet{If ye keep my commandments, ye shall abide in my love; even as I have kept my Father's commandments, and abide in his love.}}
\bv{11}{\redlet{These things have I spoken unto you, that my joy may be in you, and \supptext{that} your joy may be made full.}}
\par
\bv{12}{\redlet{This is my commandment, that ye love one another, even as I have loved you.}}
\bv{13}{\redlet{Greater love hath no man than this, that a man lay down his life for his friends.}}
\bv{14}{\redlet{Ye are my friends, if ye do the things which I command you.}}
\chapsec{New Intimacy}
\bv{15}{\redlet{No longer do I call you servants; for the servant knoweth not what his lord doeth: but I have called you friends; for all things that I heard from my Father I have made known unto you.}}
\bv{16}{\redlet{Ye did not choose me, but I chose you, and appointed you, that ye should go and bear fruit, and \supptext{that} your fruit should abide: that whatsoever ye shall ask of the Father in my name, he may give it you.}}
\bv{17}{\redlet{These things I command you, that ye may love one another.}}
\chapsec{The Believer \& the World}
\bv{18}{\redlet{If the world hateth you, ye know that it hath hated me before \supptext{it hated} you.}}
\bv{19}{\redlet{If ye were of the world, the world would love its own: but because ye are not of the world, but I chose you out of the world, therefore the world hateth you.}}
\bv{20}{\redlet{Remember the word that I said unto you, A servant is not greater than his lord. If they persecuted me, they will also persecute you; if they kept my word, they will keep yours also.}}
\bv{21}{\redlet{But all these things will they do unto you for my name's sake, because they know not him that sent me.}}
\bv{22}{\redlet{If I had not come and spoken unto them, they had not had sin: but now they have no excuse for their sin.}}
\bv{23}{\redlet{He that hateth me hateth my Father also.}}
\bv{24}{\redlet{If I had not done among them the works which none other did, they had not had sin: but now have they both seen and hated both me and my Father.}}
\bv{25}{\redlet{But \supptext{this cometh to pass}, that the word may be fulfilled that is written in their law,}}
\otQuote{Ps. 35:19}{They hated me without a cause.}
\chapsec{The Believer \& the Spirit}
\bv{26}{\redlet{But when the Comforter is come, whom I will send unto you from the Father, \supptext{even} the Spirit of truth, which proceedeth from the Father, he shall bear witness of me:}}
\bv{27}{\redlet{and ye also bear witness, because ye have been with me from the beginning.}}
\chaphead{Chapter XVI}
\chapdesc{The Disciples Warned of Persecutions}
\lettrine[image=true, lines=4, findent=3pt, nindent=0pt]{NT/John/Jn-These.eps}{\redlet{hese}} \redlet{things have I spoken unto you, that ye should not be caused to stumble.}
\bv{2}{\redlet{They shall put you out of the synagogues: yea, the hour cometh, that whosoever killeth you shall think that he offereth service unto God.}}
\bv{3}{\redlet{And these things will they do, because they have not known the Father, nor me.}}
\bv{4}{\redlet{But these things have I spoken unto you, that when their hour is come, ye may remember them, how that I told you. And these things I said not unto you from the beginning, because I was with you.}}
\bv{5}{\redlet{But now I go unto him that sent me; and none of you asketh me, `Whither goest thou?'}}
\bv{6}{\redlet{But because I have spoken these things unto you, sorrow hath filled your heart.}}
\chapsec{Threefold Work of the Spirit}
\bv{7}{\redlet{Nevertheless I tell you the truth: It is expedient for you that I go away; for if I go not away, the Comforter will not come unto you; but if I go, I will send him unto you.}}
\bv{8}{\redlet{And he, when he is come, will convict the world in respect of sin, and of righteousness, and of judgement:}}
\bv{9}{\redlet{of sin, because they believe not on me;}}
\bv{10}{\redlet{of righteousness, because I go to the Father, and ye behold me no more;}}
\bv{11}{\redlet{of judgement, because the prince of this world hath been judged.}}
\chapsec{New Truth to be Revealed by the Spirit}
\bv{12}{\redlet{I have yet many things to say unto you, but ye cannot bear them now.}}
\bv{13}{\redlet{Howbeit when he, the Spirit of truth, is come, he shall guide you into all the truth: for he shall not speak from himself; but what things soever he shall hear, \supptext{these} shall he speak: and he shall declare unto you the things that are to come.}}
\bv{14}{\redlet{He shall glorify me: for he shall take of mine, and shall declare \supptext{it} unto you.}}
\bv{15}{\redlet{All things whatsoever the Father hath are mine: therefore said I, that he taketh of mine, and shall declare \supptext{it} unto you.}}
\bv{16}{\redlet{A little while, and ye behold me no more; and again a little while, and ye shall see me.''}}
\par
\bv{17}{\supptext{Some} of his disciples therefore said one to another, ``What is this that he saith unto us, `A little while, and ye behold me not; and again `a little while, and ye shall see me:' and, `Because I go to the Father?' ''}
\bv{18}{They said therefore, ``What is this that he saith, `A little while?' We know not what he saith.''}
\bv{19}{Jesus perceived that they were desirous to ask him, and he said unto them, \redlet{Do ye inquire among yourselves concerning this, that I said, `A little while, and ye behold me not, and again a little while, and ye shall see me?'}}
\bv{20}{\redlet{Verily, verily, I say unto you, that ye shall weep and lament, but the world shall rejoice: ye shall be sorrowful, but your sorrow shall be turned into joy.}}
\bv{21}{\redlet{A woman when she is in travail hath sorrow, because her hour is come: but when she is delivered of the child, she remembereth no more the anguish, for the joy that a man is born into the world.}}
\bv{22}{\redlet{And ye therefore now have sorrow: but I will see you again, and your heart shall rejoice, and your joy no one taketh away from you.}}
\bv{23}{\redlet{And in that day ye shall ask me no question. Verily, verily, I say unto you, If ye shall ask anything of the Father, he will give it you in my name.}}
\bv{24}{\redlet{Hitherto have ye asked nothing in my name: ask, and ye shall receive, that your joy may be made full.}}
\bv{25}{\redlet{These things have I spoken unto you in dark sayings: the hour cometh, when I shall no more speak unto you in dark sayings, but shall tell you plainly of the Father.}}
\bv{26}{\redlet{In that day ye shall ask in my name: and I say not unto you, that I will pray the Father for you;}}
\bv{27}{\redlet{for the Father himself loveth you, because ye have loved me, and have believed that I came forth from the Father.}}
\bv{28}{\redlet{I came out from the Father, and am come into the world: again, I leave the world, and go unto the Father.''}}
\par
\bv{29}{His disciples say, ``Lo, now speakest thou plainly, and speakest no dark saying.}
\bv{30}{Now know we that thou knowest all things, and needest not that any man should ask thee: by this we believe that thou camest forth from God.''}
\bv{31}{Jesus answered them, \redlet{``Do ye now believe?}}
\bv{32}{\redlet{Behold, the hour cometh, yea, is come, that ye shall be scattered, every man to his own, and shall leave me alone: and \supptext{yet} I am not alone, because the Father is with me.}}
\bv{33}{\redlet{These things have I spoken unto you, that in me ye may have peace. In the world ye have tribulation: but be of good cheer; I have overcome the world.''}}
\chaphead{Chapter XVII}
\chapdesc{The High Priestly Prayer}
\lettrine[image=true, lines=4, findent=3pt, nindent=0pt]{NT/John/Jn-These.eps}{hese} things spake Jesus; and lifting up his eyes to heaven, he said, \redlet{``Father, the hour is come; glorify thy Son, that the Son may glorify thee:}
\bv{2}{\redlet{even as thou gavest him authority over all flesh, that to all whom thou hast given him, he should give eternal life.}}
\bv{3}{\redlet{And this is life eternal, that they should know thee the only true God, and him whom thou didst send, \supptext{even} Jesus Christ.}}
\bv{4}{\redlet{I glorified thee on the earth, having accomplished the work which thou hast given me to do.}}
\bv{5}{\redlet{And now, Father, glorify thou me with thine own self with the glory which I had with thee before the world was.}}
\par
\bv{6}{\redlet{I manifested thy name unto the men whom thou gavest me out of the world: thine they were, and thou gavest them to me; and they have kept thy word.}}
\bv{7}{\redlet{Now they know that all things whatsoever thou hast given me are from thee:}}
\bv{8}{\redlet{for the words which thou gavest me I have given unto them; and they received \supptext{them}, and knew of a truth that I came forth from thee, and they believed that thou didst send me.}}
\bv{9}{\redlet{I pray for them: I pray not for the world, but for those whom thou hast given me; for they are thine:}}
\bv{10}{\redlet{and all things that are mine are thine, and thine are mine: and I am glorified in them.}}
\bv{11}{\redlet{And I am no more in the world, and these are in the world, and I come to thee. Holy Father, keep them in thy name which thou hast given me, that they may be one, even as we \supptext{are}.}}
\par
\bv{12}{\redlet{While I was with them, I kept them in thy name which thou hast given me: and I guarded them, and not one of them perished, but the son of perdition; that the scripture might be fulfilled.}}
\bv{13}{\redlet{But now I come to thee; and these things I speak in the world, that they may have my joy made full in themselves.}}
\bv{14}{\redlet{I have given them thy word; and the world hated them, because they are not of the world, even as I am not of the world.}}
\bv{15}{\redlet{I pray not that thou shouldest take them from the world, but that thou shouldest keep them from the evil \supptext{one}.}}
\bv{16}{\redlet{They are not of the world, even as I am not of the world.}}
\par
\bv{17}{\redlet{Sanctify them in the truth: thy word is truth.}}
\bv{18}{\redlet{As thou didst send me into the world, even so sent I them into the world.}}
\bv{19}{\redlet{And for their sakes I sanctify myself, that they themselves also may be sanctified in truth.}}
\bv{20}{\redlet{Neither for these only do I pray, but for them also that believe on me through their word;}}
\bv{21}{\redlet{that they may all be one; even as thou, Father, \supptext{art} in me, and I in thee, that they also may be in us: that the world may believe that thou didst send me.}}
\bv{22}{\redlet{And the glory which thou hast given me I have given unto them; that they may be one, even as we \supptext{are} one;}}
\par
\bv{23}{\redlet{I in them, and thou in me, that they may be perfected into one; that the world may know that thou didst send me, and lovedst them, even as thou lovedst me.}}
\bv{24}{\redlet{Father, I desire that they also whom thou hast given me be with me where I am, that they may behold my glory, which thou hast given me: for thou lovedst me before the foundation of the world.}}
\bv{25}{\redlet{O righteous Father, the world knew thee not, but I knew thee; and these knew that thou didst send me;}}
\bv{26}{\redlet{and I made known unto them thy name, and will make it known; that the love wherewith thou lovedst me may be in them, and I in them.''}}
\chaphead{Chapter XVIII}
\chapdesc{Jesus at Gethsemane}
\lettrine[image=true, lines=4, findent=3pt, nindent=0pt]{NT/John/Jn-When.eps}{hen} Jesus had spoken these words, he went forth with his disciples over the brook Kidron, where was a garden, into which he entered, himself and his disciples.
\chapsec{Jesus' Betrayal \& Arrest}
\bv{2}{Now Judas also, who betrayed him, knew the place: for Jesus oft-times resorted thither with his disciples.}
\bv{3}{Judas then, having received the band \supptext{of soldiers}, and officers from the chief priests and the Pharisees, cometh thither with lanterns and torches and weapons.}
\bv{4}{Jesus therefore, knowing all the things that were coming upon him, went forth, and saith unto them, \redlet{``Whom seek ye?''}}
\bv{5}{They answered him, ``Jesus of Nazareth.'' Jesus saith unto them, \redlet{\scshape I am}. And Judas also, who betrayed him, was standing with them.}
\bv{6}{When therefore he said unto them, \redlet{\scshape I am}, they went backward, and fell to the ground.}
\bv{7}{Again therefore he asked them, \redlet{``Whom seek ye?''} And they said, ``Jesus of Nazareth.''}
\bv{8}{Jesus answered, \redlet{``I told you that {\scshape I am}; if therefore ye seek me, let these go their way:''}}
\bv{9}{that the word might be fulfilled which he spake, \redlet{``Of those whom thou hast given me I lost not one.''}}
\par
\bv{10}{Simon Peter therefore having a sword drew it, and struck the high priest's servant, and cut off his right ear. Now the servant's name was Malchus.}
\bv{11}{Jesus therefore said unto Peter, \redlet{``Put up the sword into the sheath: the cup which the Father hath given me, shall I not drink it?''}}
\chapsec{Jesus Brought before the High Priest}
\bv{12}{So the band and the chief captain, and the officers of the Jews, seized Jesus and bound him,}
\bv{13}{and led him to Annas first; for he was father in law to Caiaphas, who was high priest that year.}
\par
\bv{14}{Now Caiaphas was he that gave counsel to the Jews, that it was expedient that one man should die for the people.}
\bv{15}{And Simon Peter followed Jesus, and \supptext{so did} another disciple. Now that disciple was known unto the high priest, and entered in with Jesus into the court of the high priest;}
\bv{16}{but Peter was standing at the door without. So the other disciple, who was known unto the high priest, went out and spake unto her that kept the door, and brought in Peter.}
\chapsec{St. Peter's First Denial}
\bv{17}{The maid therefore that kept the door saith unto Peter, ``Art thou also \supptext{one} of this man's disciples?'' He saith, ``I am not.''}
\bv{18}{Now the servants and the officers were standing \supptext{there}, having made a fire of coals; for it was cold; and they were warming themselves: and Peter also was with them, standing and warming himself.}
\chapsec{Interrogation by the High Priest}
\bv{19}{The high priest therefore asked Jesus of his disciples, and of his teaching.}
\bv{20}{Jesus answered him, \redlet{``I have spoken openly to the world; I ever taught in synagogues, and in the temple, where all the Jews come together; and in secret spake I nothing.}}
\bv{21}{\redlet{Why askest thou me? ask them that have heard \supptext{me}, what I spake unto them: behold, these know the things which I said.''}}
\bv{22}{And when he had said this, one of the officers standing by struck Jesus with his hand, saying, ``Answerest thou the high priest so?''}
\bv{23}{Jesus answered him, \redlet{``If I have spoken evil, bear witness of the evil: but if well, why smitest thou me?''}}
\bv{24}{Annas therefore sent him bound unto Caiaphas the high priest.}
\chapsec{St. Peter's Final Denials}
\bv{25}{Now Simon Peter was standing and warming himself. They said therefore unto him, ``Art thou also \supptext{one} of his disciples?'' He denied, and said, ``I am not.''}
\bv{26}{One of the servants of the high priest, being a kinsman of him whose ear Peter cut off, saith, ``Did not I see thee in the garden with him?''}
\bv{27}{Peter therefore denied again: and straightway the cock crew.}
\chapsec{Jesus Brought before Pilate}
\bv{28}{They lead Jesus therefore from Caiaphas into the Prætorium: and it was early; and they themselves entered not into the Prætorium, that they might not be defiled, but might eat the passover.}
\bv{29}{Pilate therefore went out unto them, and saith, ``What accusation bring ye against this man?''}
\bv{30}{They answered and said unto him, ``If this man were not an evil-doer, we should not have delivered him up unto thee.''}
\bv{31}{Pilate therefore said unto them, ``Take him yourselves, and judge him according to your law.'' The Jews said unto him, ``It is not lawful for us to put any man to death:''}
\bv{32}{that the word of Jesus might be fulfilled, which he spake, signifying by what manner of death he should die.}
\bv{33}{Pilate therefore entered again into the Prætorium, and called Jesus, and said unto him, ``Art thou the King of the Jews?''}
\bv{34}{Jesus answered, \redlet{``Sayest thou this of thyself, or did others tell it thee concerning me?''}}
\bv{35}{Pilate answered, ``Am I a Jew? Thine own nation and the chief priests delivered thee unto me: what hast thou done?''}
\bv{36}{Jesus answered, \redlet{``My kingdom is not of this world: if my kingdom were of this world, then would my servants fight, that I should not be delivered to the Jews: but now is my kingdom not from hence.''}}
\bv{37}{Pilate therefore said unto him, ``Art thou a king then?'' Jesus answered, \redlet{``Thou sayest that I am a king. To this end have I been born, and to this end am I come into the world, that I should bear witness unto the truth. Every one that is of the truth heareth my voice.''}}
\bv{38}{Pilate saith unto him, ``What is truth?'' And when he had said this, he went out again unto the Jews, and saith unto them, ``I find no crime in him.}
\chapsec{Jesus Condemned \& Barabbas Released}
\bv{39}{But ye have a custom, that I should release unto you one at the passover: will ye therefore that I release unto you the King of the Jews?''}
\bv{40}{They cried out therefore again, saying, ``Not this man, but Barabbas.'' Now Barabbas was a robber.}
\chaphead{Chapter XIX}
\chapdesc{Jesus Crowned with Thorns}
\lettrine[image=true, lines=4, findent=3pt, nindent=0pt]{NT/John/Jn-The.eps}{hen} Pilate therefore took Jesus, and scourged him.
\bv{2}{And the soldiers platted a crown of thorns, and put it on his head, and arrayed him in a purple garment;}
\bv{3}{and they came unto him, and said, ``Hail, King of the Jews!'' and they struck him with their hands.}
\chapsec{Pilate Brings Jesus before the Multitude}
\bv{4}{And Pilate went out again, and saith unto them, ``Behold, I bring him out to you, that ye may know that I find no crime in him.''}
\bv{5}{Jesus therefore came out, wearing the crown of thorns and the purple garment. And \supptext{Pilate} saith unto them, ``Behold, the man!''}
\bv{6}{When therefore the chief priests and the officers saw him, they cried out, saying, ``Crucify \supptext{him}, crucify \supptext{him}!'' Pilate saith unto them, ``Take him yourselves, and crucify him: for I find no crime in him.''}
\bv{7}{The Jews answered him, ``We have a law, and by that law he ought to die, because he made himself the Son of God.''}
\bv{8}{When Pilate therefore heard this saying, he was the more afraid;}
\bv{9}{and he entered into the Prætorium again, and saith unto Jesus, ``Whence art thou?'' But Jesus gave him no answer.}
\bv{10}{Pilate therefore saith unto him, ``Speakest thou not unto me? knowest thou not that I have power to release thee, and have power to crucify thee?''}
\bv{11}{Jesus answered him, \redlet{``Thou wouldest have no power against me, except it were given thee from above: therefore he that delivered me unto thee hath greater sin.''}}
\bv{12}{Upon this Pilate sought to release him: but the Jews cried out, saying, ``If thou release this man, thou art not Cæsar's friend: every one that maketh himself a king speaketh against Cæsar.''}
\bv{13}{When Pilate therefore heard these words, he brought Jesus out, and sat down on the judgement-seat at a place called The Pavement, but in Hebrew, Gabbatha.}
\chapsec{Final Rejection of Jesus by the Jews}
\bv{14}{Now it was the Preparation of the passover: it was about the sixth hour. And he saith unto the Jews, ``Behold, your King!''}
\bv{15}{They therefore cried out, ``Away with \supptext{him}, away with \supptext{him}, crucify him!'' Pilate saith unto them, ``Shall I crucify your King?'' The chief priests answered, ``We have no king but Cæsar.''}
\chapsec{The Crucifixion of Jesus Christ}
\bv{16}{Then therefore he delivered him unto them to be crucified.}
\bv{17}{They took Jesus therefore: and he went out, bearing the cross for himself, unto the place called The place of a skull, which is called in Hebrew Golgotha:}
\bv{18}{where they crucified him, and with him two others, on either side one, and Jesus in the midst.}
\bv{19}{And Pilate wrote a title also, and put it on the cross. And there was written, ``JESUS OF NAZARETH, THE KING OF THE JEWS.''}
\bv{20}{This title therefore read many of the Jews, for the place where Jesus was crucified was nigh to the city; and it was written in Hebrew, \supptext{and} in Latin, \supptext{and} in Greek.}
\bv{21}{The chief priests of the Jews therefore said to Pilate, ``Write not, `The King of the Jews;' but, that he said, `I am King of the Jews.' ''}
\bv{22}{Pilate answered, ``What I have written I have written.''}
\par
\bv{23}{The soldiers therefore, when they had crucified Jesus, took his garments and made four parts, to every soldier a part; and also the coat: now the coat was without seam, woven from the top throughout.}
\bv{24}{They said therefore one to another, ``Let us not rend it, but cast lots for it, whose it shall be:'' that the scripture might be fulfilled, which saith,}
\otQuote{Ps. 22:18}{They parted my garments among them, And upon my vesture did they cast lots.}
\par
\bv{25}{These things therefore the soldiers did. But there were standing by the cross of Jesus his mother, and his mother's sister, Mary the \supptext{wife} of Clopas, and Mary Magdalene.}
\chapsec{Mary given to the Beloved Disciple}
\bv{26}{When Jesus therefore saw his mother, and the disciple standing by whom he loved, he saith unto his mother, \redlet{``Woman, behold, thy son!''}}
\bv{27}{Then saith he to the disciple, \redlet{``Behold, thy mother!''} And from that hour the disciple took her unto his own \supptext{home}.}
\bv{28}{After this Jesus, knowing that all things are now finished, that the scripture might be accomplished, saith, \redlet{``I thirst.''}}
\chapsec{Consummation of the New Covenant}
\bv{29}{There was set there a vessel full of vinegar: so they put a sponge full of the vinegar upon hyssop, and brought it to his mouth.}
\bv{30}{When Jesus therefore had received the vinegar, he said, \redlet{``It is finished:''} and he bowed his head, and gave up his spirit.}
\chapsec{Old Testament Fulfilment}
\bv{31}{The Jews therefore, because it was the Preparation, that the bodies should not remain on the cross upon the sabbath (for the day of that sabbath was a high \supptext{day}), asked of Pilate that their legs might be broken, and \supptext{that} they might be taken away.}
\bv{32}{The soldiers therefore came, and brake the legs of the first, and of the other that was crucified with him:}
\bv{33}{but when they came to Jesus, and saw that he was dead already, they brake not his legs:}
\bv{34}{howbeit one of the soldiers with a spear pierced his side, and straightway there came out blood and water.}
\bv{35}{And he that hath seen hath borne witness, and his witness is true: and he knoweth that he saith true, that ye also may believe.}
\bv{36}{For these things came to pass, that the scripture might be fulfilled,}
\otQuote{Ex. 12:46}{A bone of him shall not be broken.}
\bv{37}{And again another scripture saith,}
\otQuote{Zech. 12:10}{They shall look on him whom they pierced.}
\chapsec{The Entombment}
\bv{38}{And after these things Joseph of Arimathæa, being a disciple of Jesus, but secretly for fear of the Jews, asked of Pilate that he might take away the body of Jesus: and Pilate gave \supptext{him} leave. He came therefore, and took away his body.}
\bv{39}{And there came also Nicodemus, he who at the first came to him by night, bringing a mixture of myrrh and aloes, about a hundred pounds.}
\bv{40}{So they took the body of Jesus, and bound it in linen cloths with the spices, as the custom of the Jews is to bury.}
\bv{41}{Now in the place where he was crucified there was a garden; and in the garden a new tomb wherein was never man yet laid.}
\bv{42}{There then because of the Jews' Preparation (for the tomb was nigh at hand) they laid Jesus.}
\chaphead{Chapter XX}
\chapdesc{The Resurrection of Jesus Christ}
\lettrine[image=true, lines=4, findent=3pt, nindent=0pt]{NT/John/Jn-Now.eps}{ow} on the first \supptext{day} of the week cometh Mary Magdalene early, while it was yet dark, unto the tomb, and seeth the stone taken away from the tomb.
\bv{2}{She runneth therefore, and cometh to Simon Peter, and to the other disciple whom Jesus loved, and saith unto them, ``They have taken away the Lord out of the tomb, and we know not where they have laid him.''}
\bv{3}{Peter therefore went forth, and the other disciple, and they went toward the tomb.}
\bv{4}{And they ran both together: and the other disciple outran Peter, and came first to the tomb;}
\bv{5}{and stooping and looking in, he seeth the linen cloths lying; yet entered he not in.}
\bv{6}{Simon Peter therefore also cometh, following him, and entered into the tomb; and he beholdeth the linen cloths lying,}
\bv{7}{and the napkin, that was upon his head, not lying with the linen cloths, but rolled up in a place by itself.}
\bv{8}{Then entered in therefore the other disciple also, who came first to the tomb, and he saw, and believed.}
\bv{9}{For as yet they knew not the scripture, that he must rise again from the dead.}
\bv{10}{So the disciples went away again unto their own home.}
\chapsec{Jesus Appears to Mary magdalene}
\bv{11}{But Mary was standing without at the tomb weeping: so, as she wept, she stooped and looked into the tomb;}
\bv{12}{and she beholdeth two angels in white sitting, one at the head, and one at the feet, where the body of Jesus had lain.}
\bv{13}{And they say unto her, ``Woman, why weepest thou?'' She saith unto them, ``Because they have taken away my Lord, and I know not where they have laid him.''}
\bv{14}{When she had thus said, she turned herself back, and beholdeth Jesus standing, and knew not that it was Jesus.}
\bv{15}{Jesus saith unto her, \redlet{``Woman, why weepest thou? whom seekest thou?''} She, supposing him to be the gardener, saith unto him, ``Sir, if thou hast borne him hence, tell me where thou hast laid him, and I will take him away.''}
\bv{16}{Jesus saith unto her, \redlet{``Mary.''} She turneth herself, and saith unto him in Hebrew, ``Rabboni;'' which is to say, ```Teacher.''}
\bv{17}{Jesus saith to her, Touch me not; for I am not yet ascended unto the Father: but go unto my brethren, and say to them, I ascend unto my Father and your Father, and my God and your God.}
\bv{18}{Mary Magdalene cometh and telleth the disciples, ``I have seen the Lord;'' and \supptext{that} he had said these things unto her.}
\chapsec{Jesus Appears to the Disciples}
\bv{19}{When therefore it was evening, on that day, the first \supptext{day} of the week, and when the doors were shut where the disciples were, for fear of the Jews, Jesus came and stood in the midst, and saith unto them, \redlet{``Peace \supptext{be} unto you.''}}
\bv{20}{And when he had said this, he showed unto them his hands and his side. The disciples therefore were glad, when they saw the Lord.}
\bv{21}{Jesus therefore said to them again, \redlet{``Peace \supptext{be} unto you: as the Father hath sent me, even so send I you.''}}
\bv{22}{And when he had said this, he breathed on them, and saith unto them, \redlet{``Receive ye the Holy Ghost:}}
\bv{23}{\redlet{whose soever sins ye forgive, they are forgiven unto them; whose soever \supptext{sins} ye retain, they are retained.''}}
\chapsec{St. Thomas' Faith}
\bv{24}{But Thomas, one of the twelve, called Didymus, was not with them when Jesus came.}
\bv{25}{The other disciples therefore said unto him, ``We have seen the Lord.'' But he said unto them, ``Except I shall see in his hands the print of the nails, and put my finger into the print of the nails, and put my hand into his side, I will not believe.''}
\bv{26}{And after eight days again his disciples were within, and Thomas with them. Jesus cometh, the doors being shut, and stood in the midst, and said, \redlet{``Peace \supptext{be} unto you.''}}
\bv{27}{Then saith he to Thomas, \redlet{``Reach hither thy finger, and see my hands; and reach \supptext{hither} thy hand, and put it into my side: and be not faithless, but believing.''}}
\bv{28}{Thomas answered and said unto him, ``My Lord and my God.''}
\bv{29}{Jesus saith unto him, \redlet{``Because thou hast seen me, thou hast believed: blessed \supptext{are} they that have not seen, and \supptext{yet} have believed.''}}
\chapsec{Purpose of St. John's Gospel}
\bv{30}{Many other signs therefore did Jesus in the presence of the disciples, which are not written in this book:}
\bv{31}{but these are written, that ye may believe that Jesus is the Christ, the Son of God; and that believing ye may have life in his name.}
\chaphead{Chapter XXI}
\chapdesc{Epilogue}
\lettrine[image=true, lines=4, findent=3pt, nindent=0pt]{NT/John/Jn-After.eps}{fter} these things Jesus manifested himself again to the disciples at the sea of Tiberias; and he manifested \supptext{himself} on this wise.
\bv{2}{There were together Simon Peter, and Thomas called Didymus, and Nathanael of Cana in Galilee, and the \supptext{sons} of Zebedee, and two other of his disciples.}
\par
\bv{3}{Simon Peter saith unto them, ``I go a fishing.'' They say unto him, ``We also come with thee.'' They went forth, and entered into the boat; and that night they took nothing.}
\bv{4}{But when day was now breaking, Jesus stood on the beach: yet the disciples knew not that it was Jesus.}
\par
\bv{5}{Jesus therefore saith unto them, \redlet{``Children, have ye aught to eat?''} They answered him, ``No.''}
\par
\bv{6}{And he said unto them, \redlet{``Cast the net on the right side of the boat, and ye shall find.''} They cast therefore, and now they were not able to draw it for the multitude of fishes.}
\bv{7}{That disciple therefore whom Jesus loved saith unto Peter, ``It is the Lord.'' So when Simon Peter heard that it was the Lord, he girt his coat about him (for he was naked), and cast himself into the sea.}
\bv{8}{But the other disciples came in the little boat (for they were not far from the land, but about two hundred cubits off), dragging the net \supptext{full} of fishes.}
\bv{9}{So when they got out upon the land, they see a fire of coals there, and fish laid thereon, and bread.}
\bv{10}{Jesus saith unto them, \redlet{``Bring of the fish which ye have now taken.''}}
\bv{11}{Simon Peter therefore went up, and drew the net to land, full of great fishes, a hundred and fifty and three: and for all there were so many, the net was not rent.}
\par
\bv{12}{Jesus saith unto them, \redlet{``Come \supptext{and} break your fast.''} And none of the disciples durst inquire of him, ``Who art thou?'' knowing that it was the Lord.}
\bv{13}{Jesus cometh, and taketh the bread, and giveth them, and the fish likewise.}
\bv{14}{This is now the third time that Jesus was manifested to the disciples, after that he was risen from the dead.}
\chapsec{Jesus Restores St. Peter's Faith}
\bv{15}{So when they had broken their fast, Jesus saith to Simon Peter, \redlet{``Simon, \supptext{son} of John, lovest thou me more than these?''} He saith unto him, ``Yea, Lord; thou knowest that I love thee.'' He saith unto him, \redlet{``Feed my lambs.''}}
\bv{16}{He saith to him again a second time, \redlet{``Simon, \supptext{son} of John, lovest thou me?''} He saith unto him, ``Yea, Lord; thou knowest that I love thee.'' He saith unto him, \redlet{``Tend my sheep.''}}
\bv{17}{He saith unto him the third time, \redlet{``Simon, \supptext{son} of John, lovest thou me?''} Peter was grieved because he said unto him the third time, \redlet{``Lovest thou me?''} And he said unto him, ``Lord, thou knowest all things; thou knowest that I love thee.'' Jesus saith unto him, \redlet{``Feed my sheep.}}
\par
\bv{18}{\redlet{Verily, verily, I say unto thee, When thou wast young, thou girdedst thyself, and walkedst whither thou wouldest: but when thou shalt be old, thou shalt stretch forth thy hands, and another shall gird thee, and carry thee whither thou wouldest not.''}}
\bv{19}{Now this he spake, signifying by what manner of death he should glorify God. And when he had spoken this, he saith unto him, \redlet{``Follow me.''}}
\par
\bv{20}{Peter, turning about, seeth the disciple whom Jesus loved following; who also leaned back on his breast at the supper, and said, ``Lord, who is he that betrayeth thee?''}
\bv{21}{Peter therefore seeing him saith to Jesus, ``Lord, and what shall this man do?''}
\bv{22}{Jesus saith unto him, \redlet{``If I will that he tarry till I come, what \supptext{is that} to thee? follow thou me.''}}
\bv{23}{This saying therefore went forth among the brethren, that that disciple should not die: yet Jesus said not unto him, that he should not die; but, \redlet{``If I will that he tarry till I come, what \supptext{is that} to thee?''}}
\bv{24}{This is the disciple that beareth witness of these things, and wrote these things: and we know that his witness is true.}
\bv{25}{And there are also many other things which Jesus did, the which if they should be written every one, I suppose that even the world itself would not contain the books that should be written.}
\begin{center}
	{\scshape [Here Endeth the Gospel of John]}
\end{center}
\clearpage
\chapter{The Woman Caught in Adultery}
\chapdesc{The story of the woman caught in adultery \textit{(pericope adulterae)} traditionally has been inserted into either St. John's or St. Luke's Gospel. It is likely a common historical story shared by early Christians about Our Lord.}
And they went every man unto his own house: but Jesus went unto the mount of Olives. And early in the morning he came again into the temple, and all the people came unto him; and he sat down, and taught them. And the scribes and the Pharisees bring a woman taken in adultery; and having set her in the midst, they say unto him, ``Teacher, this woman hath been taken in adultery, in the very act. Now in the law Moses commanded us to stone such: what then sayest thou of her?''
\par
And this they said, trying him, that they might have \supptext{whereof} to accuse him. But Jesus stooped down, and with his finger wrote on the ground. But when they continued asking him, he lifted up himself, and said unto them, ``He that is without sin among you, let him first cast a stone at her.'' And again he stooped down, and with his finger wrote on the ground. And they, when they heard it, went out one by one, beginning from the eldest, \supptext{even} unto the last: and Jesus was left alone, and the woman, where she was, in the midst.
\par
And Jesus lifted up himself, and said unto her, ``Woman, where are they? did no man condemn thee?'' And she said, ``No man, Lord.'' And Jesus said, ``Neither do I condemn thee: go thy way; from henceforth sin no more.''
	\clearpage
	\input{./books/Acts/Acts.tex}
	\clearpage
	\chapter{The Epistle of Paul to the Romans}
\fancyhead[RE,LO]{Romans}
\chaphead{Chapter I}
\chapdesc{Introduction}
\lettrine[image=true, lines=4, findent=3pt, nindent=0pt]{NT/Romans/Paul.eps}{aul}, a servant of Jesus Christ, called \supptext{to be} an apostle, separated unto the gospel of God,
\bv{2}{which he promised afore through his prophets in the holy scriptures,}
\bv{3}{concerning his Son, who was born of the seed of David according to the flesh,}
\bv{4}{who was declared \supptext{to be} the Son of God with power, according to the spirit of holiness, by the resurrection from the dead; \supptext{even} Jesus Christ our Lord,}
\bv{5}{through whom we received grace and apostleship, unto obedience of faith among all the nations, for his name's sake;}
\bv{6}{among whom are ye also, called \supptext{to be} Jesus Christ's:}
\bv{7}{to all that are in Rome, beloved of God, called \supptext{to be} saints: Grace to you and peace from God our Father and the Lord Jesus Christ.}
\chapsec{Thanksgiving for Faith}
\bv{8}{First, I thank my God through Jesus Christ for you all, that your faith is proclaimed throughout the whole world.}
\bv{9}{For God is my witness, whom I serve in my spirit in the gospel of his Son, how unceasingly I make mention of you, always in my prayers}
\bv{10}{making request, if by any means now at length I may be prospered by the will of God to come unto you.}
\par
\bv{11}{For I long to see you, that I may impart unto you some spiritual gift, to the end ye may be established;}
\bv{12}{that is, that I with you may be comforted in you, each of us by the other's faith, both yours and mine.}
\bv{13}{And I would not have you ignorant, brethren, that oftentimes I purposed to come unto you (and was hindered hitherto), that I might have some fruit in you also, even as in the rest of the Gentiles.}
\bv{14}{I am debtor both to Greeks and to Barbarians, both to the wise and to the foolish.}
\bv{15}{So, as much as in me is, I am ready to preach the gospel to you also that are in Rome.}
\par
\bv{16}{For I am not ashamed of the gospel: for it is the power of God unto salvation to every one that believeth; to the Jew first, and also to the Greek.}
\bv{17}{For therein is revealed a righteousness of God from faith unto faith: as it is written,}
\otQuote{Hab. 2:4}{But the righteous shall live by faith.}
\chapsec{The Guilt of the World}
\bv{18}{For the wrath of God is revealed from heaven against all ungodliness and unrighteousness of men, who hinder the truth in unrighteousness;}
\chapsec{Natural Law}
\bv{19}{because that which is known of God is manifest in them; for God manifested it unto them.}
\bv{20}{For the invisible things of him since the creation of the world are clearly seen, being perceived through the things that are made, \supptext{even} his everlasting power and divinity; that they may be without excuse:}
\bv{21}{because that, knowing God, they glorified him not as God, neither gave thanks; but became vain in their reasonings, and their senseless heart was darkened.}
\chapsec{Descent into Sin}
\bv{22}{Professing themselves to be wise, they became fools,}
\bv{23}{and changed the glory of the incorruptible God for the likeness of an image of corruptible man, and of birds, and four-footed beasts, and creeping things.}
\chapsec{Consequence of Sin}
\bv{24}{Wherefore God gave them up in the lusts of their hearts unto uncleanness, that their bodies should be dishonoured among themselves:}
\bv{25}{for that they exchanged the truth of God for a lie, and worshipped and served the creature rather than the Creator, who is blessed for ever. Amen.}
\bv{26}{For this cause God gave them up unto vile passions: for their women changed the natural use into that which is against nature:}
\bv{27}{and likewise also the men, leaving the natural use of the woman, burned in their lust one toward another, men with men working unseemliness, and receiving in themselves that recompense of their error which was due.}
\bv{28}{And even as they refused to have God in \supptext{their} knowledge, God gave them up unto a reprobate mind, to do those things which are not fitting;}
\bv{29}{being filled with all unrighteousness, wickedness, covetousness, maliciousness; full of envy, murder, strife, deceit, malignity; whisperers,}
\bv{30}{backbiters, hateful to God, insolent, haughty, boastful, inventors of evil things, disobedient to parents,}
\bv{31}{without understanding, covenant-breakers, without natural affection, unmerciful:}
\bv{32}{who, knowing the ordinance of God, that they that practise such things are worthy of death, not only do the same, but also consent with them that practise them.}
\chaphead{Chapter II}
\chapdesc{Gentiles Can Have No Excuse}
\lettrine[image=true, lines=4, findent=3pt, nindent=0pt]{NT/Romans/Wherefore.eps}{herefore} thou art without excuse, O man, whosoever thou art that judgest: for wherein thou judgest another, thou condemnest thyself; for thou that judgest dost practise the same things.
\bv{2}{And we know that the judgement of God is according to truth against them that practise such things.}
\bv{3}{And reckonest thou this, O man, who judgest them that practise such things, and doest the same, that thou shalt escape the judgement of God?}
\bv{4}{Or despisest thou the riches of his goodness and forbearance and longsuffering, not knowing that the goodness of God leadeth thee to repentance?}
\bv{5}{but after thy hardness and impenitent heart treasurest up for thyself wrath in the day of wrath and revelation of the righteous judgement of God;}
\bv{6}{who will render to every man according to his works:}
\bv{7}{to them that by patience in well-doing seek for glory and honour and incorruption, eternal life:}
\bv{8}{but unto them that are factious, and obey not the truth, but obey unrighteousness, \supptext{shall be} wrath and indignation,}
\bv{9}{tribulation and anguish, upon every soul of man that worketh evil, of the Jew first, and also of the Greek;}
\bv{10}{but glory and honour and peace to every man that worketh good, to the Jew first, and also to the Greek:}
\bv{11}{for there is no respect of persons with God.}
\chapsec{Natural Law Condemns the Gentiles}
\bv{12}{For as many as have sinned without the law shall also perish without the law: and as many as have sinned under the law shall be judged by the law;}
\bv{13}{for not the hearers of the law are just before God, but the doers of the law shall be justified;}
\bv{14}{(for when Gentiles that have not the law do by nature the things of the law, these, not having the law, are the law unto themselves;}
\bv{15}{in that they show the work of the law written in their hearts, their conscience bearing witness therewith, and their thoughts one with another accusing or else excusing \supptext{them});\mref{Wisdom}}
\bv{16}{in the day when God shall judge the secrets of men, according to my gospel, by Jesus Christ.}
\chapsec{Mosaic Law Condemns the Jews}
\bv{17}{But if thou bearest the name of a Jew, and restest upon the law, and gloriest in God,}
\bv{18}{and knowest his will, and approvest the things that are excellent, being instructed out of the law,}
\bv{19}{and art confident that thou thyself art a guide of the blind, a light of them that are in darkness,}
\bv{20}{a corrector of the foolish, a teacher of babes, having in the law the form of knowledge and of the truth;}
\bv{21}{thou therefore that teachest another, teachest thou not thyself? thou that preachest a man should not steal, dost thou steal?}
\bv{22}{thou that sayest a man should not commit adultery, dost thou commit adultery? thou that abhorrest idols, dost thou rob temples?}
\bv{23}{thou who gloriest in the law, through thy transgression of the law dishonourest thou God?}
\chapsec{Guilt of the Jews' Sin}
\bv{24}{For the name of God is blasphemed among the Gentiles because of you, even as it is written.}
\bv{25}{For circumcision indeed profiteth, if thou be a doer of the law: but if thou be a transgressor of the law, thy circumcision is become uncircumcision.}
\bv{26}{If therefore the uncircumcision keep the ordinances of the law, shall not his uncircumcision be reckoned for circumcision?}
\bv{27}{and shall not the uncircumcision which is by nature, if it fulfil the law, judge thee, who with the letter and circumcision art a transgressor of the law?}
\bv{28}{For he is not a Jew who is one outwardly; neither is that circumcision which is outward in the flesh:}
\bv{29}{but he is a Jew who is one inwardly; and circumcision is that of the heart, in the spirit not in the letter; whose praise is not of men, but of God.}
\chaphead{Chapter III}
\chapdesc{The Great Guilt of the Jews}
\lettrine[image=true, lines=4, findent=3pt, nindent=0pt]{NT/Romans/What.eps}{hat} advantage then hath the Jew? or what is the profit of circumcision?
\bv{2}{Much every way: first of all, that they were intrusted with the oracles of God.}
\bv{3}{For what if some were without faith? shall their want of faith make of none effect the faithfulness of God?}
\bv{4}{God forbid: yea, let God be found true, but every man a liar; as it is written,}
\otQuote{Ps. 51:4}{That thou mightest be justified in thy words,
And mightest prevail when thou comest into judgement.}
\bv{5}{But if our unrighteousness commendeth the righteousness of God, what shall we say? Is God unrighteous who visiteth with wrath? (I speak after the manner of men.)}
\bv{6}{God forbid: for then how shall God judge the world?}
\bv{7}{But if the truth of God through my lie abounded unto his glory, why am I also still judged as a sinner?}
\bv{8}{and why not (as we are slanderously reported, and as some affirm that we say), Let us do evil, that good may come? whose condemnation is just.}
\chapsec{Law Condemns Jew \& Gentile}
\bv{9}{What then? are we better than they? No, in no wise: for we before laid to the charge both of Jews and Greeks, that they are all under sin;}
\bv{10}{as it is written,}
\otQuote{Ps. 14:1-3; 53:1-3}{There is none righteous, no, not one;
\bv{11}{There is none that understandeth,
There is none that seeketh after God;}
\bv{12}{They have all turned aside, they are together become unprofitable;
There is none that doeth good, no, not so much as one:}}
\otQuote{Ps. 5:9}{\bv{13}{Their throat is an open sepulchre;
With their tongues they have used deceit:
The poison of asps is under their lips:}}
\otQuote{Ps. 10:7}{\bv{14}{Whose mouth is full of cursing and bitterness:}}
\otQuote{Is. 59:7-8}{\bv{15}{Their feet are swift to shed blood;}
\bv{16}{Destruction and misery are in their ways;}
\bv{17}{And the way of peace have they not known:}}
\otQuote{Ps. 36:1}{\bv{18}{There is no fear of God before their eyes.}}
\bv{19}{Now we know that what things soever the law saith, it speaketh to them that are under the law; that every mouth may be stopped, and all the world may be brought under the judgement of God:}
\bv{20}{because by the works of the law shall no flesh be justified in his sight; for through the law \supptext{cometh} the knowledge of sin.}
\chapsec{Justification in Christ apart from Law}
\bv{21}{But now apart from the law a righteousness of God hath been manifested, being witnessed by the law and the prophets;\mcomm{``\supptext{We} are not justified through ourselves or through our own wisdom or understanding or piety or works which we wrought in holiness of heart, but through faith.'' -St. Clement}}
\bv{22}{even the righteousness of God through faith in Jesus Christ unto all them that believe; for there is no distinction;}
\bv{23}{for all have sinned, and fall short of the glory of God;}
\bv{24}{being justified freely by his grace through the redemption that is in Christ Jesus:}
\bv{25}{whom God set forth \supptext{to be} a propitiation, through faith, in his blood, to show his righteousness because of the passing over of the sins done aforetime, in the forbearance of God;}
\bv{26}{for the showing, \supptext{I say}, of his righteousness at this present season: that he might himself be just, and the justifier of him that hath faith in Jesus.}
\par
\bv{27}{Where then is the glorying? It is excluded. By what manner of law? of works? Nay: but by a law of faith.}
\bv{28}{We reckon therefore that a man is justified by faith apart from the works of the law.}
\chapsec{Justification for Those Condemned by Law}
\bv{29}{Or is God \supptext{the God} of Jews only? is he not \supptext{the God} of Gentiles also? Yea, of Gentiles also:}
\bv{30}{if so be that God is one, and he shall justify the circumcision by faith, and the uncircumcision through faith.}
\chapsec{Justification by Faith Honours Law}
\bv{31}{Do we then make the law of none effect through faith? God forbid: nay, we establish the law.}
\chaphead{Chapter IV}
\chapdesc{Justification by Faith Alone Illustrated}
\lettrine[image=true, lines=4, findent=3pt, nindent=0pt]{NT/Romans/What.eps}{hat} then shall we say that Abraham, our forefather, hath found according to the flesh?
\bv{2}{For if Abraham was justified by works, he hath whereof to glory; but not toward God.}
\bv{3}{For what saith the scripture? And Abraham believed God, and it was reckoned unto him for righteousness.}
\bv{4}{Now to him that worketh, the reward is not reckoned as of grace, but as of debt.\mcomm{``What could man...do of himself to recover the righteousness which he had once lost? Therefore another’s righteousness was ascribed to him who lacked his own.'' -St. Bernard}}
\chapsec{Justifying Faith Defined}
\bv{5}{But to him that worketh not, but believeth on him that justifieth the ungodly, his faith is reckoned for righteousness.}
\bv{6}{Even as David also pronounceth blessing upon the man, unto whom God reckoneth righteousness apart from works,}
\bv{7}{\supptext{saying},}
\otQuote{Ps. 32:1-2}{Blessed are they whose iniquities are forgiven,
And whose sins are covered.
\bv{8}{Blessed is the man to whom the Lord will not reckon sin.}}
\chapsec{Both Jews \& Gentiles are Justified}
\bv{9}{Is this blessing then pronounced upon the circumcision, or upon the uncircumcision also? for we say, To Abraham his faith was reckoned for righteousness.}
\bv{10}{How then was it reckoned? when he was in circumcision, or in uncircumcision? Not in circumcision, but in uncircumcision:}
\bv{11}{and he received the sign of circumcision, a seal of the righteousness of the faith which he had while he was in uncircumcision: that he might be the father of all them that believe, though they be in uncircumcision, that righteousness might be reckoned unto them;}
\bv{12}{and the father of circumcision to them who not only are of the circumcision, but who also walk in the steps of that faith of our father Abraham which he had in uncircumcision.}
\chapsec{Justification apart from Law}
\bv{13}{For not through the law was the promise to Abraham or to his seed that he should be heir of the world, but through the righteousness of faith.}
\bv{14}{For if they that are of the law are heirs, faith is made void, and the promise is made of none effect:}
\bv{15}{for the law worketh wrath; but where there is no law, neither is there transgression.}
\bv{16}{For this cause \supptext{it is} of faith, that \supptext{it may be} according to grace; to the end that the promise may be sure to all the seed; not to that only which is of the law, but to that also which is of the faith of Abraham, who is the father of us all}
\bv{17}{(as it is written, \shortQ{Gen. 17:5}{A father of many nations have I made thee}) before him whom he believed, \supptext{even} God, who giveth life to the dead, and calleth the things that are not, as though they were.}
\bv{18}{Who in hope believed against hope, to the end that he might become a father of many nations, according to that which had been spoken, So shall thy seed be.}
\bv{19}{And without being weakened in faith he considered his own body now as good as dead (he being about a hundred years old), and the deadness of Sarah's womb;}
\bv{20}{yet, looking unto the promise of God, he wavered not through unbelief, but waxed strong through faith, giving glory to God,}
\bv{21}{and being fully assured that what he had promised, he was able also to perform.}
\bv{22}{Wherefore also it was reckoned unto him for righteousness.}
\bv{23}{Now it was not written for his sake alone, that it was reckoned unto him;}
\bv{24}{but for our sake also, unto whom it shall be reckoned, who believe on him that raised Jesus our Lord from the dead,}
\bv{25}{who was delivered up for our trespasses, and was raised for our justification.}
\chaphead{Chapter V}
\chapdesc{Peace with God through Justification}
\lettrine[image=true, lines=4, findent=3pt, nindent=0pt]{NT/Romans/Being.eps}{eing} therefore justified by faith, we have peace with God through our Lord Jesus Christ;
\bv{2}{through whom also we have had our access by faith into this grace wherein we stand; and we rejoice in hope of the glory of God.}
\bv{3}{And not only so, but we also rejoice in our tribulations: knowing that tribulation worketh stedfastness;}
\bv{4}{and stedfastness, approvedness; and approvedness, hope:}
\bv{5}{and hope putteth not to shame; because the love of God hath been shed abroad in our hearts through the Holy Spirit which was given unto us.}
\bv{6}{For while we were yet weak, in due season Christ died for the ungodly.}
\bv{7}{For scarcely for a righteous man will one die: for peradventure for the good man some one would even dare to die.}
\bv{8}{But God commendeth his own love toward us, in that, while we were yet sinners, Christ died for us.}
\bv{9}{Much more then, being now justified by his blood, shall we be saved from the wrath \supptext{of God} through him.}
\bv{10}{For if, while we were enemies, we were reconciled to God through the death of his Son, much more, being reconciled, shall we be saved by his life;}
\bv{11}{and not only so, but we also rejoice in God through our Lord Jesus Christ, through whom we have now received the reconciliation.}
\chapsec{Sin from Adam}
\bv{12}{Therefore, as through one man sin entered into the world, and death through sin; and so death passed unto all men, for that all sinned:—}
\bv{13}{for until the law sin was in the world; but sin is not imputed when there is no law.}
\bv{14}{Nevertheless death reigned from Adam until Moses, even over them that had not sinned after the likeness of Adam's transgression, who is a figure of him that was to come.}
\chapsec{Grace from Jesus Christ}
\bv{15}{But not as the trespass, so also \supptext{is} the free gift. For if by the trespass of the one the many died, much more did the grace of God, and the gift by the grace of the one man, Jesus Christ, abound unto the many.}
\bv{16}{And not as through one that sinned, \supptext{so} is the gift: for the judgement \supptext{came} of many trespasses unto justification.}
\par
\bv{17}{For if, by the trespass of the one, death reigned through the one; much more shall they that receive the abundance of grace and of the gift of righteousness reign in life through the one, \supptext{even} Jesus Christ.}
\bv{18}{So then as through one trespass \supptext{the judgement came} unto all men to condemnation; even so through one act of righteousness \supptext{the free gift came} unto all men to justification of life.}
\bv{19}{For as through the one man's disobedience the many were made sinners, even so through the obedience of the one shall the many be made righteous.}
\par
\bv{20}{And the law came in besides, that the trespass might abound; but where sin abounded, grace did abound more exceedingly:}
\bv{21}{that, as sin reigned in death, even so might grace reign through righteousness unto eternal life through Jesus Christ our Lord.}
\chaphead{Chapter VI}
\chapdesc{Union with Christ Frees from Sin}
\lettrine[image=true, lines=4, findent=3pt, nindent=0pt]{NT/Romans/What.eps}{hat} shall we say then? Shall we continue in sin, that grace may abound?
\bv{2}{God forbid. We who died to sin, how shall we any longer live therein?}
\bv{3}{Or are ye ignorant that all we who were baptised into Christ Jesus were baptised into his death?}
\bv{4}{We were buried therefore with him through baptism into death: that like as Christ was raised from the dead through the glory of the Father, so we also might walk in newness of life.}
\par
\bv{5}{For if we have become united with \supptext{him} in the likeness of his death, we shall be also \supptext{in the likeness} of his resurrection;}
\bv{6}{knowing this, that our old man was crucified with \supptext{him}, that the body of sin might be done away, that so we should no longer be in bondage to sin;}
\bv{7}{for he that hath died is justified from sin.}
\bv{8}{But if we died with Christ, we believe that we shall also live with him;}
\bv{9}{knowing that Christ being raised from the dead dieth no more; death no more hath dominion over him.}
\bv{10}{For the death that he died, he died unto sin once: but the life that he liveth, he liveth unto God.}
\chapsec{Sanctification Follows Justification}
\bv{11}{Even so reckon ye also yourselves to be dead unto sin, but alive unto God in Christ Jesus.}
\bv{12}{Let not sin therefore reign in your mortal body, that ye should obey the lusts thereof:}
\bv{13}{neither present your members unto sin \supptext{as} instruments of unrighteousness; but present yourselves unto God, as alive from the dead, and your members \supptext{as} instruments of righteousness unto God.}
\chapsec{Law Condemns not the Justified}
\bv{14}{For sin shall not have dominion over you: for ye are not under law, but under grace.}
\bv{15}{What then? shall we sin, because we are not under law, but under grace? God forbid.}
\bv{16}{Know ye not, that to whom ye present yourselves \supptext{as} servants unto obedience, his servants ye are whom ye obey; whether of sin unto death, or of obedience unto righteousness?}
\bv{17}{But thanks be to God, that, whereas ye were servants of sin, ye became obedient from the heart to that form of teaching whereunto ye were delivered;}
\bv{18}{and being made free from sin, ye became servants of righteousness.}
\bv{19}{I speak after the manner of men because of the infirmity of your flesh: for as ye presented your members \supptext{as} servants to uncleanness and to iniquity unto iniquity, even so now present your members \supptext{as} servants to righteousness unto sanctification.}
\bv{20}{For when ye were servants of sin, ye were free in regard of righteousness.}
\bv{21}{What fruit then had ye at that time in the things whereof ye are now ashamed? for the end of those things is death.}
\bv{22}{But now being made free from sin and become servants to God, ye have your fruit unto sanctification, and the end eternal life.}
\bv{23}{For the wages of sin is death; but the free gift of God is eternal life in Christ Jesus our Lord.}
\chaphead{Chapter VII}
\chapdesc{Sin Remains in the Justified}
\lettrine[image=true, lines=4, findent=3pt, nindent=0pt]{NT/Romans/Or.eps}{r} are ye ignorant, brethren (for I speak to men who know the law), that the law hath dominion over a man for so long time as he liveth?
\bv{2}{For the woman that hath a husband is bound by law to the husband while he liveth; but if the husband die, she is discharged from the law of the husband.}
\bv{3}{So then if, while the husband liveth, she be joined to another man, she shall be called an adulteress: but if the husband die, she is free from the law, so that she is no adulteress, though she be joined to another man.}
\bv{4}{Wherefore, my brethren, ye also were made dead to the law through the body of Christ; that ye should be joined to another, \supptext{even} to him who was raised from the dead, that we might bring forth fruit unto God.}
\bv{5}{For when we were in the flesh, the sinful passions, which were through the law, wrought in our members to bring forth fruit unto death.}
\bv{6}{But now we have been discharged from the law, having died to that wherein we were held; so that we serve in newness of the spirit, and not in oldness of the letter.}
\chapsec{Law Does Not Sanctify}
\bv{7}{What shall we say then? Is the law sin? God forbid. Howbeit, I had not known sin, except through the law: for I had not known coveting, except the law had said, ``Thou shalt not covet:''}
\bv{8}{but sin, finding occasion, wrought in me through the commandment all manner of coveting: for apart from the law sin \supptext{is} dead.}
\bv{9}{And I was alive apart from the law once: but when the commandment came, sin revived, and I died;}
\bv{10}{and the commandment, which \supptext{was} unto life, this I found \supptext{to be} unto death:}
\bv{11}{for sin, finding occasion, through the commandment beguiled me, and through it slew me.}
\bv{12}{So that the law is holy, and the commandment holy, and righteous, and good.}
\bv{13}{Did then that which is good become death unto me? God forbid. But sin, that it might be shown to be sin, by working death to me through that which is good;---that through the commandment sin might become exceeding sinful.}
\chapsec{Man's Disordered Desires}
\bv{14}{For we know that the law is spiritual: but I am carnal, sold under sin.}
\bv{15}{For that which I do I know not: for not what I would, that do I practise; but what I hate, that I do.}
\bv{16}{But if what I would not, that I do, I consent unto the law that it is good.}
\bv{17}{So now it is no more I that do it, but sin which dwelleth in me.}
\bv{18}{For I know that in me, that is, in my flesh, dwelleth no good thing: for to will is present with me, but to do that which is good \supptext{is} not.}
\bv{19}{For the good which I would I do not: but the evil which I would not, that I practise.}
\chapsec{Concupiscence is Sin}
\bv{20}{But if what I would not, that I do, it is no more I that do it, but sin which dwelleth in me.}
\bv{21}{I find then the law, that, to me who would do good, evil is present.}
\bv{22}{For I delight in the law of God after the inward man:}
\bv{23}{but I see a different law in my members, warring against the law of my mind, and bringing me into captivity under the law of sin which is in my members.}
\bv{24}{Wretched man that I am! who shall deliver me out of the body of this death?}
\chapsec{The Just Man is Acquitted of his Sin}
\bv{25}{I thank God through Jesus Christ our Lord. So then I of myself with the mind, indeed, serve the law of God; but with the flesh the law of sin.}
\chaphead{Chapter VIII}
\lettrine[image=true, lines=4, findent=3pt, nindent=0pt]{NT/Romans/There.eps}{here} is therefore now no condemnation to them that are in Christ Jesus.
\chapsec{Faith Conquers Law}
\bv{2}{For the law of the Spirit of life in Christ Jesus made me free from the law of sin and of death.}
\bv{3}{For what the law could not do, in that it was weak through the flesh, God, sending his own Son in the likeness of sinful flesh and for sin, condemned sin in the flesh:}
\bv{4}{that the ordinance of the law might be fulfilled in us, who walk not after the flesh, but after the Spirit.}
\chapsec{Conflict of Spirit \& Flesh}
\bv{5}{For they that are after the flesh mind the things of the flesh; but they that are after the Spirit the things of the Spirit.}
\bv{6}{For the mind of the flesh is death; but the mind of the Spirit is life and peace:}
\bv{7}{because the mind of the flesh is enmity against God; for it is not subject to the law of God, neither indeed can it be:}
\bv{8}{and they that are in the flesh cannot please God.}
\bv{9}{But ye are not in the flesh but in the Spirit, if so be that the Spirit of God dwelleth in you. But if any man hath not the Spirit of Christ, he is none of his.}
\bv{10}{And if Christ is in you, the body is dead because of sin; but the spirit is life because of righteousness.}
\bv{11}{But if the Spirit of him that raised up Jesus from the dead dwelleth in you, he that raised up Christ Jesus from the dead shall give life also to your mortal bodies through his Spirit that dwelleth in you.}
\bv{12}{So then, brethren, we are debtors, not to the flesh, to live after the flesh:}
\bv{13}{for if ye live after the flesh, ye must die; but if by the Spirit ye put to death the deeds of the body, ye shall live.}
\chapsec{Born of the Spirit}
\bv{14}{For as many as are led by the Spirit of God, these are sons of God.}
\bv{15}{For ye received not the spirit of bondage again unto fear; but ye received the spirit of adoption, whereby we cry, Abba, Father.}
\bv{16}{The Spirit himself beareth witness with our spirit, that we are children of God:}
\bv{17}{and if children, then heirs; heirs of God, and joint-heirs with Christ; if so be that we suffer with \supptext{him}, that we may be also glorified with \supptext{him}.}
\chapsec{Final Deliverance}
\bv{18}{For I reckon that the sufferings of this present time are not worthy to be compared with the glory which shall be revealed to us-ward.}
\bv{19}{For the earnest expectation of the creation waiteth for the revealing of the sons of God.}
\bv{20}{For the creation was subjected to vanity, not of its own will, but by reason of him who subjected it, in hope}
\bv{21}{that the creation itself also shall be delivered from the bondage of corruption into the liberty of the glory of the children of God.}
\bv{22}{For we know that the whole creation groaneth and travaileth in pain together until now.}
\bv{23}{And not only so, but ourselves also, who have the first-fruits of the Spirit, even we ourselves groan within ourselves, waiting for \supptext{our} adoption, \supptext{to wit}, the redemption of our body.}
\bv{24}{For in hope were we saved: but hope that is seen is not hope: for who hopeth for that which he seeth?}
\bv{25}{But if we hope for that which we see not, \supptext{then} do we with patience wait for it.}
\chapsec{The Spirit Intercedes}
\bv{26}{And in like manner the Spirit also helpeth our infirmity: for we know not how to pray as we ought; but the Spirit himself maketh intercession for \supptext{us} with groanings which cannot be uttered;}
\bv{27}{and he that searcheth the hearts knoweth what is the mind of the Spirit, because he maketh intercession for the saints according to \supptext{the will of} God.}
\chapsec{God's Purpose in the Gospel}
\bv{28}{And we know that to them that love God all things work together for good, \supptext{even} to them that are called according to \supptext{his} purpose.}
\bv{29}{For whom he foreknew, he also foreordained \supptext{to be} conformed to the image of his Son, that he might be the firstborn among many brethren:}
\bv{30}{and whom he foreordained, them he also called: and whom he called, them he also justified: and whom he justified, them he also glorified.}
\bv{31}{What then shall we say to these things? If God \supptext{is} for us, who \supptext{is} against us?}
\bv{32}{He that spared not his own Son, but delivered him up for us all, how shall he not also with him freely give us all things?}
\bv{33}{Who shall lay anything to the charge of God's elect? It is God that justifieth;}
\bv{34}{who is he that condemneth? It is Christ Jesus that died, yea rather, that was raised from the dead, who is at the right hand of God, who also maketh intercession for us.}
\chapsec{Blessed Assurance}
\bv{35}{Who shall separate us from the love of Christ? shall tribulation, or anguish, or persecution, or famine, or nakedness, or peril, or sword?}
\bv{36}{Even as it is written,}
\otQuote{Ps. 44:22}{For thy sake we are killed all the day long;
We were accounted as sheep for the slaughter.}
\bv{37}{Nay, in all these things we are more than conquerors through him that loved us.}
\bv{38}{For I am persuaded, that neither death, nor life, nor angels, nor principalities, nor things present, nor things to come, nor powers,}
\bv{39}{nor height, nor depth, nor any other creature, shall be able to separate us from the love of God, which is in Christ Jesus our Lord.}
\chaphead{Chapter IX}
\chapdesc{Apostolic Solicitude for Israel}
\lettrine[image=true, lines=4, findent=3pt, nindent=0pt]{NT/Romans/I.eps}{} say the truth in Christ, I lie not, my conscience bearing witness with me in the Holy Spirit,
\bv{2}{that I have great sorrow and unceasing pain in my heart.}
\bv{3}{For I could wish that I myself were anathema from Christ for my brethren's sake, my kinsmen according to the flesh:}
\par
\bv{4}{who are Israelites; whose is the adoption, and the glory, and the covenants, and the giving of the law, and the service \supptext{of God}, and the promises;}
\bv{5}{whose are the fathers, and of whom is Christ as concerning the flesh, who is over all, God blessed for ever. Amen.}
\chapsec{True vs. False Israel}
\bv{6}{But \supptext{it is} not as though the word of God hath come to nought. For they are not all Israel, that are of Israel:}
\bv{7}{neither, because they are Abraham's seed, are they all children: but, \otQuote{Gen. 21:12}{In Isaac shall thy seed be called.}}
\par
\bv{8}{That is, it is not the children of the flesh that are children of God; but the children of the promise are reckoned for a seed.}
\bv{9}{For this is a word of promise, \otQuote{Gen. 18:10, 14}{According to this season will I come, and Sarah shall have a son.}}
\bv{10}{And not only so; but Rebecca also having conceived by one, \supptext{even} by our father Isaac---}
\bv{11}{for \supptext{the children} being not yet born, neither having done anything good or bad, that the purpose of God according to election might stand, not of works, but of him that calleth,}
\bv{12}{it was said unto her, \otQuote{Gen. 25:23}{The elder shall serve the younger.}}
\bv{13}{Even as it is written, \otQuote{Mal. 1:2-3}{Jacob I loved, but Esau I hated.}}
\chapsec{Righteousness of God}
\bv{14}{What shall we say then? Is there unrighteousness with God? God forbid.}
\bv{15}{For he saith to Moses, \otQuote{Ex. 33:19}{I will have mercy on whom I have mercy, and I will have compassion on whom I have compassion.}}
\bv{16}{So then it is not of him that willeth, nor of him that runneth, but of God that hath mercy.}
\bv{17}{For the scripture saith unto Pharaoh, \otQuote{Ex. 9:16}{For this very purpose did I raise thee up, that I might show in thee my power, and that my name might be published abroad in all the earth.}}
\bv{18}{So then he hath mercy on whom he will, and whom he will he hardeneth.}
\chapsec{God's Sovereign Election}
\bv{19}{Thou wilt say then unto me, ``Why doth he still find fault? For who withstandeth his will?''\mcomm{ENGLISH FATHER}}
\bv{20}{Nay but, O man, who art thou that repliest against God? Shall the thing formed say to him that formed it, ``Why didst thou make me thus?''}
\bv{21}{Or hath not the potter a right over the clay, from the same lump to make one part a vessel unto honour, and another unto dishonour?}
\bv{22}{What if God, willing to show his wrath, and to make his power known, endured with much longsuffering vessels of wrath fitted unto destruction:}
\bv{23}{and that he might make known the riches of his glory upon vessels of mercy, which he afore prepared unto glory,}
\bv{24}{\supptext{even} us, whom he also called, not from the Jews only, but also from the Gentiles?}
\par
\bv{25}{As he saith also in Hosea,}
\otQuote{Hos. 2:23}{I will call that my people, which was not my people;
And her beloved, that was not beloved.}
\otQuote{Hos. 1:10}{\bv{26}{And it shall be,} \supptext{that} in the place where it was said unto them, ``Ye are not my people,'' There shall they be called sons of the living God.}
\bv{27}{And Isaiah crieth concerning Israel, \otQuote{Is. 10:22-23}{If the number of the children of Israel be as the sand of the sea, it is the remnant that shall be saved:}
\bv{28}{for the Lord will execute \supptext{his} word upon the earth, finishing it and cutting it short.}}
\bv{29}{And, as Isaiah hath said before,}
\otQuote{Is. 1:9}{Except the Lord of Sabaoth had left us a seed,
We had become as Sodom, and had been made like unto Gomorrah.}
\par
\bv{30}{What shall we say then? That the Gentiles, who followed not after righteousness, attained to righteousness, even the righteousness which is of faith:}
\bv{31}{but Israel, following after a law of righteousness, did not arrive at \supptext{that} law.}
\bv{32}{Wherefore? Because \supptext{they sought it} not by faith, but as it were by works. They stumbled at the stone of stumbling;}
\bv{33}{even as it is written,}
\otQuote{Is. 28:16}{Behold, I lay in Zion a stone of stumbling and a rock of offence:
And he that believeth on him shall not be put to shame.}
\chaphead{Chapter X}
\chapdesc{Israel's Faithlessness}
\lettrine[image=true, lines=4, findent=3pt, nindent=0pt]{NT/Romans/Brethren.eps}{rethren}, my heart's desire and my supplication to God is for them, that they may be saved.
\bv{2}{For I bear them witness that they have a zeal for God, but not according to knowledge.}
\bv{3}{For being ignorant of God's righteousness, and seeking to establish their own, they did not subject themselves to the righteousness of God.}
\bv{4}{For Christ is the end of the law unto righteousness to every one that believeth.}
\bv{5}{For Moses writeth\mref{cf. Lev. 18:5} that the man that doeth the righteousness which is of the law shall live thereby.}
\bv{6}{But the righteousness which is of faith saith thus, ``Say not in thy heart, `Who shall ascend into heaven?'{''} (that is, to bring Christ down:)}
\bv{7}{``or, `Who shall descend into the abyss?'{''} (that is, to bring Christ up from the dead.)}
\chapsec{The Saving Word}
\bv{8}{But what saith it? ``The word is nigh thee, in thy mouth, and in thy heart:'' that is, the word of faith, which we preach:}
\bv{9}{because if thou shalt confess with thy mouth Jesus \supptext{as} Lord, and shalt believe in thy heart that God raised him from the dead, thou shalt be saved:}
\bv{10}{for with the heart man believeth unto righteousness; and with the mouth confession is made unto salvation.}
\bv{11}{For the scripture saith, \otQuote{Is. 28:16}{Whosoever believeth on him shall not be put to shame.}}
\bv{12}{For there is no distinction between Jew and Greek: for the same \supptext{Lord} is Lord of all, and is rich unto all that call upon him:}
\bv{13}{for, \otQuote{Jl. 2:32}{Whosoever shall call upon the name of the Lord shall be saved.}}
\bv{14}{How then shall they call on him in whom they have not believed? and how shall they believe in him whom they have not heard? and how shall they hear without a preacher?}
\bv{15}{and how shall they preach, except they be sent? even as it is written, How beautiful are the feet of them that bring glad tidings of good things!}
\bv{16}{But they did not all hearken to the glad tidings. For Isaiah saith, \otQuote{Is. 53:1}{Lord, who hath believed our report?}}
\bv{17}{So belief \supptext{cometh} of hearing, and hearing by the word of Christ.}
\bv{18}{But I say, Did they not hear? Yea, verily,}
\otQuote{Ps. 19:4}{Their sound went out into all the earth,
And their words unto the ends of the world.}
\bv{19}{But I say, Did Israel not know? First Moses saith,}
\otQuote{Deut. 32:21}{I will provoke you to jealousy with that which is no nation,
With a nation void of understanding will I anger you.}
\bv{20}{And Isaiah is very bold, and saith,}
\otQuote{Is. 65:1}{I was found of them that sought me not;
I became manifest unto them that asked not of me.}
\bv{21}{But as to Israel he saith, \otQuote{Is. 65:2}{All the day long did I spread out my hands unto a disobedient and gainsaying people.}}
\chaphead{Chapter XI}
\chapdesc{Faithful Israel is Saved}
\lettrine[image=true, lines=4, findent=3pt, nindent=0pt]{NT/Romans/I.eps}{} say then, Did God cast off his people? God forbid. For I also am an Israelite, of the seed of Abraham, of the tribe of Benjamin.
\bv{2}{God did not cast off his people which he foreknew. Or know ye not what the scripture saith of Elijah? how he pleadeth with God against Israel:}
\otQuote{1 Kgs. 19:10,14}{\bv{3}{Lord, they have killed thy prophets, they have digged down thine altars; and I am left alone, and they seek my life.}}
\bv{4}{But what saith the answer of God unto him? \otQuote{1 Kgs. 19:18}{I have left for myself seven thousand men, who have not bowed the knee to Baal.}}
\bv{5}{Even so then at this present time also there is a remnant according to the election of grace.}
\bv{6}{But if it is by grace, it is no more of works: otherwise grace is no more grace.}
\chapsec{Faithless Israel is Blind}
\bv{7}{What then? That which Israel seeketh for, that he obtained not; but the election obtained it, and the rest were hardened:}
\bv{8}{according as it is written,} \otQuote{Is. 29:10}{God gave them a spirit of stupor, eyes that they should not see, and ears that they should not hear, unto this very day.}
\bv{9}{And David saith,}
\otQuote{Ps. 69:22-23}{Let their table be made a snare, and a trap, And a stumblingblock, and a recompense unto them: \bv{10}{Let their eyes be darkened, that they may not see, And bow thou down their back always.}}
\bv{11}{I say then, Did they stumble that they might fall? God forbid: but by their fall salvation \supptext{is come} unto the Gentiles, to provoke them to jealousy.}
\bv{12}{Now if their fall is the riches of the world, and their loss the riches of the Gentiles; how much more their fulness?}
\chapsec{Warning to the Gentiles}
\bv{13}{But I speak to you that are Gentiles. Inasmuch then as I am an apostle of Gentiles, I glorify my ministry;}
\bv{14}{if by any means I may provoke to jealousy \supptext{them that are} my flesh, and may save some of them.}
\bv{15}{For if the casting away of them \supptext{is} the reconciling of the world, what \supptext{shall} the receiving \supptext{of them be}, but life from the dead?}
\par
\bv{16}{And if the firstfruit is holy, so is the lump: and if the root is holy, so are the branches.}
\bv{17}{But if some of the branches were broken off, and thou, being a wild olive, wast grafted in among them, and didst become partaker with them of the root of the fatness of the olive tree;}
\bv{18}{glory not over the branches: but if thou gloriest, it is not thou that bearest the root, but the root thee.}
\bv{19}{Thou wilt say then, ``Branches were broken off, that I might be grafted in.''}
\bv{20}{Well; by their unbelief they were broken off, and thou standest by thy faith. Be not highminded, but fear:}
\bv{21}{for if God spared not the natural branches, neither will he spare thee.}
\chapsec{The Just Can Fall Away}
\bv{22}{Behold then the goodness and severity of God: toward them that fell, severity; but toward thee, God's goodness, if thou continue in his goodness: otherwise thou also shalt be cut off.}
\bv{23}{And they also, if they continue not in their unbelief, shall be grafted in: for God is able to graft them in again.}
\bv{24}{For if thou wast cut out of that which is by nature a wild olive tree, and wast grafted contrary to nature into a good olive tree; how much more shall these, which are the natural \supptext{branches}, be grafted into their own olive tree?}
\chapsec{Israel's Salvation}
\bv{25}{For I would not, brethren, have you ignorant of this mystery, lest ye be wise in your own conceits, that a hardening in part hath befallen Israel, until the fulness of the Gentiles be come in;}
\bv{26}{and so all Israel shall be saved: even as it is written,}
\otQuote{Is. 59:20-21}{There shall come out of Zion the Deliverer; He shall turn away ungodliness from Jacob: \bv{27}{And this is my covenant unto them, When I shall take away their sins.}}
\bv{28}{As touching the gospel, they are enemies for your sake: but as touching the election, they are beloved for the fathers' sake.}
\bv{29}{For the gifts and the calling of God are not repented of.}
\bv{30}{For as ye in time past were disobedient to God, but now have obtained mercy by their disobedience,}
\bv{31}{even so have these also now been disobedient, that by the mercy shown to you they also may now obtain mercy.}
\bv{32}{For God hath shut up all unto disobedience, that he might have mercy upon all.}
\bv{33}{O the depth of the riches both of the wisdom and the knowledge of God! how unsearchable are his judgements, and his ways past tracing out!}
\bv{34}{For who hath known the mind of the Lord? or who hath been his counsellor?}
\bv{35}{or who hath first given to him, and it shall be recompensed unto him again?}
\bv{36}{For of him, and through him, and unto him, are all things. To him \supptext{be} the glory for ever. Amen.}
\chaphead{Chapter XII}
\chapdesc{Unity of the Church}
\lettrine[image=true, lines=4, findent=3pt, nindent=0pt]{NT/Romans/I.eps}{} beseech you therefore, brethren, by the mercies of God, to present your bodies a living sacrifice, holy, acceptable to God, \supptext{which is} your spiritual service.
\bv{2}{And be not fashioned according to this world: but be ye transformed by the renewing of your mind, that ye may prove what is the good and acceptable and perfect will of God.}
\bv{3}{For I say, through the grace that was given me, to every man that is among you, not to think of himself more highly than he ought to think; but so to think as to think soberly, according as God hath dealt to each man a measure of faith.}
\par
\bv{4}{For even as we have many members in one body, and all the members have not the same office:}
\bv{5}{so we, who are many, are one body in Christ, and severally members one of another.}
\bv{6}{And having gifts differing according to the grace that was given to us, whether prophecy, \supptext{let us prophesy} according to the proportion of our faith;}
\bv{7}{or ministry, \supptext{let us give ourselves} to our ministry; or he that teacheth, to his teaching;}
\bv{8}{or he that exhorteth, to his exhorting: he that giveth, \supptext{let him do it} with liberality; he that ruleth, with diligence; he that showeth mercy, with cheerfulness.}
\chapsec{Perfect Love}
\bv{9}{Let love be without hypocrisy. Abhor that which is evil; cleave to that which is good.}
\bv{10}{In love of the brethren be tenderly affectioned one to another; in honour preferring one another;}
\bv{11}{in diligence not slothful; fervent in spirit; serving the Lord;}
\bv{12}{rejoicing in hope; patient in tribulation; continuing stedfastly in prayer;}
\bv{13}{communicating to the necessities of the saints; given to hospitality.}
\bv{14}{Bless them that persecute you; bless, and curse not.}
\bv{15}{Rejoice with them that rejoice; weep with them that weep.}
\bv{16}{Be of the same mind one toward another. Set not your mind on high things, but condescend to things that are lowly. Be not wise in your own conceits.}
\par
\bv{17}{Render to no man evil for evil. Take thought for things honourable in the sight of all men.}
\bv{18}{If it be possible, as much as in you lieth, be at peace with all men.}
\bv{19}{Avenge not yourselves, beloved, but give place unto the wrath \supptext{of God}: for it is written, Vengeance belongeth unto me; I will recompense, saith the Lord.}
\bv{20}{But if thine enemy hunger, feed him; if he thirst, give him to drink: for in so doing thou shalt heap coals of fire upon his head.}
\bv{21}{Be not overcome of evil, but overcome evil with good.}
\chaphead{Chapter XIII}
\chapdesc{Holy Obedience}
\lettrine[image=true, lines=4, findent=3pt, nindent=0pt]{NT/Romans/Let.eps}{et} every soul be in subjection to the higher powers: for there is no power but of God; and the \supptext{powers} that be are ordained of God.
\bv{2}{Therefore he that resisteth the power, withstandeth the ordinance of God: and they that withstand shall receive to themselves judgement.}
\chapsec{Authority of the King}
\bv{3}{For rulers are not a terror to the good work, but to the evil. And wouldest thou have no fear of the power? do that which is good, and thou shalt have praise from the same:}
\bv{4}{for he is a minister of God to thee for good. But if thou do that which is evil, be afraid; for he beareth not the sword in vain: for he is a minister of God, an avenger for wrath to him that doeth evil.}
\bv{5}{Wherefore \supptext{ye} must needs be in subjection, not only because of the wrath, but also for conscience' sake.}
\bv{6}{For for this cause ye pay tribute also; for they are ministers of God's service, attending continually upon this very thing.}
\bv{7}{Render to all their dues: tribute to whom tribute \supptext{is due}; custom to whom custom; fear to whom fear; honour to whom honour.}
\chapsec{Love of Neighbour}
\bv{8}{Owe no man anything, save to love one another: for he that loveth his neighbour hath fulfilled the law.}
\bv{9}{For this, Thou shalt not commit adultery, Thou shalt not kill, Thou shalt not steal, Thou shalt not covet, and if there be any other commandment, it is summed up in this word, namely, Thou shalt love thy neighbour as thyself.}
\bv{10}{Love worketh no ill to his neighbour: love therefore is the fulfilment of the law.}
\bv{11}{And this, knowing the season, that already it is time for you to awake out of sleep: for now is salvation nearer to us than when we \supptext{first} believed.}
\par
\bv{12}{The night is far spent, and the day is at hand: let us therefore cast off the works of darkness, and let us put on the armor of light.}
\bv{13}{Let us walk becomingly, as in the day; not in revelling and drunkenness, not in chambering and wantonness, not in strife and jealousy.}
\bv{14}{But put ye on the Lord Jesus Christ, and make not provision for the flesh, to \supptext{fulfil} the lusts \supptext{thereof}.}
\chaphead{Chapter XIV}
\chapdesc{Bearing with Doubt}
\lettrine[image=true, lines=4, findent=3pt, nindent=0pt]{NT/Romans/But.eps}{ut} him that is weak in faith receive ye, \supptext{yet} not for decision of scruples.
\bv{2}{One man hath faith to eat all things: but he that is weak eateth herbs.}
\bv{3}{Let not him that eateth set at nought him that eateth not; and let not him that eateth not judge him that eateth: for God hath received him.}
\bv{4}{Who art thou that judgest the servant of another? to his own lord he standeth or falleth. Yea, he shall be made to stand; for the Lord hath power to make him stand.}
\chapsec{Christian Liberty}
\bv{5}{One man esteemeth one day above another: another esteemeth every day \supptext{alike}. Let each man be fully assured in his own mind.}
\bv{6}{He that regardeth the day, regardeth it unto the Lord: and he that eateth, eateth unto the Lord, for he giveth God thanks; and he that eateth not, unto the Lord he eateth not, and giveth God thanks.}
\bv{7}{For none of us liveth to himself, and none dieth to himself.}
\bv{8}{For whether we live, we live unto the Lord; or whether we die, we die unto the Lord: whether we live therefore, or die, we are the Lord's.}
\bv{9}{For to this end Christ died and lived \supptext{again}, that he might be Lord of both the dead and the living.}
\bv{10}{But thou, why dost thou judge thy brother? or thou again, why dost thou set at nought thy brother? for we shall all stand before the judgement-seat of God.}
\bv{11}{For it is written,}
\otQuote{Is. 45:23}{As I live, saith the Lord, to me every knee shall bow,
And every tongue shall confess to God.}
\bv{12}{So then each one of us shall give account of himself to God.}
\chapsec{Christian Judgement}
\bv{13}{Let us not therefore judge one another any more: but judge ye this rather, that no man put a stumblingblock in his brother's way, or an occasion of falling.}
\bv{14}{I know, and am persuaded in the Lord Jesus, that nothing is unclean of itself: save that to him who accounteth anything to be unclean, to him it is unclean.}
\bv{15}{For if because of meat thy brother is grieved, thou walkest no longer in love. Destroy not with thy meat him for whom Christ died.}
\bv{16}{Let not then your good be evil spoken of:}
\bv{17}{for the kingdom of God is not eating and drinking, but righteousness and peace and joy in the Holy Spirit.}
\bv{18}{For he that herein serveth Christ is well-pleasing to God, and approved of men.}
\bv{19}{So then let us follow after things which make for peace, and things whereby we may edify one another.}
\chapsec{Importance of Conscience}
\bv{20}{Overthrow not for meat's sake the work of God. All things indeed are clean; howbeit it is evil for that man who eateth with offence.}
\bv{21}{It is good not to eat flesh, nor to drink wine, nor \supptext{to do anything} whereby thy brother stumbleth.}
\bv{22}{The faith which thou hast, have thou to thyself before God. Happy is he that judgeth not himself in that which he approveth.}
\bv{23}{But he that doubteth is condemned if he eat, because \supptext{he eateth} not of faith; and whatsoever is not of faith is sin.}
\chaphead{Chapter XV}
\chapdesc{Burden of the Strong}
\lettrine[image=true, lines=4, findent=3pt, nindent=0pt]{NT/Romans/Now.eps}{ow} we that are strong ought to bear the infirmities of the weak, and not to please ourselves.
\bv{2}{Let each one of us please his neighbour for that which is good, unto edifying.}
\bv{3}{For Christ also pleased not himself; but, as it is written, \shortQ{Ps. 69:9}{The reproaches of them that reproached thee fell upon me.}}
\chapsec{Christian Unity}
\bv{4}{For whatsoever things were written aforetime were written for our learning, that through patience and through comfort of the scriptures we might have hope.}
\bv{5}{Now the God of patience and of comfort grant you to be of the same mind one with another according to Christ Jesus:}
\bv{6}{that with one accord ye may with one mouth glorify the God and Father of our Lord Jesus Christ.}
\bv{7}{Wherefore receive ye one another, even as Christ also received you, to the glory of God.}
\bv{8}{For I say that Christ hath been made a minister of the circumcision for the truth of God, that he might confirm the promises \supptext{given} unto the fathers,}
\bv{9}{and that the Gentiles might glorify God for his mercy; as it is written,}
\otQuote{Ps. 18:49}{Therefore will I give praise unto thee among the Gentiles,
And sing unto thy name.}
\bv{10}{And again he saith,}
\otQuote{Deut. 32:43}{Rejoice, ye Gentiles, with his people.}
\bv{11}{And again,}
\otQuote{Ps. 117:1}{Praise the Lord, all ye Gentiles; And let all the peoples praise him.}
\bv{12}{And again, Isaiah saith,}
\otQuote{Is. 11:10}{There shall be the root of Jesse, And he that ariseth to rule over the Gentiles; On him shall the Gentiles hope.}
\bv{13}{Now the God of hope fill you with all joy and peace in believing, that ye may abound in hope, in the power of the Holy Spirit.}
\chapsec{St. Paul's Ministry}
\bv{14}{And I myself also am persuaded of you, my brethren, that ye yourselves are full of goodness, filled with all knowledge, able also to admonish one another.}
\bv{15}{But I write the more boldly unto you in some measure, as putting you again in remembrance, because of the grace that was given me of God,}
\bv{16}{that I should be a minister of Christ Jesus unto the Gentiles, ministering the gospel of God, that the offering up of the Gentiles might be made acceptable, being sanctified by the Holy Spirit.}
\chapsec{St. Paul's Fruit}
\bv{17}{I have therefore my glorying in Christ Jesus in things pertaining to God.}
\bv{18}{For I will not dare to speak of any things save those which Christ wrought through me, for the obedience of the Gentiles, by word and deed,}
\bv{19}{in the power of signs and wonders, in the power of the Holy Spirit; so that from Jerusalem, and round about even unto Illyricum, I have fully preached the gospel of Christ;}
\bv{20}{yea, making it my aim so to preach the gospel, not where Christ was \supptext{already} named, that I might not build upon another man's foundation;}
\bv{21}{but, as it is written,}
\otQuote{Is. 52:15}{They shall see, to whom no tidings of him came, And they who have not heard shall understand.}
\bv{22}{Wherefore also I was hindered these many times from coming to you:}
\bv{23}{but now, having no more any place in these regions, and having these many years a longing to come unto you,}
\bv{24}{whensoever I go unto Spain (for I hope to see you in my journey, and to be brought on my way thitherward by you, if first in some measure I shall have been satisfied with your company)—}
\bv{25}{but now, \supptext{I say}, I go unto Jerusalem, ministering unto the saints.}
\bv{26}{For it hath been the good pleasure of Macedonia and Achaia to make a certain contribution for the poor among the saints that are at Jerusalem.}
\bv{27}{Yea, it hath been their good pleasure; and their debtors they are. For if the Gentiles have been made partakers of their spiritual things, they owe it \supptext{to them} also to minister unto them in carnal things.}
\bv{28}{When therefore I have accomplished this, and have sealed to them this fruit, I will go on by you unto Spain.}
\bv{29}{And I know that, when I come unto you, I shall come in the fulness of the blessing of Christ.}
\bv{30}{Now I beseech you, brethren, by our Lord Jesus Christ, and by the love of the Spirit, that ye strive together with me in your prayers to God for me;}
\bv{31}{that I may be delivered from them that are disobedient in Jud{\ae}a, and \supptext{that} my ministration which \supptext{I have} for Jerusalem may be acceptable to the saints;}
\bv{32}{that I may come unto you in joy through the will of God, and together with you find rest.}
\bv{33}{Now the God of peace be with you all. Amen.}
\chaphead{Chapter XVI}
\chapdesc{Salutations}
\lettrine[image=true, lines=4, findent=3pt, nindent=0pt]{NT/Romans/I.eps}{} commend unto you Phoebe our sister, who is a servant of the church that is at Cenchre{\ae}:
\bv{2}{that ye receive her in the Lord, worthily of the saints, and that ye assist her in whatsoever matter she may have need of you: for she herself also hath been a helper of many, and of mine own self.}
\bv{3}{Salute Prisca and Aquila my fellow-workers in Christ Jesus,}
\bv{4}{who for my life laid down their own necks; unto whom not only I give thanks, but also all the churches of the Gentiles:}
\bv{5}{and \supptext{salute} the church that is in their house. Salute Ep{\ae}netus my beloved, who is the firstfruits of Asia unto Christ.}
\par
\bv{6}{Salute Mary, who bestowed much labor on you.}
\bv{7}{Salute Andronicus and Junias, my kinsmen, and my fellow-prisoners, who are of note among the apostles, who also have been in Christ before me.}
\bv{8}{Salute Ampliatus my beloved in the Lord.}
\bv{9}{Salute Urbanus our fellow-worker in Christ, and Stachys my beloved.}
\bv{10}{Salute Apelles the approved in Christ. Salute them that are of the \supptext{household} of Aristobulus.}
\bv{11}{Salute Herodion my kinsman. Salute them of the \supptext{household} of Narcissus, that are in the Lord.}
\bv{12}{Salute Tryph{\ae}na and Tryphosa, who labor in the Lord. Salute Persis the beloved, who labored much in the Lord.}
\bv{13}{Salute Rufus the chosen in the Lord, and his mother and mine.}
\bv{14}{Salute Asyncritus, Phlegon, Hermes, Patrobas, Hermas, and the brethren that are with them.}
\bv{15}{Salute Philologus and Julia, Nereus and his sister, and Olympas, and all the saints that are with them.}
\par
\bv{16}{Salute one another with a holy kiss. All the churches of Christ salute you.}
\bv{17}{Now I beseech you, brethren, mark them that are causing the divisions and occasions of stumbling, contrary to the doctrine which ye learned: and turn away from them.}
\bv{18}{For they that are such serve not our Lord Christ, but their own belly; and by their smooth and fair speech they beguile the hearts of the innocent.}
\bv{19}{For your obedience is come abroad unto all men. I rejoice therefore over you: but I would have you wise unto that which is good, and simple unto that which is evil.}
\bv{20}{And the God of peace shall bruise Satan under your feet shortly.
The grace of our Lord Jesus Christ be with you.}
\par
\bv{21}{Timothy my fellow-worker saluteth you; and Lucius and Jason and Sosipater, my kinsmen.}
\bv{22}{I Tertius, who write the epistle, salute you in the Lord.}
\bv{23}{Gaius my host, and of the whole church, saluteth you. Erastus the treasurer of the city saluteth you, and Quartus the brother.\mcomm{The grace of our Lord Jesus Christ be with you all.}}
\bv{25}{Now to him that is able to establish you according to my gospel and the preaching of Jesus Christ, according to the revelation of the mystery which hath been kept in silence through times eternal,}
\bv{26}{but now is manifested, and by the scriptures of the prophets, according to the commandment of the eternal God, is made known unto all the nations unto obedience of faith:}
\bv{27}{to the only wise God, through Jesus Christ, to whom be the glory for ever. Amen.}
	\clearpage
	\chapter{The Epistle of St. Paul to Philemon}
\fancyhead[RE,LO]{Philemon}
\chaphead{Chapter I}
\chapdesc{Apostolic Greeting}
\lettrine[image=true, lines=4, findent=3pt, nindent=0pt]{NT/Philemon/Phil-Paul.eps}{aul}, a prisoner of Christ Jesus, and Timothy our brother, to Philemon our beloved and fellow-worker,
\bv{2}{and to Apphia our sister, and to Archippus our fellow-soldier, and to the church in thy house:}
\bv{3}{Grace to you and peace from God our Father and the Lord Jesus Christ.}
\chapsec{Character of Philemon}
\bv{4}{I thank my God always, making mention of thee in my prayers,}
\bv{5}{hearing of thy love, and of the faith which thou hast toward the Lord Jesus, and toward all the saints;}
\bv{6}{that the fellowship of thy faith may become effectual, in the knowledge of every good thing which is in you, unto Christ.}
\bv{7}{For I had much joy and comfort in thy love, because the hearts of the saints have been refreshed through thee, brother.}
\chapsec{Intercession for Onesimus}
\bv{8}{Wherefore, though I have all boldness in Christ to enjoin thee that which is befitting,}
\bv{9}{yet for love’s sake I rather beseech, being such a one as Paul the aged, and now a prisoner also of Christ Jesus:}
\bv{10}{I beseech thee for my child, whom I have begotten in my bonds, Onesimus,}
\bv{11}{who once was unprofitable to thee, but now is profitable to thee and to me:}
\bv{12}{whom I have sent back to thee in his own person, that is, my very heart:}
\bv{13}{whom I would fain have kept with me, that in thy behalf he might minister unto me in the bonds of the gospel:}
\bv{14}{but without thy mind I would do nothing; that thy goodness should not be as of necessity, but of free will.}
\par
\bv{15}{For perhaps he was therefore parted \supptext{from thee} for a season, that thou shouldest have him for ever;}
\bv{16}{no longer as a servant, but more than a servant, a brother beloved, specially to me, but how much rather to thee, both in the flesh and in the Lord.}
\bv{17}{If then thou countest me a partner, receive him as myself.}
\bv{18}{But if he hath wronged thee at all, or oweth \supptext{thee} aught, put that to mine account;}
\bv{19}{I Paul write it with mine own hand, I will repay it: that I say not unto thee that thou owest to me even thine own self besides.}
\bv{20}{Yea, brother, let me have joy of thee in the Lord: refresh my heart in Christ.}
\bv{21}{Having confidence in thine obedience I write unto thee, knowing that thou wilt do even beyond what I say.}
\chapsec{Salutations \& Conclusion}
\bv{22}{But withal prepare me also a lodging: for I hope that through your prayers I shall be granted unto you.}
\bv{23}{Epaphras, my fellow-prisoner in Christ Jesus, saluteth thee;}
\bv{24}{\supptext{and so do} Mark, Aristarchus, Demas, Luke, my fellow-workers.}
\bv{25}{The grace of our Lord Jesus Christ be with your spirit. Amen.}
	\clearpage
	\input{./books/Catholic/1John.tex}
	\clearpage
	\chapter{The Second Epistle of Saint John}
\fancyhead[RE,LO]{The Second Epistle of John}
%\chaphead{Chapter I}
\chapdesc{Truth \& Love Inseparable}
\lettrine[image=true, lines=4, findent=3pt, nindent=0pt]{T-2Jn.eps}{he} elder unto the elect lady and her children, whom I love in truth; and not I only, but also all they that know the truth;
\bv{2}{for the truth’s sake which abideth in us, and it shall be with us for ever:}
\bv{3}{Grace, mercy, peace shall be with us, from God the Father, and from Jesus Christ, the Son of the Father, in truth and love.}
\par
\bv{4}{I rejoice greatly that I have found \supptext{certain} of thy children walking in truth, even as we received commandment from the Father.}
\bv{5}{And now I beseech thee, lady, not as though I wrote to thee a new commandment, but that which we had from the beginning, that we love one another.}
\bv{6}{And this is love, that we should walk after his commandments. This is the commandment, even as ye heard from the beginning, that ye should walk in it.}
\chapsec{Doctrine: The Test of Reality}
\bv{7}{For many deceivers are gone forth into the world, \supptext{even} they that confess not that Jesus Christ cometh in the flesh. This is the deceiver and the antichrist.}
\bv{8}{Look to yourselves, that ye lose not the things which we have wrought, but that ye receive a full reward.}
\bv{9}{Whosoever goeth onward and abideth not in the teaching of Christ, hath not God: he that abideth in the teaching, the same hath both the Father and the Son.}
\bv{10}{If any one cometh unto you, and bringeth not this teaching, receive him not into \supptext{your} house, and give him no greeting:}
\bv{11}{for he that giveth him greeting partaketh in his evil works.}
\chapsec{Superscription}
\bv{12}{Having many things to write unto you, I would not \supptext{write them} with paper and ink: but I hope to come unto you, and to speak face to face, that your joy may be made full.}
\bv{13}{The children of thine elect sister salute thee.}
	\clearpage
	\input{./books/Catholic/3John.tex}
	\clearpage
	\input{./books/Catholic/Jude.tex}
	%\clearpage
	%\input{books/Revelation/Rev.tex}
%Translation decisions so far:
%sick of the palsy -> paralytic
%baptize -> baptise
%John 18:5-6: I am he to I AM
%neighbor -> neighbour
%Holy Spirit -> Holy Ghost
%Zachariah -> Zechariah
%Matthew 27:50: yielded up his spirit -> yielded up the ghost
%John 8:58: before Abraham was born -> before Abraham was
%favor -> favour
%vapor -> vapour
%honor -> honour
%judgment -> judgement
\end{document}