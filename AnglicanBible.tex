\documentclass[twocolumn,twoside,titlepage,10pt]{book}\usepackage[paperheight=210mm,paperwidth=148mm,inner=25.4mm,outer=20mm,top=15mm,bottom=15mm,heightrounded]{geometry}
\usepackage{fontspec}
\setmainfont[Numbers={Lowercase, Proportional}, Ligatures={Common, Rare}]{EB Garamond}
%Margins...
\usepackage{marginnote}
\usepackage[switch*]{lineno}
\usepackage{nicefrac, xfrac}
\usepackage{cjhebrew}
\usepackage[noorphans=true,indentfirst=false]{quoting}
\usepackage{ragged2e}
%Middle Line Separating Columns
\setlength{\columnseprule}{0.4pt}

%For First Letter Images
\usepackage{graphicx}
\graphicspath{ {images/vector}}
\usepackage{wrapfig}
\usepackage{epsfig}
\usepackage{type1cm, lettrine}

%Footnotes
\usepackage[hang, flushmargin,perpage]{footmisc}
\renewcommand{\thefootnote}{\alph{footnote}}
%\renewcommand{\thefootnote}{\fnsymbol{footnote}}

%Bible Verses
\newcommand{\bv}[2]{\textsuperscript{#1}{#2}}

%\newcommand{\qv}[4]{
	%\begin{quoting}
	%\marginnote{\textit{\small{#1}}}
	%{\textsuperscript{#3}}#2
	%\end{quoting}
	%#4}
%\newcommand{\qvv}[4]{
	%\begin{quoting}
	%{\textsuperscript{#3}}#2
	%\end{quoting}
	%\marginnote{\textit{\small{#1}}}
	%#4}
\interfootnotelinepenalty=10000

%For when a contemporary character has a long quotation
\newcommand{\canticle}[1]{
\par
\noindent
%\hspace{2em}#1
\begin{verse}
	#1
\end{verse}
%\begin{quotation}
%	#1
%\end{quotation}
}
\newcommand{\VN}[1]{
\textsuperscript{#1}%
}
%Old Testament Quote
\newcommand{\otQuote}[2]{
\par
\noindent
	%\begin{verse}
	%	#2\mref{#1}%
	%\end{verse}
	\begin{quoting}
	%\mref{#1}
	#2\footnote{#1}%
	\end{quoting}
}
\newcommand{\shortQ}[2]{
	``#2''\footnote{#1}%
}
%Marginal Glosses \& References
\usepackage{microtype}
\usepackage[Ragged, shape=up, size=script]{sidenotesplus}
\newcommand{\mcomm}[1]{\sidenote*{#1}}
\newcommand{\forcewhite}[1]{\textcolor{White}{#1}}
\newcommand{\mref}[1]{
%\marginpar{\small{\textit{#1}}}
\footnote{#1}
}
%Different font for editorial additions.
\newcommand{\supptext}[1]{%
\texttt{\footnotesize{#1}}%
}

\newcommand{\chaphead}[1]{
\fancyhead[LE,RO]{#1}
\section{#1}
}
\newcommand{\chapdesc}[1]{
	%\marginpar{\small{\textit{#1}}}
	%
	%\begin{center}
	%\small{\textit{#1}}
	%\end{center}
	%\par
	%\noindent
	\subsection{#1}
	\noindent
}
%Formatting for the description of sections within the chapter.
\newcommand{\chapsec}[1]{
	%\par
	%\marginpar{\\{\small{\textit{#1}}}}
	%\subsection{#1}
	\par
	\begin{center}
		\textit{#1}
	\end{center}
	\par
	\noindent
}
%Try to avoid chapters at bottom.
\usepackage[nobottomtitles*]{titlesec}
%Trying to fix marginnotes for left text fixing on right...

%Red Letter Option
\usepackage[dvipsnames]{xcolor}
%\newcommand{\redlet}[1]{\textcolor{BrickRed}{#1}}
\newcommand{\redlet}[1]{#1}
\newcommand{\god}[1]{#1}
%\newcommand{\ann}[1]{\footnote[8]{#1}}
%Hanging chapter descriptions
\usepackage{hanging}
\frenchspacing
%\usepackage{mparhack}
%Page Headers
\usepackage{fancyhdr}
\pagestyle{fancy}
\fancyhf{}
\fancyfoot[C]{\thepage}
\fancypagestyle{plain}{
	\fancyhf{}
	\fancyfoot[C]{}
	\renewcommand{\headrulewidth}{0pt}
}
%Adjust Headings
\usepackage{titlesec}
	\titleclass{\chapter}{straight}
	\titleformat{\chapter}[display]{\normalfont\bfseries\scshape\centering}{}{0pt}{\Large}
	\titlespacing*{\chapter}{0pt}{0pt}{5pt}
	\titleformat{\section}[hang]{\centering}{}{0pt}{\large}
	\titlespacing*{\section}{0pt}{2ex}{2ex}
	\titleformat{\subsection}[hang]{\centering}{}{0pt}{\small \itshape}
	\titlespacing*{\subsection}{0pt}{1ex}{1ex}
\begin{document}
	\begin{titlepage}
		\begin{center}
			\vspace*{2cm}
			\par
			{
			\textsc{
			\LARGE{The}\\
			\vspace{1ex}
			\Huge{New}\\
			\vspace{1.5ex}
			\huge{Testament}
			}\\
			\vspace{2ex}
			%\Large{Containing the Old Testament,\\
			%the Apocrypha,\\
			%and the New Testament}
			%\Large{Containing the New Testament}
			\Large{of Our Lord Jesus Christ}
			\par
			\vspace{.5cm}
			%\large{Revised according to the Original Languages}
			\large{\textit{Anglican Patrimony Version}}
			}
			\par
			\vspace{1cm}
			%{\textit{Anglican Patrimony Version}}
			\par
			%\vfill
			\includegraphics[scale=.3]{Pantokrator2.eps}
			\par
			\vspace{1cm}
			\textsc{Anno Domini 2022}
		\end{center}
	\end{titlepage}
    \begin{onecolumn}
	\chapter{Copyright}
	\copyright 2023 Apologia Anglicana, LLC. Some Rights Reserved. Creative Commons Attribution-ShareAlike 4.0 International (https://creativecommons.org/licenses/by-sa/4.0/).
	\end{onecolumn}
	\begin{onecolumn}
	\chapter{Why a New Edition?}
	Ever since the middle twentieth century, there has been a flood of new biblical translations. In recent years, it seems like the more the Bible is translated, the less it is read. With so many translations, what is the need for the \textit{Anglican Patrimony Version}? With all of these many great Bibles, none have sought to translate the Bible with the wisdom of the Anglican tradition. In this version, you will find a translation that is both critical and beautiful.
	\section{Critical \& Beautiful}
The translators of the King James Bible sought to create a translation that used the best original language manuscripts and was equally beautiful \& understandable by the average parishioner. Sadly, the latter has often been neglected, losing the beauty of the King James Bible for post-modern language. It is confusing why the Anglican churches would expect their parishioners to not understand the same language they hear in the liturgy every Sunday and every day during the Daily Office, such as ``thou'' and ``ye.'' Or, even worse, some churches which call themselves ``Anglican'' or ``Episcopal'' have abandoned this treasure of the Anglican patrimony altogether. It is out of love for this treasure that we produce the \textit{Anglican Patrimony Version}.
\section{Textual Basis \& Translation Decisions}
The \textit{Anglican Patrimony Version} is a fresh typeset of the American Standard Version, with light modifications. The ASV did a good job to conservatively revise the Authorised Version with the wisdom of textual criticism. It also sought, again sparingly, to update the language when it was truly necessary and expedient. The goal of this edition is to continue as close to the American Standard Version as possible and to change the base text sparingly, only when necessary to:
\begin{enumerate}
\item Conform it to the earliest manuscripts we have.
\item Replace words that truly cannot properly communicate the underlying meaning to the average church-going Anglican without a dictionary.
\end{enumerate}
We hope you come to value and appreciate God's Word by means of this version.
	\end{onecolumn}
	\twocolumn
	\newpage
	\setcounter{tocdepth}{0}
	\tableofcontents
	\newpage
	\begin{onecolumn}	
\chapter{The Problem of Textual Criticism}
The reality of textual criticism has posed quite the difficulty for many churches, leading to great confusion and schisms.
\par
The \textit{Anglican Patrimony Version} is meant to be read, both in personal use and in the parish setting. Therefore, it does not make extensive use of textual footnotes and explanations. However, great care has been taken to be faithful to the original languages, as we best understand them. Therefore, textual criticism has been essential for this edition. Due to this, I thought it critical to provide not just an academic explanation of textual criticism but a theological reflection and explanation for what textual criticism \textit{means} (instead of leaving vague footnotes to confuse the reader like many recent translations have sadly done).
\section{What is Textual Criticism?}
Textual criticism is the result of three realities:
\begin{enumerate}
	\item The original autographs of the Scriptures were written by God through human authors.
	\item We do not possess those original autographs.
	\item The manuscripts---of those autographs---which we possess differ from each other in certain places.
\end{enumerate}
The goal of the text critic is to discern which variation reflects what was actually present in the original autograph.
\section{An Unsure Text?}
Some Christians desire to dismiss textual criticism categorically. If the Catholic Church is guided by the Holy Ghost, how could it lose the original text? Also, if the Church is the guardian of the Sacred Scriptures, then how could it fail in its mission so as to not know what those Scriptures are? Even worse, does the Church then need to depend on the natural sciences to perform its supernatural mission?

These are all very good concerns, and they need to be taken seriously and answered before we move forward with making critical editions of the Scriptures.
\subsection{Was the Original Text Lost?}
To the first point, the issue our current situation presents is not having lost text and needing to recover it. It is much better! Rather, we have too much text. For example, we know that, over time, Christians were reading and using their Bibles. Sometimes they wrote theological commentary in the margin. Unfortunately, the margin is also where a scribe would write a verse he initially forgot to copy in the body of the text (they didn't have erasers or White-Out). So, sometimes, the commentary would be copied into the main text by a later copyist: intermingling the inspired text of Sacred Scripture with commentary on that Scripture. This is not the only problem that can occur in transmission, but it shows that we are not digging for ``lost books'' or ``lost verses'' to add. Likewise, we are not trying to remove inspired text from the Bible. Instead, the goal of the text critic is to untangle what scribes centuries before have tangled and intermingled.

The original text is not lost. It is simply that non-inspired words---or even sometimes mis-copying or accidentally skipping a line---have been included in some manuscripts (though those same errors do not happen in others). The issue then is detangling the text based on all of the manuscripts we have.
\subsection{How Does the Church Guard the Scriptures?}
To the second point, the Word of God constitutes the Catholic Church, and the Catholic Church receives the Word of God and cherishes it. However, at its divine founding, the Church's understanding of the Sacred Scriptures was not immediate and fully comprehensive. Instead, since the Church is guided by the Holy Ghost, it discerns the Scriptures over time: both their meaning and extent.

This is seen in the canon of the Scriptures. There was a time when there was not catholic agreement on the canon of the New Testament. Yet, over time, the Holy Ghost guided the Church, and the Church listened to the Good Shepherd's voice, receiving only those books which are inspired. Similarly, the Church right now is discerning His voice regarding the Old Testament canon, where there is not catholic agreement on exactly which books belong. In the same way, there is a discernment that happens over time regarding verses. This is not failure: it is simply the divine timescale. And part of living on the divine timescale is reforming ourselves constantly unto the law of God.

Though, it must be said, it is the sad reality that the Church can perform her mission imperfectly. Accretions can develop which obscure the face of Christ, hinder the work of the Church, and even cause theological missteps. However, at the same time, the Sacred Scriptures can be ornamented with textual jewels, so to say, of good commentary. While the technical language for these interpolations is ``corruption,'' it is the editor's opinion that most of them are better called ornamentation. And while the ornamentation itself is not bad, it is necessary to be able look at the image of Christ as it is purely, without ornamentation, lest the ornaments be confused for the image. This is the work of textual criticism: discern the inspired and non-inspired words, rejecting nothing that is good but putting it in its proper place.

\subsection{Does the Church Need Natural Science?}
To the third point, the Catholic Church has always needed natural realities for her mission. That is at the essence of the Incarnation. Natural realities are never destroyed but raised up for a supernatural purpose: Grace perfects (not destroys) nature. Therefore, the natural and supernatural ought not to be put in opposition. Instead, the natural---such as natural law, natural substances, and natural skills---are embedded and incorporated into the Scriptures: into the very promises of God. For example, to be faithful to the Sacred Scriptures, knowledge of the natural language of Greek is necessary (something the Western Church sadly neglected for some time). Likewise, to give the Scriptures to others, it requires gathering paper and collecting \& copying manuscripts and giving them to others: all of which are natural skills and substances. It is at this point where the natural science of textual criticism exists at the service of the Word for the Church.
\subsection{Conclusion}
I hope this provides a helpful primer and apologetic to textual criticism. This critical edition has been prepared out of love for the Scriptures: the key treasure \& guiding light of the Church. So, while some verses may be moved into the margins or a separate section, this is not to remove and undermine the Word of God. Instead, it is to honour the Word of God by keeping the inspired text front-and-centre and the excellent notes and spiritual commentary ready on the side for Christians to appreciate in its proper context.
\end{onecolumn}
	%\onecolumn
\vspace*{50ex}
\begin{center}
	\rule{15em}{.25mm}\\
	\vspace{1.5ex}
	{\Large{\scshape The Old Testament}}
	\par
	\rule{15em}{.25mm}
\end{center}
\vfill
\twocolumn
	%\chapter{The Book of the Prophet Malachi}
\fancyhead[RE,LO]{The Book of Malachi}
\chaphead{Chapter I}
\chapdesc{God's Love for Israel}
\lettrine[image=true, lines=4, findent=3pt, nindent=0pt]{T-Mal.eps}{he} burden of the word of Yahweh to Israel by Malachi.
\bv{2}{I have loved you, saith Yahweh. Yet ye say, Wherein hast thou loved us? Was not Esau Jacob's brother? saith Yahweh: yet I loved Jacob;}
\bv{3}{but Esau I hated, and made his mountains a desolation, and \supptext{gave} his heritage to the jackals of the wilderness.}
\bv{4}{Whereas Edom saith, We are beaten down, but we will return and build the waste places; thus saith Yahweh of hosts, They shall build, but I will throw down; and men shall call them The border of wickedness, and The people against whom Yahweh hath indignation for ever.}
\bv{5}{And your eyes shall see, and ye shall say, Yahweh be magnified beyond the border of Israel.}
\chapsec{The Sins of the Restoration Priests}
\bv{6}{A son honoreth his father, and a servant his master: if then I am a father, where is mine honor? and if I am a master, where is my fear? saith Yahweh of hosts unto you, O priests, that despise my name. And ye say, Wherein have we despised thy name?}
\bv{7}{Ye offer polluted bread upon mine altar. And ye say, Wherein have we polluted thee? In that ye say, The table of Yahweh is contemptible.}
\bv{8}{And when ye offer the blind for sacrifice, it is no evil! and when ye offer the lame and sick, it is no evil! Present it now unto thy governor; will he be pleased with thee? or will he accept thy person? saith Yahweh of hosts.}
\bv{9}{And now, I pray you, entreat the favor of God, that he may be gracious unto us: this hath been by your means: will he accept any of your persons? saith Yahweh of hosts.}
\par
\bv{10}{Oh that there were one among you that would shut the doors, that ye might not kindle \supptext{fire on} mine altar in vain! I have no pleasure in you, saith Yahweh of hosts, neither will I accept an offering at your hand.}
\bv{11}{For from the rising of the sun even unto the going down of the same my name \supptext{shall be} great among the Gentiles; and in every place incense \supptext{shall be} great among the Gentiles, saith Yahweh of hosts.}
\bv{12}{But ye profane it, in that ye say, The table of Yahweh is polluted, and the fruit thereof, even its food, is contemptible.}
\par
\bv{13}{Ye say also, Behold, what a weariness is it! and ye have snuffed at it, saith Yahweh of hosts; and ye have brought that which was taken by violence, and the lame, and the sick; thus ye bring the offering: should I accept this at your hand? saith Yahweh.}
\bv{14}{But cursed be the deceiver, who hath in his flock a male, and voweth, and sacrificeth unto the Lord a blemished thing; for I am a great King, saith Yahweh of hosts, and my name is terrible among the Gentiles.}
\chaphead{Chapter II}
\chapdesc{Message to the Priests Continued}
\lettrine[image=true, lines=4, findent=3pt, nindent=0pt]{A-Mal.eps}{nd} now, O ye priests, this commandment is for you.
\bv{2}{If ye will not hear, and if ye will not lay it to heart, to give glory unto my name, saith Yahweh of hosts, then will I send the curse upon you, and I will curse your blessings; yea, I have cursed them already, because ye do not lay it to heart.}
\bv{3}{Behold, I will rebuke your seed, and will spread dung upon your faces, even the dung of your feasts; and ye shall be taken away with it.}
\bv{4}{And ye shall know that I have sent this commandment unto you, that my covenant may be with Levi, saith Yahweh of hosts.}
\bv{5}{My covenant was with him of life and peace; and I gave them to him that he might fear; and he feared me, and stood in awe of my name.}
\bv{6}{The law of truth was in his mouth, and unrighteousness was not found in his lips: he walked with me in peace and uprightness, and turned many away from iniquity.}
\bv{7}{For the priest's lips should keep knowledge, and they should seek the law at his mouth; for he is the messenger of Yahweh of hosts.}
\bv{8}{But ye are turned aside out of the way; ye have caused many to stumble in the law; ye have corrupted the covenant of Levi, saith Yahweh of hosts.}
\bv{9}{Therefore have I also made you contemptible and base before all the people, according as ye have not kept my ways, but have had respect of persons in the law.}
\chapsec{Sins against Brotherhood}
\bv{10}{Have we not all one father? hath not one God created us? why do we deal treacherously every man against his brother, profaning the covenant of our fathers?}
\chapsec{Sins against Family}
\bv{11}{Judah hath dealt treacherously, and an abomination is committed in Israel and in Jerusalem; for Judah hath profaned the holiness of Yahweh which he loveth, and hath married the daughter of a foreign god.}
\bv{12}{Yahweh will cut off, to the man that doeth this, him that waketh and him that answereth, out of the tents of Jacob, and him that offereth an offering unto Yahweh of hosts.}
\bv{13}{And this again ye do: ye cover the altar of Yahweh with tears, with weeping, and with sighing, insomuch that he regardeth not the offering any more, neither receiveth it with good will at your hand.}
\bv{14}{Yet ye say, Wherefore? Because Yahweh hath been witness between thee and the wife of thy youth, against whom thou hast dealt treacherously, though she is thy companion, and the wife of thy covenant.}
\bv{15}{And did he not make one, although he had the residue of the Spirit? And wherefore one? He sought a godly seed. Therefore take heed to your spirit, and let none deal treacherously against the wife of his youth.}
\bv{16}{For I hate putting away, saith Yahweh, the God of Israel, and him that covereth his garment with violence, saith Yahweh of hosts: therefore take heed to your spirit, that ye deal not treacherously.}
\bv{17}{Ye have wearied Yahweh with your words. Yet ye say, Wherein have we wearied him? In that ye say, Every one that doeth evil is good in the sight of Yahweh, and he delighteth in them; or where is the God of justice?}
\chaphead{Chapter III}
\chapdesc{Ministries of St. John \& Jesus Foretold}
\lettrine[image=true, lines=4, findent=3pt, nindent=0pt]{B-Mal.eps}{ehold} I send my messenger, and he shall prepare the way before me: and the Lord, whom ye seek, will suddenly come to his temple; and the messenger of the covenant, whom ye desire, behold, he cometh, saith Yahweh of hosts.
\bv{2}{But who can abide the day of his coming? and who shall stand when he appeareth? for he is like a refiner's fire, and like fullers' soap:}
\bv{3}{and he will sit as a refiner and purifier of silver, and he will purify the sons of Levi, and refine them as gold and silver; and they shall offer unto Yahweh offerings in righteousness.}
\bv{4}{Then shall the offering of Judah and Jerusalem be pleasant unto Yahweh, as in the days of old, and as in ancient years.}
\bv{5}{And I will come near to you to judgement; and I will be a swift witness against the sorcerers, and against the adulterers, and against the false swearers, and against those that oppress the hireling in his wages, the widow, and the fatherless, and that turn aside the sojourner \supptext{from his right}, and fear not me, saith Yahweh of hosts.}
\bv{6}{For I, Yahweh, change not; therefore ye, O sons of Jacob, are not consumed.}
\chapsec{The People Have Robbed God}
\bv{7}{From the days of your fathers ye have turned aside from mine ordinances, and have not kept them. Return unto me, and I will return unto you, saith Yahweh of hosts. But ye say, Wherein shall we return?}
\bv{8}{Will a man rob God? yet ye rob me. But ye say, Wherein have we robbed thee? In tithes and offerings.}
\bv{9}{Ye are cursed with the curse; for ye rob me, even this whole nation.}
\bv{10}{Bring ye the whole tithe into the store-house, that there may be food in my house, and prove me now herewith, saith Yahweh of hosts, if I will not open you the windows of heaven, and pour you out a blessing, that there shall not be room enough \supptext{to receive it}.}
\bv{11}{And I will rebuke the devourer for your sakes, and he shall not destroy the fruits of your ground; neither shall your vine cast its fruit before the time in the field, saith Yahweh of hosts.}
\bv{12}{And all nations shall call you happy; for ye shall be a delightsome land, saith Yahweh of hosts.}
\bv{13}{Your words have been stout against me, saith Yahweh. Yet ye say, What have we spoken against thee?}
\bv{14}{Ye have said, It is vain to serve God; and what profit is it that we have kept his charge, and that we have walked mournfully before Yahweh of hosts?}
\bv{15}{and now we call the proud happy; yea, they that work wickedness are built up; yea, they tempt God, and escape.}
\chapsec{The Faithful Remnant}
\bv{16}{Then they that feared Yahweh spake one with another; and Yahweh hearkened, and heard, and a book of remembrance was written before him, for them that feared Yahweh, and that thought upon his name.}
\bv{17}{And they shall be mine, saith Yahweh of hosts, \supptext{even} mine own possession, in the day that I make; and I will spare them, as a man spareth his own son that serveth him.}
\bv{18}{Then shall ye return and discern between the righteous and the wicked, between him that serveth God and him that serveth him not.}
\chaphead{Chapter IV}
\chapdesc{The Day of Yahweh}
\lettrine[image=true, lines=4, findent=3pt, nindent=0pt]{F-Mal.eps}{or} behold, the day cometh, it burneth as a furnace; and all the proud, and all that work wickedness, shall be stubble; and the day that cometh shall burn them up, saith Yahweh of hosts, that it shall leave them neither root nor branch.
\chapsec{The Second Coming of Christ}
\bv{2}{But unto you that fear my name shall the sun of righteousness arise with healing in its wings; and ye shall go forth, and gambol as calves of the stall.}
\bv{3}{And ye shall tread down the wicked; for they shall be ashes under the soles of your feet in the day that I make, saith Yahweh of hosts.}
\bv{4}{Remember ye the law of Moses my servant, which I commanded unto him in Horeb for all Israel, even statutes and ordinances.}
\chapsec{Elijah to Come before the Day of Yahweh}
\bv{5}{Behold, I will send you Elijah the prophet before the great and terrible day of Yahweh come.}
\bv{6}{And he shall turn the heart of the fathers to the children, and the heart of the children to their fathers; lest I come and smite the earth with a curse.}
	%\fancyhead[RE,LO]{}
\fancyhead[LE,RO]{}
\onecolumn
\vspace*{50ex}
\begin{center}
	\rule{15em}{.25mm}\\
	\vspace{1.5ex}
	{\Large{\scshape Apocrypha}}
	\par
	\rule{15em}{.25mm}
\end{center}
\vfill
\twocolumn
	\fancyhead[RE,LO]{}
\fancyhead[LE,RO]{}
\onecolumn
\vspace*{50ex}
\begin{center}
	\rule{15em}{.25mm}\\
	\vspace{1.5ex}
	{\Large{\scshape Prayer before Reading Sacred Scripture}}
	\par
	\rule{15em}{.25mm}
\end{center}
\vfill
\twocolumn
	\fancyhead[RE,LO]{The Prayer of Manasseh}
\chapter{The Prayer of Manasseh}
\fancyhead[LE,RO]{}
\chapdesc{When He Was Held Captive in Babylon}
\lettrine[image=true, lines=4, findent=3pt, nindent=0pt]{O.eps}{} LORD Almighty,\mcomm{that art in heaven} thou God of our fathers, of Abraham, and Isaac, and Jacob, and of their righteous seed;
\bv{2}{who hast made heaven and earth, with all the ornament\mcomm{order or array (Gen. 2:1 LXX)} thereof;}
\bv{3}{who hast bound the sea by the word of thy commandment; who hast shut up the deep, and sealed it by thy terrible and glorious name;}
\bv{4}{whom all things fear, yea, tremble before thy power;}
\bv{5}{for the majesty of thy glory cannot be borne, and the anger of thy threatening toward sinners is importable:}
\bv{6}{thy merciful promise is unmeasurable and unsearchable;}
\bv{7}{for thou art the Lord Most High, of great compassion, longsuffering and abundant in mercy, and repentest of the evils of men.}
\par
\bv{8}{Thou, O Lord, according to thy great goodness hast promised repentance and forgiveness to them that have sinned against thee: and of thine infinite mercies\mcomm{Thou hast promised that repentance shall be the way for them to return to thee.} hast appointed repentance unto sinners, that they may be saved. Thou therefore, O Lord, that art the God of the just, hast not appointed repentance to the just, to Abraham, and Isaac, and Jacob,\mcomm{He calls their sins nothing, but attributes unto them righteousness.} which have not sinned against thee; but thou hast appointed repentance unto me that am a sinner:}
\bv{9}{for I have sinned above the number of the sands of the sea.}
\par
My transgressions are multiplied, O Lord: my transgressions are multiplied, and I am not worthy to behold and see the height of heaven for the multitude of mine iniquities.
\bv{10}{I am bowed down with many iron bands, that I cannot lift up mine head by reason of my sins, neither have I any respite: for I have provoked thy wrath, and done that which is evil before thee: I did not thy will, neither kept I thy commandments: I have set up abominations, and have multiplied detestable things\mcomm{stumbling-blocks}.}
\par
\bv{11}{Now therefore I bow the knee of mine heart, beseeching thee of grace.}
\bv{12}{I have sinned, O Lord, I have sinned, and I acknowledge mine iniquities:}
\bv{13}{but, I humbly beseech thee, forgive me, O Lord, forgive me, and destroy me not with mine iniquities. Be not angry with me for ever, by reserving evil for me; neither condemn me into the lower parts of the earth. For thou, O Lord\mcomm{O God}, art the God of them that repent;}
\bv{14}{and in me thou wilt shew all thy goodness: for thou wilt save me, that am unworthy, according to thy great mercy.}
\par
\bv{15}{And I will praise thee for ever all the days of my life: for all the host of heaven doth sing thy praise, and thine is the glory for ever and ever. Amen.}
\fancyhead[RE,LO]{}
\fancyhead[LE,RO]{}
	\clearpage
	\fancyhead[RE,LO]{}
\fancyhead[LE,RO]{}
\onecolumn
\vspace*{50ex}
\begin{center}
	\rule{15em}{.25mm}\\
	\vspace{1.5ex}
	{\Large{\scshape The New Testament}}
	\par
	\rule{15em}{.25mm}
\end{center}
\vfill
\twocolumn
	\clearpage
	\chapter{The Holy Gospel of Jesus Christ according to Saint Matthew}
\fancyhead[RE,LO]{The Gospel according to Matthew}
\chaphead{Chapter I}
\chapdesc{The Genealogy of Jesus Christ}
\lettrine[image=true, lines=4, findent=3pt, nindent=0pt]{NT/Matthew/Matt1-T.ps}{he} book of the generation of Jesus Christ, the son of David, the son of Abraham.
\bv{2}{Abraham begat Isaac; and Isaac begat Jacob; and Jacob begat Judah and his brethren;}
\bv{3}{and Judah begat Perez and Zerah of Tamar; and Perez begat Hezron; and Hezron begat Ram;}
\bv{4}{and Ram begat Amminadab; and Amminadab begat Nahshon; and Nahshon begat Salmon;}
\bv{5}{and Salmon begat Boaz of Rahab; and Boaz begat Obed of Ruth; and Obed begat Jesse;}
\bv{6}{and Jesse begat David the king.}
\par
And David begat Solomon of her \supptext{that had been the wife} of Uriah;
\bv{7}{and Solomon begat Rehoboam; and Rehoboam begat Abijah; and Abijah begat Asa;}
\bv{8}{and Asa begat Jehoshaphat; and Jehoshaphat begat Joram; and Joram begat Uzziah;}
\bv{9}{and Uzziah begat Jotham; and Jotham begat Ahaz; and Ahaz begat Hezekiah;}
\bv{10}{and Hezekiah begat Manasseh; and Manasseh begat Amon; and Amon begat Josiah;}
\bv{11}{and Josiah begat Jechoniah and his brethren, at the time of the carrying away to Babylon.}
\par
\bv{12}{And after the carrying away to Babylon, Jechoniah begat Shealtiel; and Shealtiel begat Zerubbabel;}
\bv{13}{and Zerubbabel begat Abiud; and Abiud begat Eliakim; and Eliakim begat Azor;}
\bv{14}{and Azor begat Sadoc; and Sadoc begat Achim; and Achim begat Eliud;}
\bv{15}{and Eliud begat Eleazar; and Eleazar begat Matthan; and Matthan begat Jacob;}
\bv{16}{and Jacob begat Joseph the husband of Mary, of whom was born Jesus, who is called Christ.}
\par
\bv{17}{So all the generations from Abraham unto David are fourteen generations; and from David unto the carrying away to Babylon fourteen generations; and from the carrying away to Babylon unto the Christ fourteen generations.}
\chapsec{The Birth of Jesus Christ}
\bv{18}{Now the birth of Jesus Christ was on this wise: When his mother Mary had been betrothed to Joseph, before they came together she was found with child of the Holy Ghost.}
\bv{19}{And Joseph her husband, being a righteous man, and not willing to make her a public example, was minded to put her away privily.}
\bv{20}{But when he thought on these things, behold, an angel of the Lord appeared unto him in a dream, saying, ``Joseph, thou son of David, fear not to take unto thee Mary thy wife: for that which is conceived in her is of the Holy Ghost.}
\bv{21}{And she shall bring forth a son; and thou shalt call his name {\scshape Jesus}; for it is he that shall save his people from their sins.''}
\bv{22}{Now all this is come to pass, that it might be fulfilled which was spoken by the Lord through the prophet, saying,}
\otQuote{Is. 7:14}{\bv{23}{Behold, the virgin shall be with child, and shall bring forth a son, 
And they shall call his name Immanuel;}}
which is, being interpreted, ``God with us.''
\bv{24}{And Joseph arose from his sleep, and did as the angel of the Lord commanded him, and took unto him his wife;}
\bv{25}{and knew her not till she had brought forth a son: and he called his name {\scshape Jesus}.}
\chaphead{Chapter II}
\chapdesc{Visit of the Magi}
\lettrine[image=true, lines=4, findent=3pt, nindent=0pt]{NT/Matthew/Mt-Now.eps}{ow} when Jesus was born in Bethlehem of Judæa in the days of Herod the king, behold, Wise-men from the east came to Jerusalem, saying,
\bv{2}{``Where is he that is born King of the Jews? for we saw his star in the east, and are come to worship him.''}
\bv{3}{And when Herod the king heard it, he was troubled, and all Jerusalem with him.}
\bv{4}{And gathering together all the chief priests and scribes of the people, he inquired of them where the Christ should be born.}
\bv{5}{And they said unto him, ``In Bethlehem of Judæa:'' for thus it is written through the prophet,}
\otQuote{Mic. 5:2}{\bv{6}{And thou Bethlehem, land of Judah,
Art in no wise least among the princes of Judah:
For out of thee shall come forth a governor,
Who shall be shepherd of my people Israel.}}
\bv{7}{Then Herod privily called the Wise-men, and learned of them exactly what time the star appeared.}
\bv{8}{And he sent them to Bethlehem, and said, ``Go and search out exactly concerning the young child; and when ye have found \supptext{him}, bring me word, that I also may come and worship him.''}
\par
\bv{9}{And they, having heard the king, went their way; and lo, the star, which they saw in the east, went before them, till it came and stood over where the young child was.}
\bv{10}{And when they saw the star, they rejoiced with exceeding great joy.}
\bv{11}{And they came into the house and saw the young child with Mary his mother; and they fell down and worshipped him; and opening their treasures they offered unto him gifts, gold and frankincense and myrrh.}
\bv{12}{And being warned \supptext{of God} in a dream that they should not return to Herod, they departed into their own country another way.}
\chapsec{The Flight into Egypt}
\bv{13}{Now when they were departed, behold, an angel of the Lord appeareth to Joseph in a dream, saying, ``Arise and take the young child and his mother, and flee into Egypt, and be thou there until I tell thee: for Herod will seek the young child to destroy him.''}
\bv{14}{And he arose and took the young child and his mother by night, and departed into Egypt;}
\bv{15}{and was there until the death of Herod: that it might be fulfilled which was spoken by the Lord through the prophet, saying,}
\otQuote{Hos. 11:1}{Out of Egypt did I call my son.}
\chapsec{Herod's Slaughter of the Innocents}
\bv{16}{Then Herod, when he saw that he was mocked of the Wise-men, was exceeding wroth, and sent forth, and slew all the male children that were in Bethlehem, and in all the borders thereof, from two years old and under, according to the time which he had exactly learned of the Wise-men.}
\bv{17}{Then was fulfilled that which was spoken through Jeremiah the prophet, saying,}
\otQuote{Jer. 31:15}{\bv{18}{A voice was heard in Ramah,
Weeping and great mourning,
Rachel weeping for her children;
And she would not be comforted, because they are not.}}
\chapsec{The Return to Nazareth}
\bv{19}{But when Herod was dead, behold, an angel of the Lord appeareth in a dream to Joseph in Egypt, saying,}
\bv{20}{``Arise and take the young child and his mother, and go into the land of Israel: for they are dead that sought the young child's life.''}
\bv{21}{And he arose and took the young child and his mother, and came into the land of Israel.}
\bv{22}{But when he heard that Archelaus was reigning over Judæa in the room of his father Herod, he was afraid to go thither; and being warned \supptext{of God} in a dream, he withdrew into the parts of Galilee,}
\bv{23}{and came and dwelt in a city called Nazareth; that it might be fulfilled which was spoken through the prophets, that he should be called a Nazarene.}
\chaphead{Chapter III}
\chapdesc{The Ministry of St. John the Baptist}
\lettrine[image=true, lines=4, findent=3pt, nindent=0pt]{NT/Matthew/Mt-And.eps}{nd} in those days cometh John the Baptist, preaching in the wilderness of Judæa, saying,
\bv{2}{``Repent ye; for the kingdom of heaven is at hand.''}
\bv{3}{For this is he that was spoken of through Isaiah the prophet, saying,}
\otQuote{Is. 40:3}{The voice of one crying in the wilderness,
Make ye ready the way of the Lord,
Make his paths straight.}
\bv{4}{Now John himself had his raiment of camel's hair, and a leathern girdle about his loins; and his food was locusts and wild honey.}
\bv{5}{Then went out unto him Jerusalem, and all Judæa, and all the region round about the Jordan;}
\bv{6}{and they were baptised of him in the river Jordan, confessing their sins.}
\bv{7}{But when he saw many of the Pharisees and Sadducees coming to his baptism, he said unto them, ``Ye offspring of vipers, who warned you to flee from the wrath to come?}
\bv{8}{Bring forth therefore fruit worthy of repentance:}
\bv{9}{and think not to say within yourselves, `We have Abraham to our father:' for I say unto you, that God is able of these stones to raise up children unto Abraham.}
\bv{10}{And even now the axe lieth at the root of the trees: every tree therefore that bringeth not forth good fruit is hewn down, and cast into the fire.}
\bv{11}{I indeed baptise you in water unto repentance: but he that cometh after me is mightier than I, whose shoes I am not worthy to bear: he shall baptise you in the Holy Ghost and \supptext{in} fire:}
\bv{12}{whose fan is in his hand, and he will thoroughly cleanse his threshing-floor; and he will gather his wheat into the garner, but the chaff he will burn up with unquenchable fire.''}
\chapsec{Baptism of Jesus}
\bv{13}{Then cometh Jesus from Galilee to the Jordan unto John, to be baptised of him.}
\bv{14}{But John would have hindered him, saying, ``I have need to be baptised of thee, and comest thou to me?''}
\bv{15}{But Jesus answering said unto him, ``Suffer \supptext{it} now: for thus it becometh us to fulfil all righteousness.'' Then he suffereth him.}
\bv{16}{And Jesus, when he was baptised, went up straightway from the water: and lo, the heavens were opened unto him, and he saw the Spirit of God descending as a dove, and coming upon him;}
\bv{17}{and lo, a voice out of the heavens, saying, \god{``This is my beloved Son, in whom I am well pleased.''}}
\chaphead{Chapter IV}
\chapdesc{The Temptation of Jesus}
\lettrine[image=true, lines=4, findent=3pt, nindent=0pt]{T.ps}{hen} was Jesus led up of the Spirit into the wilderness to be tempted of the devil.
\bv{2}{And when he had fasted forty days and forty nights, he afterward hungered.}
\bv{3}{And the tempter came and said unto him, ``If thou art the Son of God, command that these stones become bread.''}
\bv{4}{But he answered and said, \redlet{``It is written,}}
\otQuote{Deut. 8:3}{\redlet{Man shall not live by bread alone, but by every word that proceedeth out of the mouth of God.''}}
\bv{5}{Then the devil taketh him into the holy city; and he set him on the pinnacle of the temple,}
\bv{6}{and saith unto him, ``If thou art the Son of God, cast thyself down: for it is written,}
\otQuote{Ps. 91:11}{He shall give his angels charge concerning thee:}
{and,}
\otQuote{Ps. 91:12}{On their hands they shall bear thee up,
Lest haply thou dash thy foot against a stone.''}
\bv{7}{Jesus said unto him, \redlet{``Again it is written,}}
\otQuote{Deut. 6:16}{Thou shalt not make trial of the Lord thy God.''}
\bv{8}{Again, the devil taketh him unto an exceeding high mountain, and showeth him all the kingdoms of the world, and the glory of them;}
\bv{9}{and he said unto him, ``All these things will I give thee, if thou wilt fall down and worship me.''}
\bv{10}{Then saith Jesus unto him, \redlet{``Get thee hence, Satan: for it is written,}}
\otQuote{Deut. 6:13}{\redlet{Thou shalt worship the Lord thy God, and him only shalt thou serve.''}}
\bv{11}{Then the devil leaveth him; and behold, angels came and ministered unto him.}
\chapsec{Jesus Begins his Public Ministry}
\bv{12}{Now when he heard that John was delivered up, he withdrew into Galilee;}
\bv{13}{and leaving Nazareth, he came and dwelt in Capernaum, which is by the sea, in the borders of Zebulun and Naphtali:}
\bv{14}{that it might be fulfilled which was spoken through Isaiah the prophet, saying,}
\otQuote{Is. 9:1-2}{\bv{15}{The land of Zebulun and the land of Naphtali,
Toward the sea, beyond the Jordan,
Galilee of the Gentiles,}
\bv{16}{The people that sat in darkness
Saw a great light,
And to them that sat in the region and shadow of death,
To them did light spring up.}}
\chapsec{The Call of Sts. Peter \& Andrew}
\bv{17}{From that time began Jesus to preach, and to say, \redlet{``Repent ye; for the kingdom of heaven is at hand.''}}
\bv{18}{And walking by the sea of Galilee, he saw two brethren, Simon who is called Peter, and Andrew his brother, casting a net into the sea; for they were fishers.}
\bv{19}{And he saith unto them, \redlet{``Come ye after me, and I will make you fishers of men.''}}
\bv{20}{And they straightway left the nets, and followed him.}
\chapsec{The Call of Sts. James \& John}
\bv{21}{And going on from thence he saw two other brethren, James the \supptext{son} of Zebedee, and John his brother, in the boat with Zebedee their father, mending their nets; and he called them.}
\bv{22}{And they straightway left the boat and their father, and followed him.}
\bv{23}{And Jesus went about in all Galilee, teaching in their synagogues, and preaching the gospel of the kingdom, and healing all manner of disease and all manner of sickness among the people.}
\bv{24}{And the report of him went forth into all Syria: and they brought unto him all that were sick, holden with divers diseases and torments, possessed with demons, and epileptic, and palsied; and he healed them.}
\bv{25}{And there followed him great multitudes from Galilee and Decapolis and Jerusalem and Judæa and \supptext{from} beyond the Jordan.}
\chaphead{Chapter V}
\chapdesc{The Sermon on the Mount}\mcomm{The Beatitudes}
\lettrine[image=true, lines=4, findent=3pt, nindent=0pt]{Mt-A.eps}{nd} seeing the multitudes, he went up into the mountain: and when he had sat down, his disciples came unto him:
\bv{2}{and he opened his mouth and taught them, saying,}
\bv{3}{\redlet{``Blessed are the poor in spirit: for theirs is the kingdom of heaven.}}
\bv{4}{\redlet{Blessed are they that mourn: for they shall be comforted.}}
\bv{5}{\redlet{Blessed are the meek: for they shall inherit the earth.}}
\bv{6}{\redlet{Blessed are they that hunger and thirst after righteousness: for they shall be filled.}}
\bv{7}{\redlet{Blessed are the merciful: for they shall obtain mercy.}}
\bv{8}{\redlet{Blessed are the pure in heart: for they shall see God.}}
\bv{9}{\redlet{Blessed are the peacemakers: for they shall be called sons of God.}}
\bv{10}{\redlet{Blessed are they that have been persecuted for righteousness' sake: for theirs is the kingdom of heaven.}}
\bv{11}{\redlet{Blessed are ye when \supptext{men} shall reproach you, and persecute you, and say all manner of evil against you falsely, for my sake.}}
\bv{12}{\redlet{Rejoice, and be exceeding glad: for great is your reward in heaven: for so persecuted they the prophets that were before you.}}
\chapsec{Similitudes of the Believer}
\bv{13}{\redlet{Ye are the salt of the earth: but if the salt have lost its savor, wherewith shall it be salted? it is thenceforth good for nothing, but to be cast out and trodden under foot of men.}}
\bv{14}{\redlet{Ye are the light of the world. A city set on a hill cannot be hid.}}
\bv{15}{\redlet{Neither do \supptext{men} light a lamp, and put it under the bushel, but on the stand; and it shineth unto all that are in the house.}}
\bv{16}{\redlet{Even so let your light shine before men; that they may see your good works, and glorify your Father who is in heaven.}}
\chapsec{Relation of Christ to the Law}
\bv{17}{\redlet{Think not that I came to destroy the law or the prophets: I came not to destroy, but to fulfil.}}
\bv{18}{\redlet{For verily I say unto you, Till heaven and earth pass away, one jot or one tittle shall in no wise pass away from the law, till all things be accomplished.}}
\bv{19}{\redlet{Whosoever therefore shall break one of these least commandments, and shall teach men so, shall be called least in the kingdom of heaven: but whosoever shall do and teach them, he shall be called great in the kingdom of heaven.}}
\bv{20}{\redlet{For I say unto you, that except your righteousness shall exceed \supptext{the righteousness} of the scribes and Pharisees, ye shall in no wise enter into the kingdom of heaven.}}
\chapsec{Sinful Anger}
\bv{21}{\redlet{Ye have heard that it was said to them of old time, `Thou shalt not kill; and whosoever shall kill shall be in danger of the judgement:'}\mref{Ex. 20:13}}
\bv{22}{\redlet{but I say unto you, that every one who is angry with his brother shall be in danger of the judgement; and whosoever shall say to his brother, `Raca,' shall be in danger of the council; and whosoever shall say, `Moreh,' shall be in danger of the hell of fire.}}
\par
\bv{23}{\redlet{If therefore thou art offering thy gift at the altar, and there rememberest that thy brother hath aught against thee,}}
\bv{24}{\redlet{leave there thy gift before the altar, and go thy way, first be reconciled to thy brother, and then come and offer thy gift.}}
\bv{25}{\redlet{Agree with thine adversary quickly, while thou art with him in the way; lest haply the adversary deliver thee to the judge, and the judge deliver thee to the officer, and thou be cast into prison.}}
\bv{26}{\redlet{Verily I say unto thee, Thou shalt by no means come out thence, till thou have paid the last quadrans.}\mcomm{Quadrans: $\frac{1}{64}$\textsuperscript{th} of a denarius (a day's wages)}}
\chapsec{Adultery}
\bv{27}{\redlet{Ye have heard that it was said, `Thou shalt not commit adultery:'}\mref{Ex. 20:14}}
\bv{28}{\redlet{But I say unto you, that every one that looketh on a woman to lust after her hath committed adultery with her already in his heart.}}
\bv{29}{\redlet{And if thy right eye causeth thee to stumble, pluck it out, and cast it from thee: for it is profitable for thee that one of thy members should perish, and not thy whole body be cast into hell.}}
\bv{30}{\redlet{And if thy right hand causeth thee to stumble, cut it off, and cast it from thee: for it is profitable for thee that one of thy members should perish, and not thy whole body go into hell.}}
\chapsec{Divorce}
\bv{31}{\redlet{It was said also, `Whosoever shall put away his wife, let him give her a writing of divorcement:'}\mref{Deut. 24:1}}
\bv{32}{\redlet{but I say unto you, that every one that putteth away his wife, saving for the cause of fornication, maketh her an adulteress: and whosoever shall marry her when she is put away committeth adultery.}}
\chapsec{False Swearing}
\bv{33}{\redlet{Again, ye have heard that it was said to them of old time, `Thou shalt not forswear thyself, but shalt perform unto the Lord thine oaths:'}\mref{Lev. 19:12}}
\bv{34}{\redlet{but I say unto you, Swear not at all; neither by the heaven, for it is the throne of God;}}
\bv{35}{\redlet{nor by the earth, for it is the footstool of his feet; nor by Jerusalem, for it is the city of the great King.}}
\bv{36}{\redlet{Neither shalt thou swear by thy head, for thou canst not make one hair white or black.}}
\bv{37}{\redlet{But let your speech be, `Yea, yea;' `Nay, nay:' and whatsoever is more than these is of the evil \supptext{one}.}}
\chapsec{Hate}
\bv{38}{\redlet{Ye have heard that it was said, `An eye for an eye, and a tooth for a tooth:}'}\mref{Ex. 21:24}
\bv{39}{\redlet{but I say unto you, Resist not him that is evil: but whosoever smiteth thee on thy right cheek, turn to him the other also.}}
\bv{40}{\redlet{And if any man would go to law with thee, and take away thy coat, let him have thy cloak also.}}
\bv{41}{\redlet{And whosoever shall compel thee to go one mile, go with him two.}}
\bv{42}{\redlet{Give to him that asketh thee, and from him that would borrow of thee turn not thou away.}}
\chapsec{Call to Love}
\bv{43}{\redlet{Ye have heard that it was said, `Thou shalt love thy neighbor, and hate thine enemy:'}\mref{cf. Lev. 19:18; Ps. 139:22}}
\bv{44}{\redlet{but I say unto you, Love your enemies, and pray for them that persecute you;}}
\bv{45}{\redlet{that ye may be sons of your Father who is in heaven: for he maketh his sun to rise on the evil and the good, and sendeth rain on the just and the unjust.}}
\bv{46}{\redlet{For if ye love them that love you, what reward have ye? do not even the publicans the same?}}
\bv{47}{\redlet{And if ye salute your brethren only, what do ye more \supptext{than others}? do not even the Gentiles the same?}}
\bv{48}{\redlet{Ye therefore shall be perfect, as your heavenly Father is perfect.}}
\chaphead{Chapter VI}
\chapdesc{Jesus Condemns False Religion}
\lettrine[image=true, lines=4, findent=3pt, nindent=0pt]{T.ps}{\redlet{ake}} \redlet{heed that ye do not your righteousness before men, to be seen of them: else ye have no reward with your Father who is in heaven.}
\bv{2}{\redlet{When therefore thou doest alms, sound not a trumpet before thee, as the hypocrites do in the synagogues and in the streets, that they may have glory of men. Verily I say unto you, They have received their reward.}}
\bv{3}{\redlet{But when thou doest alms, let not thy left hand know what thy right hand doeth:}}
\bv{4}{\redlet{that thine alms may be in secret: and thy Father who seeth in secret shall recompense thee.}}
\bv{5}{\redlet{And when ye pray, ye shall not be as the hypocrites: for they love to stand and pray in the synagogues and in the corners of the streets, that they may be seen of men. Verily I say unto you, They have received their reward.}}
\bv{6}{\redlet{But thou, when thou prayest, enter into thine inner chamber, and having shut thy door, pray to thy Father who is in secret, and thy Father who seeth in secret shall recompense thee.}}
\bv{7}{\redlet{And in praying use not vain repetitions, as the Gentiles do: for they think that they shall be heard for their much speaking.}}
\chapsec{The Lord's Prayer}
\bv{8}{\redlet{Be not therefore like unto them: for your Father knoweth what things ye have need of, before ye ask him.}}
\bv{9}{\redlet{After this manner therefore pray ye:}}
\canticle{\redlet{Our Father who art in heaven, Hallowed be thy name.\\
\bv{10}{Thy kingdom come. Thy will be done, as in heaven, so on earth.}\\
\bv{11}{Give us this day our daily bread.}\\
\bv{12}{And forgive us our debts, as we also have forgiven our debtors.}\\
\bv{13}{And bring us not into temptation, but deliver us from the evil \supptext{one}.}}}\mcomm{For thine is the kingdom, and the power, and the glory, forever and ever. Amen.}
\bv{14}{\redlet{For if ye forgive men their trespasses, your heavenly Father will also forgive you.}}
\bv{15}{\redlet{But if ye forgive not men their trespasses, neither will your Father forgive your trespasses.}}
\par
\bv{16}{\redlet{Moreover when ye fast, be not, as the hypocrites, of a sad countenance: for they disfigure their faces, that they may be seen of men to fast. Verily I say unto you, They have received their reward.}}
\bv{17}{\redlet{But thou, when thou fastest, anoint thy head, and wash thy face;}}
\bv{18}{\redlet{that thou be not seen of men to fast, but of thy Father who is in secret: and thy Father, who seeth in secret, shall recompense thee.}}
\chapsec{Man's True Treasures}
\bv{19}{\redlet{Lay not up for yourselves treasures upon the earth, where moth and rust consume, and where thieves break through and steal:}}
\bv{20}{\redlet{but lay up for yourselves treasures in heaven, where neither moth nor rust doth consume, and where thieves do not break through nor steal:}}
\bv{21}{\redlet{for where thy treasure is, there will thy heart be also.}}
\chapsec{The Light of Man}
\bv{22}{\redlet{The lamp of the body is the eye: if therefore thine eye be single, thy whole body shall be full of light.}}
\bv{23}{\redlet{But if thine eye be evil, thy whole body shall be full of darkness. If therefore the light that is in thee be darkness, how great is the darkness!}}
\bv{24}{\redlet{No man can serve two masters: for either he will hate the one, and love the other; or else he will hold to one, and despise the other. Ye cannot serve God and mammon.}\mcomm{Mammon: money}}
\chapsec{Exhortation against Anxiety}
\bv{25}{\redlet{Therefore I say unto you, Be not anxious for your life, what ye shall eat, or what ye shall drink; nor yet for your body, what ye shall put on. Is not the life more than the food, and the body than the raiment?}}
\bv{26}{\redlet{Behold the birds of the heaven, that they sow not, neither do they reap, nor gather into barns; and your heavenly Father feedeth them. Are not ye of much more value than they?}}
\bv{27}{\redlet{And which of you by being anxious can add one cubit unto the measure of his life?}}
\bv{28}{\redlet{And why are ye anxious concerning raiment? Consider the lilies of the field, how they grow; they toil not, neither do they spin:}}
\bv{29}{\redlet{yet I say unto you, that even Solomon in all his glory was not arrayed like one of these.}}
\bv{30}{\redlet{But if God doth so clothe the grass of the field, which to-day is, and to-morrow is cast into the oven, \supptext{shall he} not much more \supptext{clothe} you, O ye of little faith?}}
\bv{31}{\redlet{Be not therefore anxious, saying, `What shall we eat?' or, `What shall we drink?' or, `Wherewithal shall we be clothed?'}}
\bv{32}{\redlet{For after all these things do the Gentiles seek; for your heavenly Father knoweth that ye have need of all these things.}}
\bv{33}{\redlet{But seek ye first his kingdom, and his righteousness; and all these things shall be added unto you.}}
\bv{34}{\redlet{Be not therefore anxious for the morrow: for the morrow will be anxious for itself. Sufficient unto the day is the evil thereof.}}
\chaphead{Chapter VII}
\chapdesc{False Judgement Condemned}
\lettrine[image=true, lines=4, findent=3pt, nindent=0pt]{J.eps}{\redlet{udge}} \redlet{not, that ye be not judged.}
\bv{2}{\redlet{For with what judgement ye judge, ye shall be judged: and with what measure ye mete, it shall be measured unto you.}}
\bv{3}{\redlet{And why beholdest thou the mote that is in thy brother's eye, but considerest not the beam that is in thine own eye?}}
\bv{4}{\redlet{Or how wilt thou say to thy brother, Let me cast out the mote out of thine eye; and lo, the beam is in thine own eye?}}
\bv{5}{\redlet{Thou hypocrite, cast out first the beam out of thine own eye; and then shalt thou see clearly to cast out the mote out of thy brother's eye.}}
\bv{6}{\redlet{Give not that which is holy unto the dogs, neither cast your pearls before the swine, lest haply they trample them under their feet, and turn and rend you.}}
\chapsec{Encouragement to Pray}
\bv{7}{\redlet{Ask, and it shall be given you; seek, and ye shall find; knock, and it shall be opened unto you:}}
\bv{8}{\redlet{for every one that asketh receiveth; and he that seeketh findeth; and to him that knocketh it shall be opened.}}
\bv{9}{\redlet{Or what man is there of you, who, if his son shall ask him for a loaf, will give him a stone;}}
\bv{10}{\redlet{or if he shall ask for a fish, will give him a serpent?}}
\bv{11}{\redlet{If ye then, being evil, know how to give good gifts unto your children, how much more shall your Father who is in heaven give good things to them that ask him?}}
\chapsec{Summary of Righteousness}
\bv{12}{\redlet{All things therefore whatsoever ye would that men should do unto you, even so do ye also unto them: for this is the law and the prophets.}}
\chapsec{The Two Ways}
\bv{13}{\redlet{Enter ye in by the narrow gate: for wide is the gate, and broad is the way, that leadeth to destruction, and many are they that enter in thereby.}}
\bv{14}{\redlet{For narrow is the gate, and straitened the way, that leadeth unto life, and few are they that find it.}}
\chapsec{Warning against False Teachers}
\bv{15}{\redlet{Beware of false prophets, who come to you in sheep's clothing, but inwardly are ravening wolves.}}
\bv{16}{\redlet{By their fruits ye shall know them. Do \supptext{men} gather grapes of thorns, or figs of thistles?}}
\bv{17}{\redlet{Even so every good tree bringeth forth good fruit; but the corrupt tree bringeth forth evil fruit.}}
\bv{18}{\redlet{A good tree cannot bring forth evil fruit, neither can a corrupt tree bring forth good fruit.}}
\bv{19}{\redlet{Every tree that bringeth not forth good fruit is hewn down, and cast into the fire.}}
\bv{20}{\redlet{Therefore by their fruits ye shall know them.}}
\chapsec{The Danger of Faithless Religion}
\bv{21}{\redlet{Not every one that saith unto me, `Lord, Lord,' shall enter into the kingdom of heaven; but he that doeth the will of my Father who is in heaven.}}
\bv{22}{\redlet{Many will say to me in that day, `Lord, Lord, did we not prophesy by thy name, and by thy name cast out demons, and by thy name do many mighty works?'}}
\bv{23}{\redlet{And then will I profess unto them, I never knew you: depart from me, ye that work iniquity.}}
\chapsec{The Two Foundations}
\bv{24}{\redlet{Every one therefore that heareth these words of mine, and doeth them, shall be likened unto a wise man, who built his house upon the rock:}}
\bv{25}{\redlet{and the rain descended, and the floods came, and the winds blew, and beat upon that house; and it fell not: for it was founded upon the rock.}}
\bv{26}{\redlet{And every one that heareth these words of mine, and doeth them not, shall be likened unto a foolish man, who built his house upon the sand:}}
\bv{27}{\redlet{and the rain descended, and the floods came, and the winds blew, and smote upon that house; and it fell: and great was the fall thereof.''}}
\bv{28}{And it came to pass, when Jesus had finished these words, the multitudes were astonished at his teaching:}
\bv{29}{for he taught them as \supptext{one} having authority, and not as their scribes.}
\chaphead{Chapter VIII}
\chapdesc{Jesus Heals a Leper}
\lettrine[image=true, lines=4, findent=3pt, nindent=0pt]{Mt-A.eps}{nd} when he was come down from the mountain, great multitudes followed him.
\bv{2}{And behold, there came to him a leper and worshipped him, saying, ``Lord, if thou wilt, thou canst make me clean.''}
\bv{3}{And he stretched forth his hand, and touched him, saying, \redlet{``I will; be thou made clean.''} And straightway his leprosy was cleansed.}
\bv{4}{And Jesus saith unto him, \redlet{``See thou tell no man; but go, show thyself to the priest, and offer the gift that Moses commanded, for a testimony unto them.''}}
\chapsec{Jesus Heals the Centurion's Servant}
\bv{5}{And when he was entered into Capernaum, there came unto him a centurion, beseeching him,}
\bv{6}{and saying, ``Lord, my servant lieth in the house sick of the palsy, grievously tormented.''}
\bv{7}{And he saith unto him, \redlet{``I will come and heal him.''}}
\bv{8}{And the centurion answered and said, ``Lord, I am not worthy that thou shouldest come under my roof; but only say the word, and my servant shall be healed.}
\bv{9}{For I also am a man under authority, having under myself soldiers: and I say to this one, `Go,' and he goeth; and to another, `Come,' and he cometh; and to my servant, `Do this,' and he doeth it.''}
\bv{10}{And when Jesus heard it, he marvelled, and said to them that followed, \redlet{``Verily I say unto you, I have not found so great faith, no, not in Israel.}}
\bv{11}{\redlet{And I say unto you, that many shall come from the east and the west, and shall sit down with Abraham, and Isaac, and Jacob, in the kingdom of heaven:}}
\bv{12}{\redlet{but the sons of the kingdom shall be cast forth into the outer darkness: there shall be the weeping and the gnashing of teeth.''}}
\bv{13}{And Jesus said unto the centurion, \redlet{``Go thy way; as thou hast believed, \supptext{so} be it done unto thee.''} And the servant was healed in that hour.}
\chapsec{Jesus Heals St. Peter's Mother-in-law}
\bv{14}{And when Jesus was come into Peter's house, he saw his wife's mother lying sick of a fever.}
\bv{15}{And he touched her hand, and the fever left her; and she arose, and ministered unto him.}
\bv{16}{And when even was come, they brought unto him many possessed with demons: and he cast out the spirits with a word, and healed all that were sick:}
\bv{17}{that it might be fulfilled which was spoken through Isaiah the prophet, saying,}
\otQuote{Is. 53:4}{Himself took our infirmities, and bare our diseases.}
\bv{18}{Now when Jesus saw great multitudes about him, he gave commandment to depart unto the other side.}
\chapsec{Professed Disciples Tested}
\bv{19}{And there came a scribe, and said unto him, ``Teacher, I will follow thee whithersoever thou goest.''}
\bv{20}{And Jesus saith unto him, \redlet{``The foxes have holes, and the birds of the heaven \supptext{have} nests; but the Son of man hath not where to lay his head.''}}
\bv{21}{And another of the disciples said unto him, ``Lord, suffer me first to go and bury my father.''}
\bv{22}{But Jesus saith unto him, \redlet{``Follow me; and leave the dead to bury their own dead.''}}
\chapsec{Jesus Stills the Waves}
\bv{23}{And when he was entered into a boat, his disciples followed him.}
\bv{24}{And behold, there arose a great tempest in the sea, insomuch that the boat was covered with the waves: but he was asleep.}
\bv{25}{And they came to him, and awoke him, saying, ``Save, Lord; we perish.''}
\bv{26}{And he saith unto them, \redlet{``Why are ye fearful, O ye of little faith?''} Then he arose, and rebuked the winds and the sea; and there was a great calm.}
\bv{27}{And the men marvelled, saying, ``What manner of man is this, that even the winds and the sea obey him?''}
\chapsec{Jesus Casts out Demons}
\bv{28}{And when he was come to the other side into the country of the Gadarenes, there met him two possessed with demons, coming forth out of the tombs, exceeding fierce, so that no man could pass by that way.}
\bv{29}{And behold, they cried out, saying, ``What have we to do with thee, thou Son of God? art thou come hither to torment us before the time?''}
\bv{30}{Now there was afar off from them a herd of many swine feeding.}
\bv{31}{And the demons besought him, saying, If thou cast us out, send us away into the herd of swine.}
\bv{32}{And he said unto them, \redlet{``Go.''} And they came out, and went into the swine: and behold, the whole herd rushed down the steep into the sea, and perished in the waters.}
\bv{33}{And they that fed them fled, and went away into the city, and told everything, and what was befallen to them that were possessed with demons.}
\bv{34}{And behold, all the city came out to meet Jesus: and when they saw him, they besought \supptext{him} that he would depart from their borders.}
\chaphead{Chapter IX}
\chapdesc{Jesus Heals the Paralytic}
\lettrine[image=true, lines=4, findent=3pt, nindent=0pt]{Mt-A.eps}{nd} he entered into a boat, and crossed over, and came into his own city.
\bv{2}{And behold, they brought to him a paralytic, lying on a bed: and Jesus seeing their faith said unto the paralytic, \redlet{``Son, be of good cheer; thy sins are forgiven.''}}
\bv{3}{And behold, certain of the scribes said within themselves, ``This man blasphemeth.''}
\bv{4}{And Jesus knowing their thoughts said, \redlet{``Wherefore think ye evil in your hearts?}}
\bv{5}{\redlet{For which is easier, to say, `Thy sins are forgiven;' or to say, `Arise, and walk?'}}
\bv{6}{\redlet{But that ye may know that the Son of man hath authority on earth to forgive sins''} (then saith he to the paralytic), \redlet{``Arise, and take up thy bed, and go unto thy house.''}}
\bv{7}{And he arose, and departed to his house.}
\bv{8}{But when the multitudes saw it, they were afraid, and glorified God, who had given such authority unto men.}
\chapsec{The Call of St. Matthew}
\bv{9}{And as Jesus passed by from thence, he saw a man, called Matthew, sitting at the place of toll: and he saith unto him, \redlet{``Follow me.''} And he arose, and followed him.}
\chapsec{Jesus Answers the Pharisees}
\bv{10}{And it came to pass, as he sat at meat in the house, behold, many publicans and sinners came and sat down with Jesus and his disciples.}
\bv{11}{And when the Pharisees saw it, they said unto his disciples, ``Why eateth your Teacher with the publicans and sinners?''}
\bv{12}{But when he heard it, he said, \redlet{``They that are whole have no need of a physician, but they that are sick.}}
\bv{13}{\redlet{But go ye and learn what \supptext{this} meaneth,}}
\otQuote{Hos. 6:6}{\redlet{`I desire mercy, and not sacrifice:'}}
\redlet{for I came not to call the righteous, but sinners.''}
\bv{14}{Then come to him the disciples of John, saying, ``Why do we and the Pharisees fast oft, but thy disciples fast not?''}
\bv{15}{And Jesus said unto them, \redlet{``Can the sons of the bridechamber mourn, as long as the bridegroom is with them? but the days will come, when the bridegroom shall be taken away from them, and then will they fast.}}
\chapsec{Parables of the Garment \& Bottles}
\bv{16}{\redlet{And no man putteth a piece of undressed cloth upon an old garment; for that which should fill it up taketh from the garment, and a worse rent is made.}}
\bv{17}{\redlet{Neither do \supptext{men} put new wine into old wine-skins: else the skins burst, and the wine is spilled, and the skins perish: but they put new wine into fresh wine-skins, and both are preserved.''}}
\chapsec{Jesus Heals the Woman \& Jairus' Daughter}
\bv{18}{While he spake these things unto them, behold, there came a ruler, and worshipped him, saying, ``My daughter is even now dead: but come and lay thy hand upon her, and she shall live.''}
\bv{19}{And Jesus arose, and followed him, and \supptext{so did} his disciples.}
\bv{20}{And behold, a woman, who had an issue of blood twelve years, came behind him, and touched the border of his garment:}
\bv{21}{for she said within herself, ``If I do but touch his garment, I shall be made whole.''}
\bv{22}{But Jesus turning and seeing her said, \redlet{``Daughter, be of good cheer; thy faith hath made thee whole.''} And the woman was made whole from that hour.}
\bv{23}{And when Jesus came into the ruler's house, and saw the flute-players, and the crowd making a tumult,}
\bv{24}{he said, \redlet{``Give place: for the damsel is not dead, but sleepeth.''} And they laughed him to scorn.}
\bv{25}{But when the crowd was put forth, he entered in, and took her by the hand; and the damsel arose.}
\bv{26}{And the fame hereof went forth into all that land.}
\chapsec{Two Blind Men Healed}
\bv{27}{And as Jesus passed by from thence, two blind men followed him, crying out, and saying, ``Have mercy on us, thou son of David.''}
\bv{28}{And when he was come into the house, the blind men came to him: and Jesus saith unto them, \redlet{``Believe ye that I am able to do this?''} They say unto him, ``Yea, Lord.''}
\bv{29}{Then touched he their eyes, saying, \redlet{``According to your faith be it done unto you.''}}
\bv{30}{And their eyes were opened. And Jesus strictly charged them, saying, \redlet{``See that no man know it.''}}
\bv{31}{But they went forth, and spread abroad his fame in all that land.}
\chapsec{A Demon Cast Out}
\bv{32}{And as they went forth, behold, there was brought to him a dumb man possessed with a demon.}
\bv{33}{And when the demon was cast out, the dumb man spake: and the multitudes marvelled, saying, ``It was never so seen in Israel.''}
\bv{34}{But the Pharisees said, ``By the prince of the demons casteth he out demons.''}
\chapsec{Jesus Preaches \& Heals}
\bv{35}{And Jesus went about all the cities and the villages, teaching in their synagogues, and preaching the gospel of the kingdom, and healing all manner of disease and all manner of sickness.}
\bv{36}{But when he saw the multitudes, he was moved with compassion for them, because they were distressed and scattered, as sheep not having a shepherd.}
\bv{37}{Then saith he unto his disciples, \redlet{``The harvest indeed is plenteous, but the laborers are few.}}
\bv{38}{\redlet{Pray ye therefore the Lord of the harvest, that he send forth laborers into his harvest.''}}
\chaphead{Chapter X}
\chapdesc{The Twelve Instructed \& Sent}
\lettrine[image=true, lines=4, findent=3pt, nindent=0pt]{Mt-A.eps}{nd} he called unto him his twelve disciples, and gave them authority over unclean spirits, to cast them out, and to heal all manner of disease and all manner of sickness.
\bv{2}{Now the names of the twelve apostles are these: The first, Simon, who is called Peter, and Andrew his brother; James the \supptext{son} of Zebedee, and John his brother;}
\bv{3}{Philip, and Bartholomew; Thomas, and Matthew the publican; James the \supptext{son} of Alphæus, and Thaddæus;}
\bv{4}{Simon the Cananæan, and Judas Iscariot, who also betrayed him.}
\par
\bv{5}{These twelve Jesus sent forth, and charged them, saying, \redlet{``Go not into \supptext{any} way of the Gentiles, and enter not into any city of the Samaritans:}}
\bv{6}{\redlet{but go rather to the lost sheep of the house of Israel.}}
\bv{7}{\redlet{And as ye go, preach, saying, `The kingdom of heaven is at hand.'}}
\bv{8}{\redlet{Heal the sick, raise the dead, cleanse the lepers, cast out demons: freely ye received, freely give.}}
\par
\bv{9}{\redlet{Get you no gold, nor silver, nor brass in your purses;}}
\bv{10}{\redlet{no wallet for \supptext{your} journey, neither two coats, nor shoes, nor staff: for the laborer is worthy of his food.}}
\bv{11}{\redlet{And into whatsoever city or village ye shall enter, search out who in it is worthy; and there abide till ye go forth.}}
\bv{12}{\redlet{And as ye enter into the house, salute it.}}
\par
\bv{13}{\redlet{And if the house be worthy, let your peace come upon it: but if it be not worthy, let your peace return to you.}}
\bv{14}{\redlet{And whosoever shall not receive you, nor hear your words, as ye go forth out of that house or that city, shake off the dust of your feet.}}
\par
\bv{15}{\redlet{Verily I say unto you, It shall be more tolerable for the land of Sodom and Gomorrah in the day of judgement, than for that city.}}
\chapsec{Persecution Foretold}
\bv{16}{\redlet{Behold, I send you forth as sheep in the midst of wolves: be ye therefore wise as serpents, and harmless as doves.}}
\bv{17}{\redlet{But beware of men: for they will deliver you up to councils, and in their synagogues they will scourge you;}}
\bv{18}{\redlet{yea and before governors and kings shall ye be brought for my sake, for a testimony to them and to the Gentiles.}}
\bv{19}{\redlet{But when they deliver you up, be not anxious how or what ye shall speak: for it shall be given you in that hour what ye shall speak.}}
\bv{20}{\redlet{For it is not ye that speak, but the Spirit of your Father that speaketh in you.}}
\par
\bv{21}{\redlet{And brother shall deliver up brother to death, and the father his child: and children shall rise up against parents, and cause them to be put to death.}}
\bv{22}{\redlet{And ye shall be hated of all men for my name's sake: but he that endureth to the end, the same shall be saved.}}
\bv{23}{\redlet{But when they persecute you in this city, flee into the next: for verily I say unto you, Ye shall not have gone through the cities of Israel, till the Son of man be come.}}
\par
\bv{24}{\redlet{A disciple is not above his teacher, nor a servant above his lord.}}
\bv{25}{\redlet{It is enough for the disciple that he be as his teacher, and the servant as his lord. If they have called the master of the house Beelzebub, how much more them of his household!}}
\bv{26}{\redlet{Fear them not therefore: for there is nothing covered, that shall not be revealed; and hid, that shall not be known.}}
\bv{27}{\redlet{What I tell you in the darkness, speak ye in the light; and what ye hear in the ear, proclaim upon the house-tops.}}
\bv{28}{\redlet{And be not afraid of them that kill the body, but are not able to kill the soul: but rather fear him who is able to destroy both soul and body in hell.}}
\chapsec{The Father's Love for his Creation}
\bv{29}{\redlet{Are not two sparrows sold for a penny? and not one of them shall fall on the ground without your Father:}}
\bv{30}{\redlet{but the very hairs of your head are all numbered.}}
\bv{31}{\redlet{Fear not therefore: ye are of more value than many sparrows.}}
\bv{32}{\redlet{Every one therefore who shall confess me before men, him will I also confess before my Father who is in heaven.}}
\bv{33}{\redlet{But whosoever shall deny me before men, him will I also deny before my Father who is in heaven.}}
\chapsec{The Peace of Christ}
\bv{34}{\redlet{Think not that I came to send peace on the earth: I came not to send peace, but a sword.}}
\bv{35}{\redlet{For I came to set a man at variance against his father, and the daughter against her mother, and the daughter in law against her mother in law:}}
\bv{36}{\redlet{and a man's foes \supptext{shall be} they of his own household.}}
\bv{37}{\redlet{He that loveth father or mother more than me is not worthy of me; and he that loveth son or daughter more than me is not worthy of me.}}
\bv{38}{\redlet{And he that doth not take his cross and follow after me, is not worthy of me.}}
\bv{39}{\redlet{He that findeth his life shall lose it; and he that loseth his life for my sake shall find it.}}
\par
\bv{40}{\redlet{He that receiveth you receiveth me, and he that receiveth me receiveth him that sent me.}}
\bv{41}{\redlet{He that receiveth a prophet in the name of a prophet shall receive a prophet's reward: and he that receiveth a righteous man in the name of a righteous man shall receive a righteous man's reward.}}
\bv{42}{\redlet{And whosoever shall give to drink unto one of these little ones a cup of cold water only, in the name of a disciple, verily I say unto you he shall in no wise lose his reward.''}}
\chaphead{Chapter XI}
\chapdesc{St. John the Baptist Sends his Disciples to Jesus}
\lettrine[image=true, lines=4, findent=3pt, nindent=0pt]{Mt-A.eps}{nd} it came to pass when Jesus had finished commanding his twelve disciples, he departed thence to teach and preach in their cities.
\bv{2}{Now when John heard in the prison the works of the Christ, he sent by his disciples}
\bv{3}{and said unto him, ``Art thou he that cometh, or look we for another?''}
\bv{4}{And Jesus answered and said unto them, \redlet{``Go and tell John the things which ye hear and see:}}
\bv{5}{\redlet{the blind receive their sight, and the lame walk, the lepers are cleansed, and the deaf hear, and the dead are raised up, and the poor have good tidings preached to them.}}
\bv{6}{\redlet{And blessed is he, whosoever shall find no occasion of stumbling in me.''}}
\bv{7}{And as these went their way, Jesus began to say unto the multitudes concerning John, \redlet{``What went ye out into the wilderness to behold? a reed shaken with the wind?}}
\bv{8}{\redlet{But what went ye out to see? a man clothed in soft \supptext{raiment}? Behold, they that wear soft \supptext{raiment} are in kings' houses.}}
\bv{9}{\redlet{But wherefore went ye out? to see a prophet? Yea, I say unto you, and much more than a prophet.}}
\par
\bv{10}{\redlet{This is he, of whom it is written,}}
\otQuote{Mal. 3:1}{\redlet{`Behold, I send my messenger before thy face,
Who shall prepare thy way before thee.'}}
\bv{11}{\redlet{Verily I say unto you, Among them that are born of women there hath not arisen a greater than John the Baptist: yet he that is but little in the kingdom of heaven is greater than he.}}
\bv{12}{\redlet{And from the days of John the Baptist until now the kingdom of heaven suffereth violence, and men of violence take it by force.}}
\bv{13}{\redlet{For all the prophets and the law prophesied until John.}}
\bv{14}{\redlet{And if ye are willing to receive \supptext{it}, this is Elijah, that is to come.}}
\bv{15}{\redlet{He that hath ears to hear, let him hear.}}
\chapsec{The Hypocrisy of the Faithless}
\bv{16}{\redlet{But whereunto shall I liken this generation? It is like unto children sitting in the marketplaces, who call unto their fellows}}
\bv{17}{\redlet{and say, `We piped unto you, and ye did not dance; we wailed, and ye did not mourn.'}}
\bv{18}{\redlet{For John came neither eating nor drinking, and they say, `He hath a demon.'}}
\bv{19}{\redlet{The Son of man came eating and drinking, and they say, `Behold, a gluttonous man and a winebibber, a friend of publicans and sinners!' And wisdom is justified by her works.''}}
\chapsec{Jesus Rejected \& Predicts Judgement}
\bv{20}{Then began he to upbraid the cities wherein most of his mighty works were done, because they repented not.}
\bv{21}{\redlet{``Woe unto thee, Chorazin! woe unto thee, Bethsaida! for if the mighty works had been done in Tyre and Sidon which were done in you, they would have repented long ago in sackcloth and ashes.}}
\bv{22}{\redlet{But I say unto you, it shall be more tolerable for Tyre and Sidon in the day of judgement, than for you.}}
\bv{23}{\redlet{And thou, Capernaum, shalt thou be exalted unto heaven? thou shalt go down unto Hades: for if the mighty works had been done in Sodom which were done in thee, it would have remained until this day.}}
\bv{24}{\redlet{But I say unto you that it shall be more tolerable for the land of Sodom in the day of judgement, than for thee.''}}
\bv{25}{At that season Jesus answered and said, \redlet{``I thank thee, O Father, Lord of heaven and earth, that thou didst hide these things from the wise and understanding, and didst reveal them unto babes:}}
\bv{26}{\redlet{yea, Father, for so it was well-pleasing in thy sight.}}
\bv{27}{\redlet{All things have been delivered unto me of my Father: and no one knoweth the Son, save the Father; neither doth any know the Father, save the Son, and he to whomsoever the Son willeth to reveal \supptext{him}.}}
\chapsec{Comfortable Words}
\bv{28}{\redlet{Come unto me, all ye that labor and are heavy laden, and I will give you rest.}}
\bv{29}{\redlet{Take my yoke upon you, and learn of me; for I am meek and lowly in heart: and ye shall find rest unto your souls.}}
\bv{30}{\redlet{For my yoke is easy, and my burden is light.''}}
\chaphead{Chapter XII}
\chapdesc{Jesus: Lord of the Sabbath}
\lettrine[image=true, lines=4, findent=3pt, nindent=0pt]{Mt-A.eps}{t} that season Jesus went on the sabbath day through the grainfields; and his disciples were hungry and began to pluck ears and to eat.
\bv{2}{But the Pharisees, when they saw it, said unto him, ``Behold, thy disciples do that which it is not lawful to do upon the sabbath.''}
\bv{3}{But he said unto them, \redlet{``Have ye not read what David did, when he was hungry, and they that were with him;}}
\bv{4}{\redlet{how he entered into the house of God, and ate the showbread, which it was not lawful for him to eat, neither for them that were with him, but only for the priests?}}
\bv{5}{\redlet{Or have ye not read in the law, that on the sabbath day the priests in the temple profane the sabbath, and are guiltless?}}
\bv{6}{\redlet{But I say unto you, that one greater than the temple is here.}}
\par
\bv{7}{\redlet{But if ye had known what this meaneth,}}
\otQuote{Hos. 6:6}{\redlet{`I desire mercy, and not sacrifice,'}}
\redlet{ye would not have condemned the guiltless.}
\bv{8}{\redlet{For the Son of man is lord of the sabbath.''}}
\chapsec{The Healing of the Withered Hand}
\bv{9}{And he departed thence, and went into their synagogue:}
\bv{10}{and behold, a man having a withered hand. And they asked him, saying, ``Is it lawful to heal on the sabbath day?'' that they might accuse him.}
\bv{11}{And he said unto them, \redlet{``What man shall there be of you, that shall have one sheep, and if this fall into a pit on the sabbath day, will he not lay hold on it, and lift it out?}}
\bv{12}{\redlet{How much then is a man of more value than a sheep! Wherefore it is lawful to do good on the sabbath day.''}}
\bv{13}{Then saith he to the man, \redlet{``Stretch forth thy hand.''} And he stretched it forth; and it was restored whole, as the other.}
\bv{14}{But the Pharisees went out, and took counsel against him, how they might destroy him.}
\bv{15}{And Jesus perceiving \supptext{it} withdrew from thence: and many followed him; and he healed them all,}
\bv{16}{and charged them that they should not make him known:}
\bv{17}{that it might be fulfilled which was spoken through Isaiah the prophet, saying,}
\otQuote{Is. 42:1-3}{\bv{18}{Behold, my servant whom I have chosen;
My beloved in whom my soul is well pleased:
I will put my Spirit upon him,
And he shall declare judgement to the Gentiles.}
\bv{19}{He shall not strive, nor cry aloud;
Neither shall any one hear his voice in the streets.}
\bv{20}{A bruised reed shall he not break,
And smoking flax shall he not quench,
Till he send forth judgement unto victory.}
\bv{21}{And in his name shall the Gentiles hope.}}
\chapsec{A Demoniac Healed}
\bv{22}{Then was brought unto him one possessed with a demon, blind and dumb: and he healed him, insomuch that the dumb man spake and saw.}
\bv{23}{And all the multitudes were amazed, and said, ``Can this be the son of David?''}
\bv{24}{But when the Pharisees heard it, they said, ``This man doth not cast out demons, but by Beelzebub the prince of the demons.''}
\bv{25}{And knowing their thoughts he said unto them, \redlet{``Every kingdom divided against itself is brought to desolation; and every city or house divided against itself shall not stand:}}
\bv{26}{\redlet{and if Satan casteth out Satan, he is divided against himself; how then shall his kingdom stand?}}
\bv{27}{\redlet{And if I by Beelzebub cast out demons, by whom do your sons cast them out? therefore shall they be your judges.}}
\bv{28}{\redlet{But if I by the Spirit of God cast out demons, then is the kingdom of God come upon you.}}
\chapsec{Parable of the Strong Man}
\bv{29}{\redlet{Or how can one enter into the house of the strong \supptext{man}, and spoil his goods, except he first bind the strong \supptext{man}? and then he will spoil his house.}}
\bv{30}{\redlet{He that is not with me is against me; and he that gathereth not with me scattereth.}}
\chapsec{Blasphemy against the Holy Ghost}
\bv{31}{\redlet{Therefore I say unto you, Every sin and blasphemy shall be forgiven unto men; but the blasphemy against the Spirit shall not be forgiven.}}
\bv{32}{\redlet{And whosoever shall speak a word against the Son of man, it shall be forgiven him; but whosoever shall speak against the Holy Ghost, it shall not be forgiven him, neither in this world, nor in that which is to come.}}
\chapsec{Good Fruit \& Idle Words}
\bv{33}{\redlet{Either make the tree good, and its fruit good; or make the tree corrupt, and its fruit corrupt: for the tree is known by its fruit.}}
\bv{34}{\redlet{Ye offspring of vipers, how can ye, being evil, speak good things? for out of the abundance of the heart the mouth speaketh.}}
\bv{35}{\redlet{The good man out of his good treasure bringeth forth good things: and the evil man out of his evil treasure bringeth forth evil things.}}
\bv{36}{\redlet{And I say unto you, that every idle word that men shall speak, they shall give account thereof in the day of judgement.}}
\bv{37}{\redlet{For by thy words thou shalt be justified, and by thy words thou shalt be condemned.''}}
\chapsec{The Sign of the Prophet Jonah}
\bv{38}{Then certain of the scribes and Pharisees answered him, saying, ``Teacher, we would see a sign from thee.''}
\bv{39}{But he answered and said unto them, \redlet{``An evil and adulterous generation seeketh after a sign; and there shall no sign be given to it but the sign of Jonah the prophet:}}
\bv{40}{\redlet{for as Jonah was three days and three nights in the belly of the whale; so shall the Son of man be three days and three nights in the heart of the earth.}}
\bv{41}{\redlet{The men of Nineveh shall stand up in the judgement with this generation, and shall condemn it: for they repented at the preaching of Jonah; and behold, a greater than Jonah is here.}}
\bv{42}{\redlet{The queen of the south shall rise up in the judgement with this generation, and shall condemn it: for she came from the ends of the earth to hear the wisdom of Solomon; and behold, a greater than Solomon is here.}}
\chapsec{Stages of Sanctification}
\bv{43}{\redlet{But the unclean spirit, when he is gone out of the man, passeth through waterless places, seeking rest, and findeth it not.}}
\bv{44}{\redlet{Then he saith, `I will return into my house whence I came out;' and when he is come, he findeth it empty, swept, and garnished.}}
\bv{45}{\redlet{Then goeth he, and taketh with himself seven other spirits more evil than himself, and they enter in and dwell there: and the last state of that man becometh worse than the first. Even so shall it be also unto this evil generation.''}}
\chapsec{Jesus' Mother \& Brethren}
\bv{46}{While he was yet speaking to the multitudes, behold, his mother and his brethren stood without, seeking to speak to him.}
\bv{47}{And one said unto him, ``Behold, thy mother and thy brethren stand without, seeking to speak to thee.''}
\bv{48}{But he answered and said unto him that told him, \redlet{``Who is my mother? and who are my brethren?''}}
\bv{49}{And he stretched forth his hand towards his disciples, and said, \redlet{``Behold, my mother and my brethren!}}
\bv{50}{\redlet{For whosoever shall do the will of my Father who is in heaven, he is my brother, and sister, and mother.''}}
\chaphead{Chapter XIII}
\chapdesc{The Parable of the Sower}
\lettrine[image=true, lines=4, findent=3pt, nindent=0pt]{O.eps}{n} that day went Jesus out of the house, and sat by the sea side.
\bv{2}{And there were gathered unto him great multitudes, so that he entered into a boat, and sat; and all the multitude stood on the beach.}
\bv{3}{And he spake to them many things in parables, saying, \redlet{``Behold, the sower went forth to sow;}}
\bv{4}{\redlet{and as he sowed, some \supptext{seeds} fell by the way side, and the birds came and devoured them:}}
\bv{5}{\redlet{and others fell upon the rocky places, where they had not much earth: and straightway they sprang up, because they had no deepness of earth:}}
\bv{6}{\redlet{and when the sun was risen, they were scorched; and because they had no root, they withered away.}}
\bv{7}{\redlet{And others fell upon the thorns; and the thorns grew up and choked them:}}
\bv{8}{\redlet{and others fell upon the good ground, and yielded fruit, some a hundredfold, some sixty, some thirty.}}
\bv{9}{\redlet{He that hath ears, let him hear.'}}
\chapsec{The Purpose of Parables}
\bv{10}{And the disciples came, and said unto him, ``Why speakest thou unto them in parables?''}
\bv{11}{And he answered and said unto them, \redlet{``Unto you it is given to know the mysteries of the kingdom of heaven, but to them it is not given.}}
\bv{12}{\redlet{For whosoever hath, to him shall be given, and he shall have abundance: but whosoever hath not, from him shall be taken away even that which he hath.}}
\bv{13}{\redlet{Therefore speak I to them in parables; because seeing they see not, and hearing they hear not, neither do they understand.}}
\bv{14}{\redlet{And unto them is fulfilled the prophecy of Isaiah, which saith,}}
\otQuote{Is. 6:9-10}{\redlet{By hearing ye shall hear, and shall in no wise understand;
And seeing ye shall see, and shall in no wise perceive:}
\bv{15}{\redlet{For this people's heart is waxed gross,
And their eyes they have closed;
Lest haply they should perceive with their eyes,
And hear with their ears,
And understand with their heart,
And should turn again,
And I should heal them.}}}
\bv{16}{\redlet{But blessed are your eyes, for they see; and your ears, for they hear.}}
\bv{17}{\redlet{For verily I say unto you, that many prophets and righteous men desired to see the things which ye see, and saw them not; and to hear the things which ye hear, and heard them not.}}
\chapsec{Explanation of the Parable of the Sower}
\bv{18}{\redlet{Hear then ye the parable of the sower.}}
\bv{19}{\redlet{When any one heareth the word of the kingdom, and understandeth it not, \supptext{then} cometh the evil \supptext{one}, and snatcheth away that which hath been sown in his heart. This is he that was sown by the way side.}}
\bv{20}{\redlet{And he that was sown upon the rocky places, this is he that heareth the word, and straightway with joy receiveth it;}}
\bv{21}{\redlet{yet hath he not root in himself, but endureth for a while; and when tribulation or persecution ariseth because of the word, straightway he stumbleth.}}
\bv{22}{\redlet{And he that was sown among the thorns, this is he that heareth the word; and the care of the world, and the deceitfulness of riches, choke the word, and he becometh unfruitful.}}
\bv{23}{\redlet{And he that was sown upon the good ground, this is he that heareth the word, and understandeth it; who verily beareth fruit, and bringeth forth, some a hundredfold, some sixty, some thirty.''}}
\chapsec{The Tares among the Wheat}
\bv{24}{Another parable set he before them, saying, \redlet{``The kingdom of heaven is likened unto a man that sowed good seed in his field:}}
\bv{25}{\redlet{but while men slept, his enemy came and sowed tares also among the wheat, and went away.}}
\bv{26}{\redlet{But when the blade sprang up and brought forth fruit, then appeared the tares also.}}
\bv{27}{\redlet{And the servants of the householder came and said unto him, `Sir, didst thou not sow good seed in thy field? whence then hath it tares?'}}
\bv{28}{\redlet{And he said unto them, `An enemy hath done this.' And the servants say unto him, `Wilt thou then that we go and gather them up?'}}
\bv{29}{\redlet{But he saith, `Nay; lest haply while ye gather up the tares, ye root up the wheat with them.}}
\bv{30}{\redlet{Let both grow together until the harvest: and in the time of the harvest I will say to the reapers, `Gather up first the tares, and bind them in bundles to burn them; but gather the wheat into my barn.'{'}{''}}}
\chapsec{The Grain of Mustard Seed}
\bv{31}{Another parable set he before them, saying, \redlet{``The kingdom of heaven is like unto a grain of mustard seed, which a man took, and sowed in his field:}}
\bv{32}{\redlet{which indeed is less than all seeds; but when it is grown, it is greater than the herbs, and becometh a tree, so that the birds of the heaven come and lodge in the branches thereof.''}}
\chapsec{The Leaven}
\bv{33}{Another parable spake he unto them; \redlet{``The kingdom of heaven is like unto leaven, which a woman took, and hid in three measures of meal, till it was all leavened.''}}
\bv{34}{All these things spake Jesus in parables unto the multitudes; and without a parable spake he nothing unto them:}
\bv{35}{that it might be fulfilled which was spoken through the prophet, saying,}
\otQuote{Ps. 78:2}{I will open my mouth in parables;
I will utter things hidden from the foundation of the world.}
\chapsec{Explanation of the Parables of the Tares}
\bv{36}{Then he left the multitudes, and went into the house: and his disciples came unto him, saying, ``Explain unto us the parable of the tares of the field.''}
\bv{37}{And he answered and said, \redlet{``He that soweth the good seed is the Son of man;}}
\bv{38}{\redlet{and the field is the world; and the good seed, these are the sons of the kingdom; and the tares are the sons of the evil \supptext{one};}}
\bv{39}{\redlet{and the enemy that sowed them is the devil: and the harvest is the end of the world; and the reapers are angels.}}
\bv{40}{\redlet{As therefore the tares are gathered up and burned with fire; so shall it be in the end of the world.}}
\bv{41}{\redlet{The Son of man shall send forth his angels, and they shall gather out of his kingdom all things that cause stumbling, and them that do iniquity,}}
\bv{42}{\redlet{and shall cast them into the furnace of fire: there shall be the weeping and the gnashing of teeth.}}
\bv{43}{\redlet{Then shall the righteous shine forth as the sun in the kingdom of their Father. He that hath ears, let him hear.}}
\chapsec{The Hid Treasure}
\bv{44}{\redlet{The kingdom of heaven is like unto a treasure hidden in the field; which a man found, and hid; and in his joy he goeth and selleth all that he hath, and buyeth that field.}}
\chapsec{The Pearl of Great Price}
\bv{45}{\redlet{Again, the kingdom of heaven is like unto a man that is a merchant seeking goodly pearls:}}
\bv{46}{\redlet{and having found one pearl of great price, he went and sold all that he had, and bought it.}}
\chapsec{The Net}
\bv{47}{\redlet{Again, the kingdom of heaven is like unto a net, that was cast into the sea, and gathered of every kind:}}
\bv{48}{\redlet{which, when it was filled, they drew up on the beach; and they sat down, and gathered the good into vessels, but the bad they cast away.}}
\bv{49}{\redlet{So shall it be in the end of the world: the angels shall come forth, and sever the wicked from among the righteous,}}
\bv{50}{\redlet{and shall cast them into the furnace of fire: there shall be the weeping and the gnashing of teeth.}}
\bv{51}{\redlet{Have ye understood all these things?''} They say unto him, ``Yea.''}
\chapsec{Parable of the Householder}
\bv{52}{And he said unto them, \redlet{``Therefore every scribe who hath been made a disciple to the kingdom of heaven is like unto a man that is a householder, who bringeth forth out of his treasure things new and old.''}}
\chapsec{Jesus Returns to Nazareth}
\bv{53}{And it came to pass, when Jesus had finished these parables, he departed thence.}
\bv{54}{And coming into his own country he taught them in their synagogue, insomuch that they were astonished, and said, ``Whence hath this man this wisdom, and these mighty works?}
\bv{55}{Is not this the carpenter's son? is not his mother called Mary? and his brethren, James, and Joseph, and Simon, and Judas?}
\bv{56}{And his sisters, are they not all with us? Whence then hath this man all these things?''}
\bv{57}{And they were offended in him. But Jesus said unto them, \redlet{``A prophet is not without honor, save in his own country, and in his own house.''}}
\bv{58}{And he did not many mighty works there because of their unbelief.}
\chaphead{Chapter XIV}
\chapdesc{Herod's Troubled Conscience}
\lettrine[image=true, lines=4, findent=3pt, nindent=0pt]{Mt-A.eps}{t} that season Herod the tetrarch heard the report concerning Jesus,
\bv{2}{and said unto his servants, ``This is John the Baptist; he is risen from the dead; and therefore do these powers work in him.''}
\bv{3}{For Herod had laid hold on John, and bound him, and put him in prison for the sake of Herodias, his brother Philip's wife.}
\bv{4}{For John said unto him, ``It is not lawful for thee to have her.''}
\bv{5}{And when he would have put him to death, he feared the multitude, because they counted him as a prophet.}
\chapsec{The Murder of St. John the Baptist}
\bv{6}{But when Herod's birthday came, the daughter of Herodias danced in the midst, and pleased Herod.}
\bv{7}{Whereupon he promised with an oath to give her whatsoever she should ask.}
\bv{8}{And she, being put forward by her mother, saith, ``Give me here on a platter the head of John the Baptist.''}
\bv{9}{And the king was grieved; but for the sake of his oaths, and of them that sat at meat with him, he commanded it to be given;}
\bv{10}{and he sent and beheaded John in the prison.}
\bv{11}{And his head was brought on a platter, and given to the damsel: and she brought it to her mother.}
\bv{12}{And his disciples came, and took up the corpse, and buried him; and they went and told Jesus.}
\bv{13}{Now when Jesus heard \supptext{it}, he withdrew from thence in a boat, to a desert place apart: and when the multitudes heard \supptext{thereof}, they followed him on foot from the cities.}
\bv{14}{And he came forth, and saw a great multitude, and he had compassion on them, and healed their sick.}
\chapsec{The Feeding of the 5,000}
\bv{15}{And when even was come, the disciples came to him, saying, ``The place is desert, and the time is already past; send the multitudes away, that they may go into the villages, and buy themselves food.''}
\bv{16}{But Jesus said unto them, \redlet{``They have no need to go away; give ye them to eat.''}}
\bv{17}{And they say unto him, ``We have here but five loaves, and two fishes.''}
\bv{18}{And he said, \redlet{``Bring them hither to me.''}}
\bv{19}{And he commanded the multitudes to sit down on the grass; and he took the five loaves, and the two fishes, and looking up to heaven, he blessed, and brake and gave the loaves to the disciples, and the disciples to the multitudes.}
\bv{20}{And they all ate, and were filled: and they took up that which remained over of the broken pieces, twelve baskets full.}
\bv{21}{And they that did eat were about five thousand men, besides women and children.}
\chapsec{Jesus Walks on Water}
\bv{22}{And straightway he constrained the disciples to enter into the boat, and to go before him unto the other side, till he should send the multitudes away.}
\bv{23}{And after he had sent the multitudes away, he went up into the mountain apart to pray: and when even was come, he was there alone.}
\bv{24}{But the boat was now in the midst of the sea, distressed by the waves; for the wind was contrary.}
\bv{25}{And in the fourth watch of the night he came unto them, walking upon the sea.}
\bv{26}{And when the disciples saw him walking on the sea, they were troubled, saying, ``It is a ghost;'' and they cried out for fear.}
\bv{27}{But straightway Jesus spake unto them, saying, \redlet{``Be of good cheer; it is I; be not afraid.''}}
\bv{28}{And Peter answered him and said, ``Lord, if it be thou, bid me come unto thee upon the waters.''}
\bv{29}{And he said, \redlet{``Come.''} And Peter went down from the boat, and walked upon the waters to come to Jesus.}
\bv{30}{But when he saw the wind, he was afraid; and beginning to sink, he cried out, saying, ``Lord, save me.''}
\bv{31}{And immediately Jesus stretched forth his hand, and took hold of him, and saith unto him, \redlet{``O thou of little faith, wherefore didst thou doubt?''}}
\bv{32}{And when they were gone up into the boat, the wind ceased.}
\par
\bv{33}{And they that were in the boat worshipped him, saying, ``Of a truth thou art the Son of God.''}
\bv{34}{And when they had crossed over, they came to the land, unto Gennesaret.}
\bv{35}{And when the men of that place knew him, they sent into all that region round about, and brought unto him all that were sick;}
\bv{36}{and they besought him that they might only touch the border of his garment: and as many as touched were made whole.}
\chaphead{Chapter XV}
\chapdesc{Jesus Rebukes the Scribes \& Pharisees}
\lettrine[image=true, lines=4, findent=3pt, nindent=0pt]{T.ps}{hen} there come to Jesus from Jerusalem Pharisees and scribes, saying,
\bv{2}{``Why do thy disciples transgress the tradition of the elders? for they wash not their hands when they eat bread.''}
\bv{3}{And he answered and said unto them, \redlet{``Why do ye also transgress the commandment of God because of your tradition?}}
\bv{4}{\redlet{For God said, `Honor thy father and thy mother:' and, `He that speaketh evil of father or mother, let him die the death.'}}
\bv{5}{\redlet{But ye say, `Whosoever shall say to his father or his mother, `That wherewith thou mightest have been profited by me is given \supptext{to God};'}}
\bv{6}{\redlet{he shall not honor his father.' And ye have made void the word of God because of your tradition.}}
\bv{7}{\redlet{Ye hypocrites, well did Isaiah prophesy of you, saying,}}
\otQuote{Is. 29:13}{\bv{8}{\redlet{This people honoreth me with their lips;
But their heart is far from me.}}
\bv{9}{\redlet{But in vain do they worship me,
Teaching \supptext{as their} doctrines the precepts of men.''}}}
\bv{10}{And he called to him the multitude, and said unto them, \redlet{``Hear, and understand:}}
\bv{11}{\redlet{Not that which entereth into the mouth defileth the man; but that which proceedeth out of the mouth, this defileth the man.''}}
\par
\bv{12}{Then came the disciples, and said unto him, ``Knowest thou that the Pharisees were offended, when they heard this saying?''}
\bv{13}{But he answered and said, \redlet{``Every plant which my heavenly Father planted not, shall be rooted up.}}
\bv{14}{\redlet{Let them alone: they are blind guides. And if the blind guide the blind, both shall fall into a pit.''}}
\par
\bv{15}{And Peter answered and said unto him, ``Declare unto us the parable.''}
\bv{16}{And he said, \redlet{``Are ye also even yet without understanding?}}
\bv{17}{\redlet{Perceive ye not, that whatsoever goeth into the mouth passeth into the belly, and is cast out into the draught?}}
\bv{18}{\redlet{But the things which proceed out of the mouth come forth out of the heart; and they defile the man.}}
\bv{19}{\redlet{For out of the heart come forth evil thoughts, murders, adulteries, fornications, thefts, false witness, railings:}}
\bv{20}{\redlet{these are the things which defile the man; but to eat with unwashen hands defileth not the man.''}}
\bv{21}{And Jesus went out thence, and withdrew into the parts of Tyre and Sidon.}
\chapsec{Humbling of the Canaanite Woman}
\bv{22}{And behold, a Canaanite woman came out from those borders, and cried, saying, ``Have mercy on me, O Lord, thou son of David; my daughter is grievously vexed with a demon.''}
\bv{23}{But he answered her not a word. And his disciples came and besought him, saying, ``Send her away; for she crieth after us.''}
\bv{24}{But he answered and said, \redlet{``I was not sent but unto the lost sheep of the house of Israel.''}}
\bv{25}{But she came and worshipped him, saying, ``Lord, help me.''}
\bv{26}{And he answered and said, \redlet{``It is not meet to take the children's bread and cast it to the dogs.''}}
\bv{27}{But she said, ``Yea, Lord: for even the dogs eat of the crumbs which fall from their masters' table.''}
\bv{28}{Then Jesus answered and said unto her, \redlet{``O woman, great is thy faith: be it done unto thee even as thou wilt.''} And her daughter was healed from that hour.}
\chapsec{Jesus Heals Multitudes}
\bv{29}{And Jesus departed thence, and came nigh unto the sea of Galilee; and he went up into the mountain, and sat there.}
\bv{30}{And there came unto him great multitudes, having with them the lame, blind, dumb, maimed, and many others, and they cast them down at his feet; and he healed them:}
\bv{31}{insomuch that the multitude wondered, when they saw the dumb speaking, the maimed whole, and the lame walking, and the blind seeing: and they glorified the God of Israel.}
\chapsec{The Feeding of the 4,000}
\bv{32}{And Jesus called unto him his disciples, and said, \redlet{``I have compassion on the multitude, because they continue with me now three days and have nothing to eat: and I would not send them away fasting, lest haply they faint on the way.''}}
\bv{33}{And the disciples say unto him, ``Whence should we have so many loaves in a desert place as to fill so great a multitude?''}
\bv{34}{And Jesus said unto them, \redlet{``How many loaves have ye?''} And they said, ``Seven, and a few small fishes.''}
\bv{35}{And he commanded the multitude to sit down on the ground;}
\bv{36}{and he took the seven loaves and the fishes; and he gave thanks and brake, and gave to the disciples, and the disciples to the multitudes.}
\bv{37}{And they all ate, and were filled: and they took up that which remained over of the broken pieces, seven baskets full.}
\bv{38}{And they that did eat were four thousand men, besides women and children.}
\bv{39}{And he sent away the multitudes, and entered into the boat, and came into the borders of Magadan.}
\chaphead{Chapter XVI}
\chapdesc{Jesus Rebukes the Blind Pharisees}
\lettrine[image=true, lines=4, findent=3pt, nindent=0pt]{Mt-A.eps}{nd} the Pharisees and Sadducees came, and trying him asked him to show them a sign from heaven.
\bv{2}{But he answered and said unto them, \redlet{``When it is evening, ye say, `\supptext{It will be} fair weather: for the heaven is red.'}}
\bv{3}{\redlet{And in the morning, `\supptext{It will be} foul weather to-day: for the heaven is red and lowering.' Ye know how to discern the face of the heaven; but ye cannot \supptext{discern} the signs of the times.}}
\bv{4}{\redlet{An evil and adulterous generation seeketh after a sign; and there shall no sign be given unto it, but the sign of Jonah.'' And he left them, and departed.}}
\chapsec{Explanation of the Leaven}
\bv{5}{And the disciples came to the other side and forgot to take bread.}
\bv{6}{And Jesus said unto them, \redlet{``Take heed and beware of the leaven of the Pharisees and Sadducees.''}}
\bv{7}{And they reasoned among themselves, saying, ``We took no bread.''}
\bv{8}{And Jesus perceiving it said, \redlet{``O ye of little faith, why reason ye among yourselves, because ye have no bread?}}
\bv{9}{\redlet{Do ye not yet perceive, neither remember the five loaves of the five thousand, and how many baskets ye took up?}}
\bv{10}{\redlet{Neither the seven loaves of the four thousand, and how many baskets ye took up?}}
\bv{11}{\redlet{How is it that ye do not perceive that I spake not to you concerning bread? But beware of the leaven of the Pharisees and Sadducees.''}}
\bv{12}{Then understood they that he bade them not beware of the leaven of bread, but of the teaching of the Pharisees and Sadducees.}
\chapsec{St. Peter's Confession}
\bv{13}{Now when Jesus came into the parts of Cæsarea Philippi, he asked his disciples, saying, \redlet{``Who do men say that the Son of man is?''}}
\bv{14}{And they said, ``Some \supptext{say} John the Baptist; some, Elijah; and others, Jeremiah, or one of the prophets.''}
\bv{15}{He saith unto them, \redlet{``But who say ye that I am?''}}
\bv{16}{And Simon Peter answered and said, ``Thou art the Christ, the Son of the living God.''}
\bv{17}{And Jesus answered and said unto him, \redlet{``Blessed art thou, Simon Bar-Jonah: for flesh and blood hath not revealed it unto thee, but my Father who is in heaven.}}
\bv{18}{\redlet{And I also say unto thee, that thou art Peter, and upon this rock I will build my church; and the gates of Hades shall not prevail against it.}\mcomm{``shall not prevail against it,'' that is, the church.}}
\chapsec{Promise to Give the Keys to St. Peter}
\bv{19}{\redlet{I will give unto thee the keys of the kingdom of heaven: and whatsoever thou shalt bind on earth shall be bound in heaven; and whatsoever thou shalt loose on earth shall be loosed in heaven.''}}
\bv{20}{Then charged he the disciples that they should tell no man that he was the Christ.}
\chapsec{Jesus Foretells his Death \& Resurrection}
\bv{21}{From that time began Jesus to show unto his disciples, that he must go unto Jerusalem, and suffer many things of the elders and chief priests and scribes, and be killed, and the third day be raised up.}
\bv{22}{And Peter took him, and began to rebuke him, saying, ``Be it far from thee, Lord: this shall never be unto thee.''}
\bv{23}{But he turned, and said unto Peter, \redlet{``Get thee behind me, Satan: thou art a stumbling-block unto me: for thou mindest not the things of God, but the things of men.''}}
\bv{24}{Then said Jesus unto his disciples, \redlet{``If any man would come after me, let him deny himself, and take up his cross, and follow me.}}
\bv{25}{\redlet{For whosoever would save his life shall lose it: and whosoever shall lose his life for my sake shall find it.}}
\bv{26}{\redlet{For what shall a man be profited, if he shall gain the whole world, and forfeit his life? or what shall a man give in exchange for his life?}}
\bv{27}{\redlet{For the Son of man shall come in the glory of his Father with his angels; and then shall he render unto every man according to his deeds.}}
\chapdesc{The Promise of the Transfiguration}
\bv{28}{\redlet{Verily I say unto you, There are some of them that stand here, who shall in no wise taste of death, till they see the Son of man coming in his kingdom.''}}
\chaphead{Chapter XVII}
\chapdesc{The Transfiguration}
\lettrine[image=true, lines=4, findent=3pt, nindent=0pt]{Mt-A.eps}{nd} after six days Jesus taketh with him Peter, and James, and John his brother, and bringeth them up into a high mountain apart:
\bv{2}{and he was transfigured before them; and his face did shine as the sun, and his garments became white as the light.}
\bv{3}{And behold, there appeared unto them Moses and Elijah talking with him.}
\bv{4}{And Peter answered, and said unto Jesus, ``Lord, it is good for us to be here: if thou wilt, I will make here three tabernacles; one for thee, and one for Moses, and one for Elijah.''}
\bv{5}{While he was yet speaking, behold, a bright cloud overshadowed them: and behold, a voice out of the cloud, saying, \god{``This is my beloved Son, in whom I am well pleased; hear ye him.''}}
\bv{6}{And when the disciples heard it, they fell on their face, and were sore afraid.}
\bv{7}{And Jesus came and touched them and said, \redlet{``Arise, and be not afraid.''}}
\bv{8}{And lifting up their eyes, they saw no one, save Jesus only.}
\par
\bv{9}{And as they were coming down from the mountain, Jesus commanded them, saying, \redlet{``Tell the vision to no man, until the Son of man be risen from the dead.''}}
\bv{10}{And his disciples asked him, saying, ``Why then say the scribes that Elijah must first come?''}
\bv{11}{And he answered and said, \redlet{``Elijah indeed cometh, and shall restore all things:}}
\bv{12}{\redlet{but I say unto you, that Elijah is come already, and they knew him not, but did unto him whatsoever they would. Even so shall the Son of man also suffer of them.''}}
\bv{13}{Then understood the disciples that he spake unto them of John the Baptist.}
\chapsec{The Powerless Disciples}
\bv{14}{And when they were come to the multitude, there came to him a man, kneeling to him, and saying,}
\bv{15}{``Lord, have mercy on my son: for he is epileptic, and suffereth grievously; for oft-times he falleth into the fire, and oft-times into the water.}
\bv{16}{And I brought him to thy disciples, and they could not cure him.''}
\bv{17}{And Jesus answered and said, \redlet{``O faithless and perverse generation, how long shall I be with you? how long shall I bear with you? bring him hither to me.''}}
\bv{18}{And Jesus rebuked him; and the demon went out of him: and the boy was cured from that hour.}
\bv{19}{Then came the disciples to Jesus apart, and said, ``Why could not we cast it out?''}
\bv{20}{And he saith unto them, \redlet{``Because of your little faith: for verily I say unto you, If ye have faith as a grain of mustard seed, ye shall say unto this mountain, Remove hence to yonder place; and it shall remove; and nothing shall be impossible unto you.''}\mcomm{But this kind goeth not out save by prayer and fasting.}}
\chapsec{Jesus again Foretells his Death \& Resurrection}
\bv{22}{And while they abode in Galilee, Jesus said unto them, \redlet{``The Son of man shall be delivered up into the hands of men;}}
\bv{23}{\redlet{and they shall kill him, and the third day he shall be raised up.''} And they were exceeding sorry.}
\chapsec{The Miracle of the Tribute Money}
\bv{24}{And when they were come to Capernaum, they that received the half-shekel came to Peter, and said, ``Doth not your teacher pay the half-shekel?''}
\bv{25}{He saith, ``Yea.'' And when he came into the house, Jesus spake first to him, saying, \redlet{``What thinkest thou, Simon? the kings of the earth, from whom do they receive toll or tribute? from their sons, or from strangers?''}}
\bv{26}{And when he said, ``From strangers,'' Jesus said unto him, \redlet{``Therefore the sons are free.}}
\bv{27}{\redlet{But, lest we cause them to stumble, go thou to the sea, and cast a hook, and take up the fish that first cometh up; and when thou hast opened his mouth, thou shalt find a shekel: that take, and give unto them for me and thee.''}}
\chaphead{Chapter XVIII}
\chapdesc{The Sermon on the Child}
\lettrine[image=true, lines=4, findent=3pt, nindent=0pt]{Mt-I.eps}{n} that hour came the disciples unto Jesus, saying, ``Who then is greatest in the kingdom of heaven?''
\bv{2}{And he called to him a little child, and set him in the midst of them,}
\bv{3}{and said, \redlet{``Verily I say unto you, Except ye turn, and become as little children, ye shall in no wise enter into the kingdom of heaven.}}
\bv{4}{\redlet{Whosoever therefore shall humble himself as this little child, the same is the greatest in the kingdom of heaven.}}
\bv{5}{\redlet{And whoso shall receive one such little child in my name receiveth me:}}
\bv{6}{\redlet{but whoso shall cause one of these little ones that believe on me to stumble, it is profitable for him that a great millstone should be hanged about his neck, and \supptext{that} he should be sunk in the depth of the sea.}}
\par
\bv{7}{\redlet{Woe unto the world because of occasions of stumbling! for it must needs be that the occasions come; but woe to that man through whom the occasion cometh!}}
\bv{8}{\redlet{And if thy hand or thy foot causeth thee to stumble, cut it off, and cast it from thee: it is good for thee to enter into life maimed or halt, rather than having two hands or two feet to be cast into the eternal fire.}}
\bv{9}{\redlet{And if thine eye causeth thee to stumble, pluck it out, and cast it from thee: it is good for thee to enter into life with one eye, rather than having two eyes to be cast into the hell of fire.}}
\bv{10}{\redlet{See that ye despise not one of these little ones: for I say unto you, that in heaven their angels do always behold the face of my Father who is in heaven.}\mcomm{For the son of man came to save that which was lost.}}
\chapsec{The Lost Sheep}
\bv{12}{\redlet{How think ye? if any man have a hundred sheep, and one of them be gone astray, doth he not leave the ninety and nine, and go unto the mountains, and seek that which goeth astray?}}
\bv{13}{\redlet{And if so be that he find it, verily I say unto you, he rejoiceth over it more than over the ninety and nine which have not gone astray.}}
\bv{14}{\redlet{Even so it is not the will of your Father who is in heaven, that one of these little ones should perish.}}
\chapsec{Discipline in the Church}
\bv{15}{\redlet{And if thy brother sin against thee, go, show him his fault between thee and him alone: if he hear thee, thou hast gained thy brother.}}
\bv{16}{\redlet{But if he hear \supptext{thee} not, take with thee one or two more, that at the mouth of two witnesses or three every word may be established.}}
\bv{17}{\redlet{And if he refuse to hear them, tell it unto the church: and if he refuse to hear the church also, let him be unto thee as the Gentile and the publican.}}
\chapsec{The Keys Given to the Apostles}
\bv{18}{\redlet{Verily I say unto you, What things soever ye bind on earth shall be bound in heaven; and what things soever ye loose on earth shall be loosed in heaven.}}
\bv{19}{\redlet{Again I say unto you, that if two of you agree on earth as touching anything that they ask, it shall be done for them of my Father who is in heaven.}}
\bv{20}{\redlet{For where two or three are gathered together in my name, there am I in the midst of them.''}}
\chapsec{The Law of Forgiveness}
\bv{21}{Then came Peter and said to him, ``Lord, how oft shall my brother sin against me, and I forgive him? until seven times?''}
\bv{22}{Jesus saith unto him, \redlet{``I say not unto thee, Until seven times; but, Until seventy times seven.}}
\chapsec{The Parable of the Ungrateful Servant}
\bv{23}{\redlet{Therefore is the kingdom of heaven likened unto a certain king, who would make a reckoning with his servants.}}
\bv{24}{\redlet{And when he had begun to reckon, one was brought unto him, that owed him ten thousand talents.}}
\bv{25}{\redlet{But forasmuch as he had not \supptext{wherewith} to pay, his lord commanded him to be sold, and his wife, and children, and all that he had, and payment to be made.}}
\bv{26}{\redlet{The servant therefore fell down and worshipped him, saying, `Lord, have patience with me, and I will pay thee all.'}}
\bv{27}{\redlet{And the lord of that servant, being moved with compassion, released him, and forgave him the debt.}}
\bv{28}{\redlet{But that servant went out, and found one of his fellow-servants, who owed him a hundred shillings: and he laid hold on him, and took \supptext{him} by the throat, saying, `Pay what thou owest.'}}
\bv{29}{\redlet{So his fellow-servant fell down and besought him, saying, `Have patience with me, and I will pay thee.'}}
\bv{30}{\redlet{And he would not: but went and cast him into prison, till he should pay that which was due.}}
\par
\bv{31}{\redlet{So when his fellow-servants saw what was done, they were exceeding sorry, and came and told unto their lord all that was done.}}
\bv{32}{\redlet{Then his lord called him unto him, and saith to him, `Thou wicked servant, I forgave thee all that debt, because thou besoughtest me:}}
\bv{33}{\redlet{shouldest not thou also have had mercy on thy fellow-servant, even as I had mercy on thee?'}}
\bv{34}{\redlet{And his lord was wroth, and delivered him to the tormentors, till he should pay all that was due.}}
\bv{35}{\redlet{So shall also my heavenly Father do unto you, if ye forgive not every one his brother from your hearts.''}}
\chaphead{Chapter XIX}
\chapdesc{Jesus again in Judaea}
\lettrine[image=true, lines=4, findent=3pt, nindent=0pt]{Mt-A.eps}{nd} it came to pass when Jesus had finished these words, he departed from Galilee, and came into the borders of Judæa beyond the Jordan;
\bv{2}{and great multitudes followed him; and he healed them there.}
\chapsec{Teaching on Divorce}
\bv{3}{And there came unto him Pharisees, trying him, and saying, ``Is it lawful \supptext{for a man} to put away his wife for every cause?''}
\bv{4}{And he answered and said, \redlet{``Have ye not read, that he who made \supptext{them} from the beginning made them male and female,}}
\bv{5}{\redlet{and said, `For this cause shall a man leave his father and mother, and shall cleave to his wife; and the two shall become one flesh?'\mref{Gen. 2:24}}}
\bv{6}{\redlet{So that they are no more two, but one flesh. What therefore God hath joined together, let not man put asunder.''}}
\bv{7}{They say unto him, ``Why then did Moses command to give a bill of divorcement, and to put \supptext{her} away?''}
\bv{8}{He saith unto them, \redlet{``Moses for your hardness of heart suffered you to put away your wives: but from the beginning it hath not been so.}}
\bv{9}{\redlet{And I say unto you, Whosoever shall put away his wife, except for fornication, and shall marry another, committeth adultery: and he that marrieth her when she is put away committeth adultery.''}}
\chapsec{The Vocation of Celibacy}
\bv{10}{The disciples say unto him, ``If the case of the man is so with his wife, it is not expedient to marry.''}
\bv{11}{But he said unto them, \redlet{``Not all men can receive this saying, but they to whom it is given.}}
\bv{12}{\redlet{For there are eunuchs, that were so born from their mother's womb: and there are eunuchs, that were made eunuchs by men: and there are eunuchs, that made themselves eunuchs for the kingdom of heaven's sake. He that is able to receive it, let him receive it.''}}
\chapsec{Jesus Receives the Children}
\bv{13}{Then were there brought unto him little children, that he should lay his hands on them, and pray: and the disciples rebuked them.}
\bv{14}{But Jesus said, \redlet{``Suffer the little children, and forbid them not, to come unto me: for to such belongeth the kingdom of heaven.''}}
\bv{15}{And he laid his hands on them, and departed thence.}
\chapsec{The Rich Young Ruler}
\bv{16}{And behold, one came to him and said, ``Teacher, what good thing shall I do, that I may have eternal life?''}
\bv{17}{And he said unto him, \redlet{``Why askest thou me concerning that which is good? One there is who is good: but if thou wouldest enter into life, keep the commandments.''}}
\bv{18}{He saith unto him, ``Which?'' And Jesus said, \redlet{``Thou shalt not kill, Thou shalt not commit adultery, Thou shalt not steal, Thou shalt not bear false witness,}}
\bv{19}{\redlet{Honor thy father and thy mother; and, Thou shalt love thy neighbor as thyself.''}}
\bv{20}{The young man saith unto him, ``All these things have I observed: what lack I yet?''}
\bv{21}{Jesus said unto him, \redlet{``If thou wouldest be perfect, go, sell that which thou hast, and give to the poor, and thou shalt have treasure in heaven: and come, follow me.''}}
\bv{22}{But when the young man heard the saying, he went away sorrowful; for he was one that had great possessions.}
\bv{23}{And Jesus said unto his disciples, \redlet{``Verily I say unto you, It is hard for a rich man to enter into the kingdom of heaven.}}
\bv{24}{\redlet{And again I say unto you, It is easier for a camel to go through a needle's eye, than for a rich man to enter into the kingdom of God.''}}
\bv{25}{And when the disciples heard it, they were astonished exceedingly, saying, ``Who then can be saved?''}
\bv{26}{And Jesus looking upon \supptext{them} said to them, \redlet{``With men this is impossible; but with God all things are possible.''}}
\chapsec{The Apostles' Place in the Kingdom}
\bv{27}{Then answered Peter and said unto him, ``Lo, we have left all, and followed thee; what then shall we have?''}
\bv{28}{And Jesus said unto them, \redlet{``Verily I say unto you, that ye who have followed me, in the regeneration when the Son of man shall sit on the throne of his glory, ye also shall sit upon twelve thrones, judging the twelve tribes of Israel.}}
\bv{29}{\redlet{And every one that hath left houses, or brethren, or sisters, or father, or mother, or children, or lands, for my name's sake, shall receive a hundredfold, and shall inherit eternal life.}}
\bv{30}{\redlet{But many shall be last \supptext{that are} first; and first \supptext{that are} last.''}}
\chaphead{Chapter XX}
\chapdesc{Parable of the Labourers in the Vineyard}
\lettrine[image=true, lines=4, findent=3pt, nindent=0pt]{Lk-F.eps}{\redlet{or}} \redlet{the kingdom of heaven is like unto a man that was a householder, who went out early in the morning to hire laborers into his vineyard.}
\bv{2}{\redlet{And when he had agreed with the laborers for a shilling a day, he sent them into his vineyard.}}
\bv{3}{\redlet{And he went out about the third hour, and saw others standing in the marketplace idle;}}
\bv{4}{\redlet{and to them he said, `Go ye also into the vineyard, and whatsoever is right I will give you.' And they went their way.}}
\bv{5}{\redlet{Again he went out about the sixth and the ninth hour, and did likewise.}}
\bv{6}{\redlet{And about the eleventh \supptext{hour} he went out, and found others standing; and he saith unto them, `Why stand ye here all the day idle?'}}
\bv{7}{\redlet{They say unto him, `Because no man hath hired us.' He saith unto them, `Go ye also into the vineyard.'}}
\par
\bv{8}{\redlet{And when even was come, the lord of the vineyard saith unto his steward, `Call the laborers, and pay them their hire, beginning from the last unto the first.'}}
\bv{9}{\redlet{And when they came that \supptext{were hired} about the eleventh hour, they received every man a denarius.}}
\bv{10}{\redlet{And when the first came, they supposed that they would receive more; and they likewise received every man a denarius.}}
\bv{11}{\redlet{And when they received it, they murmured against the householder,}}
\bv{12}{\redlet{saying, `These last have spent \supptext{but} one hour, and thou hast made them equal unto us, who have borne the burden of the day and the scorching heat.'}}
\bv{13}{\redlet{But he answered and said to one of them, `Friend, I do thee no wrong: didst not thou agree with me for a denarius?}}
\bv{14}{\redlet{Take up that which is thine, and go thy way; it is my will to give unto this last, even as unto thee.}}
\bv{15}{\redlet{Is it not lawful for me to do what I will with mine own? or is thine eye evil, because I am good?'}}
\bv{16}{\redlet{So the last shall be first, and the first last.''}}
\par
\bv{17}{And as Jesus was going up to Jerusalem, he took the twelve disciples apart, and on the way he said unto them,}
\bv{18}{\redlet{``Behold, we go up to Jerusalem; and the Son of man shall be delivered unto the chief priests and scribes; and they shall condemn him to death,}}
\bv{19}{\redlet{and shall deliver him unto the Gentiles to mock, and to scourge, and to crucify: and the third day he shall be raised up.''}}
\chapsec{Sts. James \& John with their Mother}
\bv{20}{Then came to him the mother of the sons of Zebedee with her sons, worshipping \supptext{him}, and asking a certain thing of him.}
\bv{21}{And he said unto her, \redlet{``What wouldest thou?''} She saith unto him, ``Command that these my two sons may sit, one on thy right hand, and one on thy left hand, in thy kingdom.''}
\bv{22}{But Jesus answered and said, \redlet{``Ye know not what ye ask. Are ye able to drink the cup that I am about to drink?''} They say unto him, ``We are able.''}
\bv{23}{He saith unto them, \redlet{``My cup indeed ye shall drink: but to sit on my right hand, and on \supptext{my} left hand, is not mine to give; but \supptext{it is for them} for whom it hath been prepared of my Father.''}}
\par
\bv{24}{And when the ten heard it, they were moved with indignation concerning the two brethren.}
\bv{25}{But Jesus called them unto him, and said, \redlet{``Ye know that the rulers of the Gentiles lord it over them, and their great ones exercise authority over them.}}
\bv{26}{\redlet{Not so shall it be among you: but whosoever would become great among you shall be your minister;}}
\bv{27}{\redlet{and whosoever would be first among you shall be your servant:}}
\bv{28}{\redlet{even as the Son of man came not to be ministered unto, but to minister, and to give his life a ransom for many.''}}
\chapsec{The Healing of the Two Blind Men}
\bv{29}{And as they went out from Jericho, a great multitude followed him.}
\bv{30}{And behold, two blind men sitting by the way side, when they heard that Jesus was passing by, cried out, saying, ``Lord, have mercy on us, thou son of David.''}
\bv{31}{And the multitude rebuked them, that they should hold their peace: but they cried out the more, saying, ``Lord, have mercy on us, thou son of David.''}
\bv{32}{And Jesus stood still, and called them, and said, \redlet{``What will ye that I should do unto you?''}}
\bv{33}{They say unto him, ``Lord, that our eyes may be opened.''}
\bv{34}{And Jesus, being moved with compassion, touched their eyes; and straightway they received their sight, and followed him.}
\chaphead{Chapter XXI}
\chapdesc{Preparation for the Entrance into Jerusalem}
\lettrine[image=true, lines=4, findent=3pt, nindent=0pt]{Mt-A.eps}{nd} when they drew nigh unto Jerusalem, and came unto Bethphage, unto the mount of Olives, then Jesus sent two disciples,
\bv{2}{saying unto them, \redlet{``Go into the village that is over against you, and straightway ye shall find an ass tied, and a colt with her: loose \supptext{them}, and bring \supptext{them} unto me.}}
\bv{3}{\redlet{And if any one say aught unto you, ye shall say, `The Lord hath need of them; and straightway he will send them.'{''}}}
\bv{4}{Now this is come to pass, that it might be fulfilled which was spoken through the prophet, saying,}
\otQuote{Zech. 9:9}{\bv{5}{Tell ye the daughter of Zion,
Behold, thy King cometh unto thee,
Meek, and riding upon an ass,
And upon a colt the foal of an ass.}}
\bv{6}{And the disciples went, and did even as Jesus appointed them,}
\bv{7}{and brought the ass, and the colt, and put on them their garments; and he sat thereon.}
\chapsec{Entrance into Jerusalem}
\bv{8}{And the most part of the multitude spread their garments in the way; and others cut branches from the trees, and spread them in the way.}
\bv{9}{And the multitudes that went before him, and that followed, cried, saying, ``Hosanna to the son of David: Blessed \supptext{is} he that cometh in the name of the Lord; Hosanna in the highest.''}
\bv{10}{And when he was come into Jerusalem, all the city was stirred, saying, ``Who is this?''}
\bv{11}{And the multitudes said, ``This is the prophet, Jesus, from Nazareth of Galilee.''}
\chapsec{Jesus' Second Purification of the Temple}
\bv{12}{And Jesus entered into the temple of God, and cast out all them that sold and bought in the temple, and overthrew the tables of the money-changers, and the seats of them that sold the doves;}
\bv{13}{and he saith unto them, \redlet{``It is written, `My house shall be called a house of prayer: but ye make it a den of robbers.'{''}}}
\bv{14}{And the blind and the lame came to him in the temple; and he healed them.}
\bv{15}{But when the chief priests and the scribes saw the wonderful things that he did, and the children that were crying in the temple and saying, ``Hosanna to the son of David;'' they were moved with indignation,}
\bv{16}{and said unto him, ``Hearest thou what these are saying?'' And Jesus saith unto them, \redlet{``Yea: did ye never read, `Out of the mouth of babes and sucklings thou hast perfected praise?'{''}}}
\bv{17}{And he left them, and went forth out of the city to Bethany, and lodged there.}
\chapsec{The Barren Fig Tree Cursed}
\bv{18}{Now in the morning as he returned to the city, he hungered.}
\bv{19}{And seeing a fig tree by the way side, he came to it, and found nothing thereon, but leaves only; and he saith unto it, \redlet{``Let there be no fruit from thee henceforward for ever.''} And immediately the fig tree withered away.}
\bv{20}{And when the disciples saw it, they marvelled, saying, ``How did the fig tree immediately wither away?''}
\bv{21}{And Jesus answered and said unto them, \redlet{``Verily I say unto you, If ye have faith, and doubt not, ye shall not only do what is done to the fig tree, but even if ye shall say unto this mountain, `Be thou taken up and cast into the sea,' it shall be done.}}
\bv{22}{\redlet{And all things, whatsoever ye shall ask in prayer, believing, ye shall receive.''}}
\chapsec{Jesus' Authority Questioned}
\bv{23}{And when he was come into the temple, the chief priests and the elders of the people came unto him as he was teaching, and said, ``By what authority doest thou these things? and who gave thee this authority?''}
\bv{24}{And Jesus answered and said unto them, \redlet{``I also will ask you one question, which if ye tell me, I likewise will tell you by what authority I do these things.}}
\bv{25}{\redlet{The baptism of John, whence was it? from heaven or from men?''} And they reasoned with themselves, saying, ``If we shall say, `From heaven;' he will say unto us, `Why then did ye not believe him?'}
\bv{26}{But if we shall say, `From men;' we fear the multitude; for all hold John as a prophet.''}
\bv{27}{And they answered Jesus, and said, ``We know not.'' He also said unto them, \redlet{``Neither tell I you by what authority I do these things.}}
\chapsec{Parable of the Two Sons}
\bv{28}{\redlet{But what think ye? A man had two sons; and he came to the first, and said, `Son, go work to-day in the vineyard.'}}
\bv{29}{\redlet{And he answered and said, `I will not:' but afterward he repented himself, and went.}}
\bv{30}{\redlet{And he came to the second, and said likewise. And he answered and said, `I \supptext{go}, sir:' and went not.}}
\bv{31}{\redlet{Which of the two did the will of his father?''} They say, ``The first.'' Jesus saith unto them, \redlet{``Verily I say unto you, that the publicans and the harlots go into the kingdom of God before you.}}
\bv{32}{\redlet{For John came unto you in the way of righteousness, and ye believed him not; but the publicans and the harlots believed him: and ye, when ye saw it, did not even repent yourselves afterward, that ye might believe him.}}
\chapsec{Parable of the Householder}
\bv{33}{\redlet{Hear another parable: There was a man that was a householder, who planted a vineyard, and set a hedge about it, and digged a winepress in it, and built a tower, and let it out to husbandmen, and went into another country.}}
\bv{34}{\redlet{And when the season of the fruits drew near, he sent his servants to the husbandmen, to receive his fruits.}}
\bv{35}{\redlet{And the husbandmen took his servants, and beat one, and killed another, and stoned another.}}
\bv{36}{\redlet{Again, he sent other servants more than the first: and they did unto them in like manner.}}
\bv{37}{\redlet{But afterward he sent unto them his son, saying, `They will reverence my son.'}}
\bv{38}{\redlet{But the husbandmen, when they saw the son, said among themselves, `This is the heir; come, let us kill him, and take his inheritance.'}}
\bv{39}{\redlet{And they took him, and cast him forth out of the vineyard, and killed him.}}
\bv{40}{\redlet{When therefore the lord of the vineyard shall come, what will he do unto those husbandmen?''}}
\bv{41}{They say unto him, ``He will miserably destroy those miserable men, and will let out the vineyard unto other husbandmen, who shall render him the fruits in their seasons.''}
\bv{42}{Jesus saith unto them, \redlet{``Did ye never read in the scriptures,}}
\otQuote{Ps. 118:22-3}{\redlet{The stone which the builders rejected,
The same was made the head of the corner;
This was from the Lord,
And it is marvellous in our eyes?}}
\bv{43}{\redlet{Therefore say I unto you, The kingdom of God shall be taken away from you, and shall be given to a nation bringing forth the fruits thereof.}}
\bv{44}{\redlet{And he that falleth on this stone shall be broken to pieces: but on whomsoever it shall fall, it will scatter him as dust.''}}
\bv{45}{And when the chief priests and the Pharisees heard his parables, they perceived that he spake of them.}
\bv{46}{And when they sought to lay hold on him, they feared the multitudes, because they took him for a prophet.}
\chaphead{Chapter XXII}
\chapdesc{Parable of the Marriage Feast}
\lettrine[image=true, lines=4, findent=3pt, nindent=0pt]{Mt-A.eps}{nd} Jesus answered and spake again in parables unto them, saying,
\bv{2}{\redlet{``The kingdom of heaven is likened unto a certain king, who made a marriage feast for his son,}}
\bv{3}{\redlet{and sent forth his servants to call them that were bidden to the marriage feast: and they would not come.}}
\bv{4}{\redlet{Again he sent forth other servants, saying, `Tell them that are bidden, `Behold, I have made ready my dinner; my oxen and my fatlings are killed, and all things are ready: come to the marriage feast.'{'}}}
\bv{5}{\redlet{But they made light of it, and went their ways, one to his own farm, another to his merchandise;}}
\bv{6}{\redlet{and the rest laid hold on his servants, and treated them shamefully, and killed them.}}
\bv{7}{\redlet{But the king was wroth; and he sent his armies, and destroyed those murderers, and burned their city.}}
\bv{8}{\redlet{Then saith he to his servants, `The wedding is ready, but they that were bidden were not worthy.}}
\bv{9}{\redlet{Go ye therefore unto the partings of the highways, and as many as ye shall find, bid to the marriage feast.'}}
\par
\bv{10}{\redlet{And those servants went out into the highways, and gathered together all as many as they found, both bad and good: and the wedding was filled with guests.}}
\bv{11}{\redlet{But when the king came in to behold the guests, he saw there a man who had not on a wedding-garment:}}
\bv{12}{\redlet{and he saith unto him, `Friend, how camest thou in hither not having a wedding-garment?' And he was speechless.}}
\bv{13}{\redlet{Then the king said to the servants, `Bind him hand and foot, and cast him out into the outer darkness; there shall be the weeping and the gnashing of teeth.'}}
\bv{14}{\redlet{For many are called, but few chosen.''}}
\chapsec{Jesus Answers the Herodians}
\bv{15}{Then went the Pharisees, and took counsel how they might ensnare him in \supptext{his} talk.}
\bv{16}{And they send to him their disciples, with the Herodians, saying, ``Teacher, we know that thou art true, and teachest the way of God in truth, and carest not for any one: for thou regardest not the person of men.}
\bv{17}{Tell us therefore, What thinkest thou? Is it lawful to give tribute unto Cæsar, or not?''}
\bv{18}{But Jesus perceived their wickedness, and said, \redlet{``Why make ye trial of me, ye hypocrites?}}
\bv{19}{\redlet{Show me the tribute money.''} And they brought unto him a denarius.}
\bv{20}{And he saith unto them, \redlet{``Whose is this image and superscription?''}}
\bv{21}{They say unto him, ``Cæsar's.'' Then saith he unto them, \redlet{``Render therefore unto Cæsar the things that are Cæsar's; and unto God the things that are God's.''}}
\bv{22}{And when they heard it, they marvelled, and left him, and went away.}
\chapsec{Jesus Answers the Sadducees}
\bv{23}{On that day there came to him Sadducees, they that say that there is no resurrection: and they asked him,}
\bv{24}{saying, ``Teacher, Moses said, `If a man die, having no children, his brother shall marry his wife, and raise up seed unto his brother.'}
\bv{25}{Now there were with us seven brethren: and the first married and deceased, and having no seed left his wife unto his brother;}
\bv{26}{in like manner the second also, and the third, unto the seventh.}
\bv{27}{And after them all, the woman died.}
\bv{28}{In the resurrection therefore whose wife shall she be of the seven? for they all had her.''}
\bv{29}{But Jesus answered and said unto them, \redlet{``Ye do err, not knowing the scriptures, nor the power of God.}}
\bv{30}{\redlet{For in the resurrection they neither marry, nor are given in marriage, but are as angels in heaven.}}
\bv{31}{\redlet{But as touching the resurrection of the dead, have ye not read that which was spoken unto you by God, saying,}}
\bv{32}{\redlet{`I am the God of Abraham, and the God of Isaac, and the God of Jacob?' God is not \supptext{the God} of the dead, but of the living.''}}
\bv{33}{And when the multitudes heard it, they were astonished at his teaching.}
\chapsec{Jesus Answers the Pharisees}
\bv{34}{But the Pharisees, when they heard that he had put the Sadducees to silence, gathered themselves together.}
\bv{35}{And one of them, a lawyer, asked him a question, trying him:}
\bv{36}{``Teacher, which is the great commandment in the law?''}
\bv{37}{And he said unto him, \redlet{``Thou shalt love the Lord thy God with all thy heart, and with all thy soul, and with all thy mind.}}
\bv{38}{\redlet{This is the great and first commandment.}}
\bv{39}{\redlet{And a second like \supptext{unto it} is this, `Thou shalt love thy neighbor as thyself.'}}
\bv{40}{\redlet{On these two commandments the whole law hangeth, and the prophets.''}}
\chapsec{Jesus Questions the Pharisees}
\bv{41}{Now while the Pharisees were gathered together, Jesus asked them a question,}
\bv{42}{saying, \redlet{``What think ye of the Christ? whose son is he?''} They say unto him, ``\supptext{The son} of David.''}
\bv{43}{He saith unto them, \redlet{``How then doth David in the Spirit call him Lord, saying,}}
\otQuote{Ps. 110:1}{\redlet{\bv{44}{The Lord said unto my Lord,
Sit thou on my right hand,
Till I put thine enemies underneath thy feet?}}}
\bv{45}{\redlet{If David then calleth him Lord, how is he his son?''}}
\bv{46}{And no one was able to answer him a word, neither durst any man from that day forth ask him any more questions.}
\chaphead{Chapter XXIII}
\chapdesc{The Marks of a Pharisee}
\lettrine[image=true, lines=4, findent=3pt, nindent=0pt]{T.ps}{hen} spake Jesus to the multitudes and to his disciples,
\bv{2}{saying, \redlet{``The scribes and the Pharisees sit on Moses' seat:}}
\bv{3}{\redlet{all things therefore whatsoever they bid you, \supptext{these} do and observe: but do not ye after their works; for they say, and do not.}}
\bv{4}{\redlet{Yea, they bind heavy burdens and grievous to be borne, and lay them on men's shoulders; but they themselves will not move them with their finger.}}
\bv{5}{\redlet{But all their works they do to be seen of men: for they make broad their phylacteries, and enlarge the borders \supptext{of their garments},}}
\bv{6}{\redlet{and love the chief place at feasts, and the chief seats in the synagogues,}}
\bv{7}{\redlet{and the salutations in the marketplaces, and to be called of men, Rabbi.}}
\bv{8}{\redlet{But be not ye called Rabbi: for one is your teacher, and all ye are brethren.}}
\bv{9}{\redlet{And call no man your father on the earth: for one is your Father, \supptext{even} he who is in heaven.}}
\bv{10}{\redlet{Neither be ye called masters: for one is your master, \supptext{even} the Christ.}}
\bv{11}{\redlet{But he that is greatest among you shall be your servant.}}
\bv{12}{\redlet{And whosoever shall exalt himself shall be humbled; and whosoever shall humble himself shall be exalted.}}
\chapsec{Jesus Denounces Woe upon the Pharisees}
\bv{13}{\redlet{But woe unto you, scribes and Pharisees, hypocrites! because ye shut the kingdom of heaven against men: for ye enter not in yourselves, neither suffer ye them that are entering in to enter.\mcomm{Woe unto you scribes and Pharisees, hypocrites! for you devour widows' houses, even while for a pretence ye make long prayes: therefore ye shall receive greater condemnation.}}}
\bv{15}{\redlet{Woe unto you, scribes and Pharisees, hypocrites! for ye compass sea and land to make one proselyte; and when he is become so, ye make him twofold more a son of hell than yourselves.}}
\bv{16}{\redlet{Woe unto you, ye blind guides, that say, Whosoever shall swear by the temple, it is nothing; but whosoever shall swear by the gold of the temple, he is a debtor.}}
\par
\bv{17}{\redlet{Ye fools and blind: for which is greater, the gold, or the temple that hath sanctified the gold?}}
\bv{18}{\redlet{And, Whosoever shall swear by the altar, it is nothing; but whosoever shall swear by the gift that is upon it, he is a debtor.}}
\bv{19}{\redlet{Ye blind: for which is greater, the gift, or the altar that sanctifieth the gift?}}
\bv{20}{\redlet{He therefore that sweareth by the altar, sweareth by it, and by all things thereon.}}
\bv{21}{\redlet{And he that sweareth by the temple, sweareth by it, and by him that dwelleth therein.}}
\bv{22}{\redlet{And he that sweareth by the heaven, sweareth by the throne of God, and by him that sitteth thereon.}}
\par
\bv{23}{\redlet{Woe unto you, scribes and Pharisees, hypocrites! for ye tithe mint and anise and cummin, and have left undone the weightier matters of the law, justice, and mercy, and faith: but these ye ought to have done, and not to have left the other undone.}}
\bv{24}{\redlet{Ye blind guides, that strain out the gnat, and swallow the camel!}}
\bv{25}{\redlet{Woe unto you, scribes and Pharisees, hypocrites! for ye cleanse the outside of the cup and of the platter, but within they are full from extortion and excess.}}
\bv{26}{\redlet{Thou blind Pharisee, cleanse first the inside of the cup and of the platter, that the outside thereof may become clean also.}}
\par
\bv{27}{\redlet{Woe unto you, scribes and Pharisees, hypocrites! for ye are like unto whited sepulchres, which outwardly appear beautiful, but inwardly are full of dead men's bones, and of all uncleanness.}}
\bv{28}{\redlet{Even so ye also outwardly appear righteous unto men, but inwardly ye are full of hypocrisy and iniquity.}}
\bv{29}{\redlet{Woe unto you, scribes and Pharisees, hypocrites! for ye build the sepulchres of the prophets, and garnish the tombs of the righteous,}}
\bv{30}{\redlet{and say, `If we had been in the days of our fathers, we should not have been partakers with them in the blood of the prophets.'}}
\bv{31}{\redlet{Wherefore ye witness to yourselves, that ye are sons of them that slew the prophets.}}
\bv{32}{\redlet{Fill ye up then the measure of your fathers.}}
\par
\bv{33}{\redlet{Ye serpents, ye offspring of vipers, how shall ye escape the judgement of hell?}}
\bv{34}{\redlet{Therefore, behold, I send unto you prophets, and wise men, and scribes: some of them shall ye kill and crucify; and some of them shall ye scourge in your synagogues, and persecute from city to city:}}
\bv{35}{\redlet{that upon you may come all the righteous blood shed on the earth, from the blood of Abel the righteous unto the blood of Zechariah son of Barachiah, whom ye slew between the sanctuary and the altar.}}
\bv{36}{\redlet{Verily I say unto you, All these things shall come upon this generation.}}
\chapsec{The Lament over Jerusalem}
\bv{37}{\redlet{O Jerusalem, Jerusalem, that killeth the prophets, and stoneth them that are sent unto her! how often would I have gathered thy children together, even as a hen gathereth her chickens under her wings, and ye would not!}}
\bv{38}{\redlet{Behold, your house is left unto you desolate.}}
\bv{39}{\redlet{For I say unto you, Ye shall not see me henceforth, till ye shall say, `Blessed \supptext{is} he that cometh in the name of the Lord.'{''}}}
\chaphead{Chapter XXIV}
\chapdesc{The Olivet Discourse}
\lettrine[image=true, lines=4, findent=3pt, nindent=0pt]{Mt-A.eps}{nd} Jesus went out from the temple, and was going on his way; and his disciples came to him to show him the buildings of the temple.
\bv{2}{But he answered and said unto them, \redlet{``See ye not all these things? verily I say unto you, There shall not be left here one stone upon another, that shall not be thrown down.''}}
\par
\bv{3}{And as he sat on the mount of Olives, the disciples came unto him privately, saying, ``Tell us, when shall these things be? and what \supptext{shall be} the sign of thy coming, and of the end of the world?''}
\par
\bv{4}{And Jesus answered and said unto them, \redlet{``Take heed that no man lead you astray.}}
\bv{5}{\redlet{For many shall come in my name, saying, `I am the Christ;' and shall lead many astray.}}
\bv{6}{\redlet{And ye shall hear of wars and rumors of wars; see that ye be not troubled: for \supptext{these things} must needs come to pass; but the end is not yet.}}
\bv{7}{\redlet{For nation shall rise against nation, and kingdom against kingdom; and there shall be famines and earthquakes in divers places.}}
\bv{8}{\redlet{But all these things are the beginning of travail.}}
\bv{9}{\redlet{Then shall they deliver you up unto tribulation, and shall kill you: and ye shall be hated of all the nations for my name's sake.}}
\bv{10}{\redlet{And then shall many stumble, and shall deliver up one another, and shall hate one another.}}
\bv{11}{\redlet{And many false prophets shall arise, and shall lead many astray.}}
\bv{12}{\redlet{And because iniquity shall be multiplied, the love of the many shall wax cold.}}
\bv{13}{\redlet{But he that endureth to the end, the same shall be saved.}}
\bv{14}{\redlet{And this gospel of the kingdom shall be preached in the whole world for a testimony unto all the nations; and then shall the end come.}}
\chapsec{The Great Tribulation}
\bv{15}{\redlet{When therefore ye see the abomination of desolation, which was spoken of through Daniel the prophet, standing in the holy place (let him that readeth understand),}}
\bv{16}{\redlet{then let them that are in Judæa flee unto the mountains:}}
\bv{17}{\redlet{let him that is on the housetop not go down to take out the things that are in his house:}}
\bv{18}{\redlet{and let him that is in the field not return back to take his cloak.}}
\bv{19}{\redlet{But woe unto them that are with child and to them that give suck in those days!}}
\bv{20}{\redlet{And pray ye that your flight be not in the winter, neither on a sabbath:}}
\bv{21}{\redlet{for then shall be great tribulation, such as hath not been from the beginning of the world until now, no, nor ever shall be.}}
\bv{22}{\redlet{And except those days had been shortened, no flesh would have been saved: but for the elect's sake those days shall be shortened.}}
\par
\bv{23}{\redlet{Then if any man shall say unto you, `Lo, here is the Christ,' or, `Here;' believe \supptext{it} not.}}
\bv{24}{\redlet{For there shall arise false Christs, and false prophets, and shall show great signs and wonders; so as to lead astray, if possible, even the elect.}}
\bv{25}{\redlet{Behold, I have told you beforehand.}}
\bv{26}{\redlet{If therefore they shall say unto you, `Behold, he is in the wilderness;' go not forth: `Behold, he is in the inner chambers;' believe \supptext{it} not.}}
\chapsec{The Return of the King in Glory}
\bv{27}{\redlet{For as the lightning cometh forth from the east, and is seen even unto the west; so shall be the coming of the Son of man.}}
\bv{28}{\redlet{Wheresoever the carcase is, there will the eagles be gathered together.}}
\bv{29}{\redlet{But immediately after the tribulation of those days the sun shall be darkened, and the moon shall not give her light, and the stars shall fall from heaven, and the powers of the heavens shall be shaken:}}
\bv{30}{\redlet{and then shall appear the sign of the Son of man in heaven: and then shall all the tribes of the earth mourn, and they shall see the Son of man coming on the clouds of heaven with power and great glory.}}
\bv{31}{\redlet{And he shall send forth his angels with a great sound of a trumpet, and they shall gather together his elect from the four winds, from one end of heaven to the other.}}
\chapsec{Parable of the Fig Tree}
\bv{32}{\redlet{Now from the fig tree learn her parable: when her branch is now become tender, and putteth forth its leaves, ye know that the summer is nigh;}}
\bv{33}{\redlet{even so ye also, when ye see all these things, know ye that he is nigh, \supptext{even} at the doors.}}
\bv{34}{\redlet{Verily I say unto you, This generation shall not pass away, till all these things be accomplished.}}
\bv{35}{\redlet{Heaven and earth shall pass away, but my words shall not pass away.}}
\bv{36}{\redlet{But of that day and hour knoweth no one, not even the angels of heaven, neither the Son, but the Father only.}}
\par
\bv{37}{\redlet{And as \supptext{were} the days of Noah, so shall be the coming of the Son of man.}}
\bv{38}{\redlet{For as in those days which were before the flood they were eating and drinking, marrying and giving in marriage, until the day that Noah entered into the ark,}}
\bv{39}{\redlet{and they knew not until the flood came, and took them all away; so shall be the coming of the Son of man.}}
\bv{40}{\redlet{Then shall two men be in the field; one is taken, and one is left:}}
\bv{41}{\redlet{two women \supptext{shall be} grinding at the mill; one is taken, and one is left.}}
\bv{42}{\redlet{Watch therefore: for ye know not on what day your Lord cometh.}}
\bv{43}{\redlet{But know this, that if the master of the house had known in what watch the thief was coming, he would have watched, and would not have suffered his house to be broken through.}}
\bv{44}{\redlet{Therefore be ye also ready; for in an hour that ye think not the Son of man cometh.}}
\par
\bv{45}{\redlet{Who then is the faithful and wise servant, whom his lord hath set over his household, to give them their food in due season?}}
\bv{46}{\redlet{Blessed is that servant, whom his lord when he cometh shall find so doing.}}
\bv{47}{\redlet{Verily I say unto you, that he will set him over all that he hath.}}
\bv{48}{\redlet{But if that evil servant shall say in his heart, `My lord tarrieth;'}}
\bv{49}{\redlet{and shall begin to beat his fellow-servants, and shall eat and drink with the drunken;}}
\bv{50}{\redlet{the lord of that servant shall come in a day when he expecteth not, and in an hour when he knoweth not,}}
\bv{51}{\redlet{and shall cut him asunder, and appoint his portion with the hypocrites: there shall be the weeping and the gnashing of teeth.}}
\chaphead{Chapter XXV}
\chapdesc{Parable of the Foolish Virgins}
\lettrine[image=true, lines=4, findent=3pt, nindent=0pt]{T.ps}{\redlet{hen}} \redlet{shall the kingdom of heaven be likened unto ten virgins, who took their lamps, and went forth to meet the bridegroom.}
\bv{2}{\redlet{And five of them were foolish, and five were wise.}}
\bv{3}{\redlet{For the foolish, when they took their lamps, took no oil with them:}}
\bv{4}{\redlet{but the wise took oil in their vessels with their lamps.}}
\bv{5}{\redlet{Now while the bridegroom tarried, they all slumbered and slept.}}
\bv{6}{\redlet{But at midnight there is a cry, `Behold, the bridegroom! Come ye forth to meet him.'}}
\bv{7}{\redlet{Then all those virgins arose, and trimmed their lamps.}}
\bv{8}{\redlet{And the foolish said unto the wise, `Give us of your oil; for our lamps are going out.'}}
\bv{9}{\redlet{But the wise answered, saying, `Peradventure there will not be enough for us and you: go ye rather to them that sell, and buy for yourselves.'}}
\bv{10}{\redlet{And while they went away to buy, the bridegroom came; and they that were ready went in with him to the marriage feast: and the door was shut.}}
\bv{11}{\redlet{Afterward came also the other virgins, saying, `Lord, Lord, open to us.'}}
\bv{12}{\redlet{But he answered and said, `Verily I say unto you, I know you not.'}}
\bv{13}{\redlet{Watch therefore, for ye know not the day nor the hour.}}\
\chapsec{The Parable of the Servants}
\bv{14}{\redlet{For \supptext{it is} as \supptext{when} a man, going into another country, called his own servants, and delivered unto them his goods.}}
\bv{15}{\redlet{And unto one he gave five talents,\mcomm{Talent: Unit of Weight or Measurement} to another two, to another one; to each according to his several ability; and he went on his journey.}}
\bv{16}{\redlet{Straightway he that received the five talents went and traded with them, and made other five talents.}}
\bv{17}{\redlet{In like manner he also that \supptext{received} the two gained other two.}}
\bv{18}{\redlet{But he that received the one went away and digged in the earth, and hid his lord's money.}}
\bv{19}{\redlet{Now after a long time the lord of those servants cometh, and maketh a reckoning with them.}}
\par
\bv{20}{\redlet{And he that received the five talents came and brought other five talents, saying, `Lord, thou deliveredst unto me five talents: lo, I have gained other five talents.'}}
\bv{21}{\redlet{His lord said unto him, `Well done, good and faithful servant: thou hast been faithful over a few things, I will set thee over many things; enter thou into the joy of thy lord.'}}
\bv{22}{\redlet{And he also that \supptext{received} the two talents came and said, `Lord, thou deliveredst unto me two talents: lo, I have gained other two talents.'}}
\bv{23}{\redlet{His lord said unto him, `Well done, good and faithful servant: thou hast been faithful over a few things, I will set thee over many things; enter thou into the joy of thy lord.'}}
\bv{24}{\redlet{And he also that had received the one talent came and said, `Lord, I knew thee that thou art a hard man, reaping where thou didst not sow, and gathering where thou didst not scatter;}}
\bv{25}{\redlet{and I was afraid, and went away and hid thy talent in the earth: lo, thou hast thine own.'}}
\bv{26}{\redlet{But his lord answered and said unto him, `Thou wicked and slothful servant, thou knewest that I reap where I sowed not, and gather where I did not scatter;}}
\bv{27}{\redlet{thou oughtest therefore to have put my money to the bankers, and at my coming I should have received back mine own with interest.}}
\bv{28}{\redlet{Take ye away therefore the talent from him, and give it unto him that hath the ten talents.'}}
\bv{29}{\redlet{For unto every one that hath shall be given, and he shall have abundance: but from him that hath not, even that which he hath shall be taken away.}}
\bv{30}{\redlet{And cast ye out the unprofitable servant into the outer darkness: there shall be the weeping and the gnashing of teeth.}}
\chapsec{Final Judgement}
\bv{31}{\redlet{But when the Son of man shall come in his glory, and all the angels with him, then shall he sit on the throne of his glory:}}
\bv{32}{\redlet{and before him shall be gathered all the nations: and he shall separate them one from another, as the shepherd separateth the sheep from the goats;}}
\bv{33}{\redlet{and he shall set the sheep on his right hand, but the goats on the left.}}
\bv{34}{\redlet{Then shall the King say unto them on his right hand, `Come, ye blessed of my Father, inherit the kingdom prepared for you from the foundation of the world:}}
\bv{35}{\redlet{for I was hungry, and ye gave me to eat; I was thirsty, and ye gave me drink; I was a stranger, and ye took me in;}}
\bv{36}{\redlet{naked, and ye clothed me; I was sick, and ye visited me; I was in prison, and ye came unto me.'}}
\bv{37}{\redlet{Then shall the righteous answer him, saying, `Lord, when saw we thee hungry, and fed thee? or athirst, and gave thee drink?}}
\bv{38}{\redlet{And when saw we thee a stranger, and took thee in? or naked, and clothed thee?}}
\bv{39}{\redlet{And when saw we thee sick, or in prison, and came unto thee?'}}
\bv{40}{\redlet{And the King shall answer and say unto them, `Verily I say unto you, Inasmuch as ye did it unto one of these my brethren, \supptext{even} these least, ye did it unto me.'}}
\par
\bv{41}{\redlet{Then shall he say also unto them on the left hand, `Depart from me, ye cursed, into the eternal fire which is prepared for the devil and his angels:}}
\bv{42}{\redlet{for I was hungry, and ye did not give me to eat; I was thirsty, and ye gave me no drink;}}
\bv{43}{\redlet{I was a stranger, and ye took me not in; naked, and ye clothed me not; sick, and in prison, and ye visited me not.'}}
\bv{44}{\redlet{Then shall they also answer, saying, `Lord, when saw we thee hungry, or athirst, or a stranger, or naked, or sick, or in prison, and did not minister unto thee?'}}
\bv{45}{\redlet{Then shall he answer them, saying, `Verily I say unto you, Inasmuch as ye did it not unto one of these least, ye did it not unto me.'}}
\bv{46}{\redlet{And these shall go away into eternal punishment: but the righteous into eternal life.''}}
\chaphead{Chapter XXVI}
\chapdesc{The Jews Conspire to Put Jesus to Death}
\lettrine[image=true, lines=4, findent=3pt, nindent=0pt]{Mt-A.eps}{nd} it came to pass, when Jesus had finished all these words, he said unto his disciples,
\bv{2}{\redlet{``Ye know that after two days the passover cometh, and the Son of man is delivered up to be crucified.''}}
\bv{3}{Then were gathered together the chief priests, and the elders of the people, unto the court of the high priest, who was called Caiaphas;}
\bv{4}{and they took counsel together that they might take Jesus by subtlety, and kill him.}
\bv{5}{But they said, ``Not during the feast, lest a tumult arise among the people.''}
\chapsec{Mary of Bethany Anoints Jesus}
\bv{6}{Now when Jesus was in Bethany, in the house of Simon the leper,}
\bv{7}{there came unto him a woman having an alabaster cruse of exceeding precious ointment, and she poured it upon his head, as he sat at meat.}
\bv{8}{But when the disciples saw it, they had indignation, saying, ``To what purpose is this waste?}
\bv{9}{For this \supptext{ointment} might have been sold for much, and given to the poor.''}
\bv{10}{But Jesus perceiving it said unto them, \redlet{	``Why trouble ye the woman? for she hath wrought a good work upon me.}}
\bv{11}{\redlet{For ye have the poor always with you; but me ye have not always.}}
\bv{12}{\redlet{For in that she poured this ointment upon my body, she did it to prepare me for burial.}}
\bv{13}{\redlet{Verily I say unto you, Wheresoever this gospel shall be preached in the whole world, that also which this woman hath done shall be spoken of for a memorial of her.''}}
\chapsec{Judas Iscariot Sells the Lord}
\bv{14}{Then one of the twelve, who was called Judas Iscariot, went unto the chief priests,}
\bv{15}{and said, ``What are ye willing to give me, and I will deliver him unto you?'' And they weighed unto him thirty pieces of silver.}
\bv{16}{And from that time he sought opportunity to deliver him \supptext{unto them}.}
\chapsec{Preparation of the Passover}
\bv{17}{Now on the first \supptext{day} of unleavened bread the disciples came to Jesus, saying, \redlet{``Where wilt thou that we make ready for thee to eat the passover?''}}
\bv{18}{And he said, \redlet{``Go into the city to such a man, and say unto him, `The Teacher saith, `My time is at hand; I keep the passover at thy house with my disciples.'{'}{''}}}
\bv{19}{And the disciples did as Jesus appointed them; and they made ready the passover.}
\chapsec{The Last Passover}
\bv{20}{Now when even was come, he was sitting at meat with the twelve disciples;}
\bv{21}{and as they were eating, he said, \redlet{``Verily I say unto you, that one of you shall betray me.''}}
\bv{22}{And they were exceeding sorrowful, and began to say unto him every one, ``Is it I, Lord?''}
\bv{23}{And he answered and said, \redlet{``He that dipped his hand with me in the dish, the same shall betray me.}}
\bv{24}{\redlet{The Son of man goeth, even as it is written of him: but woe unto that man through whom the Son of man is betrayed! good were it for that man if he had not been born.''}}
\bv{25}{And Judas, who betrayed him, answered and said, ``Is it I, Rabbi?'' He saith unto him, \redlet{``Thou hast said.''}}
\chapsec{Institution of the Eucharist}
\bv{26}{And as they were eating, Jesus took bread, and blessed, and brake it; and he gave to the disciples, and said, \redlet{``Take, eat; this is my body.''}}
\bv{27}{And he took a cup, and gave thanks, and gave to them, saying, \redlet{``Drink ye all of it;}}
\bv{28}{\redlet{for this is my blood of the covenant, which is poured out for many unto remission of sins.}}
\bv{29}{\redlet{But I say unto you, I shall not drink henceforth of this fruit of the vine, until that day when I drink it new with you in my Father's kingdom.''}}
\chapsec{Jesus Foretells St. Peter's Denial}
\bv{30}{And when they had sung a hymn, they went out into the mount of Olives.}
\bv{31}{Then saith Jesus unto them, \redlet{``All ye shall be offended in me this night: for it is written,}}
\otQuote{Zech. 13:7}{\redlet{I will smite the shepherd, and the sheep of the flock shall be scattered abroad.}}
\bv{32}{\redlet{But after I am raised up, I will go before you into Galilee.''}}
\bv{33}{But Peter answered and said unto him, ``If all shall be offended in thee, I will never be offended.''}
\bv{34}{Jesus said unto him, \redlet{``Verily I say unto thee, that this night, before the cock crow, thou shalt deny me thrice.''}}
\bv{35}{Peter saith unto him, ``Even if I must die with thee, \supptext{yet} will I not deny thee.'' Likewise also said all the disciples.}
\chapsec{The Agony in the Garden}
\bv{36}{Then cometh Jesus with them unto a place called Gethsemane, and saith unto his disciples, \redlet{``Sit ye here, while I go yonder and pray.''}}
\bv{37}{And he took with him Peter and the two sons of Zebedee, and began to be sorrowful and sore troubled.}
\bv{38}{Then saith he unto them, \redlet{``My soul is exceeding sorrowful, even unto death: abide ye here, and watch with me.''}}
\par
\bv{39}{And he went forward a little, and fell on his face, and prayed, saying, \redlet{``My Father, if it be possible, let this cup pass away from me: nevertheless, not as I will, but as thou wilt.''}}
\par
\bv{40}{And he cometh unto the disciples, and findeth them sleeping, and saith unto Peter, \redlet{``What, could ye not watch with me one hour?}}
\bv{41}{\redlet{Watch and pray, that ye enter not into temptation: the spirit indeed is willing, but the flesh is weak.''}}
\par
\bv{42}{Again a second time he went away, and prayed, saying, \redlet{``My Father, if this cannot pass away, except I drink it, thy will be done.''}}
\bv{43}{And he came again and found them sleeping, for their eyes were heavy.}
\par
\bv{44}{And he left them again, and went away, and prayed a third time, saying again the same words.}
\bv{45}{Then cometh he to the disciples, and saith unto them, \redlet{``Sleep on now, and take your rest: behold, the hour is at hand, and the Son of man is betrayed into the hands of sinners.}}
\bv{46}{\redlet{Arise, let us be going: behold, he is at hand that betrayeth me.''}}
\chapsec{The Betrayal \& Arrest of Jesus}
\bv{47}{And while he yet spake, lo, Judas, one of the twelve, came, and with him a great multitude with swords and staves, from the chief priests and elders of the people.}
\bv{48}{Now he that betrayed him gave them a sign, saying, ``Whomsoever I shall kiss, that is he: take him.''}
\bv{49}{And straightway he came to Jesus, and said, ``Hail, Rabbi;'' and kissed him.}
\bv{50}{And Jesus said unto him, \redlet{``Friend, \supptext{do} that for which thou art come.''} Then they came and laid hands on Jesus, and took him.}
\bv{51}{And behold, one of them that were with Jesus stretched out his hand, and drew his sword, and smote the servant of the high priest, and struck off his ear.}
\bv{52}{Then saith Jesus unto him, \redlet{``Put up again thy sword into its place: for all they that take the sword shall perish with the sword.}}
\bv{53}{\redlet{Or thinkest thou that I cannot beseech my Father, and he shall even now send me more than twelve legions of angels?}}
\bv{54}{\redlet{How then should the scriptures be fulfilled, that thus it must be?''}}
\bv{55}{In that hour said Jesus to the multitudes, \redlet{``Are ye come out as against a robber with swords and staves to seize me? I sat daily in the temple teaching, and ye took me not.}}
\bv{56}{\redlet{But all this is come to pass, that the scriptures of the prophets might be fulfilled.''} Then all the disciples left him, and fled.}
\chapsec{Jesus Brought before Caiaphas}
\bv{57}{And they that had taken Jesus led him away to \supptext{the house of} Caiaphas the high priest, where the scribes and the elders were gathered together.}
\bv{58}{But Peter followed him afar off, unto the court of the high priest, and entered in, and sat with the officers, to see the end.}
\bv{59}{Now the chief priests and the whole council sought false witness against Jesus, that they might put him to death;}
\bv{60}{and they found it not, though many false witnesses came. But afterward came two,}
\bv{61}{and said, ``This man said, `I am able to destroy the temple of God, and to build it in three days.'{''}}
\bv{62}{And the high priest stood up, and said unto him, ``Answerest thou nothing? what is it which these witness against thee?''}
\bv{63}{But Jesus held his peace. And the high priest said unto him, ``I adjure thee by the living God, that thou tell us whether thou art the Christ, the Son of God.''}
\bv{64}{Jesus saith unto him, \redlet{``Thou hast said: nevertheless I say unto you, Henceforth ye shall see the Son of man sitting at the right hand of Power, and coming on the clouds of heaven.''}}
\par
\bv{65}{Then the high priest rent his garments, saying, ``He hath spoken blasphemy: what further need have we of witnesses? behold, now ye have heard the blasphemy:}
\bv{66}{what think ye?'' They answered and said, ``He is worthy of death.''}
\bv{67}{Then did they spit in his face and buffet him: and some smote him with the palms of their hands,}
\bv{68}{saying, ``Prophesy unto us, thou Christ: who is he that struck thee?''}
\chapsec{St. Peter Denies the Lord}
\bv{69}{Now Peter was sitting without in the court: and a maid came unto him, saying, ``Thou also wast with Jesus the Galilæan.''}
\bv{70}{But he denied before them all, saying, ``I know not what thou sayest.''}
\bv{71}{And when he was gone out into the porch, another \supptext{maid} saw him, and saith unto them that were there, ``This man also was with Jesus of Nazareth.''}
\bv{72}{And again he denied with an oath, ``I know not the man.''}
\bv{73}{And after a little while they that stood by came and said to Peter, ``Of a truth thou also art \supptext{one} of them; for thy speech maketh thee known.''}
\bv{74}{Then began he to curse and to swear, ``I know not the man.'' And straightway the cock crew.}
\bv{75}{And Peter remembered the word which Jesus had said, \redlet{``Before the cock crow, thou shalt deny me thrice.''} And he went out, and wept bitterly.}
\chaphead{Chapter XXVII}
\chapdesc{The Sanhedrin Deliver Jesus to Pilate}
\lettrine[image=true, lines=4, findent=3pt, nindent=0pt]{Mt-N.eps}{ow} when morning was come, all the chief priests and the elders of the people took counsel against Jesus to put him to death:
\bv{2}{and they bound him, and led him away, and delivered him up to Pilate the governor.}
\chapsec{Judas Iscariot's Proud Repentance}
\bv{3}{Then Judas, who betrayed him, when he saw that he was condemned, repented himself, and brought back the thirty pieces of silver to the chief priests and elders,}
\bv{4}{saying, ``I have sinned in that I betrayed innocent blood.'' But they said, ``What is that to us? see thou \supptext{to it}.''}
\bv{5}{And he cast down the pieces of silver into the sanctuary, and departed; and he went away and hanged himself.}
\bv{6}{And the chief priests took the pieces of silver, and said, ``It is not lawful to put them into the treasury, since it is the price of blood.''}
\bv{7}{And they took counsel, and bought with them the potter's field, to bury strangers in.}
\bv{8}{Wherefore that field was called, ``The field of blood,'' unto this day.}
\bv{9}{Then was fulfilled that which was spoken through Jeremiah the prophet, saying,}
\otQuote{Zech. 11:13}{And they took the thirty pieces of silver, the price of him that was priced, whom \supptext{certain} of the children of Israel did price; \bv{10}{and they gave them for the potter's field, as the Lord appointed me.}}
\chapsec{Jesus Interrogated by Pilate}
\bv{11}{Now Jesus stood before the governor: and the governor asked him, saying, ``Art thou the King of the Jews?'' And Jesus said unto him, \redlet{``Thou sayest.''}}
\bv{12}{And when he was accused by the chief priests and elders, he answered nothing.}
\bv{13}{Then saith Pilate unto him, ``Hearest thou not how many things they witness against thee?''}
\bv{14}{And he gave him no answer, not even to one word: insomuch that the governor marvelled greatly.}
\chapsec{Jesus or Barabbas?}
\bv{15}{Now at the feast the governor was wont to release unto the multitude one prisoner, whom they would.}
\bv{16}{And they had then a notable prisoner, called Barabbas.}
\bv{17}{When therefore they were gathered together, Pilate said unto them, ``Whom will ye that I release unto you? Barabbas, or Jesus who is called Christ?''}
\bv{18}{For he knew that for envy they had delivered him up.}
\bv{19}{And while he was sitting on the judgement-seat, his wife sent unto him, saying, ``Have thou nothing to do with that righteous man; for I have suffered many things this day in a dream because of him.''}
\bv{20}{Now the chief priests and the elders persuaded the multitudes that they should ask for Barabbas, and destroy Jesus.}
\chapsec{Jewish Apostacy}
\bv{21}{But the governor answered and said unto them, ``Which of the two will ye that I release unto you?'' And they said, ``Barabbas.''}
\bv{22}{Pilate saith unto them, ``What then shall I do unto Jesus who is called Christ?'' They all say, ``Let him be crucified.''}
\bv{23}{And he said, ``Why, what evil hath he done?'' But they cried out exceedingly, saying, ``Let him be crucified.''}
\bv{24}{So when Pilate saw that he prevailed nothing, but rather that a tumult was arising, he took water, and washed his hands before the multitude, saying, ``I am innocent of the blood of this righteous man; see ye \supptext{to it}.''}
\bv{25}{And all the people answered and said, ``His blood \supptext{be} on us, and on our children.''}
\chapsec{Barabbas Released}
\bv{26}{Then released he unto them Barabbas; but Jesus he scourged and delivered to be crucified.}
\chapsec{Crowning with Thorns}
\bv{27}{Then the soldiers of the governor took Jesus into the Prætorium, and gathered unto him the whole band.}
\bv{28}{And they stripped him, and put on him a scarlet robe.}
\bv{29}{And they platted a crown of thorns and put it upon his head, and a reed in his right hand; and they kneeled down before him, and mocked him, saying, ``Hail, King of the Jews!''}
\bv{30}{And they spat upon him, and took the reed and smote him on the head.}
\bv{31}{And when they had mocked him, they took off from him the robe, and put on him his garments, and led him away to crucify him.}
\bv{32}{And as they came out, they found a man of Cyrene, Simon by name: him they compelled to go \supptext{with them}, that he might bear his cross.}
\chapsec{The Crucifixion}
\bv{33}{And when they were come unto a place called Golgotha, that is to say, ``The place of a skull,''}
\bv{34}{they gave him wine to drink mingled with gall: and when he had tasted it, he would not drink.}
\par
\bv{35}{And when they had crucified him, they parted his garments among them, casting lots;}
\bv{36}{and they sat and watched him there.}
\bv{37}{And they set up over his head his accusation written, {\scshape This is Jesus the King of the Jews.}}
\bv{38}{Then are there crucified with him two robbers, one on the right hand and one on the left.}
\bv{39}{And they that passed by railed on him, wagging their heads,}
\bv{40}{and saying, ``Thou that destroyest the temple, and buildest it in three days, save thyself: if thou art the Son of God, come down from the cross.''}
\bv{41}{In like manner also the chief priests mocking \supptext{him}, with the scribes and elders, said,}
\bv{42}{``He saved others; himself he cannot save. He is the King of Israel; let him now come down from the cross, and we will believe on him.}
\bv{43}{He trusteth on God; let him deliver him now, if he desireth him: for he said, `I am the Son of God.'{''}}
\bv{44}{And the robbers also that were crucified with him cast upon him the same reproach.}
\chapsec{The Death of Jesus Christ}
\bv{45}{Now from the sixth hour there was darkness over all the land until the ninth hour.}
\bv{46}{And about the ninth hour Jesus cried with a loud voice, saying, \redlet{``Eli, Eli, lama sabachthani?''}\mref{Ps. 22} that is, \redlet{``My God, my God, why hast thou forsaken me?''}}
\bv{47}{And some of them that stood there, when they heard it, said, ``This man calleth Elijah.''}
\bv{48}{And straightway one of them ran, and took a sponge, and filled it with vinegar, and put it on a reed, and gave him to drink.}
\bv{49}{And the rest said, ``Let be; let us see whether Elijah cometh to save him.''}
\bv{50}{And Jesus cried again with a loud voice, and yielded up the ghost.}
\chapsec{The Fulfilment of the Law}
\bv{51}{And behold, the veil of the temple was rent in two from the top to the bottom; and the earth did quake; and the rocks were rent;}
\bv{52}{and the tombs were opened; and many bodies of the saints that had fallen asleep were raised;}
\bv{53}{and coming forth out of the tombs after his resurrection they entered into the holy city and appeared unto many.}
\bv{54}{Now the centurion, and they that were with him watching Jesus, when they saw the earthquake, and the things that were done, feared exceedingly, saying, ``Truly this was the Son of God.''}
\bv{55}{And many women were there beholding from afar, who had followed Jesus from Galilee, ministering unto him:}
\bv{56}{among whom was Mary Magdalene, and Mary the mother of James and Joses, and the mother of the sons of Zebedee.}
\chapsec{Christ's Burial}
\bv{57}{And when even was come, there came a rich man from Arimathæa, named Joseph, who also himself was Jesus' disciple:}
\bv{58}{this man went to Pilate, and asked for the body of Jesus. Then Pilate commanded it to be given up.}
\bv{59}{And Joseph took the body, and wrapped it in a clean linen cloth,}
\bv{60}{and laid it in his own new tomb, which he had hewn out in the rock: and he rolled a great stone to the door of the tomb, and departed.}
\bv{61}{And Mary Magdalene was there, and the other Mary, sitting over against the sepulchre.}
\chapsec{The Sepulchre Sealed \& Guarded}
\bv{62}{Now on the morrow, which is \supptext{the day} after the Preparation, the chief priests and the Pharisees were gathered together unto Pilate,}
\bv{63}{saying, ``Sir, we remember that that deceiver said while he was yet alive, `After three days I rise again.'}
\bv{64}{Command therefore that the sepulchre be made sure until the third day, lest haply his disciples come and steal him away, and say unto the people, `He is risen from the dead:' and the last error will be worse than the first.''}
\bv{65}{Pilate said unto them, ``Ye have a guard: go, make it \supptext{as} sure as ye can.''}
\bv{66}{So they went, and made the sepulchre sure, sealing the stone, the guard being with them.}
\chaphead{Chapter XXVIII}
\chapdesc{The Resurrection of Jesus Christ}
\lettrine[image=true, lines=4, findent=3pt, nindent=0pt]{Mt-N.eps}{ow} late on the sabbath day, as it began to dawn toward the first \supptext{day} of the week, came Mary Magdalene and the other Mary to see the sepulchre.
\bv{2}{And behold, there was a great earthquake; for an angel of the Lord descended from heaven, and came and rolled away the stone, and sat upon it.}
\bv{3}{His appearance was as lightning, and his raiment white as snow:}
\bv{4}{and for fear of him the watchers did quake, and became as dead men.}
\bv{5}{And the angel answered and said unto the women, ``Fear not ye; for I know that ye seek Jesus, who hath been crucified.}
\bv{6}{He is not here; for he is risen, even as he said. Come, see the place where the Lord lay.}
\bv{7}{And go quickly, and tell his disciples, `He is risen from the dead;' and lo, he goeth before you into Galilee; there shall ye see him: lo, I have told you.''}
\bv{8}{And they departed quickly from the tomb with fear and great joy, and ran to bring his disciples word.}
\bv{9}{And behold, Jesus met them, saying, \redlet{``All hail.''} And they came and took hold of his feet, and worshipped him.}
\bv{10}{Then saith Jesus unto them, \redlet{``Fear not: go tell my brethren that they depart into Galilee, and there shall they see me.''}}
\bv{11}{Now while they were going, behold, some of the guard came into the city, and told unto the chief priests all the things that were come to pass.}
\bv{12}{And when they were assembled with the elders, and had taken counsel, they gave much money unto the soldiers,}
\bv{13}{saying, ``Say ye, `His disciples came by night, and stole him away while we slept.'}
\bv{14}{And if this come to the governor's ears, we will persuade him, and rid you of care.''}
\bv{15}{So they took the money, and did as they were taught: and this saying was spread abroad among the Jews, \supptext{and continueth} until this day.}
\bv{16}{But the eleven disciples went into Galilee, unto the mountain where Jesus had appointed them.}
\bv{17}{And when they saw him, they worshipped \supptext{him}; but some doubted.}
\bv{18}{And Jesus came to them and spake unto them, saying, \redlet{``All authority hath been given unto me in heaven and on earth.}}
\bv{19}{\redlet{Go ye therefore, and make disciples of all the nations, baptising them into the name of the Father and of the Son and of the Holy Ghost:}}
\bv{20}{\redlet{teaching them to observe all things whatsoever I commanded you: and lo, I am with you always, even unto the end of the world.''}}
\begin{center}
	{\scshape [Here Endeth the Gospel of Matthew]}
\end{center}
	\clearpage
	\chapter{The Holy Gospel of Jesus Christ according to Saint Mark}
\fancyhead[RE,LO]{The Gospel according to Mark}
\chaphead{Chapter I}
\chapdesc{The Ministry of St. John the Baptist}
\lettrine[image=true, lines=4, findent=3pt, nindent=0pt]{NT/Mark/Mk1-T.eps}{he} beginning of the gospel of Jesus Christ, the Son of God.
\bv{2}{Even as it is written in Isaiah the prophet,}
\otQuote{Malachi 3:1; Isaiah 40:3}{``Behold, I send my messenger before thy face, Who shall prepare thy way; The voice of one crying in the wilderness, Make ye ready the way of the Lord, Make his paths straight;''}
\bv{4}{John came, who baptised in the wilderness and preached the baptism of repentance unto remission of sins.}
\bv{5}{And there went out unto him all the country of Judæa, and all they of Jerusalem; and they were baptised of him in the river Jordan, confessing their sins.}
\bv{6}{And John was clothed with camel's hair, and \supptext{had} a leathern girdle about his loins, and did eat locusts and wild honey.}
\bv{7}{And he preached, saying, ``There cometh after me he that is mightier than I, the latchet of whose shoes I am not worthy to stoop down and unloose.}
\bv{8}{I baptised you in water; but he shall baptise you in the Holy Ghost.''}
\chapsec{The Baptism of Jesus}
\bv{9}{And it came to pass in those days, that Jesus came from Nazareth of Galilee, and was baptised of John in the Jordan.}
\bv{10}{And straightway coming up out of the water, he saw the heavens rent asunder, and the Spirit as a dove descending upon him:}
\bv{11}{and a voice came out of the heavens, ``Thou art my beloved Son, in thee I am well pleased.''}
\chapsec{The Temptation of Jesus}
\bv{12}{And straightway the Spirit driveth him forth into the wilderness.}
\bv{13}{And he was in the wilderness forty days tempted of Satan; and he was with the wild beasts; and the angels ministered unto him.}
\chapsec{The First Galilean Ministry}
\bv{14}{Now after John was delivered up, Jesus came into Galilee, preaching the gospel of God,}
\bv{15}{and saying, \redlet{``The time is fulfilled, and the kingdom of God is at hand: repent ye, and believe in the gospel.''}}
\chapsec{The Call of Sts. Peter \& Andrew}
\bv{16}{And passing along by the sea of Galilee, he saw Simon and Andrew the brother of Simon casting a net in the sea; for they were fishers.}
\bv{17}{And Jesus said unto them, \redlet{``Come ye after me, and I will make you to become fishers of men.''}}
\bv{18}{And straightway they left the nets, and followed him.}
\bv{19}{And going on a little further, he saw James the \supptext{son} of Zebedee, and John his brother, who also were in the boat mending the nets.}
\bv{20}{And straightway he called them: and they left their father Zebedee in the boat with the hired servants, and went after him.}
\chapsec{Jesus Casts out Demons in Capernaum}
\bv{21}{And they go into Capernaum; and straightway on the sabbath day he entered into the synagogue and taught.}
\bv{22}{And they were astonished at his teaching: for he taught them as having authority, and not as the scribes.}
\bv{23}{And straightway there was in their synagogue a man with an unclean spirit; and he cried out,}
\bv{24}{saying, ``What have we to do with thee, Jesus thou Nazarene? art thou come to destroy us? I know thee who thou art, the Holy One of God.''}
\bv{25}{And Jesus rebuked him, saying, \redlet{``Hold thy peace, and come out of him.''}}
\bv{26}{And the unclean spirit, tearing him and crying with a loud voice, came out of him.}
\bv{27}{And they were all amazed, insomuch that they questioned among themselves, saying, ``What is this? a new teaching! with authority he commandeth even the unclean spirits, and they obey him.''}
\bv{28}{And the report of him went out straightway everywhere into all the region of Galilee round about.}
\chapsec{Simon's Mother-in-Law Healed of a Fever}
\bv{29}{And straightway, when they were come out of the synagogue, they came into the house of Simon and Andrew, with James and John.}
\bv{30}{Now Simon's wife's mother lay sick of a fever; and straightway they tell him of her:}
\bv{31}{and he came and took her by the hand, and raised her up; and the fever left her, and she ministered unto them.}
\chapsec{Demons Rebuked \& Many Healed}
\bv{32}{And at even, when the sun did set, they brought unto him all that were sick, and them that were possessed with demons.}
\bv{33}{And all the city was gathered together at the door.}
\bv{34}{And he healed many that were sick with divers diseases, and cast out many demons; and he suffered not the demons to speak, because they knew him.}
\chapsec{Jesus Prays \& Preaches in Galilee}
\bv{35}{And in the morning, a great while before day, he rose up and went out, and departed into a desert place, and there prayed.}
\bv{36}{And Simon and they that were with him followed after him;}
\bv{37}{and they found him, and say unto him, ``All are seeking thee.''}
\bv{38}{And he saith unto them, \redlet{``Let us go elsewhere into the next towns, that I may preach there also; for to this end came I forth.''}}
\bv{39}{And he went into their synagogues throughout all Galilee, preaching and casting out demons.}
\chapsec{Jesus Heals a Leper}
\bv{40}{And there cometh to him a leper, beseeching him, and kneeling down to him, and saying unto him, ``If thou wilt, thou canst make me clean.''}
\bv{41}{And being moved with compassion, he stretched forth his hand, and touched him, and saith unto him, \redlet{``I will; be thou made clean.''}}
\bv{42}{And straightway the leprosy departed from him, and he was made clean.}
\bv{43}{And he strictly charged him, and straightway sent him out,}
\bv{44}{and saith unto him, \redlet{``See thou say nothing to any man: but go show thyself to the priest, and offer for thy cleansing the things which Moses commanded, for a testimony unto them.''}}
\bv{45}{But he went out, and began to publish it much, and to spread abroad the matter, insomuch that Jesus could no more openly enter into a city, but was without in desert places: and they came to him from every quarter.}
\chaphead{Chapter II}
\chapdesc{Jesus Heals a Paralytic}
\lettrine[image=true, lines=4, findent=3pt, nindent=0pt]{NT/Mark/Mk-And.eps}{nd} when he entered again into Capernaum after some days, it was noised that he was in the house.
\bv{2}{And many were gathered together, so that there was no longer room \supptext{for them}, no, not even about the door: and he spake the word unto them.}
\bv{3}{And they come, bringing unto him a paralytic, borne of four.}
\bv{4}{And when they could not come nigh unto him for the crowd, they uncovered the roof where he was: and when they had broken it up, they let down the bed whereon the paralytic lay.}
\bv{5}{And Jesus seeing their faith saith unto the paralytic, \redlet{``Son, thy sins are forgiven.''}}
\bv{6}{But there were certain of the scribes sitting there, and reasoning in their hearts,}
\bv{7}{``Why doth this man thus speak? he blasphemeth: who can forgive sins but one, \supptext{even} God?''}
\bv{8}{And straightway Jesus, perceiving in his spirit that they so reasoned within themselves, saith unto them, \redlet{``Why reason ye these things in your hearts?}}
\bv{9}{\redlet{Which is easier, to say to the paralytic, Thy sins are forgiven; or to say, Arise, and take up thy bed, and walk?}}
\bv{10}{\redlet{But that ye may know that the Son of man hath authority on earth to forgive sins''} (he saith to the paralytic),}
\bv{11}{\redlet{``I say unto thee, Arise, take up thy bed, and go unto thy house.''}}
\bv{12}{And he arose, and straightway took up the bed, and went forth before them all; insomuch that they were all amazed, and glorified God, saying, ``We never saw it on this fashion.''}
\chapsec{Christ Calls St. Matthew (Levi)}
\bv{13}{And he went forth again by the sea side; and all the multitude resorted unto him, and he taught them.}
\bv{14}{And as he passed by, he saw Levi the \supptext{son} of Alphæus sitting at the place of toll, and he saith unto him, Follow me. And he arose and followed him.}
\bv{15}{And it came to pass, that he was sitting at meat in his house, and many publicans and sinners sat down with Jesus and his disciples: for there were many, and they followed him.}
\bv{16}{And the scribes of the Pharisees, when they saw that he was eating with the sinners and publicans, said unto his disciples, ``\supptext{How is it} that he eateth and drinketh with publicans and sinners?''}
\bv{17}{And when Jesus heard it, he saith unto them, \redlet{``They that are whole have no need of a physician, but they that are sick: I came not to call the righteous, but sinners.''}\mref{cf. Prayer of Manasseh 8}}
\chapsec{A Teaching on Fasting}
\bv{18}{And John's disciples and the Pharisees were fasting: and they come and say unto him, ``Why do John's disciples and the disciples of the Pharisees fast, but thy disciples fast not?''}
\bv{19}{And Jesus said unto them, \redlet{``Can the sons of the bridechamber fast, while the bridegroom is with them? as long as they have the bridegroom with them, they cannot fast.}}
\bv{20}{\redlet{But the days will come, when the bridegroom shall be taken away from them, and then will they fast in that day.}}
\chapsec{Parables of the Cloths \& Bottles}
\bv{21}{\redlet{No man seweth a piece of undressed cloth on an old garment: else that which should fill it up taketh from it, the new from the old, and a worse rent is made.}}
\bv{22}{\redlet{And no man putteth new wine into old wine-skins; else the wine will burst the skins, and the wine perisheth, and the skins: but \supptext{they put} new wine into fresh wine-skins.''}}
\chapsec{Jesus Lord of the Sabbath}
\bv{23}{And it came to pass, that he was going on the sabbath day through the grainfields; and his disciples began, as they went, to pluck the ears.}
\bv{24}{And the Pharisees said unto him, ``Behold, why do they on the sabbath day that which is not lawful?''}
\bv{25}{And he said unto them, \redlet{``Did ye never read what David did, when he had need, and was hungry, he, and they that were with him?}}
\bv{26}{\redlet{How he entered into the house of God when Abiathar was high priest, and ate the showbread, which it is not lawful to eat save for the priests, and gave also to them that were with him?''}}
\bv{27}{And he said unto them, \redlet{``The sabbath was made for man, and not man for the sabbath:}}
\bv{28}{\redlet{so that the Son of man is lord even of the sabbath.''}}
\chaphead{Chapter III}
\chapdesc{Jesus Heals the Withered Hand on the Sabbath}
\lettrine[image=true, lines=4, findent=3pt, nindent=0pt]{NT/Mark/Mk-And.eps}{nd} he entered again into the synagogue; and there was a man there who had his hand withered.
\bv{2}{And they watched him, whether he would heal him on the sabbath day; that they might accuse him.}
\bv{3}{And he saith unto the man that had his hand withered, \redlet{``Stand forth.''}}
\bv{4}{And he saith unto them, \redlet{``Is it lawful on the sabbath day to do good, or to do harm? to save a life, or to kill?''} But they held their peace.}
\bv{5}{And when he had looked round about on them with anger, being grieved at the hardening of their heart, he saith unto the man, \redlet{``Stretch forth thy hand.''} And he stretched it forth; and his hand was restored.}
\bv{6}{And the Pharisees went out, and straightway with the Herodians took counsel against him, how they might destroy him.}
\chapsec{A Great Crowd Follows Jesus}
\bv{7}{And Jesus with his disciples withdrew to the sea: and a great multitude from Galilee followed; and from Judæa,}
\bv{8}{and from Jerusalem, and from Idumæa, and beyond the Jordan, and about Tyre and Sidon, a great multitude, hearing what great things he did, came unto him.}
\bv{9}{And he spake to his disciples, that a little boat should wait on him because of the crowd, lest they should throng him:}
\bv{10}{for he had healed many; insomuch that as many as had plagues pressed upon him that they might touch him.}
\bv{11}{And the unclean spirits, whensoever they beheld him, fell down before him, and cried, saying, ``Thou art the Son of God.''}
\bv{12}{And he charged them much that they should not make him known.}
\chapsec{The Twelve Apostles}
\bv{13}{And he goeth up into the mountain, and calleth unto him whom he himself would; and they went unto him.}
\bv{14}{And he appointed twelve, that they might be with him, and that he might send them forth to preach,}
\bv{15}{and to have authority to cast out demons:}
\bv{16}{and Simon he surnamed Peter;}
\bv{17}{and James the \supptext{son} of Zebedee, and John the brother of James; and them he surnamed Boanerges, which is, Sons of thunder:}
\bv{18}{and Andrew, and Philip, and Bartholomew, and Matthew, and Thomas, and James the \supptext{son} of Alphæus, and Thaddæus, and Simon the Cananæan,}
\bv{19}{and Judas Iscariot, who also betrayed him. And he cometh into a house.}
\bv{20}{And the multitude cometh together again, so that they could not so much as eat bread.}
\bv{21}{And when his friends heard it, they went out to lay hold on him: for they said, ``He is beside himself.''}
\chapsec{Blasphemy Against the Holy Ghost}
\bv{22}{And the scribes that came down from Jerusalem said, ``He hath Beelzebub, and, By the prince of the demons casteth he out the demons.''}
\bv{23}{And he called them unto him, and said unto them in parables, \redlet{``How can Satan cast out Satan?}}
\bv{24}{\redlet{And if a kingdom be divided against itself, that kingdom cannot stand.}}
\bv{25}{\redlet{And if a house be divided against itself, that house will not be able to stand.}}
\bv{26}{\redlet{And if Satan hath risen up against himself, and is divided, he cannot stand, but hath an end.}}
\bv{27}{\redlet{But no one can enter into the house of the strong \supptext{man}, and spoil his goods, except he first bind the strong \supptext{man}; and then he will spoil his house.}}
\bv{28}{\redlet{Verily I say unto you, All their sins shall be forgiven unto the sons of men, and their blasphemies wherewith soever they shall blaspheme:}}
\bv{29}{\redlet{but whosoever shall blaspheme against the Holy Ghost hath never forgiveness, but is guilty of an eternal sin:''}}
\bv{30}{because they said, ``He hath an unclean spirit.''}
\chapsec{Jesus’ Mother and Brothers}
\bv{31}{And there come his mother and his brethren; and, standing without, they sent unto him, calling him.}
\bv{32}{And a multitude was sitting about him; and they say unto him, ``Behold, thy mother and thy brethren without seek for thee.''}
\bv{33}{And he answereth them, and saith, \redlet{``Who is my mother and my brethren?''}}
\bv{34}{And looking round on them that sat round about him, he saith, \redlet{``Behold, my mother and my brethren!}}
\bv{35}{\redlet{For whosoever shall do the will of God, the same is my brother, and sister, and mother.''}}
\chaphead{Chapter IV}
\chapdesc{The Parable of the Sower}
\lettrine[image=true, lines=4, findent=3pt, nindent=0pt]{NT/Mark/Mk-And.eps}{nd} again he began to teach by the sea side. And there is gathered unto him a very great multitude, so that he entered into a boat, and sat in the sea; and all the multitude were by the sea on the land.
\bv{2}{And he taught them many things in parables, and said unto them in his teaching,}
\bv{3}{\redlet{``Hearken: Behold, the sower went forth to sow:}}
\bv{4}{\redlet{and it came to pass, as he sowed, some \supptext{seed} fell by the way side, and the birds came and devoured it.}}
\bv{5}{\redlet{And other fell on the rocky \supptext{ground}, where it had not much earth; and straightway it sprang up, because it had no deepness of earth:}}
\bv{6}{\redlet{and when the sun was risen, it was scorched; and because it had no root, it withered away.}}
\bv{7}{\redlet{And other fell among the thorns, and the thorns grew up, and choked it, and it yielded no fruit.}}
\bv{8}{\redlet{And others fell into the good ground, and yielded fruit, growing up and increasing; and brought forth, thirtyfold, and sixtyfold, and a hundredfold.''}}
\bv{9}{And he said, \redlet{``Who hath ears to hear, let him hear.''}}
\chapsec{The Purpose of the Parables}
\bv{10}{And when he was alone, they that were about him with the twelve asked of him the parables.}
\bv{11}{And he said unto them, \redlet{``Unto you is given the mystery of the kingdom of God: but unto them that are without, all things are done in parables:}}
\bv{12}{\redlet{that seeing they may see, and not perceive; and hearing they may hear, and not understand; lest haply they should turn again, and it should be forgiven them.''}}
\bv{13}{And he saith unto them, \redlet{``Know ye not this parable? and how shall ye know all the parables?}}
\bv{14}{\redlet{The sower soweth the word.}}
\bv{15}{\redlet{And these are they by the way side, where the word is sown; and when they have heard, straightway cometh Satan, and taketh away the word which hath been sown in them.}}
\bv{16}{\redlet{And these in like manner are they that are sown upon the rocky \supptext{places}, who, when they have heard the word, straightway receive it with joy;}}
\bv{17}{\redlet{and they have no root in themselves, but endure for a while; then, when tribulation or persecution ariseth because of the word, straightway they stumble.}}
\bv{18}{\redlet{And others are they that are sown among the thorns; these are they that have heard the word,}}
\bv{19}{\redlet{and the cares of the world, and the deceitfulness of riches, and the lusts of other things entering in, choke the word, and it becometh unfruitful.}}
\bv{20}{\redlet{And those are they that were sown upon the good ground; such as hear the word, and accept it, and bear fruit, thirtyfold, and sixtyfold, and a hundredfold.''}}
\chapsec{A Lamp Under a Basket}
\bv{21}{And he said unto them, \redlet{``Is the lamp brought to be put under the bushel, or under the bed, \supptext{and} not to be put on the stand?}}
\bv{22}{\redlet{For there is nothing hid, save that it should be manifested; neither was \supptext{anything} made secret, but that it should come to light.}}
\bv{23}{\redlet{If any man hath ears to hear, let him hear.''}}
\bv{24}{And he said unto them, \redlet{``Take heed what ye hear: with what measure ye mete it shall be measured unto you; and more shall be given unto you.}}
\bv{25}{\redlet{For he that hath, to him shall be given: and he that hath not, from him shall be taken away even that which he hath.''}}
\chapsec{The Parable of the Seed Growing}
\bv{26}{And he said, \redlet{``So is the kingdom of God, as if a man should cast seed upon the earth;}}
\bv{27}{\redlet{and should sleep and rise night and day, and the seed should spring up and grow, he knoweth not how.}}
\bv{28}{\redlet{The earth beareth fruit of herself; first the blade, then the ear, then the full grain in the ear.}}
\bv{29}{\redlet{But when the fruit is ripe, straightway he putteth forth the sickle, because the harvest is come.''}}
\chapsec{The Parable of the Mustard Seed}
\bv{30}{And he said, \redlet{``How shall we liken the kingdom of God? or in what parable shall we set it forth?}}
\bv{31}{\redlet{It is like a grain of mustard seed, which, when it is sown upon the earth, though it be less than all the seeds that are upon the earth,}}
\bv{32}{\redlet{yet when it is sown, groweth up, and becometh greater than all the herbs, and putteth out great branches; so that the birds of the heaven can lodge under the shadow thereof.''}}
\bv{33}{And with many such parables spake he the word unto them, as they were able to hear it;}
\bv{34}{and without a parable spake he not unto them: but privately to his own disciples he expounded all things.}
\chapsec{Jesus Calms a Storm}
\bv{35}{And on that day, when even was come, he saith unto them, \redlet{``Let us go over unto the other side.''}}
\bv{36}{And leaving the multitude, they take him with them, even as he was, in the boat. And other boats were with him.}
\bv{37}{And there ariseth a great storm of wind, and the waves beat into the boat, insomuch that the boat was now filling.}
\bv{38}{And he himself was in the stern, asleep on the cushion: and they awake him, and say unto him, ``Teacher, carest thou not that we perish?''}
\bv{39}{And he awoke, and rebuked the wind, and said unto the sea, \redlet{``Peace, be still.''} And the wind ceased, and there was a great calm.}
\bv{40}{And he said unto them, \redlet{``Why are ye fearful? have ye not yet faith?''}}
\bv{41}{And they feared exceedingly, and said one to another, ``Who then is this, that even the wind and the sea obey him?''}
\chaphead{Chapter V}
\chapdesc{Jesus Heals a Man with a Demon}
\lettrine[image=true, lines=4, findent=3pt, nindent=0pt]{NT/Mark/Mk-And.eps}{nd} they came to the other side of the sea, into the country of the Gerasenes.
\bv{2}{And when he was come out of the boat, straightway there met him out of the tombs a man with an unclean spirit,}
\bv{3}{who had his dwelling in the tombs: and no man could any more bind him, no, not with a chain;}
\bv{4}{because that he had been often bound with fetters and chains, and the chains had been rent asunder by him, and the fetters broken in pieces: and no man had strength to tame him.}
\bv{5}{And always, night and day, in the tombs and in the mountains, he was crying out, and cutting himself with stones.}
\bv{6}{And when he saw Jesus from afar, he ran and worshipped him;}
\bv{7}{and crying out with a loud voice, he saith, ``What have I to do with thee, Jesus, thou Son of the Most High God? I adjure thee by God, torment me not.''}
\bv{8}{For he said unto him, \redlet{``Come forth, thou unclean spirit, out of the man.''}}
\bv{9}{And he asked him, \redlet{``What is thy name?''} And he saith unto him, ``My name is Legion; for we are many.''}
\bv{10}{And he besought him much that he would not send them away out of the country.}
\bv{11}{Now there was there on the mountain side a great herd of swine feeding.}
\bv{12}{And they besought him, saying, ``Send us into the swine, that we may enter into them.''}
\bv{13}{And he gave them leave. And the unclean spirits came out, and entered into the swine: and the herd rushed down the steep into the sea, \supptext{in number} about two thousand; and they were drowned in the sea.}
\par
\bv{14}{And they that fed them fled, and told it in the city, and in the country. And they came to see what it was that had come to pass.}
\bv{15}{And they come to Jesus, and behold him that was possessed with demons sitting, clothed and in his right mind, \supptext{even} him that had the legion: and they were afraid.}
\bv{16}{And they that saw it declared unto them how it befell him that was possessed with demons, and concerning the swine.}
\bv{17}{And they began to beseech him to depart from their borders.}
\bv{18}{And as he was entering into the boat, he that had been possessed with demons besought him that he might be with him.}
\bv{19}{And he suffered him not, but saith unto him, \redlet{``Go to thy house unto thy friends, and tell them how great things the Lord hath done for thee, and \supptext{how} he had mercy on thee.''}}
\bv{20}{And he went his way, and began to publish in Decapolis how great things Jesus had done for him: and all men marvelled.}
\chapsec{Jesus Heals a Woman and Jairus’ Daughter}
\bv{21}{And when Jesus had crossed over again in the boat unto the other side, a great multitude was gathered unto him; and he was by the sea.}
\bv{22}{And there cometh one of the rulers of the synagogue, Jaïrus by name; and seeing him, he falleth at his feet,}
\bv{23}{and beseecheth him much, saying, ``My little daughter is at the point of death: \supptext{I pray thee}, that thou come and lay thy hands on her, that she may be made whole, and live.''}
\bv{24}{And he went with him; and a great multitude followed him, and they thronged him.}
\par
\bv{25}{And a woman, who had an issue of blood twelve years,}
\bv{26}{and had suffered many things of many physicians, and had spent all that she had, and was nothing bettered, but rather grew worse,}
\bv{27}{having heard the things concerning Jesus, came in the crowd behind, and touched his garment.}
\bv{28}{For she said, ``If I touch but his garments, I shall be made whole.''}
\bv{29}{And straightway the fountain of her blood was dried up; and she felt in her body that she was healed of her plague.}
\bv{30}{And straightway Jesus, perceiving in himself that the power \supptext{proceeding} from him had gone forth, turned him about in the crowd, and said, \redlet{``Who touched my garments?''}}
\bv{31}{And his disciples said unto him, ``Thou seest the multitude thronging thee, and sayest thou, `Who touched me?'{''}}
\bv{32}{And he looked round about to see her that had done this thing.}
\bv{33}{But the woman fearing and trembling, knowing what had been done to her, came and fell down before him, and told him all the truth.}
\bv{34}{And he said unto her, \redlet{``Daughter, thy faith hath made thee whole; go in peace, and be whole of thy plague.''}}
\par
\bv{35}{While he yet spake, they come from the ruler of the synagogue's \supptext{house}, saying, ``Thy daughter is dead: why troublest thou the Teacher any further?''}
\bv{36}{But Jesus, not heeding the word spoken, saith unto the ruler of the synagogue, \redlet{``Fear not, only believe.''}}
\bv{37}{And he suffered no man to follow with him, save Peter, and James, and John the brother of James.}
\bv{38}{And they come to the house of the ruler of the synagogue; and he beholdeth a tumult, and \supptext{many} weeping and wailing greatly.}
\bv{39}{And when he was entered in, he saith unto them, \redlet{``Why make ye a tumult, and weep? the child is not dead, but sleepeth.''}}
\bv{40}{And they laughed him to scorn. But he, having put them all forth, taketh the father of the child and her mother and them that were with him, and goeth in where the child was.}
\bv{41}{And taking the child by the hand, he saith unto her, \redlet{``Talitha cumi;''} which is, being interpreted, \redlet{``Damsel, I say unto thee, Arise.''}}
\bv{42}{And straightway the damsel rose up, and walked; for she was twelve years old. And they were amazed straightway with a great amazement.}
\bv{43}{And he charged them much that no man should know this: and he commanded that \supptext{something} should be given her to eat.}
\chaphead{Chapter VI}
\chapdesc{Jesus Rejected at Nazareth}
\lettrine[image=true, lines=4, findent=3pt, nindent=0pt]{NT/Mark/Mk-And.eps}{nd} he went out from thence; and he cometh into his own country; and his disciples follow him.
\bv{2}{And when the sabbath was come, he began to teach in the synagogue: and many hearing him were astonished, saying, ``Whence hath this man these things?'' and, ``What is the wisdom that is given unto this man,'' and ``\supptext{what mean} such mighty works wrought by his hands?}
\bv{3}{Is not this the carpenter, the son of Mary, and brother of James, and Joses, and Judas, and Simon? and are not his sisters here with us?'' And they were offended in him.}
\bv{4}{And Jesus said unto them, \redlet{``A prophet is not without honor, save in his own country, and among his own kin, and in his own house.''}}
\bv{5}{And he could there do no mighty work, save that he laid his hands upon a few sick folk, and healed them.}
\bv{6}{And he marvelled because of their unbelief. And he went round about the villages teaching.}
\chapsec{Jesus Sends Out the Twelve Apostles}
\bv{7}{And he calleth unto him the twelve, and began to send them forth by two and two; and he gave them authority over the unclean spirits;}
\bv{8}{and he charged them that they should take nothing for \supptext{their} journey, save a staff only; no bread, no wallet, no money in their purse;}
\bv{9}{but \supptext{to go} shod with sandals: and, \supptext{said he}, put not on two coats.}
\bv{10}{And he said unto them, \redlet{``Wheresoever ye enter into a house, there abide till ye depart thence.}}
\bv{11}{\redlet{And whatsoever place shall not receive you, and they hear you not, as ye go forth thence, shake off the dust that is under your feet for a testimony unto them.''}}\mcomm{Verily I say unto you, It shall be more tolerable for Sodom and Gomorrha in the day of judgement, than for that city.}
\bv{12}{And they went out, and preached that \supptext{men} should repent.}
\bv{13}{And they cast out many demons, and anointed with oil many that were sick, and healed them.}
\chapsec{The Death of St. John the Baptist}
\bv{14}{And king Herod heard \supptext{thereof}; for his name had become known: and he said, ``John the baptiser is risen from the dead, and therefore do these powers work in him.''}
\bv{15}{But others said, ``It is Elijah.'' And others said, ``\supptext{It is} a prophet, \supptext{even} as one of the prophets.''}
\bv{16}{But Herod, when he heard \supptext{thereof}, said, ``John, whom I beheaded, he is risen.''}
\bv{17}{For Herod himself had sent forth and laid hold upon John, and bound him in prison for the sake of Herodias, his brother Philip's wife; for he had married her.}
\bv{18}{For John said unto Herod, ``It is not lawful for thee to have thy brother's wife.''}
\bv{19}{And Herodias set herself against him, and desired to kill him; and she could not;}
\bv{20}{for Herod feared John, knowing that he was a righteous and holy man, and kept him safe. And when he heard him, he was much perplexed; and he heard him gladly.}
\bv{21}{And when a convenient day was come, that Herod on his birthday made a supper to his lords, and the high captains, and the chief men of Galilee;}
\bv{22}{and when the daughter of Herodias herself came in and danced, she pleased Herod and them that sat at meat with him; and the king said unto the damsel, ``Ask of me whatsoever thou wilt, and I will give it thee.''}
\bv{23}{And he sware unto her, ``Whatsoever thou shalt ask of me, I will give it thee, unto the half of my kingdom.''}
\bv{24}{And she went out, and said unto her mother, ``What shall I ask?'' And she said, ``The head of John the baptiser.''}
\bv{25}{And she came in straightway with haste unto the king, and asked, saying, ``I will that thou forthwith give me on a platter the head of John the Baptist.''}
\bv{26}{And the king was exceeding sorry; but for the sake of his oaths, and of them that sat at meat, he would not reject her.}
\bv{27}{And straightway the king sent forth a soldier of his guard, and commanded to bring his head: and he went and beheaded him in the prison,}
\bv{28}{and brought his head on a platter, and gave it to the damsel; and the damsel gave it to her mother.}
\bv{29}{And when his disciples heard \supptext{thereof}, they came and took up his corpse, and laid it in a tomb.}
\chapsec{Jesus Feeds the Five Thousand}
\bv{30}{And the apostles gather themselves together unto Jesus; and they told him all things, whatsoever they had done, and whatsoever they had taught.}
\bv{31}{And he saith unto them, \redlet{``Come ye yourselves apart into a desert place, and rest a while.''} For there were many coming and going, and they had no leisure so much as to eat.}
\bv{32}{And they went away in the boat to a desert place apart.}
\bv{33}{And \supptext{the people} saw them going, and many knew \supptext{them}, and they ran together there on foot from all the cities, and outwent them.}
\bv{34}{And he came forth and saw a great multitude, and he had compassion on them, because they were as sheep not having a shepherd: and he began to teach them many things.}
\bv{35}{And when the day was now far spent, his disciples came unto him, and said, ``The place is desert, and the day is now far spent;}
\bv{36}{send them away, that they may go into the country and villages round about, and buy themselves somewhat to eat.''}
\bv{37}{But he answered and said unto them, \redlet{``Give ye them to eat.''} And they say unto him, ``Shall we go and buy two hundred denarii's\mcomm{Denarius: a day's wages.} worth of bread, and give them to eat?''}
\bv{38}{And he saith unto them, \redlet{``How many loaves have ye? go \supptext{and} see.''} And when they knew, they say, ``Five, and two fishes.''}
\bv{39}{And he commanded them that all should sit down by companies upon the green grass.}
\bv{40}{And they sat down in ranks, by hundreds, and by fifties.}
\bv{41}{And he took the five loaves and the two fishes, and looking up to heaven, he blessed, and brake the loaves; and he gave to the disciples to set before them; and the two fishes divided he among them all.}
\bv{42}{And they all ate, and were filled.}
\bv{43}{And they took up broken pieces, twelve basketfuls, and also of the fishes.}
\bv{44}{And they that ate the loaves were five thousand men.}
\chapsec{Jesus Walks on the Water}
\bv{45}{And straightway he constrained his disciples to enter into the boat, and to go before \supptext{him} unto the other side to Bethsaida, while he himself sendeth the multitude away.}
\bv{46}{And after he had taken leave of them, he departed into the mountain to pray.}
\bv{47}{And when even was come, the boat was in the midst of the sea, and he alone on the land.}
\bv{48}{And seeing them distressed in rowing, for the wind was contrary unto them, about the fourth watch of the night he cometh unto them, walking on the sea; and he would have passed by them:}
\bv{49}{but they, when they saw him walking on the sea, supposed that it was a ghost, and cried out;}
\bv{50}{for they all saw him, and were troubled. But he straightway spake with them, and saith unto them, \redlet{``Be of good cheer: it is I; be not afraid.''}}
\bv{51}{And he went up unto them into the boat; and the wind ceased: and they were sore amazed in themselves;}
\bv{52}{for they understood not concerning the loaves, but their heart was hardened.}
\chapsec{Jesus Heals the Sick in Gennesaret}
\bv{53}{And when they had crossed over, they came to the land unto Gennesaret, and moored to the shore.}
\bv{54}{And when they were come out of the boat, straightway \supptext{the people} knew him,}
\bv{55}{and ran round about that whole region, and began to carry about on their beds those that were sick, where they heard he was.}
\bv{56}{And wheresoever he entered, into villages, or into cities, or into the country, they laid the sick in the marketplaces, and besought him that they might touch if it were but the border of his garment: and as many as touched him were made whole.}
\chaphead{Chapter VII}
\chapdesc{Traditions and Commandments}
\lettrine[image=true, lines=4, findent=3pt, nindent=0pt]{NT/Mark/Mk-And.eps}{nd} there are gathered together unto him the Pharisees, and certain of the scribes, who had come from Jerusalem,
\bv{2}{and had seen that some of his disciples ate their bread with defiled, that is, unwashen, hands.}
\bv{3}{(For the Pharisees, and all the Jews, except they wash their hands diligently, eat not, holding the tradition of the elders;}
\bv{4}{and \supptext{when they come} from the marketplace, except they bathe themselves, they eat not; and many other things there are, which they have received to hold, washings of cups, and pots, and brasen vessels.)}
\bv{5}{And the Pharisees and the scribes ask him, ``Why walk not thy disciples according to the tradition of the elders, but eat their bread with defiled hands?''}
\bv{6}{And he said unto them, \redlet{``Well did Isaiah prophesy of you hypocrites, as it is written,}}
\otQuote{Isaiah 29:13 LXX}{\redlet{This people honoreth me with their lips, But their heart is far from me. \bv{7}{But in vain do they worship me, Teaching \supptext{as their} doctrines the precepts of men.}}}
\bv{8}{\redlet{Ye leave the commandment of God, and hold fast the tradition of men.''}}\mcomm{the washing of pitchers and cups, and you do many other such things.}
\bv{9}{And he said unto them, \redlet{``Full well do ye reject the commandment of God, that ye may keep your tradition.}}
\bv{10}{\redlet{For Moses said, `Honor thy father and thy mother;' and, `He that speaketh evil of father or mother, let him die the death:'}}
\bv{11}{\redlet{but ye say, `If a man shall say to his father or his mother, That wherewith thou mightest have been profited by me is Corban, that is to say, Given \supptext{to God};}}
\bv{12}{\redlet{ye no longer suffer him to do aught for his father or his mother;'}}
\bv{13}{\redlet{making void the word of God by your tradition, which ye have delivered: and many such like things ye do.''}}
\chapsec{What Defiles a Person}
\bv{14}{And he called to him the multitude again, and said unto them, \redlet{``Hear me all of you, and understand:}}
\bv{15}{\redlet{there is nothing from without the man, that going into him can defile him; but the things which proceed out of the man are those that defile the man.''}}\mcomm{If any man hath ears to hear, let him hear.}
\bv{17}{And when he was entered into the house from the multitude, his disciples asked of him the parable.}
\bv{18}{And he saith unto them, \redlet{``Are ye so without understanding also? Perceive ye not, that whatsoever from without goeth into the man, \supptext{it} cannot defile him;}}
\bv{19}{\redlet{because it goeth not into his heart, but into his belly, and goeth out into the draught?''} \supptext{This he said}, making all meats clean.}
\bv{20}{And he said, \redlet{``That which proceedeth out of the man, that defileth the man.}}
\bv{21}{\redlet{For from within, out of the heart of men, evil thoughts proceed, fornications, thefts, murders, adulteries,}}
\bv{22}{\redlet{covetings, wickednesses, deceit, lasciviousness, an evil eye, railing, pride, foolishness:}}
\bv{23}{\redlet{all these evil things proceed from within, and defile the man.''}}
\chapsec{The Syrophoenician Woman's Faith}
\bv{24}{And from thence he arose, and went away into the borders of Tyre and Sidon. And he entered into a house, and would have no man know it; and he could not be hid.}
\bv{25}{But straightway a woman, whose little daughter had an unclean spirit, having heard of him, came and fell down at his feet.}
\bv{26}{Now the woman was a Greek, a Syrophoenician by race. And she besought him that he would cast forth the demon out of her daughter.}
\bv{27}{And he said unto her, \redlet{``Let the children first be filled: for it is not meet to take the children's bread and cast it to the dogs.''}}
\bv{28}{But she answered and saith unto him, ``Yea, Lord; even the dogs under the table eat of the children's crumbs.''}\mcomm{The proud woman is brought to humility by Our Lord.}
\bv{29}{And he said unto her, \redlet{``For this saying go thy way; the demon is gone out of thy daughter.''}}
\bv{30}{And she went away unto her house, and found the child laid upon the bed, and the demon gone out.}
\chapsec{Jesus Heals a Deaf Man}
\bv{31}{And again he went out from the borders of Tyre, and came through Sidon unto the sea of Galilee, through the midst of the borders of Decapolis.}
\bv{32}{And they bring unto him one that was deaf, and had an impediment in his speech; and they beseech him to lay his hand upon him.}
\bv{33}{And he took him aside from the multitude privately, and put his fingers into his ears, and he spat, and touched his tongue;}
\bv{34}{and looking up to heaven, he sighed, and saith unto him, \redlet{``Ephphatha,''} that is, \redlet{``Be opened.''}}
\bv{35}{And his ears were opened, and the bond of his tongue was loosed, and he spake plain.}
\bv{36}{And he charged them that they should tell no man: but the more he charged them, so much the more a great deal they published it.}
\bv{37}{And they were beyond measure astonished, saying, ``He hath done all things well; he maketh even the deaf to hear, and the dumb to speak.''}
\chaphead{Chapter VIII}
\chapdesc{Jesus Feeds the Four Thousand}
\lettrine[image=true, lines=4, findent=3pt, nindent=0pt]{NT/Mark/Mk-In.eps}{n} those days, when there was again a great multitude, and they had nothing to eat, he called unto him his disciples, and saith unto them,
\bv{2}{\redlet{``I have compassion on the multitude, because they continue with me now three days, and have nothing to eat:}}
\bv{3}{\redlet{and if I send them away fasting to their home, they will faint on the way; and some of them are come from far.''}}
\bv{4}{And his disciples answered him, ``Whence shall one be able to fill these men with bread here in a desert place?''}
\bv{5}{And he asked them, \redlet{``How many loaves have ye?''} And they said, ``Seven.''}
\bv{6}{And he commandeth the multitude to sit down on the ground: and he took the seven loaves, and having given thanks, he brake, and gave to his disciples, to set before them; and they set them before the multitude.}
\bv{7}{And they had a few small fishes: and having blessed them, he commanded to set these also before them.}
\bv{8}{And they ate, and were filled: and they took up, of broken pieces that remained over, seven baskets.}
\bv{9}{And they were about four thousand: and he sent them away.}
\bv{10}{And straightway he entered into the boat with his disciples, and came into the parts of Dalmanutha.}
\chapsec{The Pharisees Demand a Sign}
\bv{11}{And the Pharisees came forth, and began to question with him, seeking of him a sign from heaven, trying him.}
\bv{12}{And he sighed deeply in his spirit, and saith, \redlet{``Why doth this generation seek a sign? verily I say unto you, There shall no sign be given unto this generation.''}}
\bv{13}{And he left them, and again entering into \supptext{the boat} departed to the other side.}
\chapsec{The Leaven of the Pharisees and Herod}
\bv{14}{And they forgot to take bread; and they had not in the boat with them more than one loaf.}
\bv{15}{And he charged them, saying, \redlet{``Take heed, beware of the leaven of the Pharisees and the leaven of Herod.''}}
\bv{16}{And they reasoned one with another, saying, `We have no bread.'}
\bv{17}{And Jesus perceiving it saith unto them, \redlet{``Why reason ye, because ye have no bread? do ye not yet perceive, neither understand? have ye your heart hardened?}}
\bv{18}{\redlet{Having eyes, see ye not? and having ears, hear ye not? and do ye not remember?}}
\bv{19}{\redlet{When I brake the five loaves among the five thousand, how many baskets full of broken pieces took ye up?''} They say unto him, ``Twelve.''}
\bv{20}{\redlet{``And when the seven among the four thousand, how many basketfuls of broken pieces took ye up?''} And they say unto him, ``Seven.''}
\bv{21}{And he said unto them, \redlet{``Do ye not yet understand?''}}
\chapsec{Jesus Heals a Blind Man at Bethsaida}
\bv{22}{And they come unto Bethsaida. And they bring to him a blind man, and beseech him to touch him.}
\bv{23}{And he took hold of the blind man by the hand, and brought him out of the village; and when he had spit on his eyes, and laid his hands upon him, he asked him, \redlet{``Seest thou aught?''}}
\bv{24}{And he looked up, and said, ``I see men; for I behold \supptext{them} as trees, walking.''}
\bv{25}{Then again he laid his hands upon his eyes; and he looked stedfastly, and was restored, and saw all things clearly.}
\bv{26}{And he sent him away to his home, saying, \redlet{``Do not even enter into the village.''}}
\chapsec{Peter Confesses Jesus as the Christ}
\bv{27}{And Jesus went forth, and his disciples, into the villages of Cæsarea Philippi: and on the way he asked his disciples, saying unto them, \redlet{``Who do men say that I am?''}}
\bv{28}{And they told him, saying, ``John the Baptist; and others, Elijah; but others, One of the prophets.''}
\bv{29}{And he asked them, \redlet{``But who say ye that I am?''} Peter answereth and saith unto him, ``Thou art the Christ.''}
\bv{30}{And he charged them that they should tell no man of him.}
\chapsec{Jesus Foretells His Death and Resurrection}
\bv{31}{And he began to teach them, that the Son of man must suffer many things, and be rejected by the elders, and the chief priests, and the scribes, and be killed, and after three days rise again.}
\bv{32}{And he spake the saying openly. And Peter took him, and began to rebuke him.}
\bv{33}{But he turning about, and seeing his disciples, rebuked Peter, and saith, \redlet{``Get thee behind me, Satan; for thou mindest not the things of God, but the things of men.''}}
\bv{34}{And he called unto him the multitude with his disciples, and said unto them, \redlet{``If any man would come after me, let him deny himself, and take up his cross, and follow me.}}
\bv{35}{\redlet{For whosoever would save his life shall lose it; and whosoever shall lose his life for my sake and the gospel's shall save it.}}
\bv{36}{\redlet{For what doth it profit a man, to gain the whole world, and forfeit his life?}}
\bv{37}{\redlet{For what should a man give in exchange for his life?}}
\bv{38}{\redlet{For whosoever shall be ashamed of me and of my words in this adulterous and sinful generation, the Son of man also shall be ashamed of him, when he cometh in the glory of his Father with the holy angels.''}}
\chaphead{Chapter IX}
\chapdesc{The Transfiguration}
\lettrine[image=true, lines=4, findent=3pt, nindent=0pt]{NT/Mark/Mk-And.eps}{nd} he said unto them, \redlet{``Verily I say unto you, There are some here of them that stand \supptext{by}, who shall in no wise taste of death, till they see the kingdom of God come with power.''}
\bv{2}{And after six days Jesus taketh with him Peter, and James, and John, and bringeth them up into a high mountain apart by themselves: and he was transfigured before them;}
\bv{3}{and his garments became glistering, exceeding white, so as no fuller on earth can whiten them.}
\bv{4}{And there appeared unto them Elijah with Moses: and they were talking with Jesus.}
\bv{5}{And Peter answereth and saith to Jesus, ``Rabbi, it is good for us to be here: and let us make three tabernacles; one for thee, and one for Moses, and one for Elijah.''}
\bv{6}{For he knew not what to answer; for they became sore afraid.}
\bv{7}{And there came a cloud overshadowing them: and there came a voice out of the cloud, \god{This is my beloved Son: hear ye him.}}
\bv{8}{And suddenly looking round about, they saw no one any more, save Jesus only with themselves.}
\bv{9}{And as they were coming down from the mountain, he charged them that they should tell no man what things they had seen, save when the Son of man should have risen again from the dead.}
\bv{10}{And they kept the saying, questioning among themselves what the rising again from the dead should mean.}
\chapsec{Instruction of the Disciples concerning Elijah}
\bv{11}{And they asked him, saying, ``\supptext{How is it} that the scribes say that Elijah must first come?''}
\bv{12}{And he said unto them, \redlet{``Elijah indeed cometh first, and restoreth all things: and how is it written of the Son of man, that he should suffer many things and be set at nought?}}
\bv{13}{\redlet{But I say unto you, that Elijah is come, and they have also done unto him whatsoever they would, even as it is written of him.''}}
\chapsec{Jesus Rebukes a Deaf \& Dumb Spirit}
\bv{14}{And when they came to the disciples, they saw a great multitude about them, and scribes questioning with them.}
\bv{15}{And straightway all the multitude, when they saw him, were greatly amazed, and running to him saluted him.}
\bv{16}{And he asked them, \redlet{``What question ye with them?''}}
\bv{17}{And one of the multitude answered him, ``Teacher, I brought unto thee my son, who hath a dumb spirit;}
\bv{18}{and wheresoever it taketh him, it dasheth him down: and he foameth, and grindeth his teeth, and pineth away: and I spake to thy disciples that they should cast it out; and they were not able.''}
\bv{19}{And he answereth them and saith, \redlet{``O faithless generation, how long shall I be with you? how long shall I bear with you? bring him unto me.''}}
\bv{20}{And they brought him unto him: and when he saw him, straightway the spirit tare him grievously; and he fell on the ground, and wallowed foaming.}
\bv{21}{And he asked his father, \redlet{``How long time is it since this hath come unto him?''} And he said, ``From a child.}
\bv{22}{And oft-times it hath cast him both into the fire and into the waters, to destroy him: but if thou canst do anything, have compassion on us, and help us.''}
\bv{23}{And Jesus said unto him, \redlet{``If thou canst! All things are possible to him that believeth.''}}
\bv{24}{Straightway the father of the child cried out, and said, ``I believe; help thou mine unbelief.''}
\bv{25}{And when Jesus saw that a multitude came running together, he rebuked the unclean spirit, saying unto him, \redlet{``Thou dumb and deaf spirit, I command thee, come out of him, and enter no more into him.''}}
\bv{26}{And having cried out, and torn him much, he came out: and \supptext{the boy} became as one dead; insomuch that the more part said, ``He is dead.''}
\bv{27}{But Jesus took him by the hand, and raised him up; and he arose.}
\bv{28}{And when he was come into the house, his disciples asked him privately, ``\supptext{How is it} that we could not cast it out?''}
\bv{29}{And he said unto them, \redlet{``This kind can come out by nothing, save by prayer.''}}
\chapsec{Jesus Fortells his Death \& Resurrection}
\bv{30}{And they went forth from thence, and passed through Galilee; and he would not that any man should know it.}
\bv{31}{For he taught his disciples, and said unto them, \redlet{``The Son of man is delivered up into the hands of men, and they shall kill him; and when he is killed, after three days he shall rise again.''}}
\bv{32}{But they understood not the saying, and were afraid to ask him.}
\chapsec{Jesus Exhorts his Disciples to Humility}
\bv{33}{And they came to Capernaum: and when he was in the house he asked them, \redlet{``What were ye reasoning on the way?}}
\bv{34}{But they held their peace: for they had disputed one with another on the way, who \supptext{was} the greatest.}
\bv{35}{And he sat down, and called the twelve; and he saith unto them, \redlet{``If any man would be first, he shall be last of all, and servant of all.''}}
\bv{36}{And he took a little child, and set him in the midst of them: and taking him in his arms, he said unto them,}
\bv{37}{\redlet{``Whosoever shall receive one of such little children in my name, receiveth me: and whosoever receiveth me, receiveth not me, but him that sent me.''}}
\chapsec{The Rebuke of Sectarianism}
\bv{38}{John said unto him, ``Teacher, we saw one casting out demons in thy name;\mcomm{and he followeth not us:} and we forbade him, because he followed not us.''}
\bv{39}{But Jesus said, \redlet{``Forbid him not: for there is no man who shall do a mighty work in my name, and be able quickly to speak evil of me.}}
\bv{40}{\redlet{For he that is not against us is for us.}}
\bv{41}{\redlet{For whosoever shall give you a cup of water to drink, because ye are Christ's, verily I say unto you, he shall in no wise lose his reward.}}
\chapsec{Jesus' Solemn Warning of Hell}
\bv{42}{\redlet{And whosoever shall cause one of these little ones that believe on me to stumble, it were better for him if a great millstone were hanged about his neck, and he were cast into the sea.}}
\bv{43}{\redlet{And if thy hand cause thee to stumble, cut it off: it is good for thee to enter into life maimed, rather than having thy two hands to go into hell, into the unquenchable fire.}}\mcomm{Where their worm dieth not, and the fire is not quenched.}
\bv{45}{\redlet{And if thy foot cause thee to stumble, cut it off: it is good for thee to enter into life halt, rather than having thy two feet to be cast into hell.}}\mcomm{Where their worm dieth not, and the fire is not quenched.}
\bv{47}{\redlet{And if thine eye cause thee to stumble, cast it out: it is good for thee to enter into the kingdom of God with one eye, rather than having two eyes to be cast into hell;}}
\bv{48}{\redlet{where their worm dieth not, and the fire is not quenched.}}
\bv{49}{\redlet{For every one shall be salted with fire.}}\mcomm{and every sacrifice shall be salted with salt.}
\bv{50}{\redlet{Salt is good: but if the salt have lost its saltness, wherewith will ye season it? Have salt in yourselves, and be at peace one with another.''}}
\chaphead{Chapter X}
\chapdesc{Jesus' Law of Divorce}
\lettrine[image=true, lines=4, findent=3pt, nindent=0pt]{NT/Mark/Mk-And.eps}{nd} he arose from thence, and cometh into the borders of Judæa and beyond the Jordan: and multitudes come together unto him again; and, as he was wont, he taught them again.
\bv{2}{And there came unto him Pharisees, and asked him, ``Is it lawful for a man to put away \supptext{his} wife?'' trying him.}
\bv{3}{And he answered and said unto them, \redlet{``What did Moses command you?''}}
\bv{4}{And they said, ``Moses suffered to write a bill of divorcement, and to put her away.''}
\bv{5}{But Jesus said unto them, \redlet{``For your hardness of heart he wrote you this commandment.}}
\bv{6}{\redlet{But from the beginning of the creation, Male and female made he them.}}
\bv{7}{\redlet{For this cause shall a man leave his father and mother, and shall cleave to his wife;}}
\bv{8}{\redlet{and the two shall become one flesh: so that they are no more two, but one flesh.}}
\bv{9}{\redlet{What therefore God hath joined together, let not man put asunder.''}}
\bv{10}{And in the house the disciples asked him again of this matter.}
\bv{11}{And he saith unto them, \redlet{``Whosoever shall put away his wife, and marry another, committeth adultery against her:}}
\bv{12}{\redlet{and if she herself shall put away her husband, and marry another, she committeth adultery.''}}
\chapsec{Jesus Blesses Little Children}
\bv{13}{And they were bringing unto him little children, that he should touch them: and the disciples rebuked them.}
\bv{14}{But when Jesus saw it, he was moved with indignation, and said unto them, \redlet{``Suffer the little children to come unto me; forbid them not: for to such belongeth the kingdom of God.}}
\bv{15}{\redlet{Verily I say unto you, Whosoever shall not receive the kingdom of God as a little child, he shall in no wise enter therein.''}}
\bv{16}{And he took them in his arms, and blessed them, laying his hands upon them.}
\chapsec{The Rich Young Ruler}
\bv{17}{And as he was going forth into the way, there ran one to him, and kneeled to him, and asked him, ``Good Teacher, what shall I do that I may inherit eternal life?''}
\bv{18}{And Jesus said unto him, \redlet{Why callest thou me good? none is good save one, \supptext{even} God.}}
\bv{19}{\redlet{Thou knowest the commandments, ``Do not kill, Do not commit adultery, Do not steal, Do not bear false witness, Do not defraud, Honor thy father and mother.''}}
\bv{20}{And he said unto him, ``Teacher, all these things have I observed from my youth.''}
\bv{21}{And Jesus looking upon him loved him, and said unto him, \redlet{``One thing thou lackest: go, sell whatsoever thou hast, and give to the poor, and thou shalt have treasure in heaven: and come, follow me.''}}
\bv{22}{But his countenance fell at the saying, and he went away sorrowful: for he was one that had great possessions.}
\chapsec{The Warning against Riches}
\bv{23}{And Jesus looked round about, and saith unto his disciples, \redlet{``How hardly shall they that have riches enter into the kingdom of God!''}}
\bv{24}{And the disciples were amazed at his words. But Jesus answereth again, and saith unto them, \redlet{``Children, how hard is it for them that trust in riches to enter into the kingdom of God!}}
\bv{25}{\redlet{It is easier for a camel to go through a needle's eye, than for a rich man to enter into the kingdom of God.''}}
\bv{26}{And they were astonished exceedingly, saying unto him, ``Then who can be saved?''}
\bv{27}{Jesus looking upon them saith, \redlet{``With men it is impossible, but not with God: for all things are possible with God.''}}
\bv{28}{Peter began to say unto him, ``Lo, we have left all, and have followed thee.''}
\bv{29}{Jesus said, \redlet{``Verily I say unto you, There is no man that hath left house, or brethren, or sisters, or mother, or father, or children, or lands, for my sake, and for the gospel's sake,}}
\bv{30}{\redlet{but he shall receive a hundredfold now in this time, houses, and brethren, and sisters, and mothers, and children, and lands, with persecutions; and in the world to come eternal life.}}
\bv{31}{\redlet{But many \supptext{that are} first shall be last; and the last first.''}}
\chapsec{Jesus again Foretells his Death \& Resurrection}
\bv{32}{And they were on the way, going up to Jerusalem; and Jesus was going before them: and they were amazed; and they that followed were afraid. And he took again the twelve, and began to tell them the things that were to happen unto him,}
\bv{33}{\supptext{saying}, \redlet{``Behold, we go up to Jerusalem; and the Son of man shall be delivered unto the chief priests and the scribes; and they shall condemn him to death, and shall deliver him unto the Gentiles:}}
\bv{34}{\redlet{and they shall mock him, and shall spit upon him, and shall scourge him, and shall kill him; and after three days he shall rise again.''}}
\chapsec{The Desire of Sts. James \& John}
\bv{35}{And there come near unto him James and John, the sons of Zebedee, saying unto him, ``Teacher, we would that thou shouldest do for us whatsoever we shall ask of thee.''}
\bv{36}{And he said unto them, \redlet{``What would ye that I should do for you?''}}
\bv{37}{And they said unto him, ``Grant unto us that we may sit, one on thy right hand, and one on \supptext{thy} left hand, in thy glory.''}
\bv{38}{But Jesus said unto them, \redlet{``Ye know not what ye ask. Are ye able to drink the cup that I drink? or to be baptised with the baptism that I am baptised with?''}}
\bv{39}{And they said unto him, ``We are able.'' And Jesus said unto them, \redlet{``The cup that I drink ye shall drink; and with the baptism that I am baptised withal shall ye be baptised:}}
\bv{40}{\redlet{but to sit on my right hand or on \supptext{my} left hand is not mine to give; but \supptext{it is for them} for whom it hath been prepared.''}}
\bv{41}{And when the ten heard it, they began to be moved with indignation concerning James and John.}
\bv{42}{And Jesus called them to him, and saith unto them, \redlet{``Ye know that they who are accounted to rule over the Gentiles lord it over them; and their great ones exercise authority over them.}}
\bv{43}{\redlet{But it is not so among you: but whosoever would become great among you, shall be your minister;}}
\bv{44}{\redlet{and whosoever would be first among you, shall be servant of all.}}
\bv{45}{\redlet{For the Son of man also came not to be ministered unto, but to minister, and to give his life a ransom for many.''}}
\chapsec{Bartimæus Receives his Sight}
\bv{46}{And they come to Jericho: and as he went out from Jericho, with his disciples and a great multitude, the son of Timæus, Bartimæus, a blind beggar, was sitting by the way side.}
\bv{47}{And when he heard that it was Jesus the Nazarene, he began to cry out, and say, ``Jesus, thou son of David, have mercy on me.''}
\bv{48}{And many rebuked him, that he should hold his peace: but he cried out the more a great deal, ``Thou son of David, have mercy on me.''}
\bv{49}{And Jesus stood still, and said, \redlet{``Call ye him.''} And they call the blind man, saying unto him, ``Be of good cheer: rise, he calleth thee.''}
\bv{50}{And he, casting away his garment, sprang up, and came to Jesus.}
\bv{51}{And Jesus answered him, and said, \redlet{``What wilt thou that I should do unto thee?''} And the blind man said unto him, ``Rabboni, that I may receive my sight.''}
\bv{52}{And Jesus said unto him, \redlet{``Go thy way; thy faith hath made thee whole.''} And straightway he received his sight, and followed him in the way.}
\chaphead{Chapter XI}
\chapdesc{The Official Presentation of Jesus as King}
\lettrine[image=true, lines=4, findent=3pt, nindent=0pt]{NT/Mark/Mk-And.eps}{nd} when they draw nigh unto Jerusalem, unto Bethphage and Bethany, at the mount of Olives, he sendeth two of his disciples,
\bv{2}{and saith unto them, \redlet{``Go your way into the village that is over against you: and straightway as ye enter into it, ye shall find a colt tied, whereon no man ever yet sat; loose him, and bring him.}}
\bv{3}{\redlet{And if any one say unto you, `Why do ye this?' say ye, `The Lord hath need of him; and straightway he will send him back hither.'{''}}}
\bv{4}{And they went away, and found a colt tied at the door without in the open street; and they loose him.}
\bv{5}{And certain of them that stood there said unto them, ``What do ye, loosing the colt?''}
\bv{6}{And they said unto them even as Jesus had said: and they let them go.}
\bv{7}{And they bring the colt unto Jesus, and cast on him their garments; and he sat upon him.}
\bv{8}{And many spread their garments upon the way; and others branches, which they had cut from the fields.}
\bv{9}{And they that went before, and they that followed, cried, ``Hosanna; Blessed \supptext{is} he that cometh in the name of the Lord:}
\bv{10}{Blessed \supptext{is} the kingdom that cometh, \supptext{the kingdom} of our father David: Hosanna in the highest.''}
\bv{11}{And he entered into Jerusalem, into the temple; and when he had looked round about upon all things, it being now eventide, he went out unto Bethany with the twelve.}
\chapsec{The Barren Fig Tree}
\bv{12}{And on the morrow, when they were come out from Bethany, he hungered.}
\bv{13}{And seeing a fig tree afar off having leaves, he came, if haply he might find anything thereon: and when he came to it, he found nothing but leaves; for it was not the season of figs.}
\bv{14}{And he answered and said unto it, \redlet{``No man eat fruit from thee henceforward for ever.''} And his disciples heard it.}
\chapsec{Jesus Purifies the Temple}
\bv{15}{And they come to Jerusalem: and he entered into the temple, and began to cast out them that sold and them that bought in the temple, and overthrew the tables of the money-changers, and the seats of them that sold the doves;}
\bv{16}{and he would not suffer that any man should carry a vessel through the temple.}
\bv{17}{And he taught, and said unto them, \redlet{``Is it not written, `My house shall be called a house of prayer for all the nations?'\mref{Is. 56:7} but ye have made it a den of robbers.}}
\bv{18}{And the chief priests and the scribes heard it, and sought how they might destroy him: for they feared him, for all the multitude was astonished at his teaching.}
\bv{19}{And every evening he went forth out of the city.}
\bv{20}{And as they passed by in the morning, they saw the fig tree withered away from the roots.}
\bv{21}{And Peter calling to remembrance saith unto him, ``Rabbi, behold, the fig tree which thou cursedst is withered away.''}
\chapsec{The Prayer of Faith}
\bv{22}{And Jesus answering saith unto them, \redlet{``Have faith in God.}}
\bv{23}{\redlet{Verily I say unto you, Whosoever shall say unto this mountain, `Be thou taken up and cast into the sea;' and shall not doubt in his heart, but shall believe that what he saith cometh to pass; he shall have it.}}
\bv{24}{\redlet{Therefore I say unto you, All things whatsoever ye pray and ask for, believe that ye receive them, and ye shall have them.}}
\bv{25}{\redlet{And whensoever ye stand praying, forgive, if ye have aught against any one; that your Father also who is in heaven may forgive you your trespasses.}}\mcomm{But if ye do not forgive, neither will your Father who is in heaven forgive your trespasses.}
\chapsec{Jesus' Authority Questioned}
\bv{27}{And they come again to Jerusalem: and as he was walking in the temple, there come to him the chief priests, and the scribes, and the elders;}
\bv{28}{and they said unto him, ``By what authority doest thou these things? or who gave thee this authority to do these things?''}
\bv{29}{And Jesus said unto them, \redlet{``I will ask of you one question, and answer me, and I will tell you by what authority I do these things.}}
\bv{30}{\redlet{The baptism of John, was it from heaven, or from men? answer me.''}}
\bv{31}{And they reasoned with themselves, saying, ``If we shall say, `From heaven;' he will say, `Why then did ye not believe him?'}
\bv{32}{But should we say, `From men'{''}---they feared the people: for all verily held John to be a prophet.}
\bv{33}{And they answered Jesus and say, ``We know not.'' And Jesus saith unto them, \redlet{``Neither tell I you by what authority I do these things.''}}
\chaphead{Chapter XII}
\chapdesc{Parable of the Householder}
\lettrine[image=true, lines=4, findent=3pt, nindent=0pt]{NT/Mark/Mk-And.eps}{nd} he began to speak unto them in parables. \redlet{``A man planted a vineyard, and set a hedge about it, and digged a pit for the winepress, and built a tower, and let it out to husbandmen, and went into another country.}
\bv{2}{\redlet{And at the season he sent to the husbandmen a servant, that he might receive from the husbandmen of the fruits of the vineyard.}}
\bv{3}{\redlet{And they took him, and beat him, and sent him away empty.}}
\bv{4}{\redlet{And again he sent unto them another servant; and him they wounded in the head, and handled shamefully.}}
\bv{5}{\redlet{And he sent another; and him they killed: and many others; beating some, and killing some.}}
\bv{6}{\redlet{He had yet one, a beloved son: he sent him last unto them, saying, `They will reverence my son.'}}
\bv{7}{\redlet{But those husbandmen said among themselves, `This is the heir; come, let us kill him, and the inheritance shall be ours.'}}
\bv{8}{\redlet{And they took him, and killed him, and cast him forth out of the vineyard.}}
\bv{9}{\redlet{What therefore will the lord of the vineyard do? he will come and destroy the husbandmen, and will give the vineyard unto others.}}
\bv{10}{\redlet{Have ye not read even this scripture:}}
\otQuote{Ps. 118:22-3}{\redlet{The stone which the builders rejected, The same was made the head of the corner; {\textsuperscript{11}}This was from the Lord, And it is marvelous in our eyes?''}}
\bv{12}{And they sought to lay hold on him; and they feared the multitude; for they perceived that he spake the parable against them: and they left him, and went away.}
\chapsec{The Question of Tribute}
\bv{13}{And they send unto him certain of the Pharisees and of the Herodians, that they might catch him in talk.}
\bv{14}{And when they were come, they say unto him, ``Teacher, we know that thou art true, and carest not for any one; for thou regardest not the person of men, but of a truth teachest the way of God: Is it lawful to give tribute unto Cæsar, or not?}
\bv{15}{Shall we give, or shall we not give?'' But he, knowing their hypocrisy, said unto them, \redlet{``Why make ye trial of me? bring me a denarius, that I may see it.''}}
\bv{16}{And they brought it. And he saith unto them, \redlet{``Whose is this image and superscription?''} And they said unto him, ``Cæsar's.''}
\bv{17}{And Jesus said unto them, \redlet{``Render unto Cæsar the things that are Cæsar's, and unto God the things that are God's.''} And they marvelled greatly at him.}
\chapsec{Jesus Answers the Sadducees}
\bv{18}{And there come unto him Sadducees, who say that there is no resurrection; and they asked him, saying,}
\bv{19}{``Teacher, Moses wrote unto us, If a man's brother die, and leave a wife behind him, and leave no child, that his brother should take his wife, and raise up seed unto his brother.}
\bv{20}{There were seven brethren: and the first took a wife, and dying left no seed;}
\bv{21}{and the second took her, and died, leaving no seed behind him; and the third likewise:}
\bv{22}{and the seven left no seed. Last of all the woman also died.}
\bv{23}{In the resurrection whose wife shall she be of them? for the seven had her to wife.''}
\bv{24}{Jesus said unto them, \redlet{``Is it not for this cause that ye err, that ye know not the scriptures, nor the power of God?}}
\bv{25}{\redlet{For when they shall rise from the dead, they neither marry, nor are given in marriage; but are as angels in heaven.}}
\bv{26}{\redlet{But as touching the dead, that they are raised; have ye not read in the book of Moses, in \supptext{the place concerning} the Bush, how God spake unto him, saying,}}
\otQuote{}{\redlet{I \supptext{am} the God of Abraham, and the God of Isaac, and the God of Jacob?}}
\bv{27}{\redlet{He is not the God of the dead, but of the living: ye do greatly err.''}}
\chapsec{The Great Commandments}
\bv{28}{And one of the scribes came, and heard them questioning together, and knowing that he had answered them well, asked him, ``What commandment is the first of all?''}
\bv{29}{Jesus answered, \redlet{``The first is,}}
\otQuote{Deut. 6:4-5}{\redlet{Hear, O Israel; The Lord our God, the Lord is one:
\bv{30}{and thou shalt love the Lord thy God with all thy heart, and with all thy soul, and with all thy mind, and with all thy strength.}}}
\bv{31}{\redlet{The second is this,}}
\otQuote{Lev. 19:18}{\redlet{Thou shalt love thy neighbor as thyself.}}
\redlet{There is none other commandment greater than these.}
\bv{32}{And the scribe said unto him, ``Of a truth, Teacher, thou hast well said that he is one; and there is none other but he:}
\bv{33}{and to love him with all the heart, and with all the understanding, and with all the strength, and to love his neighbor as himself, is much more than all whole burnt-offerings and sacrifices.''}
\bv{34}{And when Jesus saw that he answered discreetly, he said unto him, \redlet{``Thou art not far from the kingdom of God.''} And no man after that durst ask him any question.}
\chapsec{Jesus Questions the Pharisees}
\bv{35}{And Jesus answered and said, as he taught in the temple, \redlet{``How say the scribes that the Christ is the son of David?}}
\bv{36}{\redlet{David himself said in the Holy Ghost,}}
\otQuote{Ps. 110:1}{\redlet{The Lord said unto my Lord, Sit thou on my right hand, Till I make thine enemies the footstool of thy feet.}}
\bv{37}{\redlet{David himself calleth him Lord; and whence is he his son?''} And the common people heard him gladly.}
\bv{38}{And in his teaching he said, \redlet{``Beware of the scribes, who desire to walk in long robes, and \supptext{to have} salutations in the marketplaces,}}
\bv{39}{\redlet{and chief seats in the synagogues, and chief places at feasts:}}
\bv{40}{\redlet{they that devour widows' houses, and for a pretence make long prayers; these shall receive greater condemnation.''}}
\chapsec{Jesus \& the Widow's Lepta}
\bv{41}{And he sat down over against the treasury, and beheld how the multitude cast money into the treasury: and many that were rich cast in much.}
\bv{42}{And there came a poor widow, and she cast in two lepta, which make a quadrans.\mcomm{Lepton: $\frac{1}{128}$\textsuperscript{th} of a denarius (see Mark 6:37)\\Quadrans: $\frac{1}{64}$\textsuperscript{th} of a denarius}}
\bv{43}{And he called unto him his disciples, and said unto them, \redlet{``Verily I say unto you, This poor widow cast in more than all they that are casting into the treasury:}}
\bv{44}{\redlet{for they all did cast in of their superfluity; but she of her want did cast in all that she had, \supptext{even} all her living.''}}
\chaphead{Chapter XIII}
\chapdesc{The Olivet Discourse}
\lettrine[image=true, lines=4, findent=3pt, nindent=0pt]{NT/Mark/Mk-And.eps}{nd} as he went forth out of the temple, one of his disciples saith unto him, Teacher, behold, what manner of stones and what manner of buildings!
\bv{2}{And Jesus said unto him, \redlet{``Seest thou these great buildings? there shall not be left here one stone upon another, which shall not be thrown down.''}}
\bv{3}{And as he sat on the mount of Olives over against the temple, Peter and James and John and Andrew asked him privately,}
\bv{4}{``Tell us, when shall these things be? and what \supptext{shall be} the sign when these things are all about to be accomplished?''}
\chapsec{The Course of this Age}
\bv{5}{And Jesus began to say unto them, \redlet{Take heed that no man lead you astray.}}
\bv{6}{\redlet{Many shall come in my name, saying, `I am \supptext{he};' and shall lead many astray.}}
\bv{7}{\redlet{And when ye shall hear of wars and rumors of wars, be not troubled: \supptext{these things} must needs come to pass; but the end is not yet.}}
\bv{8}{\redlet{For nation shall rise against nation, and kingdom against kingdom; there shall be earthquakes in divers places; there shall be famines: these things are the beginning of travail.}}
\bv{9}{\redlet{But take ye heed to yourselves: for they shall deliver you up to councils; and in synagogues shall ye be beaten; and before governors and kings shall ye stand for my sake, for a testimony unto them.}}
\bv{10}{\redlet{And the gospel must first be preached unto all the nations.}}
\bv{11}{\redlet{And when they lead you \supptext{to judgement}, and deliver you up, be not anxious beforehand what ye shall speak: but whatsoever shall be given you in that hour, that speak ye; for it is not ye that speak, but the Holy Ghost.}}
\bv{12}{\redlet{And brother shall deliver up brother to death, and the father his child; and children shall rise up against parents, and cause them to be put to death.}}
\bv{13}{\redlet{And ye shall be hated of all men for my name's sake: but he that endureth to the end, the same shall be saved.}}
\chapsec{The Great Tribulation}
\bv{14}{\redlet{But when ye see the abomination of desolation}\mcomm{spoken of by Daniel the prophet [Dan. 9:17]} \redlet{standing where he ought not (let him that readeth understand), then let them that are in Judæa flee unto the mountains:}}
\bv{15}{\redlet{and let him that is on the housetop not go down, nor enter in, to take anything out of his house:}}
\bv{16}{\redlet{and let him that is in the field not return back to take his cloak.}}
\bv{17}{\redlet{But woe unto them that are with child and to them that give suck in those days!}}
\bv{18}{\redlet{And pray ye that it be not in the winter.}}
\bv{19}{\redlet{For those days shall be tribulation, such as there hath not been the like from the beginning of the creation which God created until now, and never shall be.}}
\bv{20}{\redlet{And except the Lord had shortened the days, no flesh would have been saved; but for the elect's sake, whom he chose, he shortened the days.}}
\bv{21}{\redlet{And then if any man shall say unto you, `Lo, here is the Christ;' or, `Lo, there;' believe \supptext{it} not:}}
\bv{22}{\redlet{for there shall arise false Christs and false prophets, and shall show signs and wonders, that they may lead astray, if possible, the elect.}}
\bv{23}{\redlet{But take ye heed: behold, I have told you all things beforehand.}}
\chapsec{The Lord's Return in Glory}
\bv{24}{\redlet{But in those days, after that tribulation, the sun shall be darkened, and the moon shall not give her light,}}
\bv{25}{\redlet{and the stars shall be falling from heaven, and the powers that are in the heavens shall be shaken.}}
\bv{26}{\redlet{And then shall they see the Son of man coming in clouds with great power and glory.}}
\bv{27}{\redlet{And then shall he send forth the angels, and shall gather together his elect from the four winds, from the uttermost part of the earth to the uttermost part of heaven.}}
\chapsec{Parable of the Fig Tree}
\bv{28}{\redlet{Now from the fig tree learn her parable: when her branch is now become tender, and putteth forth its leaves, ye know that the summer is nigh;}}
\bv{29}{\redlet{even so ye also, when ye see these things coming to pass, know ye that he is nigh, \supptext{even} at the doors.}}
\bv{30}{\redlet{Verily I say unto you, This generation shall not pass away, until all these things be accomplished.}}
\bv{31}{\redlet{Heaven and earth shall pass away: but my words shall not pass away.}}
\bv{32}{\redlet{But of that day or that hour knoweth no one, not even the angels in heaven, neither the Son, but the Father.}}
\bv{33}{\redlet{Take ye heed, watch and pray: for ye know not when the time is.}}
\chapsec{Watchfulness in View of the Lord's Return}
\bv{34}{\redlet{\supptext{It is} as \supptext{when} a man, sojourning in another country, having left his house, and given authority to his servants, to each one his work, commanded also the porter to watch.}}
\bv{35}{\redlet{Watch therefore: for ye know not when the lord of the house cometh, whether at even, or at midnight, or at cockcrowing, or in the morning;}}
\bv{36}{\redlet{lest coming suddenly he find you sleeping.}}
\bv{37}{\redlet{And what I say unto you I say unto all, Watch.}}
\chaphead{Chapter XIV}
\chapdesc{The Plot to Put Jesus to Death}
\lettrine[image=true, lines=4, findent=3pt, nindent=0pt]{NT/Mark/Mk-Now.eps}{ow} after two days was \supptext{the feast of} the passover and the unleavened bread: and the chief priests and the scribes sought how they might take him with subtlety, and kill him:
\bv{2}{for they said, ``Not during the feast, lest haply there shall be a tumult of the people.''}
\chapsec{Jesus Anointed by Mary of Bethany}
\bv{3}{And while he was in Bethany in the house of Simon the leper, as he sat at meat, there came a woman having an alabaster cruse of ointment of pure nard very costly; \supptext{and} she brake the cruse, and poured it over his head.}
\bv{4}{But there were some that had indignation among themselves, \supptext{saying}, ``To what purpose hath this waste of the ointment been made?}
\bv{5}{For this ointment might have been sold for above three hundred denarii, and given to the poor.'' And they murmured against her.}
\bv{6}{But Jesus said, \redlet{``Let her alone; why trouble ye her? she hath wrought a good work on me.}}
\bv{7}{\redlet{For ye have the poor always with you, and whensoever ye will ye can do them good: but me ye have not always.}}
\bv{8}{\redlet{She hath done what she could; she hath anointed my body beforehand for the burying.}}
\bv{9}{\redlet{And verily I say unto you, Wheresoever the gospel shall be preached throughout the whole world, that also which this woman hath done shall be spoken of for a memorial of her.''}}
\chapsec{Judas Covenants to Betray Jesus}
\bv{10}{And Judas Iscariot, he that was one of the twelve, went away unto the chief priests, that he might deliver him unto them.}
\bv{11}{And they, when they heard it, were glad, and promised to give him money. And he sought how he might conveniently deliver him \supptext{unto them}.}
\chapsec{The Preparation of the Passover}
\bv{12}{And on the first day of unleavened bread, when they sacrificed the passover, his disciples say unto him, ``Where wilt thou that we go and make ready that thou mayest eat the passover?''}
\bv{13}{And he sendeth two of his disciples, and saith unto them, \redlet{``Go into the city, and there shall meet you a man bearing a pitcher of water: follow him;}}
\bv{14}{\redlet{and wheresoever he shall enter in, say to the master of the house, `The Teacher saith, `Where is my guest-chamber, where I shall eat the passover with my disciples?'{'}}}
\bv{15}{\redlet{And he will himself show you a large upper room furnished \supptext{and} ready: and there make ready for us.''}}
\bv{16}{And the disciples went forth, and came into the city, and found as he had said unto them: and they made ready the passover.}
\chapsec{The Last Passover}
\bv{17}{And when it was evening he cometh with the twelve.}
\bv{18}{And as they sat and were eating, Jesus said, \redlet{``Verily I say unto you, One of you shall betray me, \supptext{even} he that eateth with me.''}}
\bv{19}{They began to be sorrowful, and to say unto him one by one, ``Is it I?''}
\bv{20}{And he said unto them, \redlet{``\supptext{It is} one of the twelve, he that dippeth with me in the dish.}}
\bv{21}{\redlet{For the Son of man goeth, even as it is written of him: but woe unto that man through whom the Son of man is betrayed! good were it for that man if he had not been born.''}}
\chapsec{Institution of the Eucharist}
\bv{22}{And as they were eating, he took bread, and when he had blessed, he brake it, and gave to them, and said, \redlet{``Take ye: this is my body.''}}
\bv{23}{And he took a cup, and when he had given thanks, he gave to them: and they all drank of it.}
\bv{24}{And he said unto them, \redlet{``This is my blood of the covenant,}\mcomm{the new covenant} \redlet{which is poured out for many.}}
\bv{25}{\redlet{Verily I say unto you, I shall no more drink of the fruit of the vine, until that day when I drink it new in the kingdom of God.''}}
\chapsec{St. Peter's Denial Foretold}
\bv{26}{And when they had sung a hymn, they went out unto the mount of Olives.}
\bv{27}{And Jesus saith unto them, \redlet{``All ye shall be offended:}}\mcomm{because of me tonight} \redlet{for it is written, `I will smite the shepherd, and the sheep shall be scattered abroad.'\mref{Zech. 13:7}}
\bv{28}{\redlet{Howbeit, after I am raised up, I will go before you into Galilee.''}}
\bv{29}{But Peter said unto him, ``Although all shall be offended, yet will not I.''}
\bv{30}{And Jesus saith unto him, \redlet{``Verily I say unto thee, that thou to-day, \supptext{even} this night, before the cock crow twice, shalt deny me thrice.''}}
\bv{31}{But he spake exceeding vehemently, ``If I must die with thee, I will not deny thee.'' And in like manner also said they all.}
\chapsec{The Agony in the Garden}
\bv{32}{And they come unto a place which was named Gethsemane: and he saith unto his disciples, \redlet{``Sit ye here, while I pray.''}}
\bv{33}{And he taketh with him Peter and James and John, and began to be greatly amazed, and sore troubled.}
\bv{34}{And he saith unto them, \redlet{``My soul is exceeding sorrowful even unto death: abide ye here, and watch.''}}
\chapsec{The First Prayer}
\bv{35}{And he went forward a little, and fell on the ground, and prayed that, if it were possible, the hour might pass away from him.}
\bv{36}{And he said, \redlet{``Abba, Father, all things are possible unto thee; remove this cup from me: howbeit not what I will, but what thou wilt.''}}
\bv{37}{And he cometh, and findeth them sleeping, and saith unto Peter, \redlet{``Simon, sleepest thou? couldest thou not watch one hour?}}
\bv{38}{\redlet{Watch and pray, that ye enter not into temptation: the spirit indeed is willing, but the flesh is weak.''}}
\chapsec{The Second Prayer}
\bv{39}{And again he went away, and prayed, saying the same words.}
\bv{40}{And again he came, and found them sleeping, for their eyes were very heavy; and they knew not what to answer him.}
\chapsec{The Third Prayer}
\bv{41}{And he cometh the third time, and saith unto them, \redlet{``Sleep on now, and take your rest: it is enough; the hour is come; behold, the Son of man is betrayed into the hands of sinners.}}
\bv{42}{\redlet{Arise, let us be going: behold, he that betrayeth me is at hand.''}}
\chapsec{The Betrayal \& Arrest of Jesus}
\bv{43}{And straightway, while he yet spake, cometh Judas, one of the twelve, and with him a multitude with swords and staves, from the chief priests and the scribes and the elders.}
\bv{44}{Now he that betrayed him had given them a token, saying, ``Whomsoever I shall kiss, that is he; take him, and lead him away safely.''}
\bv{45}{And when he was come, straightway he came to him, and saith, ``Rabbi;'' and kissed him.}
\bv{46}{And they laid hands on him, and took him.}
\chapsec{St. Peter Smites the Servant}
\bv{47}{But a certain one of them that stood by drew his sword, and smote the servant of the high priest, and struck off his ear.}
\bv{48}{And Jesus answered and said unto them, \redlet{``Are ye come out, as against a robber, with swords and staves to seize me?}}
\bv{49}{\redlet{I was daily with you in the temple teaching, and ye took me not: but \supptext{this is done} that the scriptures might be fulfilled.''}}
\bv{50}{And they all left him, and fled.}
\bv{51}{And a certain young man followed with him, having a linen cloth cast about him, over \supptext{his} naked \supptext{body}: and they lay hold on him;}
\bv{52}{but he left the linen cloth, and fled naked.}
\chapsec{Jesus is Brought before the Sanhedrin}
\bv{53}{And they led Jesus away to the high priest: and there come together with him all the chief priests and the elders and the scribes.}
\bv{54}{And Peter had followed him afar off, even within, into the court of the high priest; and he was sitting with the officers, and warming himself in the light \supptext{of the fire}.}
\bv{55}{Now the chief priests and the whole council sought witness against Jesus to put him to death; and found it not.}
\bv{56}{For many bare false witness against him, and their witness agreed not together.}
\bv{57}{And there stood up certain, and bare false witness against him, saying,}
\bv{58}{``We heard him say, I will destroy this temple that is made with hands, and in three days I will build another made without hands.''}
\bv{59}{And not even so did their witness agree together.}
\bv{60}{And the high priest stood up in the midst, and asked Jesus, saying, ``Answerest thou nothing? what is it which these witness against thee?''}
\bv{61}{But he held his peace, and answered nothing. Again the high priest asked him, and saith unto him, ``Art thou the Christ, the Son of the Blessed?''}
\bv{62}{And Jesus said, \redlet{``I am: and ye shall see the Son of man sitting at the right hand of Power, and coming with the clouds of heaven.''}}
\bv{63}{And the high priest rent his clothes, and saith, ``What further need have we of witnesses?}
\bv{64}{Ye have heard the blasphemy: what think ye?'' And they all condemned him to be worthy of death.}
\bv{65}{And some began to spit on him, and to cover his face, and to buffet him, and to say unto him, ``Prophesy:'' and the officers received him with blows of their hands.}
\chapsec{St. Peter Denies his Lord}
\bv{66}{And as Peter was beneath in the court, there cometh one of the maids of the high priest;}
\bv{67}{and seeing Peter warming himself, she looked upon him, and saith, ``Thou also wast with the Nazarene, \supptext{even} Jesus.''}
\bv{68}{But he denied, saying, ``I neither know, nor understand what thou sayest:'' and he went out into the porch; and the cock crew.}
\bv{69}{And the maid saw him, and began again to say to them that stood by, ``This is \supptext{one} of them.''}
\bv{70}{But he again denied it. And after a little while again they that stood by said to Peter, ``Of a truth thou art \supptext{one} of them; for thou art a Galilæan.''\mcomm{and thy speech agreeth thereto}}
\bv{71}{But he began to curse, and to swear, ``I know not this man of whom ye speak.''}
\bv{72}{And straightway the second time the cock crew. And Peter called to mind the word, how that Jesus said unto him, \redlet{``Before the cock crow twice, thou shalt deny me thrice.''} And when he thought thereon, he wept.}
\chaphead{Chapter XV}
\chapdesc{Jesus Sent before Pilate}
\lettrine[image=true, lines=4, findent=3pt, nindent=0pt]{NT/Mark/Mk-And.eps}{nd} straightway in the morning the chief priests with the elders and scribes, and the whole council, held a consultation, and bound Jesus, and carried him away, and delivered him up to Pilate.
\bv{2}{And Pilate asked him, ``Art thou the King of the Jews?'' And he answering saith unto him, \redlet{``Thou sayest.''}}
\bv{3}{And the chief priests accused him of many things.\mcomm{but he answered nothing.}}
\bv{4}{And Pilate again asked him, saying, ``Answerest thou nothing? behold how many things they accuse thee of.''}
\bv{5}{But Jesus no more answered anything; insomuch that Pilate marvelled.}
\bv{6}{Now at the feast he used to release unto them one prisoner, whom they asked of him.}
\chapsec{Not Jesus but Barabbas}
\bv{7}{And there was one called Barabbas, \supptext{lying} bound with them that had made insurrection, men who in the insurrection had committed murder.}
\bv{8}{And the multitude went up and began to ask him \supptext{to do} as he was wont to do unto them.}
\bv{9}{And Pilate answered them, saying, ``Will ye that I release unto you the King of the Jews?''}
\bv{10}{For he perceived that for envy the chief priests had delivered him up.}
\bv{11}{But the chief priests stirred up the multitude, that he should rather release Barabbas unto them.}
\bv{12}{And Pilate again answered and said unto them, ``What then shall I do unto him whom ye call the King of the Jews?''}
\bv{13}{And they cried out again, ``Crucify him.''}
\bv{14}{And Pilate said unto them, ``Why, what evil hath he done?'' But they cried out exceedingly, ``Crucify him.''}
\bv{15}{And Pilate, wishing to content the multitude, released unto them Barabbas, and delivered Jesus, when he had scourged him, to be crucified.}
\chapsec{Jesus Crowned with Thorns}
\bv{16}{And the soldiers led him away within the court, which is the Prætorium; and they call together the whole band.}
\bv{17}{And they clothe him with purple, and platting a crown of thorns, they put it on him;}
\bv{18}{and they began to salute him, ``Hail, King of the Jews!''}
\bv{19}{And they smote his head with a reed, and spat upon him, and bowing their knees worshipped him.}
\bv{20}{And when they had mocked him, they took off from him the purple, and put on him his garments. And they lead him out to crucify him.}
\bv{21}{And they compel one passing by, Simon of Cyrene, coming from the country, the father of Alexander and Rufus, to go \supptext{with them}, that he might bear his cross.}
\bv{22}{And they bring him unto the place Golgotha, which is, being interpreted, The place of a skull.}
\bv{23}{And they offered him wine mingled with myrrh: but he received it not.}
\chapsec{Jesus Crucified}
\bv{24}{And they crucify him, and part his garments among them, casting lots upon them, what each should take.}
\bv{25}{And it was the third hour, and they crucified him.}
\bv{26}{And the superscription of his accusation was written over, ``THE KING OF THE JEWS.''}
\bv{27}{And with him they crucify two robbers; one on his right hand, and one on his left.\mcomm{And the scripture was fulfilled, which saith, And he was reckoned with transgressors. [Is. 53:12]}}
\bv{29}{And they that passed by railed on him, wagging their heads, and saying, ``Ha! thou that destroyest the temple, and buildest it in three days,}
\bv{30}{save thyself, and come down from the cross.''}
\bv{31}{In like manner also the chief priests mocking \supptext{him} among themselves with the scribes said, ``He saved others; himself he cannot save.}
\bv{32}{Let the Christ, the King of Israel, now come down from the cross, that we may see and believe.'' And they that were crucified with him reproached him.}
\bv{33}{And when the sixth hour was come, there was darkness over the whole land until the ninth hour.}
\chapsec{Jesus Christ Gives up the Ghost}
\bv{34}{And at the ninth hour Jesus cried with a loud voice, \redlet{``Eloi, Eloi, lama sabachthani?''} which is, being interpreted, \redlet{``My God, my God, why hast thou forsaken me?''}\mref{cf. Ps. 22:1}}
\bv{35}{And some of them that stood by, when they heard it, said, ``Behold, he calleth Elijah.''}
\bv{36}{And one ran, and filling a sponge full of vinegar, put it on a reed, and gave him to drink, saying, ``Let be; let us see whether Elijah cometh to take him down.''}
\bv{37}{And Jesus uttered a loud voice, and gave up the ghost.}
\bv{38}{And the veil of the temple was rent in two from the top to the bottom.}
\bv{39}{And when the centurion, who stood by over against him, saw that he so gave up the ghost, he said, ``Truly this man was the Son of God.''}
\bv{40}{And there were also women beholding from afar: among whom \supptext{were} both Mary Magdalene, and Mary the mother of James the less and of Joses, and Salome;}
\bv{41}{who, when he was in Galilee, followed him, and ministered unto him; and many other women that came up with him unto Jerusalem.}
\chapsec{The Entombment}
\bv{42}{And when even was now come, because it was the Preparation, that is, the day before the sabbath,}
\bv{43}{there came Joseph of Arimathæa, a councillor of honorable estate, who also himself was looking for the kingdom of God; and he boldly went in unto Pilate, and asked for the body of Jesus.}
\bv{44}{And Pilate marvelled if he were already dead: and calling unto him the centurion, he asked him whether he had been any while dead.}
\bv{45}{And when he learned it of the centurion, he granted the corpse to Joseph.}
\bv{46}{And he bought a linen cloth, and taking him down, wound him in the linen cloth, and laid him in a tomb which had been hewn out of a rock; and he rolled a stone against the door of the tomb.}
\bv{47}{And Mary Magdalene and Mary the \supptext{mother} of Joses beheld where he was laid.}
\chaphead{Chapter XVI}
\chapdesc{The Resurrection of Jesus Christ}
\lettrine[image=true, lines=4, findent=3pt, nindent=0pt]{NT/Mark/Mk-And.eps}{nd} when the sabbath was past, Mary Magdalene, and Mary the \supptext{mother} of James, and Salome, bought spices, that they might come and anoint him.
\bv{2}{And very early on the first day of the week, they come to the tomb when the sun was risen.}
\bv{3}{And they were saying among themselves, ``Who shall roll us away the stone from the door of the tomb?''}
\bv{4}{and looking up, they see that the stone is rolled back: for it was exceeding great.}
\bv{5}{And entering into the tomb, they saw a young man sitting on the right side, arrayed in a white robe; and they were amazed.}
\bv{6}{And he saith unto them, ``Be not amazed: ye seek Jesus, the Nazarene, who hath been crucified: he is risen; he is not here: behold, the place where they laid him!}
\bv{7}{But go, tell his disciples and Peter, He goeth before you into Galilee: there shall ye see him, as he said unto you.''}
\bv{8}{And they went out, and fled from the tomb; for trembling and astonishment had come upon them: and they said nothing to any one; for they were afraid.}
\begin{center}
	{\scshape [Here Endeth the Gospel of Mark]}
\end{center}
\clearpage
\chapter{Traditional Additions to Mark}
\chapdesc{Due to the short and abrupt ending of St. Mark's Gospel, these different endings have been added over time.}
\textit{The Shorter Ending}\\
They told all that had been commanded them briefly to those around Peter. After that, Jesus himself sent them out, from east to west, with the sacred and imperishable proclamation of eternal salvation.
\par
\noindent
\textit{The ``Freer Logion'' Ending}\\
And they made excuse, saying, ``This age of lawlessness and unbelief is under Satan, who by the unclean spirits does not allow men power to comprehend the truth of God. For this reason reveal thy righteousness now,'' they said to Christ.
\par
And Christ replied to them, ``The limit of the years of the power of Satan has been fulfilled, but other terrible things are near at hand. And I was delivered unto death on behalf of those who sinned, in order that they may return to the truth and sin no more, to the end that they may inherit the spiritual and incorruptible glory of righteousness (which glory is) in heaven. But go ye into all the world.''\par
\noindent
\textit{The Longer Ending}\\
{Now when he was risen early on the first day of the week, he appeared first to Mary Magdalene, from whom he had cast out seven demons.}
{She went and told them that had been with him, as they mourned and wept.}
{And they, when they heard that he was alive, and had been seen of her, disbelieved.}
{And after these things he was manifested in another form unto two of them, as they walked, on their way into the country.}
{And they went away and told it unto the rest: neither believed they them.}
\par
{And afterward he was manifested unto the eleven themselves as they sat at meat; and he upbraided them with their unbelief and hardness of heart, because they believed not them that had seen him after he was risen.}
{And he said unto them, ``Go ye into all the world, and preach the gospel to the whole creation.}
{He that believeth and is baptised shall be saved; but he that disbelieveth shall be condemned.}
{And these signs shall accompany them that believe: in my name shall they cast out demons; they shall speak with new tongues;}
{they shall take up serpents, and if they drink any deadly thing, it shall in no wise hurt them; they shall lay hands on the sick, and they shall recover.''}
\par
{So then the Lord Jesus, after he had spoken unto them, was received up into heaven, and sat down at the right hand of God.}
{And they went forth, and preached everywhere, the Lord working with them, and confirming the word by the signs that followed. Amen.}
	\clearpage
	\chapter{The Holy Gospel of Jesus Christ according to Saint Luke}
\fancyhead[RE,LO]{The Gospel according to Luke}
\chaphead{Chapter I}
\chapdesc{Introduction}
\lettrine[image=true, lines=4, findent=3pt, nindent=0pt]{NT/Luke/Lk-F.eps}{orasmuch} as many have taken in hand to draw up a narrative concerning those matters which have been fulfilled among us,
\bv{2}{even as they delivered them unto us, who from the beginning were eyewitnesses and ministers of the word,}
\bv{3}{it seemed good to me also, having traced the course of all things accurately from the first, to write unto thee in order, most excellent Theophilus;}
\bv{4}{that thou mightest know the certainty concerning the things wherein thou wast instructed.}
\chapsec{Birth of St. John the Baptist Foretold}
\bv{5}{There was in the days of Herod, king of Judæa, a certain priest named Zacharias, of the course of Abijah: and he had a wife of the daughters of Aaron, and her name was Elisabeth.}
\bv{6}{And they were both righteous before God, walking in all the commandments and ordinances of the Lord blameless.}
\bv{7}{And they had no child, because that Elisabeth was barren, and they both were \supptext{now} well stricken in years.}
\par
\bv{8}{Now it came to pass, while he executed the priest's office before God in the order of his course,}
\bv{9}{according to the custom of the priest's office, his lot was to enter into the temple of the Lord and burn incense.}
\bv{10}{And the whole multitude of the people were praying without at the hour of incense.}
\bv{11}{And there appeared unto him an angel of the Lord standing on the right side of the altar of incense.}
\bv{12}{And Zacharias was troubled when he saw \supptext{him}, and fear fell upon him.}
\bv{13}{But the angel said unto him, ``Fear not, Zacharias: because thy supplication is heard, and thy wife Elisabeth shall bear thee a son, and thou shalt call his name `John.'}
\bv{14}{And thou shalt have joy and gladness; and many shall rejoice at his birth.}
\bv{15}{For he shall be great in the sight of the Lord, and he shall drink no wine nor strong drink; and he shall be filled with the Holy Ghost, even from his mother's womb.}
\bv{16}{And many of the children of Israel shall he turn unto the Lord their God.}
\bv{17}{And he shall go before his face in the spirit and power of Elijah, to turn the hearts of the fathers to the children, and the disobedient \supptext{to walk} in the wisdom of the just; to make ready for the Lord a people prepared \supptext{for him}.''}
\bv{18}{And Zacharias said unto the angel, ``Whereby shall I know this? for I am an old man, and my wife well stricken in years.''}
\bv{19}{And the angel answering said unto him, ``I am Gabriel, that stand in the presence of God; and I was sent to speak unto thee, and to bring thee these good tidings.}
\bv{20}{And behold, thou shalt be silent and not able to speak, until the day that these things shall come to pass, because thou believedst not my words, which shall be fulfilled in their season.''}
\bv{21}{And the people were waiting for Zacharias, and they marvelled while he tarried in the temple.}
\bv{22}{And when he came out, he could not speak unto them: and they perceived that he had seen a vision in the temple: and he continued making signs unto them, and remained dumb.}
\bv{23}{And it came to pass, when the days of his ministration were fulfilled, he departed unto his house.}
\par
\bv{24}{And after these days Elisabeth his wife conceived; and she hid herself five months, saying,}
\bv{25}{``Thus hath the Lord done unto me in the days wherein he looked upon \supptext{me}, to take away my reproach among men.''}
\chapsec{The Annunciation}
\bv{26}{Now in the sixth month the angel Gabriel was sent from God unto a city of Galilee, named Nazareth,}
\bv{27}{to a virgin betrothed to a man whose name was Joseph, of the house of David; and the virgin's name was Mary.}
\bv{28}{And he came in unto her, and said, ``Hail, thou that art highly favoured, the Lord \supptext{is} with thee.''}
\bv{29}{But she was greatly troubled at the saying, and cast in her mind what manner of salutation this might be.}
\bv{30}{And the angel said unto her, ``Fear not, Mary: for thou hast found favour with God.}
\bv{31}{And behold, thou shalt conceive in thy womb, and bring forth a son, and shalt call his name {\scshape Jesus}.}
\bv{32}{He shall be great, and shall be called the Son of the Most High: and the Lord God shall give unto him the throne of his father David:}
\bv{33}{and he shall reign over the house of Jacob for ever; and of his kingdom there shall be no end.''}
\bv{34}{And Mary said unto the angel, ``How shall this be, seeing I know not a man?''}
\bv{35}{And the angel answered and said unto her, ``The Holy Ghost shall come upon thee, and the power of the Most High shall overshadow thee: wherefore also the holy thing which is begotten shall be called the Son of God.}
\bv{36}{And behold, Elisabeth thy kinswoman, she also hath conceived a son in her old age; and this is the sixth month with her that was called barren.}
\bv{37}{For no word from God shall be void of power.''}
\bv{38}{And Mary said, ``Behold, the handmaid of the Lord; be it unto me according to thy word. And the angel departed from her.''}
\chapsec{St. Mary Visits St. Elisabeth}
\bv{39}{And Mary arose in these days and went into the hill country with haste, into a city of Judah;}
\bv{40}{and entered into the house of Zacharias and saluted Elisabeth.}
\bv{41}{And it came to pass, when Elisabeth heard the salutation of Mary, the babe leaped in her womb; and Elisabeth was filled with the Holy Ghost;}
\bv{42}{and she lifted up her voice with a loud cry, and said, ``Blessed \supptext{art} thou among women, and blessed \supptext{is} the fruit of thy womb.}
\bv{43}{And whence is this to me, that the mother of my Lord should come unto me?}
\bv{44}{For behold, when the voice of thy salutation came into mine ears, the babe leaped in my womb for joy.}
\bv{45}{And blessed \supptext{is} she that believed; for there shall be a fulfilment of the things which have been spoken to her from the Lord.''}
\chapsec{The Magnificat}
\bv{46}{And Mary said,}
\canticle{My soul doth magnify the Lord,
\bv{47}{And my spirit hath rejoiced in God my Saviour.}\\
\bv{48}{For he hath looked upon the low estate of his handmaid:\\
For behold, from henceforth all generations shall call me blessed.}\\
\bv{49}{For he that is mighty hath done to me great things;
And holy is his name.}\\
\bv{50}{And his mercy is unto generations and generations
On them that fear him.}\\
\bv{51}{He hath showed strength with his arm;\\
He hath scattered the proud in the imagination of their heart.}\\
\bv{52}{He hath put down princes from} \supptext{their} thrones,\\
And hath exalted them of low degree.\\
\bv{53}{The hungry he hath filled with good things;\\
And the rich he hath sent empty away.}\\
\bv{54}{He hath given help to Israel his servant,\\
That he might remember mercy}\\
\bv{55}{(As he spake unto our fathers)\\
Toward Abraham and his seed for ever.\mcomm{Glory be to the Father and to the Son and to the Holy Ghost.}}}
\bv{56}{And Mary abode with her about three months, and returned unto her house.}
\chapsec{Birth of St. John the Baptist}
\bv{57}{Now Elisabeth's time was fulfilled that she should be delivered; and she brought forth a son.}
\bv{58}{And her neighbours and her kinsfolk heard that the Lord had magnified his mercy towards her; and they rejoiced with her.}
\bv{59}{And it came to pass on the eighth day, that they came to circumcise the child; and they would have called him Zacharias, after the name of his father.}
\bv{60}{And his mother answered and said, ``Not so; but he shall be called John.''}
\bv{61}{And they said unto her, ``There is none of thy kindred that is called by this name.''}
\bv{62}{And they made signs to his father, what he would have him called.}
\bv{63}{And he asked for a writing tablet, and wrote, saying, ``His name is John.'' And they marvelled all.}
\bv{64}{And his mouth was opened immediately, and his tongue \supptext{loosed}, and he spake, blessing God.}
\bv{65}{And fear came on all that dwelt round about them: and all these sayings were noised abroad throughout all the hill country of Judæa.}
\bv{66}{And all that heard them laid them up in their heart, saying, ``What then shall this child be?'' For the hand of the Lord was with him.}
\chapsec{The Benedictus}
\bv{67}{And his father Zacharias was filled with the Holy Ghost, and prophesied, saying,}
\canticle{\bv{68}{Blessed \supptext{be} the Lord, the God of Israel; For he hath visited and wrought redemption for his people,}\\
\bv{69}{And hath raised up a horn of salvation for us
In the house of his servant David}\\
\bv{70}{(As he spake by the mouth of his holy prophets that have been from of old),}\\
\bv{71}{Salvation from our enemies, and from the hand of all that hate us;}\\
\bv{72}{To show mercy towards our fathers,
And to remember his holy covenant;}\\
\bv{73}{The oath which he sware unto Abraham our father,}\\
\bv{74}{To grant unto us that we being delivered out of the hand of our enemies
Should serve him without fear,}\\
\bv{75}{In holiness and righteousness before him all our days.}\\
\bv{76}{Yea and thou, child, shalt be called the prophet of the Most High:
For thou shalt go before the face of the Lord to make ready his ways;}\\
\bv{77}{To give knowledge of salvation unto his people
In the remission of their sins,}\\
\bv{78}{Because of the tender mercy of our God,
Whereby the dayspring from on high shall visit us,}\\
\bv{79}{To shine upon them that sit in darkness and the shadow of death;\\
To guide our feet into the way of peace.\mcomm{Glory be to the Father and to the Son and to the Holy Ghost.}}}
\bv{80}{And the child grew, and waxed strong in spirit, and was in the deserts till the day of his showing unto Israel.}
\chaphead{Chapter II}
\chapdesc{The Birth of Jesus}
\lettrine[image=true, lines=4, findent=3pt, nindent=0pt]{NT/Luke/Lk-N.eps}{ow} it came to pass in those days, there went out a decree from Cæsar Augustus, that all the world should be enrolled.
\bv{2}{This was the first enrolment made when Quirinius was governor of Syria.}
\bv{3}{And all went to enrol themselves, every one to his own city.}
\bv{4}{And Joseph also went up from Galilee, out of the city of Nazareth, into Judæa, to the city of David, which is called Bethlehem, because he was of the house and family of David;}
\bv{5}{to enrol himself with Mary, who was betrothed to him, being great with child.}
\bv{6}{And it came to pass, while they were there, the days were fulfilled that she should be delivered.}
\bv{7}{And she brought forth her firstborn son; and she wrapped him in swaddling clothes, and laid him in a manger, because there was no room for them in the inn.}
\chapsec{Adoration of the Shepherds}
\bv{8}{And there were shepherds in the same country abiding in the field, and keeping watch by night over their flock.}
\bv{9}{And an angel of the Lord stood by them, and the glory of the Lord shone round about them: and they were sore afraid.}
\bv{10}{And the angel said unto them, Be not afraid; for behold, I bring you good tidings of great joy which shall be to all the people:}
\bv{11}{for there is born to you this day in the city of David a Saviour, who is Christ the Lord.}
\bv{12}{And this \supptext{is} the sign unto you: Ye shall find a babe wrapped in swaddling clothes, and lying in a manger.}
\bv{13}{And suddenly there was with the angel a multitude of the heavenly host praising God, and saying,}
\canticle{\bv{14}{Glory to God in the highest,\\
And on earth peace among men in whom he is well pleased.}}
\bv{15}{And it came to pass, when the angels went away from them into heaven, the shepherds said one to another, Let us now go even unto Bethlehem, and see this thing that is come to pass, which the Lord hath made known unto us.}
\bv{16}{And they came with haste, and found both Mary and Joseph, and the babe lying in the manger.}
\bv{17}{And when they saw it, they made known concerning the saying which was spoken to them about this child.}
\bv{18}{And all that heard it wondered at the things which were spoken unto them by the shepherds.}
\bv{19}{But Mary kept all these sayings, pondering them in her heart.}
\bv{20}{And the shepherds returned, glorifying and praising God for all the things that they had heard and seen, even as it was spoken unto them.}
\chapsec{Circumcision of Jesus}
\bv{21}{And when eight days were fulfilled for circumcising him, his name was called JESUS, which was so called by the angel before he was conceived in the womb.}
\bv{22}{And when the days of their purification according to the law of Moses were fulfilled, they brought him up to Jerusalem, to present him to the Lord}
\bv{23}{(as it is written in the law of the Lord, Every male that openeth the womb shall be called holy to the Lord),}
\bv{24}{and to offer a sacrifice according to that which is said in the law of the Lord, A pair of turtledoves, or two young pigeons.}
\chapsec{Adoration \& Prophecy of Simeon}
\bv{25}{And behold, there was a man in Jerusalem, whose name was Simeon; and this man was righteous and devout, looking for the consolation of Israel: and the Holy Ghost was upon him.}
\bv{26}{And it had been revealed unto him by the Holy Ghost, that he should not see death, before he had seen the Lord's Christ.}
\bv{27}{And he came in the Spirit into the temple: and when the parents brought in the child Jesus, that they might do concerning him after the custom of the law,}
\bv{28}{then he received him into his arms, and blessed God, and said,}
\canticle{\bv{29}{Now lettest thou thy servant depart, Lord,
According to thy word, in peace;}\\
\bv{30}{For mine eyes have seen thy salvation,}\\
\bv{31}{Which thou hast prepared before the face of all peoples;}\\
\bv{32}{A light for revelation to the Gentiles,
And the glory of thy people Israel.\mcomm{Glory be to the Father and to the Son and to the Holy Ghost.}}}
\bv{33}{And his father and his mother were marvelling at the things which were spoken concerning him;}
\bv{34}{and Simeon blessed them, and said unto Mary his mother, Behold, this \supptext{child} is set for the falling and the rising of many in Israel; and for a sign which is spoken against;}
\bv{35}{yea and a sword shall pierce through thine own soul; that thoughts out of many hearts may be revealed.}
\chapsec{Adoration of Anna}
\bv{36}{And there was one Anna, a prophetess, the daughter of Phanuel, of the tribe of Asher (she was of a great age, having lived with a husband seven years from her virginity,}
\bv{37}{and she had been a widow even unto fourscore and four years), who departed not from the temple, worshipping with fastings and supplications night and day.}
\bv{38}{And coming up at that very hour she gave thanks unto God, and spake of him to all them that were looking for the redemption of Jerusalem.}
\chapsec{Return to Nazareth: the Silent Years}
\bv{39}{And when they had accomplished all things that were according to the law of the Lord, they returned into Galilee, to their own city Nazareth.}
\bv{40}{And the child grew, and waxed strong, filled with wisdom: and the grace of God was upon him.}
\chapsec{Jesus \& his Parents at Passover}
\bv{41}{And his parents went every year to Jerusalem at the feast of the passover.}
\bv{42}{And when he was twelve years old, they went up after the custom of the feast;}
\bv{43}{and when they had fulfilled the days, as they were returning, the boy Jesus tarried behind in Jerusalem; and his parents knew it not;}
\bv{44}{but supposing him to be in the company, they went a day's journey; and they sought for him among their kinsfolk and acquaintance:}
\bv{45}{and when they found him not, they returned to Jerusalem, seeking for him.}
\bv{46}{And it came to pass, after three days they found him in the temple, sitting in the midst of the teachers, both hearing them, and asking them questions:}
\bv{47}{and all that heard him were amazed at his understanding and his answers.}
\bv{48}{And when they saw him, they were astonished; and his mother said unto him, Son, why hast thou thus dealt with us? behold, thy father and I sought thee sorrowing.}
\bv{49}{And he said unto them, How is it that ye sought me? knew ye not that I must be in my Father's house?}
\bv{50}{And they understood not the saying which he spake unto them.}
\bv{51}{And he went down with them, and came to Nazareth; and he was subject unto them: and his mother kept all \supptext{these} sayings in her heart.}
\bv{52}{And Jesus advanced in wisdom and stature, and in favour with God and men.}
\chaphead{Chapter III}
\chapdesc{The Ministry of St. John the Baptist}
\lettrine[image=true, lines=4, findent=3pt, nindent=0pt]{NT/Luke/Lk-N.eps}{ow} in the fifteenth year of the reign of Tiberius Cæsar, Pontius Pilate being governor of Judæa, and Herod being tetrarch of Galilee, and his brother Philip tetrarch of the region of Ituræa and Trachonitis, and Lysanias tetrarch of Abilene,
\bv{2}{in the high-priesthood of Annas and Caiaphas, the word of God came unto John the son of Zacharias in the wilderness.}
\bv{3}{And he came into all the region round about the Jordan, preaching the baptism of repentance unto remission of sins;}
\bv{4}{as it is written in the book of the words of Isaiah the prophet,}
\otQuote{Is. 40:3-5}{The voice of one crying in the wilderness,
Make ye ready the way of the Lord,
Make his paths straight.}
\otQuote{Is. 57:14}{\bv{5}{Every valley shall be filled,
And every mountain and hill shall be brought low;
And the crooked shall become straight,
And the rough ways smooth;}}
\otQuote{Is. 52:10}{\bv{6}{And all flesh shall see the salvation of God.}}
\bv{7}{He said therefore to the multitudes that went out to be baptised of him, ``Ye offspring of vipers, who warned you to flee from the wrath to come?}
\bv{8}{Bring forth therefore fruits worthy of repentance, and begin not to say within yourselves, `We have Abraham to our father:' for I say unto you, that God is able of these stones to raise up children unto Abraham.}
\bv{9}{And even now the axe also lieth at the root of the trees: every tree therefore that bringeth not forth good fruit is hewn down, and cast into the fire.''}
\bv{10}{And the multitudes asked him, saying, ``What then must we do?''}
\bv{11}{And he answered and said unto them, ``He that hath two coats, let him impart to him that hath none; and he that hath food, let him do likewise.''}
\bv{12}{And there came also publicans to be baptised, and they said unto him, ``Teacher, what must we do?''}
\bv{13}{And he said unto them, ``Extort no more than that which is appointed you.''}
\bv{14}{And soldiers also asked him, saying, ``And we, what must we do?'' And he said unto them, ``Extort from no man by violence, neither accuse \supptext{any one} wrongfully; and be content with your wages.''}
\bv{15}{And as the people were in expectation, and all men reasoned in their hearts concerning John, whether haply he were the Christ;}
\bv{16}{John answered, saying unto them all, ``I indeed baptise you with water; but there cometh he that is mightier than I, the latchet of whose shoes I am not worthy to unloose: he shall baptise you in the Holy Ghost and \supptext{in} fire:}
\bv{17}{whose fan is in his hand, thoroughly to cleanse his threshing-floor, and to gather the wheat into his garner; but the chaff he will burn up with unquenchable fire.''}
\bv{18}{With many other exhortations therefore preached he good tidings unto the people;}
\bv{19}{but Herod the tetrarch, being reproved by him for Herodias his brother's wife, and for all the evil things which Herod had done,}
\bv{20}{added this also to them all, that he shut up John in prison.}
\chapsec{The Baptism of Jesus}
\bv{21}{Now it came to pass, when all the people were baptised, that, Jesus also having been baptised, and praying, the heaven was opened,}
\bv{22}{and the Holy Ghost descended in a bodily form, as a dove, upon him, and a voice came out of heaven, \god{``Thou art my beloved Son; in thee I am well pleased.''}}
\chapsec{The Genealogy of St. Mary}
\bv{23}{And Jesus himself, when he began \supptext{to teach}, was about thirty years of age, being the son (as was supposed) of Joseph, the \supptext{son} of Heli,\mcomm{St. Joseph, espoused to St. Mary, was the son-in-law of Heli.}}
\bv{24}{the
\supptext{son} of Matthat, the
\supptext{son} of Levi, the
\supptext{son} of Melchi, the
\supptext{son} of Jannai, the
\supptext{son} of Joseph,}
\par
\bv{25}{the
\supptext{son} of Mattathias, the
\supptext{son} of Amos, the
\supptext{son} of Nahum, the
\supptext{son} of Esli, the
\supptext{son} of Naggai,}
\par
\bv{26}{the
\supptext{son} of Maath, the
\supptext{son} of Mattathias, the
\supptext{son} of Semein, the
\supptext{son} of Josech, the
\supptext{son} of Joda,}
\par
\bv{27}{the
\supptext{son} of Joanan, the
\supptext{son} of Rhesa, the
\supptext{son} of Zerubbabel, the
\supptext{son} of Shealtiel, the
\supptext{son} of Neri,}
\par
\bv{28}{the
\supptext{son} of Melchi, the
\supptext{son} of Addi, the
\supptext{son} of Cosam, the
\supptext{son} of Elmadam, the
\supptext{son} of Er,}
\par
\bv{29}{the
\supptext{son} of Jesus, the
\supptext{son} of Eliezer, the
\supptext{son} of Jorim, the
\supptext{son} of Matthat, the
\supptext{son} of Levi,}
\par
\bv{30}{the
\supptext{son} of Symeon, the
\supptext{son} of Judas, the
\supptext{son} of Joseph, the
\supptext{son} of Jonam, the
\supptext{son} of Eliakim,}
\par
\bv{31}{the
\supptext{son} of Melea, the
\supptext{son} of Menna, the
\supptext{son} of Mattatha, the
\supptext{son} of Nathan, the
\supptext{son} of David,}
\par
\bv{32}{the
\supptext{son} of Jesse, the
\supptext{son} of Obed, the
\supptext{son} of Boaz, the
\supptext{son} of Salmon, the
\supptext{son} of Nahshon,}
\par
\bv{33}{the
\supptext{son} of Amminadab, the
\supptext{son} of Arni, the
\supptext{son} of Hezron, the
\supptext{son} of Perez, the
\supptext{son} of Judah,}
\par
\bv{34}{the
\supptext{son} of Jacob, the
\supptext{son} of Isaac, the
\supptext{son} of Abraham, the
\supptext{son} of Terah, the
\supptext{son} of Nahor,}
\par
\bv{35}{the
\supptext{son} of Serug, the
\supptext{son} of Reu, the
\supptext{son} of Peleg, the
\supptext{son} of Eber, the
\supptext{son} of Shelah,}
\par
\bv{36}{the
\supptext{son} of Cainan, the
\supptext{son} of Arphaxad, the
\supptext{son} of Shem, the
\supptext{son} of Noah, the
\supptext{son} of Lamech,}
\par
\bv{37}{the
\supptext{son} of Methuselah, the
\supptext{son} of Enoch, the
\supptext{son} of Jared, the
\supptext{son} of Mahalaleel, the
\supptext{son} of Cainan,}
\par
\bv{38}{the
\supptext{son} of Enos, the
\supptext{son} of Seth, the
\supptext{son} of Adam, the
\supptext{son} of God.}
\chaphead{Chapter IV}
\chapdesc{The Temptation of Christ}
\lettrine[image=true, lines=4, findent=3pt, nindent=0pt]{Lk-A.eps}{nd} Jesus, full of the Holy Ghost, returned from the Jordan, and was led in the Spirit in the wilderness
\bv{2}{during forty days, being tempted of the devil. And he did eat nothing in those days: and when they were completed, he hungered.}
\bv{3}{And the devil said unto him, ``If thou art the Son of God, command this stone that it become bread.''}
\bv{4}{And Jesus answered unto him,} \redlet{``It is written,}
\otQuote{Deut. 8:3}{\redlet{Man shall not live by bread alone.''}}
\bv{5}{And he led him up, and showed him all the kingdoms of the world in a moment of time.}
\bv{6}{And the devil said unto him, ``To thee will I give all this authority, and the glory of them: for it hath been delivered unto me; and to whomsoever I will I give it.}
\bv{7}{If thou therefore wilt worship before me, it shall all be thine.''}
\bv{8}{And Jesus answered and said unto him,} \redlet{``It is written,}
\otQuote{Deut. 6:13}{\redlet{Thou shalt worship the Lord thy God, and him only shalt thou serve.''}}
\bv{9}{And he led him to Jerusalem, and set him on the pinnacle of the temple, and said unto him, ``If thou art the Son of God, cast thyself down from hence:}
\bv{10}{for it is written,}
\otQuote{Ps. 91:11-2}{He shall give his angels charge concerning thee, to guard thee:}
\bv{11}{and,}
\otQuote{Ps. 91:11-2}{On their hands they shall bear thee up,
Lest haply thou dash thy foot against a stone.''}
\bv{12}{And Jesus answering said unto him,} \redlet{``It is said,}
\otQuote{Deut. 6:16}{\redlet{Thou shalt not make trial of the Lord thy God.''}}
\bv{13}{And when the devil had completed every temptation, he departed from him for a season.}
\chapsec{Jesus Returns to Galilee}
\bv{14}{And Jesus returned in the power of the Spirit into Galilee: and a fame went out concerning him through all the region round about.}
\bv{15}{And he taught in their synagogues, being glorified of all.}
\chapsec{Jesus in the Synagogue at Nazareth}
\bv{16}{And he came to Nazareth, where he had been brought up: and he entered, as his custom was, into the synagogue on the sabbath day, and stood up to read.}
\bv{17}{And there was delivered unto him the book of the prophet Isaiah. And he opened the book, and found the place where it was written,}
\otQuote{Is. 61:1-2}{\bv{18}{The Spirit of the Lord is upon me,
Because he anointed me to preach good tidings to the poor:
He hath sent me to proclaim release to the captives,
And recovering of sight to the blind,
To set at liberty them that are bruised,}
\bv{19}{To proclaim the acceptable year of the Lord.}}
\bv{20}{And he closed the book, and gave it back to the attendant, and sat down: and the eyes of all in the synagogue were fastened on him.}
\bv{21}{And he began to say unto them, \redlet{``To-day hath this scripture been fulfilled in your ears.''}}
\bv{22}{And all bare him witness, and wondered at the words of grace which proceeded out of his mouth: and they said, ``Is not this Joseph's son?''}
\bv{23}{And he said unto them, \redlet{``Doubtless ye will say unto me this parable, `Physician, heal thyself:' whatsoever we have heard done at Capernaum, do also here in thine own country.''}}
\bv{24}{And he said, \redlet{``Verily I say unto you, No prophet is acceptable in his own country.}}
\bv{25}{\redlet{But of a truth I say unto you, There were many widows in Israel in the days of Elijah, when the heaven was shut up three years and six months, when there came a great famine over all the land;}}
\bv{26}{\redlet{and unto none of them was Elijah sent, but only to Zarephath, in the land of Sidon, unto a woman that was a widow.}}
\bv{27}{\redlet{And there were many lepers in Israel in the time of Elisha the prophet; and none of them was cleansed, but only Naaman the Syrian.''}}
\bv{28}{And they were all filled with wrath in the synagogue, as they heard these things;}
\bv{29}{and they rose up, and cast him forth out of the city, and led him unto the brow of the hill whereon their city was built, that they might throw him down headlong.}
\bv{30}{But he passing through the midst of them went his way.}
\chapsec{Jesus Goes to Capernaum}
\bv{31}{And he came down to Capernaum, a city of Galilee. And he was teaching them on the sabbath day:}
\bv{32}{and they were astonished at his teaching; for his word was with authority.}
\bv{33}{And in the synagogue there was a man, that had a spirit of an unclean demon; and he cried out with a loud voice,}
\bv{34}{``Ah! what have we to do with thee, Jesus thou Nazarene? art thou come to destroy us? I know thee who thou art, the Holy One of God.''}
\bv{35}{And Jesus rebuked him, saying, \redlet{``Hold thy peace, and come out of him.''} And when the demon had thrown him down in the midst, he came out of him, having done him no hurt.}
\bv{36}{And amazement came upon all, and they spake together, one with another, saying, ``What is this word? for with authority and power he commandeth the unclean spirits, and they come out.''}
\bv{37}{And there went forth a rumor concerning him into every place of the region round about.}
\chapsec{Jesus Heals St. Peter's Mother-in-law}
\bv{38}{And he rose up from the synagogue, and entered into the house of Simon. And Simon's wife's mother was holden with a great fever; and they besought him for her.}
\bv{39}{And he stood over her, and rebuked the fever; and it left her: and immediately she rose up and ministered unto them.}
\bv{40}{And when the sun was setting, all they that had any sick with divers diseases brought them unto him; and he laid his hands on every one of them, and healed them.}
\bv{41}{And demons also came out from many, crying out, and saying, ``Thou art the Son of God.'' And rebuking them, he suffered them not to speak, because they knew that he was the Christ.}
\bv{42}{And when it was day, he came out and went into a desert place: and the multitudes sought after him, and came unto him, and would have stayed him, that he should not go from them.}
\bv{43}{But he said unto them, \redlet{``I must preach the good tidings of the kingdom of God to the other cities also: for therefore was I sent.''}}
\bv{44}{And he was preaching in the synagogues of Galilee.}
\chaphead{Chapter V}
\chapdesc{Miraculous Catch of Fish}
\lettrine[image=true, lines=4, findent=3pt, nindent=0pt]{NT/Luke/Lk-N.eps}{ow} it came to pass, while the multitude pressed upon him and heard the word of God, that he was standing by the lake of Gennesaret;
\bv{2}{and he saw two boats standing by the lake: but the fishermen had gone out of them, and were washing their nets.}
\bv{3}{And he entered into one of the boats, which was Simon's, and asked him to put out a little from the land. And he sat down and taught the multitudes out of the boat.}
\bv{4}{And when he had left speaking, he said unto Simon, \redlet{``Put out into the deep, and let down your nets for a draught.''}}
\bv{5}{And Simon answered and said, ``Master, we toiled all night, and took nothing: but at thy word I will let down the nets.''}
\bv{6}{And when they had done this, they inclosed a great multitude of fishes; and their nets were breaking;}
\bv{7}{and they beckoned unto their partners in the other boat, that they should come and help them. And they came, and filled both the boats, so that they began to sink.}
\bv{8}{But Simon Peter, when he saw it, fell down at Jesus' knees, saying, ``Depart from me; for I am a sinful man, O Lord.''}
\bv{9}{For he was amazed, and all that were with him, at the draught of the fishes which they had taken;}
\bv{10}{and so were also James and John, sons of Zebedee, who were partners with Simon. And Jesus said unto Simon, \redlet{``Fear not; from henceforth thou shalt catch men.''}}
\bv{11}{And when they had brought their boats to land, they left all, and followed him.}
\chapsec{Jesus Heals a Leper}
\bv{12}{And it came to pass, while he was in one of the cities, behold, a man full of leprosy: and when he saw Jesus, he fell on his face, and besought him, saying, ``Lord, if thou wilt, thou canst make me clean.''}
\bv{13}{And he stretched forth his hand, and touched him, saying, \redlet{``I will; be thou made clean.''} And straightway the leprosy departed from him.}
\bv{14}{And he charged him to tell no man: but go thy way, and show thyself to the priest, and offer for thy cleansing, according as Moses commanded, for a testimony unto them.}
\bv{15}{But so much the more went abroad the report concerning him: and great multitudes came together to hear, and to be healed of their infirmities.}
\bv{16}{But he withdrew himself in the deserts, and prayed.}
\chapsec{A Paralytic Healed}
\bv{17}{And it came to pass on one of those days, that he was teaching; and there were Pharisees and doctors of the law sitting by, who were come out of every village of Galilee and Judæa and Jerusalem: and the power of the Lord was with him to heal.}
\bv{18}{And behold, men bring on a bed a man that was paralysed: and they sought to bring him in, and to lay him before him.}
\bv{19}{And not finding by what \supptext{way} they might bring him in because of the multitude, they went up to the housetop, and let him down through the tiles with his couch into the midst before Jesus.}
\par
\bv{20}{And seeing their faith, he said, \redlet{``Man, thy sins are forgiven thee.''}}
\bv{21}{And the scribes and the Pharisees began to reason, saying, ``Who is this that speaketh blasphemies? Who can forgive sins, but God alone?''}
\bv{22}{But Jesus perceiving their reasonings, answered and said unto them, \redlet{``Why reason ye in your hearts?}}
\bv{23}{\redlet{Which is easier, to say, `Thy sins are forgiven thee;' or to say, `Arise and walk?'}}
\bv{24}{\redlet{But that ye may know that the Son of man hath authority on earth to forgive sins (he said unto him that was paralysed), I say unto thee, Arise, and take up thy couch, and go unto thy house.''}}
\bv{25}{And immediately he rose up before them, and took up that whereon he lay, and departed to his house, glorifying God.}
\bv{26}{And amazement took hold on all, and they glorified God; and they were filled with fear, saying, ``We have seen strange things to-day.''}
\chapsec{The Call of St. Matthew (Levi)}
\bv{27}{And after these things he went forth, and beheld a publican, named Levi, sitting at the place of toll, and said unto him, \redlet{``Follow me.''}}
\bv{28}{And he forsook all, and rose up and followed him.}
\bv{29}{And Levi made him a great feast in his house: and there was a great multitude of publicans and of others that were sitting at meat with them.}
\chapsec{Jesus Answers the Scribes \& Pharisees}
\bv{30}{And the Pharisees and their scribes murmured against his disciples, saying, ``Why do ye eat and drink with the publicans and sinners?''}
\bv{31}{And Jesus answering said unto them, \redlet{``They that are in health have no need of a physician; but they that are sick.}}
\bv{32}{\redlet{I am not come to call the righteous but sinners to repentance.''}}
\bv{33}{And they said unto him, ``The disciples of John fast often, and make supplications; likewise also the \supptext{disciples} of the Pharisees; but thine eat and drink.''}
\bv{34}{And Jesus said unto them, \redlet{``Can ye make the sons of the bride-chamber fast, while the bridegroom is with them?}}
\bv{35}{\redlet{But the days will come; and when the bridegroom shall be taken away from them, then will they fast in those days.''}}
\chapsec{Parables of the Garment \& Bottles}
\bv{36}{And he spake also a parable unto them: \redlet{``No man rendeth a piece from a new garment and putteth it upon an old garment; else he will rend the new, and also the piece from the new will not agree with the old.}}
\bv{37}{\redlet{And no man putteth new wine into old wine-skins; else the new wine will burst the skins, and itself will be spilled, and the skins will perish.}}
\bv{38}{\redlet{But new wine must be put into fresh wine-skins.}}
\bv{39}{\redlet{And no man having drunk old \supptext{wine} desireth new; for he saith, `The old is good.'{''}}}
\chaphead{Chapter VI}
\chapdesc{Jesus \& the Sabbath}
\lettrine[image=true, lines=4, findent=3pt, nindent=0pt]{NT/Luke/Lk-N.eps}{ow} it came to pass on a sabbath, that he was going through the grainfields; and his disciples plucked the ears, and did eat, rubbing them in their hands.
\bv{2}{But certain of the Pharisees said, ``Why do ye that which it is not lawful to do on the sabbath day?''}
\bv{3}{And Jesus answering them said, \redlet{``Have ye not read even this, what David did, when he was hungry, he, and they that were with him;}}
\bv{4}{\redlet{how he entered into the house of God, and took and ate the showbread, and gave also to them that were with him; which it is not lawful to eat save for the priests alone?''}}
\bv{5}{And he said unto them, \redlet{``The Son of man is lord of the sabbath.''}}
\chapsec{The Withered Hand Healed}
\bv{6}{And it came to pass on another sabbath, that he entered into the synagogue and taught: and there was a man there, and his right hand was withered.}
\bv{7}{And the scribes and the Pharisees watched him, whether he would heal on the sabbath; that they might find how to accuse him.}
\bv{8}{But he knew their thoughts; and he said to the man that had his hand withered, \redlet{``Rise up, and stand forth in the midst.''} And he arose and stood forth.}
\bv{9}{And Jesus said unto them, \redlet{``I ask you, Is it lawful on the sabbath to do good, or to do harm? to save a life, or to destroy it?''}}
\bv{10}{And he looked round about on them all, and said unto him, \redlet{``Stretch forth thy hand.''} And he did \supptext{so}: and his hand was restored.}
\bv{11}{But they were filled with madness; and communed one with another what they might do to Jesus.}
\chapsec{The Twelve Chosen}
\bv{12}{And it came to pass in these days, that he went out into the mountain to pray; and he continued all night in prayer to God.}
\bv{13}{And when it was day, he called his disciples; and he chose from them twelve, whom also he named apostles:}
\bv{14}{Simon, whom he also named Peter, and Andrew his brother, and James and John, and Philip and Bartholomew,}
\bv{15}{and Matthew and Thomas, and James \supptext{the son} of Alphæus, and Simon who was called the Zealot,}
\bv{16}{and Judas \supptext{the son} of James, and Judas Iscariot, who became a traitor;}
\bv{17}{and he came down with them, and stood on a level place, and a great multitude of his disciples, and a great number of the people from all Judæa and Jerusalem, and the sea coast of Tyre and Sidon, who came to hear him, and to be healed of their diseases;}
\bv{18}{and they that were troubled with unclean spirits were healed.}
\bv{19}{And all the multitude sought to touch him; for power came forth from him, and healed \supptext{them} all.}
\chapsec{The Sermon on the Mount}
\bv{20}{And he lifted up his eyes on his disciples, and said, \redlet{``Blessed \supptext{are} ye poor: for yours is the kingdom of God.}}
\bv{21}{\redlet{Blessed \supptext{are} ye that hunger now: for ye shall be filled. Blessed \supptext{are} ye that weep now: for ye shall laugh.}}
\bv{22}{\redlet{Blessed are ye, when men shall hate you, and when they shall separate you \supptext{from their company}, and reproach you, and cast out your name as evil, for the Son of man's sake.}}
\bv{23}{\redlet{Rejoice in that day, and leap \supptext{for joy}: for behold, your reward is great in heaven; for in the same manner did their fathers unto the prophets.}}
\bv{24}{\redlet{But woe unto you that are rich! for ye have received your consolation.}}
\bv{25}{\redlet{Woe unto you, ye that are full now! for ye shall hunger. Woe \supptext{unto you}, ye that laugh now! for ye shall mourn and weep.}}
\bv{26}{\redlet{Woe \supptext{unto you}, when all men shall speak well of you! for in the same manner did their fathers to the false prophets.}}
\par
\bv{27}{\redlet{But I say unto you that hear, Love your enemies, do good to them that hate you,}}
\bv{28}{\redlet{bless them that curse you, pray for them that despitefully use you.}}
\bv{29}{\redlet{To him that smiteth thee on the \supptext{one} cheek offer also the other; and from him that taketh away thy cloak withhold not thy coat also.}}
\bv{30}{\redlet{Give to every one that asketh thee; and of him that taketh away thy goods ask them not again.}}
\bv{31}{\redlet{And as ye would that men should do to you, do ye also to them likewise.}}
\bv{32}{\redlet{And if ye love them that love you, what thank have ye? for even sinners love those that love them.}}
\bv{33}{\redlet{And if ye do good to them that do good to you, what thank have ye? for even sinners do the same.}}
\bv{34}{\redlet{And if ye lend to them of whom ye hope to receive, what thank have ye? even sinners lend to sinners, to receive again as much.}}
\bv{35}{\redlet{But love your enemies, and do \supptext{them} good, and lend, never despairing; and your reward shall be great, and ye shall be sons of the Most High: for he is kind toward the unthankful and evil.}}
\bv{36}{\redlet{Be ye merciful, even as your Father is merciful.}}
\bv{37}{\redlet{And judge not, and ye shall not be judged: and condemn not, and ye shall not be condemned: release, and ye shall be released:}}
\bv{38}{\redlet{give, and it shall be given unto you; good measure, pressed down, shaken together, running over, shall they give into your bosom. For with what measure ye mete it shall be measured to you again.''}}
\par
\bv{39}{And he spake also a parable unto them, \redlet{``Can the blind guide the blind? shall they not both fall into a pit?}}
\bv{40}{\redlet{The disciple is not above his teacher: but every one when he is perfected shall be as his teacher.}}
\bv{41}{\redlet{And why beholdest thou the mote that is in thy brother's eye, but considerest not the beam that is in thine own eye?}}
\bv{42}{\redlet{Or how canst thou say to thy brother, `Brother, let me cast out the mote that is in thine eye,' when thou thyself beholdest not the beam that is in thine own eye? Thou hypocrite, cast out first the beam out of thine own eye, and then shalt thou see clearly to cast out the mote that is in thy brother's eye.}}
\par
\bv{43}{\redlet{For there is no good tree that bringeth forth corrupt fruit; nor again a corrupt tree that bringeth forth good fruit.}}
\bv{44}{\redlet{For each tree is known by its own fruit. For of thorns men do not gather figs, nor of a bramble bush gather they grapes.}}
\bv{45}{\redlet{The good man out of the good treasure of his heart bringeth forth that which is good; and the evil \supptext{man} out of the evil \supptext{treasure} bringeth forth that which is evil: for out of the abundance of the heart his mouth speaketh.}}
\bv{46}{\redlet{And why call ye me, `Lord, Lord,' and do not the things which I say?}}
\chapsec{Parable of the House Built on Rock}
\bv{47}{\redlet{Every one that cometh unto me, and heareth my words, and doeth them, I will show you to whom he is like:}}
\bv{48}{\redlet{he is like a man building a house, who digged and went deep, and laid a foundation upon the rock: and when a flood arose, the stream brake against that house, and could not shake it: because it had been well builded.}}
\bv{49}{\redlet{But he that heareth, and doeth not, is like a man that built a house upon the earth without a foundation; against which the stream brake, and straightway it fell in; and the ruin of that house was great.''}}
\chaphead{Chapter VII}
\chapdesc{The Centurion's Servant Healed}
\lettrine[image=true, lines=4, findent=3pt, nindent=0pt]{NT/Luke/Lk-A.eps}{fter} he had ended all his sayings in the ears of the people, he entered into Capernaum.
\bv{2}{And a certain centurion's servant, who was dear unto him, was sick and at the point of death.}
\bv{3}{And when he heard concerning Jesus, he sent unto him elders of the Jews, asking him that he would come and save his servant.}
\bv{4}{And they, when they came to Jesus, besought him earnestly, saying, ``He is worthy that thou shouldest do this for him;}
\bv{5}{for he loveth our nation, and himself built us our synagogue.''}
\bv{6}{And Jesus went with them. And when he was now not far from the house, the centurion sent friends to him, saying unto him, ``Lord, trouble not thyself; for I am not worthy that thou shouldest come under my roof:}
\bv{7}{wherefore neither thought I myself worthy to come unto thee: but say the word, and my servant shall be healed.}
\bv{8}{For I also am a man set under authority, having under myself soldiers: and I say to this one, `Go,' and he goeth; and to another, `Come,' and he cometh; and to my servant, `Do this,' and he doeth it.''}
\bv{9}{And when Jesus heard these things, he marvelled at him, and turned and said unto the multitude that followed him, \redlet{``I say unto you, I have not found so great faith, no, not in Israel.''}}
\bv{10}{And they that were sent, returning to the house, found the servant whole.}
\chapsec{The Widow's Son Raised}
\bv{11}{And it came to pass soon afterwards, that he went to a city called Nain; and his disciples went with him, and a great multitude.}
\bv{12}{Now when he drew near to the gate of the city, behold, there was carried out one that was dead, the only son of his mother, and she was a widow: and much people of the city was with her.}
\bv{13}{And when the Lord saw her, he had compassion on her, and said unto her, \redlet{``Weep not.''}}
\bv{14}{And he came nigh and touched the bier: and the bearers stood still. And he said, \redlet{``Young man, I say unto thee, Arise.''}}
\bv{15}{And he that was dead sat up, and began to speak. And he gave him to his mother.}
\bv{16}{And fear took hold on all: and they glorified God, saying, ``A great prophet is arisen among us:'' and, ``God hath visited his people.''}
\bv{17}{And this report went forth concerning him in the whole of Judæa, and all the region round about.}
\bv{18}{And the disciples of John told him of all these things.}
\chapsec{St. John the Baptist Inquires of Jesus}
\bv{19}{And John calling unto him two of his disciples sent them to the Lord, saying, ``Art thou he that cometh, or look we for another?''}
\bv{20}{And when the men were come unto him, they said, ``John the Baptist hath sent us unto thee, saying, `Art thou he that cometh, or look we for another?'{''}}
\bv{21}{In that hour he cured many of diseases and plagues and evil spirits; and on many that were blind he bestowed sight.}
\bv{22}{And he answered and said unto them, \redlet{``Go and tell John the things which ye have seen and heard; the blind receive their sight, the lame walk, the lepers are cleansed, and the deaf hear, the dead are raised up, the poor have good tidings preached to them.}}
\bv{23}{\redlet{And blessed is he, whosoever shall find no occasion of stumbling in me.''}}
\chapsec{Jesus' Testimony to St. John the Baptist}
\bv{24}{And when the messengers of John were departed, he began to say unto the multitudes concerning John, \redlet{``What went ye out into the wilderness to behold? a reed shaken with the wind?}}
\bv{25}{\redlet{But what went ye out to see? a man clothed in soft raiment? Behold, they that are gorgeously apparelled, and live delicately, are in kings' courts.}}
\bv{26}{\redlet{But what went ye out to see? a prophet? Yea, I say unto you, and much more than a prophet.}}
\bv{27}{\redlet{This is he of whom it is written,}}
\otQuote{Mal. 3:1}{\redlet{Behold, I send my messenger before thy face,
Who shall prepare thy way before thee.}}
\bv{28}{\redlet{I say unto you, Among them that are born of women there is none greater than John: yet he that is but little in the kingdom of God is greater than he.}}
\bv{29}{And all the people when they heard, and the publicans, justified God, being baptised with the baptism of John.}
\chapsec{Jesus Exposes the Unreason of Unbelief}
\bv{30}{\redlet{But the Pharisees and the lawyers rejected for themselves the counsel of God, being not baptised of him.}}
\bv{31}{\redlet{Whereunto then shall I liken the men of this generation, and to what are they like?}}
\bv{32}{\redlet{They are like unto children that sit in the marketplace, and call one to another; who say, `We piped unto you, and ye did not dance; we wailed, and ye did not weep.'}}
\bv{33}{\redlet{For John the Baptist is come eating no bread nor drinking wine; and ye say, `He hath a demon.'}}
\bv{34}{\redlet{The Son of man is come eating and drinking; and ye say, `Behold, a gluttonous man, and a winebibber, a friend of publicans and sinners!'}}
\bv{35}{\redlet{And wisdom is justified of all her children.''}}
\chapsec{Jesus in the Pharisee's House}
\bv{36}{And one of the Pharisees desired him that he would eat with him. And he entered into the Pharisee's house, and sat down to meat.}
\bv{37}{And behold, a woman who was in the city, a sinner; and when she knew that he was sitting at meat in the Pharisee's house, she brought an alabaster cruse of ointment,}
\bv{38}{and standing behind at his feet, weeping, she began to wet his feet with her tears, and wiped them with the hair of her head, and kissed his feet, and anointed them with the ointment.}
\bv{39}{Now when the Pharisee that had bidden him saw it, he spake within himself, saying, ``This man, if he were a prophet, would have perceived who and what manner of woman this is that toucheth him, that she is a sinner.''}
\bv{40}{And Jesus answering said unto him, \redlet{``Simon, I have somewhat to say unto thee.''} And he saith, ``Teacher, say on.''}
\chapsec{Parable of the Creditor \& Two Debtors}
\bv{41}{\redlet{``A certain lender had two debtors: the one owed five hundred denarii,}}\mcomm{One denarius was given for a day's wages.} \redlet{and the other fifty.}
\bv{42}{\redlet{When they had not \supptext{wherewith} to pay, he forgave them both. Which of them therefore will love him most?''}}
\bv{43}{Simon answered and said, ``He, I suppose, to whom he forgave the most.'' And he said unto him, \redlet{``Thou hast rightly judged.''}}
\bv{44}{And turning to the woman, he said unto Simon, \redlet{``Seest thou this woman?'' I entered into thy house, thou gavest me no water for my feet: but she hath wetted my feet with her tears, and wiped them with her hair.}}
\bv{45}{\redlet{Thou gavest me no kiss: but she, since the time I came in, hath not ceased to kiss my feet.}}
\bv{46}{\redlet{My head with oil thou didst not anoint: but she hath anointed my feet with ointment.}}
\bv{47}{\redlet{Wherefore I say unto thee, Her sins, which are many, are forgiven; for she loved much: but to whom little is forgiven, \supptext{the same} loveth little.''}}
\bv{48}{And he said unto her, \redlet{``Thy sins are forgiven.''}}
\bv{49}{And they that sat at meat with him began to say within themselves, ``Who is this that even forgiveth sins?''}
\bv{50}{And he said unto the woman, \redlet{``Thy faith hath saved thee; go in peace.''}}
\chaphead{Chapter VIII}
\chapdesc{Jesus Preaches \& Heals in Galilee}
\lettrine[image=true, lines=4, findent=3pt, nindent=0pt]{NT/Luke/Lk-A.eps}{nd} it came to pass soon afterwards, that he went about through cities and villages, preaching and bringing the good tidings of the kingdom of God, and with him the twelve,
\bv{2}{and certain women who had been healed of evil spirits and infirmities: Mary that was called Magdalene, from whom seven demons had gone out,}
\bv{3}{and Joanna the wife of Chuzas Herod's steward, and Susanna, and many others, who ministered unto them of their substance.}
\chapsec{Parable of the Sower}
\bv{4}{And when a great multitude came together, and they of every city resorted unto him, he spake by a parable:}
\bv{5}{\redlet{``The sower went forth to sow his seed: and as he sowed, some fell by the way side; and it was trodden under foot, and the birds of the heaven devoured it.}}
\bv{6}{\redlet{And other fell on the rock; and as soon as it grew, it withered away, because it had no moisture.}}
\bv{7}{\redlet{And other fell amidst the thorns; and the thorns grew with it, and choked it.}}
\bv{8}{\redlet{And other fell into the good ground, and grew, and brought forth fruit a hundredfold. As he said these things, he cried, He that hath ears to hear, let him hear.''}}
\par
\bv{9}{And his disciples asked him what this parable might be.}
\bv{10}{And he said, \redlet{``Unto you it is given to know the mysteries of the kingdom of God: but to the rest in parables; that seeing they may not see, and hearing they may not understand.}}
\bv{11}{\redlet{Now the parable is this: The seed is the word of God.}}
\bv{12}{\redlet{And those by the way side are they that have heard; then cometh the devil, and taketh away the word from their heart, that they may not believe and be saved.}}
\bv{13}{\redlet{And those on the rock \supptext{are} they who, when they have heard, receive the word with joy; and these have no root, who for a while believe, and in time of temptation fall away.}}
\bv{14}{\redlet{And that which fell among the thorns, these are they that have heard, and as they go on their way they are choked with cares and riches and pleasures of \supptext{this} life, and bring no fruit to perfection.}}
\bv{15}{\redlet{And that in the good ground, these are such as in an honest and good heart, having heard the word, hold it fast, and bring forth fruit with patience.}}
\chapsec{Parable of the Lighted Candle}
\bv{16}{\redlet{And no man, when he hath lighted a lamp, covereth it with a vessel, or putteth it under a bed; but putteth it on a stand, that they that enter in may see the light.}}
\bv{17}{\redlet{For nothing is hid, that shall not be made manifest; nor \supptext{anything} secret, that shall not be known and come to light.}}
\bv{18}{\redlet{Take heed therefore how ye hear: for whosoever hath, to him shall be given; and whosoever hath not, from him shall be taken away even that which he thinketh he hath.''}}
\chapsec{The New Relationships}
\bv{19}{And there came to him his mother and brethren, and they could not come at him for the crowd.}
\bv{20}{And it was told him, ``Thy mother and thy brethren stand without, desiring to see thee.''}
\bv{21}{But he answered and said unto them, \redlet{``My mother and my brethren are these that hear the word of God, and do it.''}}
\chapsec{Jesus Stills the Waves}
\bv{22}{Now it came to pass on one of those days, that he entered into a boat, himself and his disciples; and he said unto them, \redlet{``Let us go over unto the other side of the lake:''} and they launched forth.}
\bv{23}{But as they sailed he fell asleep: and there came down a storm of wind on the lake; and they were filling \supptext{with water}, and were in jeopardy.}
\bv{24}{And they came to him, and awoke him, saying, ``Master, master, we perish.'' And he awoke, and rebuked the wind and the raging of the water: and they ceased, and there was a calm.}
\bv{25}{And he said unto them, \redlet{``Where is your faith?''} And being afraid they marvelled, saying one to another, ``Who then is this, that he commandeth even the winds and the water, and they obey him?''}
\chapsec{Demons Cast out of the Maniac}
\bv{26}{And they arrived at the country of the Gerasenes, which is over against Galilee.}
\bv{27}{And when he was come forth upon the land, there met him a certain man out of the city, who had demons; and for a long time he had worn no clothes, and abode not in \supptext{any} house, but in the tombs.}
\bv{28}{And when he saw Jesus, he cried out, and fell down before him, and with a loud voice said, ``What have I to do with thee, Jesus, thou Son of the Most High God? I beseech thee, torment me not.''}
\bv{29}{For he was commanding the unclean spirit to come out from the man. For oftentimes it had seized him: and he was kept under guard, and bound with chains and fetters; and breaking the bands asunder, he was driven of the demon into the deserts.}
\bv{30}{And Jesus asked him, \redlet{``What is thy name?''} And he said, ``Legion;'' for many demons were entered into him.}
\bv{31}{And they entreated him that he would not command them to depart into the abyss.}
\bv{32}{Now there was there a herd of many swine feeding on the mountain: and they entreated him that he would give them leave to enter into them. And he gave them leave.}
\bv{33}{And the demons came out from the man, and entered into the swine: and the herd rushed down the steep into the lake, and were drowned.}
\bv{34}{And when they that fed them saw what had come to pass, they fled, and told it in the city and in the country.}
\bv{35}{And they went out to see what had come to pass; and they came to Jesus, and found the man, from whom the demons were gone out, sitting, clothed and in his right mind, at the feet of Jesus: and they were afraid.}
\bv{36}{And they that saw it told them how he that was possessed with demons was made whole.}
\bv{37}{And all the people of the country of the Gerasenes round about asked him to depart from them; for they were holden with great fear: and he entered into a boat, and returned.}
\bv{38}{But the man from whom the demons were gone out prayed him that he might be with him: but he sent him away, saying,}
\bv{39}{\redlet{``Return to thy house, and declare how great things God hath done for thee.''} And he went his way, publishing throughout the whole city how great things Jesus had done for him.}
\chapsec{A Woman Healed \& Jairus' Daughter Raised}
\bv{40}{And as Jesus returned, the multitude welcomed him; for they were all waiting for him.}
\bv{41}{And behold, there came a man named Jairus, and he was a ruler of the synagogue: and he fell down at Jesus' feet, and besought him to come into his house;}
\bv{42}{for he had an only daughter, about twelve years of age, and she was dying. But as he went the multitudes thronged him.}
\par
\bv{43}{And a woman having an issue of blood twelve years, who had spent all her living upon physicians, and could not be healed of any,}
\bv{44}{came behind him, and touched the border of his garment: and immediately the issue of her blood stanched.}
\bv{45}{And Jesus said, \redlet{``Who is it that touched me?''} And when all denied, Peter said, and they that were with him, ``Master, the multitudes press thee and crush \supptext{thee}.''}
\bv{46}{But Jesus said, \redlet{``Some one did touch me; for I perceived that power had gone forth from me.''}}
\bv{47}{And when the woman saw that she was not hid, she came trembling, and falling down before him declared in the presence of all the people for what cause she touched him, and how she was healed immediately.}
\bv{48}{And he said unto her, \redlet{``Daughter, thy faith hath made thee whole; go in peace.''}}
\par
\bv{49}{While he yet spake, there cometh one from the ruler of the synagogue's \supptext{house}, saying, ``Thy daughter is dead; trouble not the Teacher.''}
\bv{50}{But Jesus hearing it, answered him, Fear not: only believe, and she shall be made whole.}
\bv{51}{And when he came to the house, he suffered not any man to enter in with him, save Peter, and John, and James, and the father of the maiden and her mother.}
\bv{52}{And all were weeping, and bewailing her: but he said, \redlet{``Weep not; for she is not dead, but sleepeth.''}}
\bv{53}{And they laughed him to scorn, knowing that she was dead.}
\bv{54}{But he, taking her by the hand, called, saying, \redlet{``Maiden, arise.''}}
\bv{55}{And her spirit returned, and she rose up immediately: and he commanded that \supptext{something} be given her to eat.}
\bv{56}{And her parents were amazed: but he charged them to tell no man what had been done.}
\chaphead{Chapter IX}
\chapdesc{The Twelve Sent Forth to Preach}
\lettrine[image=true, lines=4, findent=3pt, nindent=0pt]{NT/Luke/Lk-A.eps}{nd} he called the twelve together, and gave them power and authority over all demons, and to cure diseases.
\bv{2}{And he sent them forth to preach the kingdom of God, and to heal the sick.}
\bv{3}{And he said unto them, \redlet{``Take nothing for your journey, neither staff, nor wallet, nor bread, nor money; neither have two coats.}}
\bv{4}{\redlet{And into whatsoever house ye enter, there abide, and thence depart.}}
\bv{5}{\redlet{And as many as receive you not, when ye depart from that city, shake off the dust from your feet for a testimony against them.''}}
\bv{6}{And they departed, and went throughout the villages, preaching the gospel, and healing everywhere.}
\par
\bv{7}{Now Herod the tetrarch heard of all that was done: and he was much perplexed, because that it was said by some, that John was risen from the dead;}
\bv{8}{and by some, that Elijah had appeared; and by others, that one of the old prophets was risen again.}
\bv{9}{And Herod said, ``John I beheaded: but who is this, about whom I hear such things?'' And he sought to see him.}
\chapsec{The Feeding of the 5,000}
\bv{10}{And the apostles, when they were returned, declared unto him what things they had done. And he took them, and withdrew apart to a city called Bethsaida.}
\bv{11}{But the multitudes perceiving it followed him: and he welcomed them, and spake to them of the kingdom of God, and them that had need of healing he cured.}
\bv{12}{And the day began to wear away; and the twelve came, and said unto him, ``Send the multitude away, that they may go into the villages and country round about, and lodge, and get provisions: for we are here in a desert place.''}
\bv{13}{But he said unto them, \redlet{``Give ye them to eat.''} And they said, ``We have no more than five loaves and two fishes; except we should go and buy food for all this people.''}
\bv{14}{For they were about five thousand men. And he said unto his disciples, \redlet{``Make them sit down in companies, about fifty each.''}}
\bv{15}{And they did so, and made them all sit down.}
\bv{16}{And he took the five loaves and the two fishes, and looking up to heaven, he blessed them, and brake; and gave to the disciples to set before the multitude.}
\bv{17}{And they ate, and were all filled: and there was taken up that which remained over to them of broken pieces, twelve baskets.}
\chapsec{St. Peter's Confession of Christ}
\bv{18}{And it came to pass, as he was praying apart, the disciples were with him: and he asked them, saying, \redlet{``Who do the multitudes say that I am?''}}
\bv{19}{And they answering said, ``John the Baptist; but others \supptext{say}, Elijah; and others, that one of the old prophets is risen again.''}
\bv{20}{And he said unto them, \redlet{``But who say ye that I am?''} And Peter answering said, ``The Christ of God.''}
\chapsec{Jesus Foretells his Death \& Resurrection}
\bv{21}{But he charged them, and commanded \supptext{them} to tell this to no man;}
\bv{22}{saying, \redlet{``The Son of man must suffer many things, and be rejected of the elders and chief priests and scribes, and be killed, and the third day be raised up.''}}
\chapsec{The Test of Discipleship}
\bv{23}{And he said unto all, \redlet{``If any man would come after me, let him deny himself, and take up his cross daily, and follow me.}}
\bv{24}{\redlet{For whosoever would save his life shall lose it; but whosoever shall lose his life for my sake, the same shall save it.}}
\bv{25}{\redlet{For what is a man profited, if he gain the whole world, and lose or forfeit his own self?}}
\bv{26}{\redlet{For whosoever shall be ashamed of me and of my words, of him shall the Son of man be ashamed, when he cometh in his own glory, and \supptext{the glory} of the Father, and of the holy angels.}}
\bv{27}{\redlet{But I tell you of a truth, There are some of them that stand here, who shall in no wise taste of death, till they see the kingdom of God.''}}
\chapsec{The Transfiguration}
\bv{28}{And it came to pass about eight days after these sayings, that he took with him Peter and John and James, and went up into the mountain to pray.}
\bv{29}{And as he was praying, the fashion of his countenance was altered, and his raiment \supptext{became} white \supptext{and} dazzling.}
\bv{30}{And behold, there talked with him two men, who were Moses and Elijah;}
\bv{31}{who appeared in glory, and spake of his decease which he was about to accomplish at Jerusalem.}
\bv{32}{Now Peter and they that were with him were heavy with sleep: but when they were fully awake, they saw his glory, and the two men that stood with him.}
\bv{33}{And it came to pass, as they were parting from him, Peter said unto Jesus, ``Master, it is good for us to be here: and let us make three tabernacles; one for thee, and one for Moses, and one for Elijah:'' not knowing what he said.}
\bv{34}{And while he said these things, there came a cloud, and overshadowed them: and they feared as they entered into the cloud.}
\bv{35}{And a voice came out of the cloud, saying, \god{This is my Son, my chosen: hear ye him.}}
\bv{36}{And when the voice came, Jesus was found alone. And they held their peace, and told no man in those days any of the things which they had seen.}
\chapsec{The Powerless Disciples}
\bv{37}{And it came to pass, on the next day, when they were come down from the mountain, a great multitude met him.}
\bv{38}{And behold, a man from the multitude cried, saying, ``Teacher, I beseech thee to look upon my son; for he is mine only child:}
\bv{39}{and behold, a spirit taketh him, and he suddenly crieth out; and it teareth him that he foameth, and it hardly departeth from him, bruising him sorely.}
\bv{40}{And I besought thy disciples to cast it out; and they could not.''}
\bv{41}{And Jesus answered and said, \redlet{``O faithless and perverse generation, how long shall I be with you, and bear with you? bring hither thy son.''}}
\bv{42}{And as he was yet a coming, the demon dashed him down, and tare \supptext{him} grievously. But Jesus rebuked the unclean spirit, and healed the boy, and gave him back to his father.}
\bv{43}{And they were all astonished at the majesty of God.
But while all were marvelling at all the things which he did, he said unto his disciples,}
\chapsec{Jesus Again Foretells his Death}
\bv{44}{\redlet{``Let these words sink into your ears: for the Son of man shall be delivered up into the hands of men.''}}
\bv{45}{But they understood not this saying, and it was concealed from them, that they should not perceive it; and they were afraid to ask him about this saying.}
\chapsec{The Sermon on the Child}
\bv{46}{And there arose a reasoning among them, which of them was the greatest.}
\bv{47}{But when Jesus saw the reasoning of their heart, he took a little child, and set him by his side,}
\bv{48}{and said unto them, \redlet{``Whosoever shall receive this little child in my name receiveth me: and whosoever shall receive me receiveth him that sent me: for he that is least among you all, the same is great.}}
\chapsec{The Rebuke of Sectarianism}
\bv{49}{And John answered and said, ``Master, we saw one casting out demons in thy name; and we forbade him, because he followeth not with us.''}
\bv{50}{But Jesus said unto him, \redlet{``Forbid \supptext{him} not: for he that is not against you is for you.''}}
\chapsec{Final Departure from Galilee}
\bv{51}{And it came to pass, when the days were well-nigh come that he should be received up, he stedfastly set his face to go to Jerusalem,}
\bv{52}{and sent messengers before his face: and they went, and entered into a village of the Samaritans, to make ready for him.}
\bv{53}{And they did not receive him, because his face was \supptext{as though he were} going to Jerusalem.}
\bv{54}{And when his disciples James and John saw \supptext{this}, they said, ``Lord, wilt thou that we bid fire to come down from heaven, and consume them?''}
\bv{55}{But he turned, and rebuked them.}
\bv{56}{And they went to another village.}
\chapsec{Another Test of Discipleship}
\bv{57}{And as they went on the way, a certain man said unto him, ``I will follow thee whithersoever thou goest.''}
\bv{58}{And Jesus said unto him, \redlet{``The foxes have holes, and the birds of the heaven \supptext{have} nests; but the Son of man hath not where to lay his head.''}}
\bv{59}{And he said unto another, \redlet{``Follow me.''} But he said, ``Lord, suffer me first to go and bury my father.''}
\bv{60}{But he said unto him, \redlet{``Leave the dead to bury their own dead; but go thou and publish abroad the kingdom of God.''}}
\bv{61}{And another also said, ``I will follow thee, Lord; but first suffer me to bid farewell to them that are at my house.''}
\bv{62}{But Jesus said unto him, \redlet{``No man, having put his hand to the plow, and looking back, is fit for the kingdom of God.''}}
\chaphead{Chapter X}
\chapdesc{Jesus Sends the Seventy}
\lettrine[image=true, lines=4, findent=3pt, nindent=0pt]{NT/Luke/Lk-N.eps}{ow} after these things the Lord appointed seventy others, and sent them two and two before his face into every city and place, whither he himself was about to come.
\bv{2}{And he said unto them, \redlet{``The harvest indeed is plenteous, but the laborers are few: pray ye therefore the Lord of the harvest, that he send forth laborers into his harvest.}}
\bv{3}{\redlet{Go your ways; behold, I send you forth as lambs in the midst of wolves.}}
\bv{4}{\redlet{Carry no purse, no wallet, no shoes; and salute no man on the way.}}
\bv{5}{\redlet{And into whatsoever house ye shall enter, first say, Peace \supptext{be} to this house.}}
\bv{6}{\redlet{And if a son of peace be there, your peace shall rest upon him: but if not, it shall turn to you again.}}
\bv{7}{\redlet{And in that same house remain, eating and drinking such things as they give: for the laborer is worthy of his hire. Go not from house to house.}}
\bv{8}{\redlet{And into whatsoever city ye enter, and they receive you, eat such things as are set before you:}}
\bv{9}{\redlet{and heal the sick that are therein, and say unto them, `The kingdom of God is come nigh unto you.'}}
\par
\bv{10}{\redlet{But into whatsoever city ye shall enter, and they receive you not, go out into the streets thereof and say,}}
\bv{11}{\redlet{`Even the dust from your city, that cleaveth to our feet, we wipe off against you: nevertheless know this, that the kingdom of God is come nigh.'}}
\bv{12}{\redlet{I say unto you, It shall be more tolerable in that day for Sodom, than for that city.}}
\chapsec{Jesus Denounces Judgement on the Cities}
\bv{13}{\redlet{Woe unto thee, Chorazin! woe unto thee, Bethsaida! for if the mighty works had been done in Tyre and Sidon, which were done in you, they would have repented long ago, sitting in sackcloth and ashes.}}
\bv{14}{\redlet{But it shall be more tolerable for Tyre and Sidon in the judgement, than for you.}}
\bv{15}{\redlet{And thou, Capernaum, shalt thou be exalted unto heaven? thou shalt be brought down unto Hades.}}
\bv{16}{\redlet{He that heareth you heareth me; and he that rejecteth you rejecteth me; and he that rejecteth me rejecteth him that sent me.''}}
\par
\bv{17}{And the seventy returned with joy, saying, ``Lord, even the demons are subject unto us in thy name.''}
\bv{18}{And he said unto them, \redlet{``I beheld Satan fallen as lightning from heaven.}}
\bv{19}{\redlet{Behold, I have given you authority to tread upon serpents and scorpions, and over all the power of the enemy: and nothing shall in any wise hurt you.}}
\bv{20}{\redlet{Nevertheless in this rejoice not, that the spirits are subject unto you; but rejoice that your names are written in heaven.''}}
\par
\bv{21}{In that same hour he rejoiced in the Holy Ghost, and said, \redlet{``I thank thee, O Father, Lord of heaven and earth, that thou didst hide these things from the wise and understanding, and didst reveal them unto babes: yea, Father; for so it was well-pleasing in thy sight.}}
\bv{22}{\redlet{All things have been delivered unto me of my Father: and no one knoweth who the Son is, save the Father; and who the Father is, save the Son, and he to whomsoever the Son willeth to reveal \supptext{him}.''}}
\bv{23}{And turning to the disciples, he said privately, \redlet{``Blessed \supptext{are} the eyes which see the things that ye see:}}
\bv{24}{\redlet{for I say unto you, that many prophets and kings desired to see the things which ye see, and saw them not; and to hear the things which ye hear, and heard them not.''}}
\chapsec{A Lawyer Questions Jesus}
\bv{25}{And behold, a certain lawyer stood up and made trial of him, saying, ``Teacher, what shall I do to inherit eternal life?''}
\bv{26}{And he said unto him, \redlet{``What is written in the law? how readest thou?''}}
\bv{27}{And he answering said, ``Thou shalt love the Lord thy God with all thy heart, and with all thy soul, and with all thy strength, and with all thy mind; and thy neighbour as thyself.''}
\bv{28}{And he said unto him, \redlet{``Thou hast answered right: this do, and thou shalt live.''}}
\bv{29}{But he, desiring to justify himself, said unto Jesus, ``And who is my neighbour?''}
\chapsec{Parable of the Good Samaritan}
\bv{30}{Jesus made answer and said, \redlet{``A certain man was going down from Jerusalem to Jericho; and he fell among robbers, who both stripped him and beat him, and departed, leaving him half dead.}}
\bv{31}{\redlet{And by chance a certain priest was going down that way: and when he saw him, he passed by on the other side.}}
\bv{32}{\redlet{And in like manner a Levite also, when he came to the place, and saw him, passed by on the other side.}}
\bv{33}{\redlet{But a certain Samaritan, as he journeyed, came where he was: and when he saw him, he was moved with compassion,}}
\bv{34}{\redlet{and came to him, and bound up his wounds, pouring on \supptext{them} oil and wine; and he set him on his own beast, and brought him to an inn, and took care of him.}}
\bv{35}{\redlet{And on the morrow he took out two shillings, and gave them to the host, and said, `Take care of him; and whatsoever thou spendest more, I, when I come back again, will repay thee.'}}
\bv{36}{\redlet{Which of these three, thinkest thou, proved neighbour unto him that fell among the robbers?''}}
\bv{37}{And he said, ``He that showed mercy on him.'' And Jesus said unto him, \redlet{``Go, and do thou likewise.''}}
\chapsec{Martha \& Mary}
\bv{38}{Now as they went on their way, he entered into a certain village: and a certain woman named Martha received him into her house.}
\bv{39}{And she had a sister called Mary, who also sat at the Lord's feet, and heard his word.}
\bv{40}{But Martha was cumbered about much serving; and she came up to him, and said, ``Lord, dost thou not care that my sister did leave me to serve alone? bid her therefore that she help me.''}
\bv{41}{But the Lord answered and said unto her, \redlet{``Martha, Martha, thou art anxious and troubled about many things:}}
\bv{42}{\redlet{but one thing is needful: for Mary hath chosen the good part, which shall not be taken away from her.''}}
\chaphead{Chapter XI}
\chapdesc{Jesus' Teaching on Prayer}
\lettrine[image=true, lines=4, findent=3pt, nindent=0pt]{NT/Luke/Lk-A.eps}{nd} it came to pass, as he was praying in a certain place, that when he ceased, one of his disciples said unto him, ``Lord, teach us to pray, even as John also taught his disciples.''
\bv{2}{And he said unto them, \redlet{``When ye pray, say,}}
\canticle{\redlet{Father, Hallowed be thy name.\\
Thy kingdom come.\\
\bv{3}{Give us day by day our daily bread.}\\
\bv{4}{And forgive us our sins;\\
for we ourselves also forgive every one that is indebted to us.\\
And bring us not into temptation.''}}}
\chapsec{Parable of the Importunate Friend}
\bv{5}{And he said unto them, \redlet{``Which of you shall have a friend, and shall go unto him at midnight, and say to him, `Friend, lend me three loaves;}}
\bv{6}{\redlet{for a friend of mine is come to me from a journey, and I have nothing to set before him;'}}
\bv{7}{\redlet{and he from within shall answer and say, `Trouble me not: the door is now shut, and my children are with me in bed; I cannot rise and give thee?'}}
\bv{8}{\redlet{I say unto you, Though he will not rise and give him because he is his friend, yet because of his importunity he will arise and give him as many as he needeth.}}
\bv{9}{\redlet{And I say unto you, Ask, and it shall be given you; seek, and ye shall find; knock, and it shall be opened unto you.}}
\bv{10}{\redlet{For every one that asketh receiveth; and he that seeketh findeth; and to him that knocketh it shall be opened.}}
\chapsec{Parable of the Father}
\bv{11}{\redlet{And of which of you that is a father shall his son ask a loaf, and he give him a stone? or a fish, and he for a fish give him a serpent?}}
\bv{12}{\redlet{Or \supptext{if} he shall ask an egg, will he give him a scorpion?}}
\bv{13}{\redlet{If ye then, being evil, know how to give good gifts unto your children, how much more shall \supptext{your} heavenly Father give the Holy Ghost to them that ask him?''}}
\chapsec{Jesus Charged with Demoncraft}
\bv{14}{And he was casting out a demon \supptext{that was} dumb. And it came to pass, when the demon was gone out, the dumb man spake; and the multitudes marvelled.}
\bv{15}{But some of them said, ``By Beelzebub the prince of the demons casteth he out demons.''}
\bv{16}{And others, trying \supptext{him}, sought of him a sign from heaven.}
\bv{17}{But he, knowing their thoughts, said unto them, \redlet{``Every kingdom divided against itself is brought to desolation; and a house \supptext{divided} against a house falleth.}}
\bv{18}{\redlet{And if Satan also is divided against himself, how shall his kingdom stand? because ye say that I cast out demons by Beelzebub.}}
\bv{19}{\redlet{And if I by Beelzebub cast out demons, by whom do your sons cast them out? therefore shall they be your judges.}}
\bv{20}{\redlet{But if I by the finger of God cast out demons, then is the kingdom of God come upon you.}}
\chapsec{Parable of the Strongman}
\bv{21}{\redlet{When the strong \supptext{man} fully armed guardeth his own court, his goods are in peace:}}
\bv{22}{\redlet{but when a stronger than he shall come upon him, and overcome him, he taketh from him his whole armor wherein he trusted, and divideth his spoils.}}
\bv{23}{\redlet{He that is not with me is against me; and he that gathereth not with me scattereth.}}
\chapsec{Stages of Sanctification}
\bv{24}{\redlet{The unclean spirit when he is gone out of the man, passeth through waterless places, seeking rest, and finding none, he saith, `I will turn back unto my house whence I came out.'}}
\bv{25}{\redlet{And when he is come, he findeth it swept and garnished.}}
\bv{26}{\redlet{Then goeth he, and taketh \supptext{to him} seven other spirits more evil than himself; and they enter in and dwell there: and the last state of that man becometh worse than the first.''}}
\chapsec{The True Family of Christ}
\bv{27}{And it came to pass, as he said these things, a certain woman out of the multitude lifted up her voice, and said unto him, ``Blessed is the womb that bare thee, and the breasts which thou didst suck.''}
\bv{28}{But he said, \redlet{``Yea rather, blessed are they that hear the word of God, and keep it.''}}
\chapsec{The Sign of Jonah}
\bv{29}{And when the multitudes were gathering together unto him, he began to say, \redlet{``This generation is an evil generation: it seeketh after a sign; and there shall no sign be given to it but the sign of Jonah.}}
\bv{30}{\redlet{For even as Jonah became a sign unto the Ninevites, so shall also the Son of man be to this generation.}}
\bv{31}{\redlet{The queen of the south shall rise up in the judgement with the men of this generation, and shall condemn them: for she came from the ends of the earth to hear the wisdom of Solomon; and behold, a greater than Solomon is here.}}
\bv{32}{\redlet{The men of Nineveh shall stand up in the judgement with this generation, and shall condemn it: for they repented at the preaching of Jonah; and behold, a greater than Jonah is here.}}
\chapsec{Parable of the Lighted Candle}
\bv{33}{\redlet{No man, when he hath lighted a lamp, putteth it in a cellar, neither under the bushel, but on the stand, that they which enter in may see the light.}}
\bv{34}{\redlet{The lamp of thy body is thine eye: when thine eye is single, thy whole body also is full of light; but when it is evil, thy body also is full of darkness.}}
\bv{35}{\redlet{Look therefore whether the light that is in thee be not darkness.}}
\bv{36}{\redlet{If therefore thy whole body be full of light, having no part dark, it shall be wholly full of light, as when the lamp with its bright shining doth give thee light.''}}
\bv{37}{Now as he spake, a Pharisee asketh him to dine with him: and he went in, and sat down to meat.}
\bv{38}{And when the Pharisee saw it, he marvelled that he had not first bathed himself before dinner.}
\chapsec{Jesus Denounces Woes upon the Pharisees}
\bv{39}{And the Lord said unto him, \redlet{``Now ye the Pharisees cleanse the outside of the cup and of the platter; but your inward part is full of extortion and wickedness.}}
\bv{40}{\redlet{Ye foolish ones, did not he that made the outside make the inside also?}}
\bv{41}{\redlet{But give for alms those things which are within; and behold, all things are clean unto you.}}
\bv{42}{\redlet{But woe unto you Pharisees! for ye tithe mint and rue and every herb, and pass over justice and the love of God: but these ought ye to have done, and not to leave the other undone.}}
\bv{43}{\redlet{Woe unto you Pharisees! for ye love the chief seats in the synagogues, and the salutations in the marketplaces.}}
\bv{44}{\redlet{Woe unto you! for ye are as the tombs which appear not, and the men that walk over \supptext{them} know it not.''}}
\chapsec{Jesus Denounces Woes upon the Lawyers}
\bv{45}{And one of the lawyers answering saith unto him, ``Teacher, in saying this thou reproachest us also.''}
\bv{46}{And he said, \redlet{``Woe unto you lawyers also! for ye load men with burdens grievous to be borne, and ye yourselves touch not the burdens with one of your fingers.}}
\bv{47}{\redlet{Woe unto you! for ye build the tombs of the prophets, and your fathers killed them.}}
\bv{48}{\redlet{So ye are witnesses and consent unto the works of your fathers: for they killed them, and ye build \supptext{their tombs}.}}
\bv{49}{\redlet{Therefore also said the wisdom of God, `I will send unto them prophets and apostles; and \supptext{some} of them they shall kill and persecute;'}}
\bv{50}{\redlet{that the blood of all the prophets, which was shed from the foundation of the world, may be required of this generation;}}
\bv{51}{\redlet{from the blood of Abel unto the blood of Zachariah, who perished between the altar and the sanctuary: yea, I say unto you, it shall be required of this generation.}}
\bv{52}{\redlet{Woe unto you lawyers! for ye took away the key of knowledge: ye entered not in yourselves, and them that were entering in ye hindered.''}}
\bv{53}{And when he was come out from thence, the scribes and the Pharisees began to press upon \supptext{him} vehemently, and to provoke him to speak of many things;}
\bv{54}{laying wait for him, to catch something out of his mouth.}
\chaphead{Chapter XII}
\chapdesc{Jesus Warns of the Leaven of the Pharisees}
\lettrine[image=true, lines=4, findent=3pt, nindent=0pt]{NT/Luke/Lk-I.eps}{n} the mean time, when the many thousands of the multitude were gathered together, insomuch that they trod one upon another, he began to say unto his disciples first of all, \redlet{``Beware ye of the leaven of the Pharisees, which is hypocrisy.}
\bv{2}{\redlet{But there is nothing covered up, that shall not be revealed; and hid, that shall not be known.}}
\bv{3}{\redlet{Wherefore whatsoever ye have said in the darkness shall be heard in the light; and what ye have spoken in the ear in the inner chambers shall be proclaimed upon the housetops.}}
\bv{4}{\redlet{And I say unto you my friends, Be not afraid of them that kill the body, and after that have no more that they can do.}}
\bv{5}{\redlet{But I will warn you whom ye shall fear: Fear him, who after he hath killed hath power to cast into hell; yea, I say unto you, Fear him.}}
\bv{6}{\redlet{Are not five sparrows sold for two assaria?}}\mcomm{Assarion: $\frac{1}{16}$\textsuperscript{th} of a denarius (see Matt. 20:2)} \redlet{and not one of them is forgotten in the sight of God.}
\bv{7}{\redlet{But the very hairs of your head are all numbered. Fear not: ye are of more value than many sparrows.}}
\bv{8}{\redlet{And I say unto you, Every one who shall confess me before men, him shall the Son of man also confess before the angels of God:}}
\bv{9}{\redlet{but he that denieth me in the presence of men shall be denied in the presence of the angels of God.}}
\bv{10}{\redlet{And every one who shall speak a word against the Son of man, it shall be forgiven him: but unto him that blasphemeth against the Holy Ghost it shall not be forgiven.}}
\bv{11}{\redlet{And when they bring you before the synagogues, and the rulers, and the authorities, be not anxious how or what ye shall answer, or what ye shall say:}}
\bv{12}{\redlet{for the Holy Ghost shall teach you in that very hour what ye ought to say.''}}
\bv{13}{And one out of the multitude said unto him, ``Teacher, bid my brother divide the inheritance with me.''}
\bv{14}{But he said unto him, \redlet{``Man, who made me a judge or a divider over you?''}}
\bv{15}{And he said unto them, \redlet{``Take heed, and keep yourselves from all covetousness: for a man's life consisteth not in the abundance of the things which he possesseth.''}}
\chapsec{Parable of the Rich Fool}
\bv{16}{And he spake a parable unto them, saying, \redlet{``The ground of a certain rich man brought forth plentifully:}}
\bv{17}{\redlet{and he reasoned within himself, saying, `What shall I do, because I have not where to bestow my fruits?'}}
\bv{18}{\redlet{And he said, `This will I do: I will pull down my barns, and build greater; and there will I bestow all my grain and my goods.}}
\bv{19}{\redlet{And I will say to my soul, `Soul, thou hast much goods laid up for many years; take thine ease, eat, drink, be merry.' '}}
\bv{20}{\redlet{But God said unto him, `Thou foolish one, this night is thy soul required of thee; and the things which thou hast prepared, whose shall they be?'}}
\bv{21}{\redlet{So is he that layeth up treasure for himself, and is not rich toward God.''}}
\chapsec{Jesus' Teaching against Anxiety}
\bv{22}{And he said unto his disciples, \redlet{``Therefore I say unto you, Be not anxious for \supptext{your} life, what ye shall eat; nor yet for your body, what ye shall put on.}}
\bv{23}{\redlet{For the life is more than the food, and the body than the raiment.}}
\bv{24}{\redlet{Consider the ravens, that they sow not, neither reap; which have no store-chamber nor barn; and God feedeth them: of how much more value are ye than the birds!}}
\bv{25}{\redlet{And which of you by being anxious can add a cubit}}\mcomm{Cubit: based on the length from the elbow to the tip of the middle finger, usually about 18 inches.} \redlet{unto the measure of his life?}
\bv{26}{\redlet{If then ye are not able to do even that which is least, why are ye anxious concerning the rest?}}
\par
\bv{27}{\redlet{Consider the lilies, how they grow: they toil not, neither do they spin; yet I say unto you, Even Solomon in all his glory was not arrayed like one of these.}}
\bv{28}{\redlet{But if God doth so clothe the grass in the field, which to-day is, and to-morrow is cast into the oven; how much more \supptext{shall he clothe} you, O ye of little faith?}}
\bv{29}{\redlet{And seek not ye what ye shall eat, and what ye shall drink, neither be ye of doubtful mind.}}
\bv{30}{\redlet{For all these things do the nations of the world seek after: but your Father knoweth that ye have need of these things.}}
\bv{31}{\redlet{Yet seek ye his kingdom, and these things shall be added unto you.}}
\bv{32}{\redlet{Fear not, little flock; for it is your Father's good pleasure to give you the kingdom.}}
\bv{33}{\redlet{Sell that which ye have, and give alms; make for yourselves purses which wax not old, a treasure in the heavens that faileth not, where no thief draweth near, neither moth destroyeth.}}
\bv{34}{\redlet{For where your treasure is, there will your heart be also.}}
\chapsec{Parable of the Second Coming}
\bv{35}{\redlet{Let your loins be girded about, and your lamps burning;}}
\bv{36}{\redlet{and be ye yourselves like unto men looking for their lord, when he shall return from the marriage feast; that, when he cometh and knocketh, they may straightway open unto him.}}
\bv{37}{\redlet{Blessed are those servants, whom the lord when he cometh shall find watching: verily I say unto you, that he shall gird himself, and make them sit down to meat, and shall come and serve them.}}
\bv{38}{\redlet{And if he shall come in the second watch, and if in the third, and find \supptext{them} so, blessed are those \supptext{servants}.}}
\bv{39}{\redlet{But know this, that if the master of the house had known in what hour the thief was coming, he would have watched, and not have left his house to be broken through.}}
\bv{40}{\redlet{Be ye also ready: for in an hour that ye think not the Son of man cometh.''}}
\bv{41}{And Peter said, ``Lord, speakest thou this parable unto us, or even unto all?''}
\chapsec{Parable of the Steward \& his Servants}
\bv{42}{And the Lord said, \redlet{``Who then is the faithful and wise steward, whom his lord shall set over his household, to give them their portion of food in due season?}}
\bv{43}{\redlet{Blessed is that servant, whom his lord when he cometh shall find so doing.}}
\bv{44}{\redlet{Of a truth I say unto you, that he will set him over all that he hath.}}
\bv{45}{\redlet{But if that servant shall say in his heart, `My lord delayeth his coming;' and shall begin to beat the menservants and the maidservants, and to eat and drink, and to be drunken;}}
\bv{46}{\redlet{the lord of that servant shall come in a day when he expecteth not, and in an hour when he knoweth not, and shall cut him asunder, and appoint his portion with the unfaithful.}}
\bv{47}{\redlet{And that servant, who knew his lord's will, and made not ready, nor did according to his will, shall be beaten with many \supptext{stripes};}}
\bv{48}{\redlet{but he that knew not, and did things worthy of stripes, shall be beaten with few \supptext{stripes}. And to whomsoever much is given, of him shall much be required: and to whom they commit much, of him will they ask the more.}}
\chapsec{Christ: A Divider of Men}
\bv{49}{\redlet{I came to cast fire upon the earth; and what do I desire, if it is already kindled?}}
\bv{50}{\redlet{But I have a baptism to be baptised with; and how am I straitened till it be accomplished!}}
\bv{51}{\redlet{Think ye that I am come to give peace in the earth? I tell you, Nay; but rather division:}}
\bv{52}{\redlet{for there shall be from henceforth five in one house divided, three against two, and two against three.}}
\bv{53}{\redlet{They shall be divided, father against son, and son against father; mother against daughter, and daughter against her mother; mother in law against her daughter in law, and daughter in law against her mother in law.''}}
\par
\bv{54}{And he said to the multitudes also, \redlet{``When ye see a cloud rising in the west, straightway ye say, `There cometh a shower;' and so it cometh to pass.}}
\bv{55}{\redlet{And when \supptext{ye see} a south wind blowing, ye say, `There will be a scorching heat;' and it cometh to pass.}}
\bv{56}{\redlet{Ye hypocrites, ye know how to interpret the face of the earth and the heaven; but how is it that ye know not how to interpret this time?}}
\bv{57}{\redlet{And why even of yourselves judge ye not what is right?}}
\bv{58}{\redlet{For as thou art going with thine adversary before the magistrate, on the way give diligence to be quit of him; lest haply he drag thee unto the judge, and the judge shall deliver thee to the officer, and the officer shall cast thee into prison.}}
\bv{59}{\redlet{I say unto thee, Thou shalt by no means come out thence, till thou have paid the very last lepton.''}\mcomm{Lepton: $\frac{1}{128}$\textsuperscript{th} of a denarius (see v. 6)}}
\chaphead{Chapter XIII}
\chapdesc{Men are not to Judge but Repent}
\lettrine[image=true, lines=4, findent=3pt, nindent=0pt]{NT/Luke/Lk-N.eps}{ow} there were some present at that very season who told him of the Galilæans, whose blood Pilate had mingled with their sacrifices.
\bv{2}{And he answered and said unto them, \redlet{``Think ye that these Galilæans were sinners above all the Galilæans, because they have suffered these things?}}
\bv{3}{\redlet{I tell you, Nay: but, except ye repent, ye shall all in like manner perish.}}
\bv{4}{\redlet{Or those eighteen, upon whom the tower in Siloam fell, and killed them, think ye that they were offenders above all the men that dwell in Jerusalem?}}
\bv{5}{\redlet{I tell you, Nay: but, except ye repent, ye shall all likewise perish.''}}
\chapsec{Parable of the Barren Fig Tree}
\bv{6}{And he spake this parable; \redlet{A certain man had a fig tree planted in his vineyard; and he came seeking fruit thereon, and found none.}}
\bv{7}{\redlet{And he said unto the vinedresser, `Behold, these three years I come seeking fruit on this fig tree, and find none: cut it down; why doth it also cumber the ground?'}}
\bv{8}{\redlet{And he answering saith unto him, `Lord, let it alone this year also, till I shall dig about it, and dung it:}}
\bv{9}{\redlet{and if it bear fruit thenceforth, \supptext{well}; but if not, thou shalt cut it down.'{''}}}
\chapsec{The Woman Loosed from her Infirmity}
\bv{10}{And he was teaching in one of the synagogues on the sabbath day.}
\bv{11}{And behold, a woman that had a spirit of infirmity eighteen years; and she was bowed together, and could in no wise lift herself up.}
\bv{12}{And when Jesus saw her, he called her, and said to her, \redlet{``Woman, thou art loosed from thine infirmity.''}}
\bv{13}{And he laid his hands upon her: and immediately she was made straight, and glorified God.}
\bv{14}{And the ruler of the synagogue, being moved with indignation because Jesus had healed on the sabbath, answered and said to the multitude, ``There are six days in which men ought to work: in them therefore come and be healed, and not on the day of the sabbath.''}
\bv{15}{But the Lord answered him, and said, \redlet{``Ye hypocrites, doth not each one of you on the sabbath loose his ox or his ass from the stall, and lead him away to watering?}}
\bv{16}{\redlet{And ought not this woman, being a daughter of Abraham, whom Satan had bound, lo, \supptext{these} eighteen years, to have been loosed from this bond on the day of the sabbath?''}}
\bv{17}{And as he said these things, all his adversaries were put to shame: and all the multitude rejoiced for all the glorious things that were done by him.}
\chapsec{Parable of the Mustard Seed}
\bv{18}{He said therefore, \redlet{``Unto what is the kingdom of God like? and whereunto shall I liken it?}}
\bv{19}{\redlet{It is like unto a grain of mustard seed, which a man took, and cast into his own garden; and it grew, and became a tree; and the birds of the heaven lodged in the branches thereof.''}}
\chapsec{Parable of the Leaven}
\bv{20}{And again he said, \redlet{Whereunto shall I liken the kingdom of God?}}
\bv{21}{\redlet{It is like unto leaven, which a woman took and hid in three measures of meal, till it was all leavened.''}}
\chapsec{Teachings on the Way to Jerusalem}
\bv{22}{And he went on his way through cities and villages, teaching, and journeying on unto Jerusalem.}
\bv{23}{And one said unto him, ``Lord, are they few that are saved?'' And he said unto them,}
\bv{24}{\redlet{``Strive to enter in by the narrow door: for many, I say unto you, shall seek to enter in, and shall not be able.}}
\bv{25}{\redlet{When once the master of the house is risen up, and hath shut to the door, and ye begin to stand without, and to knock at the door, saying, `Lord, open to us;' and he shall answer and say to you, `I know you not whence ye are;'}}
\bv{26}{\redlet{then shall ye begin to say, `We did eat and drink in thy presence, and thou didst teach in our streets;'}}
\bv{27}{\redlet{and he shall say, `I tell you, I know not whence ye are; depart from me, all ye workers of iniquity.'}}
\bv{28}{\redlet{There shall be the weeping and the gnashing of teeth, when ye shall see Abraham, and Isaac, and Jacob, and all the prophets, in the kingdom of God, and yourselves cast forth without.}}
\bv{29}{\redlet{And they shall come from the east and west, and from the north and south, and shall sit down in the kingdom of God.}}
\bv{30}{\redlet{And behold, there are last who shall be first, and there are first who shall be last.''}}
\par
\bv{31}{In that very hour there came certain Pharisees, saying to him, ``Get thee out, and go hence: for Herod would fain kill thee.''}
\bv{32}{And he said unto them, \redlet{``Go and say to that fox, `Behold, I cast out demons and perform cures to-day and to-morrow, and the third \supptext{day} I am perfected.'}}
\bv{33}{\redlet{Nevertheless I must go on my way to-day and to-morrow and the \supptext{day} following: for it cannot be that a prophet perish out of Jerusalem.}}
\chapsec{Jesus' Lament over Jerusalem}
\bv{34}{\redlet{O Jerusalem, Jerusalem, that killeth the prophets, and stoneth them that are sent unto her! how often would I have gathered thy children together, even as a hen \supptext{gathereth} her own brood under her wings, and ye would not!}}
\bv{35}{\redlet{Behold, your house is left unto you \supptext{desolate}: and I say unto you, Ye shall not see me, until ye shall say, `Blessed \supptext{is} he that cometh in the name of the Lord.'}}
\chaphead{Chapter XIV}
\chapdesc{Jesus Heals on the Sabbath}
\lettrine[image=true, lines=4, findent=3pt, nindent=0pt]{NT/Luke/Lk-A.eps}{nd} it came to pass, when he went into the house of one of the rulers of the Pharisees on a sabbath to eat bread, that they were watching him.
\bv{2}{And behold, there was before him a certain man that had the dropsy.}
\bv{3}{And Jesus answering spake unto the lawyers and Pharisees, saying, \redlet{``Is it lawful to heal on the sabbath, or not?''}}
\bv{4}{But they held their peace. And he took him, and healed him, and let him go.}
\bv{5}{And he said unto them, \redlet{``Which of you shall have an ass or an ox fallen into a well, and will not straightway draw him up on a sabbath day?''}}
\bv{6}{And they could not answer again unto these things.}
\chapsec{Parable of the Ambitious Guest}
\bv{7}{And he spake a parable unto those that were bidden, when he marked how they chose out the chief seats; saying unto them,}
\bv{8}{\redlet{``When thou art bidden of any man to a marriage feast, sit not down in the chief seat; lest haply a more honorable man than thou be bidden of him,}}
\bv{9}{\redlet{and he that bade thee and him shall come and say to thee, `Give this man place;' and then thou shalt begin with shame to take the lowest place.}}
\bv{10}{\redlet{But when thou art bidden, go and sit down in the lowest place; that when he that hath bidden thee cometh, he may say to thee, `Friend, go up higher:' then shalt thou have glory in the presence of all that sit at meat with thee.}}
\bv{11}{\redlet{For every one that exalteth himself shall be humbled; and he that humbleth himself shall be exalted.''}}
\par
\bv{12}{And he said to him also that had bidden him, \redlet{``When thou makest a dinner or a supper, call not thy friends, nor thy brethren, nor thy kinsmen, nor rich neighbours; lest haply they also bid thee again, and a recompense be made thee.}}
\bv{13}{\redlet{But when thou makest a feast, bid the poor, the maimed, the lame, the blind:}}
\bv{14}{\redlet{and thou shalt be blessed; because they have not \supptext{wherewith} to recompense thee: for thou shalt be recompensed in the resurrection of the just.''}}
\bv{15}{And when one of them that sat at meat with him heard these things, he said unto him, ``Blessed is he that shall eat bread in the kingdom of God.''}
\chapsec{Parable of the Great Supper}
\bv{16}{But he said unto him, \redlet{``A certain man made a great supper; and he bade many:}}
\bv{17}{\redlet{and he sent forth his servant at supper time to say to them that were bidden, `Come; for \supptext{all} things are now ready.'}}
\bv{18}{\redlet{And they all with one \supptext{consent} began to make excuse. The first said unto him, `I have bought a field, and I must needs go out and see it; I pray thee have me excused.'}}
\bv{19}{\redlet{And another said, `I have bought five yoke of oxen, and I go to prove them; I pray thee have me excused.'}}
\bv{20}{\redlet{And another said, `I have married a wife, and therefore I cannot come.'}}
\bv{21}{\redlet{And the servant came, and told his lord these things. Then the master of the house being angry said to his servant, `Go out quickly into the streets and lanes of the city, and bring in hither the poor and maimed and blind and lame.'}}
\bv{22}{\redlet{And the servant said, `Lord, what thou didst command is done, and yet there is room.'}}
\bv{23}{\redlet{And the lord said unto the servant, `Go out into the highways and hedges, and constrain \supptext{them} to come in, that my house may be filled.}}
\bv{24}{\redlet{For I say unto you, that none of those men that were bidden shall taste of my supper.'{''}}}
\chapsec{Discipleship Again Tested}
\bv{25}{Now there went with him great multitudes: and he turned, and said unto them,}
\bv{26}{\redlet{``If any man cometh unto me, and hateth not his own father, and mother, and wife, and children, and brethren, and sisters, yea, and his own life also, he cannot be my disciple.}}
\bv{27}{\redlet{Whosoever doth not bear his own cross, and come after me, cannot be my disciple.}}
\chapsec{Parable of the Tower}
\bv{28}{\redlet{For which of you, desiring to build a tower, doth not first sit down and count the cost, whether he have \supptext{wherewith} to complete it?}}
\bv{29}{\redlet{Lest haply, when he hath laid a foundation, and is not able to finish, all that behold begin to mock him,}}
\bv{30}{\redlet{saying, `This man began to build, and was not able to finish.'}}
\chapsec{Parable of the King Going to War}
\bv{31}{\redlet{Or what king, as he goeth to encounter another king in war, will not sit down first and take counsel whether he is able with ten thousand to meet him that cometh against him with twenty thousand?}}
\bv{32}{\redlet{Or else, while the other is yet a great way off, he sendeth an ambassage, and asketh conditions of peace.}}
\bv{33}{\redlet{So therefore whosoever he be of you that renounceth not all that he hath, he cannot be my disciple.}}
\chapsec{Parable of the Savourless Salt}
\bv{34}{\redlet{Salt therefore is good: but if even the salt have lost its savor, wherewith shall it be seasoned?}}
\bv{35}{\redlet{It is fit neither for the land nor for the dunghill: \supptext{men} cast it out. He that hath ears to hear, let him hear.''}}
\chaphead{Chapter XV}
\chapdesc{The Murmuring Pharisees}
\lettrine[image=true, lines=4, findent=3pt, nindent=0pt]{NT/Luke/Lk-N.eps}{ow} all the publicans and sinners were drawing near unto him to hear him.
\bv{2}{And both the Pharisees and the scribes murmured, saying, ``This man receiveth sinners, and eateth with them.''}
\chapsec{Parable of the Lost Sheep}
\bv{3}{And he spake unto them this parable, saying,}
\bv{4}{\redlet{``What man of you, having a hundred sheep, and having lost one of them, doth not leave the ninety and nine in the wilderness, and go after that which is lost, until he find it?}}
\bv{5}{\redlet{And when he hath found it, he layeth it on his shoulders, rejoicing.}}
\bv{6}{\redlet{And when he cometh home, he calleth together his friends and his neighbours, saying unto them, `Rejoice with me, for I have found my sheep which was lost.'}}
\bv{7}{\redlet{I say unto you, that even so there shall be joy in heaven over one sinner that repenteth, \supptext{more} than over ninety and nine righteous persons, who need no repentance.}}
\chapsec{Parable of the Lost Coin}
\bv{8}{\redlet{Or what woman having ten pieces of silver, if she lose one piece, doth not light a lamp, and sweep the house, and seek diligently until she find it?}}
\bv{9}{\redlet{And when she hath found it, she calleth together her friends and neighbours, saying, `Rejoice with me, for I have found the piece which I had lost.'}}
\bv{10}{\redlet{Even so, I say unto you, there is joy in the presence of the angels of God over one sinner that repenteth.''}}
\chapsec{Parable of the Prodigal Son}
\bv{11}{And he said, \redlet{``A certain man had two sons:}}
\chapsec{(The Departure)}
\bv{12}{\redlet{and the younger of them said to his father, `Father, give me the portion of \supptext{thy} substance that falleth to me.' And he divided unto them his living.}}
\bv{13}{\redlet{And not many days after, the younger son gathered all together and took his journey into a far country; and there he wasted his substance with riotous living.}}
\chapsec{(The Misery of the Far Country)}
\bv{14}{\redlet{And when he had spent all, there arose a mighty famine in that country; and he began to be in want.}}
\bv{15}{\redlet{And he went and joined himself to one of the citizens of that country; and he sent him into his fields to feed swine.}}
\bv{16}{\redlet{And he would fain have filled his belly with the husks that the swine did eat: and no man gave unto him.}}
\chapsec{(The Repentance)}
\bv{17}{\redlet{But when he came to himself he said, `How many hired servants of my father's have bread enough and to spare, and I perish here with hunger!}}
\bv{18}{\redlet{I will arise and go to my father, and will say unto him, Father, I have sinned against heaven, and in thy sight:}}
\bv{19}{\redlet{I am no more worthy to be called thy son: make me as one of thy hired servants.'}}
\chapsec{(The Return \& the Father)}
\bv{20}{\redlet{And he arose, and came to his father. But while he was yet afar off, his father saw him, and was moved with compassion, and ran, and fell on his neck, and kissed him.}}
\bv{21}{\redlet{And the son said unto him, `Father, I have sinned against heaven, and in thy sight: I am no more worthy to be called thy son.'}}
\bv{22}{\redlet{But the father said to his servants, `Bring forth quickly the best robe, and put it on him; and put a ring on his hand, and shoes on his feet:}}
\chapsec{(The Rejoicing)}
\bv{23}{\redlet{and bring the fatted calf, \supptext{and} kill it, and let us eat, and make merry:}}
\bv{24}{\redlet{for this my son was dead, and is alive again; he was lost, and is found.' And they began to be merry.}}
\bv{25}{\redlet{Now his elder son was in the field: and as he came and drew nigh to the house, he heard music and dancing.}}
\chapsec{(The Envy of the Eldest Son)}
\bv{26}{\redlet{And he called to him one of the servants, and inquired what these things might be.}}
\bv{27}{\redlet{And he said unto him, `Thy brother is come; and thy father hath killed the fatted calf, because he hath received him safe and sound.'}}
\bv{28}{\redlet{But he was angry, and would not go in: and his father came out, and entreated him.}}
\bv{29}{\redlet{But he answered and said to his father, `Lo, these many years do I serve thee, and I never transgressed a commandment of thine; and \supptext{yet} thou never gavest me a kid, that I might make merry with my friends:}}
\bv{30}{\redlet{but when this thy son came, who hath devoured thy living with harlots, thou killedst for him the fatted calf.'}}
\bv{31}{\redlet{And he said unto him, `Son, thou art ever with me, and all that is mine is thine.}}
\bv{32}{\redlet{But it was meet to make merry and be glad: for this thy brother was dead, and is alive \supptext{again}; and \supptext{was} lost, and is found.'{''}}}
\chaphead{Chapter XVI}
\chapdesc{Parable of the Unjust Steward}
\lettrine[image=true, lines=4, findent=3pt, nindent=0pt]{NT/Luke/Lk-A.eps}{nd} he said also unto the disciples, \redlet{``There was a certain rich man, who had a steward; and the same was accused unto him that he was wasting his goods.}
\bv{2}{\redlet{And he called him, and said unto him, `What is this that I hear of thee? render the account of thy stewardship; for thou canst be no longer steward.'}}
\bv{3}{\redlet{And the steward said within himself, `What shall I do, seeing that my lord taketh away the stewardship from me? I have not strength to dig; to beg I am ashamed.}}
\bv{4}{\redlet{I am resolved what to do, that, when I am put out of the stewardship, they may receive me into their houses.'}}
\par
\bv{5}{\redlet{And calling to him each one of his lord's debtors, he said to the first, `How much owest thou unto my lord?'}}
\bv{6}{\redlet{And he said, `A hundred measures of oil.' And he said unto him, `Take thy bond, and sit down quickly and write fifty.'}}
\bv{7}{\redlet{Then said he to another, `And how much owest thou?' And he said, `A hundred measures of wheat.' He saith unto him, `Take thy bond, and write fourscore.'}}
\bv{8}{\redlet{And his lord commended the unrighteous steward because he had done wisely: for the sons of this world are for their own generation wiser than the sons of the light.}}
\par
\bv{9}{\redlet{And I say unto you, Make to yourselves friends by means of the mammon of unrighteousness; that, when it shall fail, they may receive you into the eternal tabernacles.}}
\bv{10}{\redlet{He that is faithful in a very little is faithful also in much: and he that is unrighteous in a very little is unrighteous also in much.}}
\bv{11}{\redlet{If therefore ye have not been faithful in the unrighteous mammon, who will commit to your trust the true \supptext{riches}?}}
\bv{12}{\redlet{And if ye have not been faithful in that which is another's, who will give you that which is your own?}}
\bv{13}{\redlet{No servant can serve two masters: for either he will hate the one, and love the other; or else he will hold to one, and despise the other. Ye cannot serve God and mammon.''}\mcomm{Mammon, that is, money.}}
\chapsec{Jesus Answers the Pharisees}
\bv{14}{And the Pharisees, who were lovers of money, heard all these things; and they scoffed at him.}
\bv{15}{And he said unto them, \redlet{``Ye are they that justify yourselves in the sight of men; but God knoweth your hearts: for that which is exalted among men is an abomination in the sight of God.}}
\bv{16}{\redlet{The law and the prophets \supptext{were} until John: from that time the gospel of the kingdom of God is preached, and every man entereth violently into it.}}
\bv{17}{\redlet{But it is easier for heaven and earth to pass away, than for one tittle of the law to fall.}}
\chapsec{Jesus \& Divorce}
\bv{18}{\redlet{Every one that putteth away his wife, and marrieth another, committeth adultery: and he that marrieth one that is put away from a husband committeth adultery.}}
\chapsec{The Rich Man \& Lazarus}
\bv{19}{\redlet{Now there was a certain rich man, and he was clothed in purple and fine linen, faring sumptuously every day:}}
\bv{20}{\redlet{and a certain beggar named Lazarus was laid at his gate, full of sores,}}
\bv{21}{\redlet{and desiring to be fed with the \supptext{crumbs} that fell from the rich man's table; yea, even the dogs came and licked his sores.}}
\bv{22}{\redlet{And it came to pass, that the beggar died, and that he was carried away by the angels into Abraham's bosom: and the rich man also died, and was buried.}}
\bv{23}{\redlet{And in Hades he lifted up his eyes, being in torments, and seeth Abraham afar off, and Lazarus in his bosom.}}
\par
\bv{24}{\redlet{And he cried and said, `Father Abraham, have mercy on me, and send Lazarus, that he may dip the tip of his finger in water, and cool my tongue; for I am in anguish in this flame.'}}
\bv{25}{\redlet{But Abraham said, `Son, remember that thou in thy lifetime receivedst thy good things, and Lazarus in like manner evil things: but now here he is comforted, and thou art in anguish.}}
\bv{26}{\redlet{And besides all this, between us and you there is a great gulf fixed, that they that would pass from hence to you may not be able, and that none may cross over from thence to us.'}}
\bv{27}{\redlet{And he said, `I pray thee therefore, father, that thou wouldest send him to my father's house;}}
\bv{28}{\redlet{for I have five brethren; that he may testify unto them, lest they also come into this place of torment.'}}
\bv{29}{\redlet{But Abraham saith, `They have Moses and the prophets; let them hear them.'}}
\bv{30}{\redlet{And he said, `Nay, father Abraham: but if one go to them from the dead, they will repent.'}}
\bv{31}{\redlet{And he said unto him, `If they hear not Moses and the prophets,}\mcomm{``To be sure, the sacred and divinely inspired Scriptures are sufficient for the exposition of the truth.'' -St. Athanasius} \redlet{neither will they be persuaded, if one rise from the dead.'{''}}}
\chaphead{Chapter XVII}
\chapdesc{An Instruction in Forgiveness}
\lettrine[image=true, lines=4, findent=3pt, nindent=0pt]{NT/Luke/Lk-A.eps}{nd} he said unto his disciples, \redlet{``It is impossible but that occasions of stumbling should come; but woe unto him, through whom they come!}
\bv{2}{\redlet{It were well for him if a millstone were hanged about his neck, and he were thrown into the sea, rather than that he should cause one of these little ones to stumble.}}
\bv{3}{\redlet{Take heed to yourselves: if thy brother sin, rebuke him; and if he repent, forgive him.}}
\bv{4}{\redlet{And if he sin against thee seven times in the day, and seven times turn again to thee, saying, `I repent;' thou shalt forgive him.''}}
\bv{5}{And the apostles said unto the Lord, ``Increase our faith.''}
\bv{6}{And the Lord said, \redlet{``If ye had faith as a grain of mustard seed, ye would say unto this sycamine tree, `Be thou rooted up, and be thou planted in the sea;' and it would obey you.}}
\chapsec{A Parable of Service}
\bv{7}{\redlet{But who is there of you, having a servant plowing or keeping sheep, that will say unto him, when he is come in from the field, `Come straightway and sit down to meat;'}}
\bv{8}{\redlet{and will not rather say unto him, `Make ready wherewith I may sup, and gird thyself, and serve me, till I have eaten and drunken; and afterward thou shalt eat and drink?'}}
\bv{9}{\redlet{Doth he thank the servant because he did the things that were commanded?}}
\bv{10}{\redlet{Even so ye also, when ye shall have done all the things that are commanded you, say, `We are unprofitable servants; we have done that which it was our duty to do.'{''}}\mcomm{The unworthiness of man and the impiety of supererogatory merit.}}
\chapsec{Ten Lepers Healed}
\bv{11}{And it came to pass, as they were on the way to Jerusalem, that he was passing along the borders of Samaria and Galilee.}
\bv{12}{And as he entered into a certain village, there met him ten men that were lepers, who stood afar off:}
\bv{13}{and they lifted up their voices, saying, ``Jesus, Master, have mercy on us.''}
\bv{14}{And when he saw them, he said unto them, \redlet{``Go and show yourselves unto the priests.''} And it came to pass, as they went, they were cleansed.}
\bv{15}{And one of them, when he saw that he was healed, turned back, with a loud voice glorifying God;}
\bv{16}{and he fell upon his face at his feet, giving him thanks: and he was a Samaritan.}
\bv{17}{And Jesus answering said, \redlet{``Were not the ten cleansed? but where are the nine?}}
\bv{18}{\redlet{Were there none found that returned to give glory to God, save this stranger?''}}
\bv{19}{And he said unto him, \redlet{``Arise, and go thy way: thy faith hath made thee whole.''}}
\chapsec{The Coming of the Kingdom}
\bv{20}{And being asked by the Pharisees, when the kingdom of God cometh, he answered them and said, \redlet{``The kingdom of God cometh not with observation:}}
\bv{21}{\redlet{neither shall they say, `Lo, here!' or, `There!' for lo, the kingdom of God is within you.}}
\chapsec{Jesus Foretells his Second Coming}
\bv{22}{And he said unto the disciples, \redlet{``The days will come, when ye shall desire to see one of the days of the Son of man, and ye shall not see it.}}
\bv{23}{\redlet{And they shall say to you, `Lo, there! Lo, here!' go not away, nor follow after \supptext{them}:}}
\bv{24}{\redlet{for as the lightning, when it lighteneth out of the one part under the heaven, shineth unto the other part under heaven; so shall the Son of man be in his day.}}
\bv{25}{\redlet{But first must he suffer many things and be rejected of this generation.}}
\bv{26}{\redlet{And as it came to pass in the days of Noah, even so shall it be also in the days of the Son of man.}}
\bv{27}{\redlet{They ate, they drank, they married, they were given in marriage, until the day that Noah entered into the ark, and the flood came, and destroyed them all.}}
\bv{28}{\redlet{Likewise even as it came to pass in the days of Lot; they ate, they drank, they bought, they sold, they planted, they builded;}}
\bv{29}{\redlet{but in the day that Lot went out from Sodom it rained fire and brimstone from heaven, and destroyed them all:}}
\bv{30}{\redlet{after the same manner shall it be in the day that the Son of man is revealed.}}
\bv{31}{\redlet{In that day, he that shall be on the housetop, and his goods in the house, let him not go down to take them away: and let him that is in the field likewise not return back.}}
\par
\bv{32}{\redlet{Remember Lot's wife.}}
\bv{33}{\redlet{Whosoever shall seek to gain his life shall lose it: but whosoever shall lose \supptext{his life} shall preserve it.}}
\bv{34}{\redlet{I say unto you, In that night there shall be two men on one bed; the one shall be taken, and the other shall be left.}}
\bv{35}{\redlet{There shall be two women grinding together; the one shall be taken, and the other shall be left.''}}\mcomm{There shall be two men in the field; the one shall be taken and the other shall be left.}
\bv{37}{And they answering say unto him, ``Where, Lord?'' And he said unto them, \redlet{``Where the body \supptext{is}, thither will the eagles also be gathered together.''}}
\chaphead{Chapter XVIII}
\chapdesc{Parable of the Unjust Judge}
\lettrine[image=true, lines=4, findent=3pt, nindent=0pt]{NT/Luke/Lk-A.eps}{nd} he spake a parable unto them to the end that they ought always to pray, and not to faint;
\bv{2}{saying, \redlet{``There was in a city a judge, who feared not God, and regarded not man:}}
\bv{3}{\redlet{and there was a widow in that city; and she came oft unto him, saying, `Avenge me of mine adversary.'}}
\bv{4}{\redlet{And he would not for a while: but afterward he said within himself, `Though I fear not God, nor regard man;}}
\bv{5}{\redlet{yet because this widow troubleth me, I will avenge her, lest she wear me out by her continual coming.'{''}}}
\bv{6}{And the Lord said, \redlet{``Hear what the unrighteous judge saith.}}
\par
\bv{7}{\redlet{And shall not God avenge his elect, that cry to him day and night, and \supptext{yet} he is longsuffering over them?}}
\bv{8}{\redlet{I say unto you, that he will avenge them speedily. Nevertheless, when the Son of man cometh, shall he find faith on the earth?''}}
\chapsec{Parable of the Pharisee \& Publican}
\bv{9}{And he spake also this parable unto certain who trusted in themselves that they were righteous, and set all others at nought:}
\bv{10}{\redlet{Two men went up into the temple to pray; the one a Pharisee, and the other a publican.}}
\bv{11}{\redlet{The Pharisee stood and prayed thus with himself, `God, I thank thee, that I am not as the rest of men, extortioners, unjust, adulterers, or even as this publican.}}
\bv{12}{\redlet{I fast twice in the week; I give tithes of all that I get.'}}
\bv{13}{\redlet{But the publican, standing afar off, would not lift up so much as his eyes unto heaven, but smote his breast, saying, `God, be thou merciful to me a sinner.'}}
\bv{14}{\redlet{I say unto you, This man went down to his house justified rather than the other: for every one that exalteth himself shall be humbled; but he that humbleth himself shall be exalted.''}}
\chapsec{Jesus Blesses Little Children}
\bv{15}{And they were bringing unto him also their babes, that he should touch them: but when the disciples saw it, they rebuked them.}
\bv{16}{But Jesus called them unto him, saying, \redlet{``Suffer the little children to come unto me, and forbid them not: for to such belongeth the kingdom of God.}}
\bv{17}{\redlet{Verily I say unto you, Whosoever shall not receive the kingdom of God as a little child, he shall in no wise enter therein.''}}
\chapsec{The Rich Young Ruler}
\bv{18}{And a certain ruler asked him, saying, ``Good Teacher, what shall I do to inherit eternal life?''}
\bv{19}{And Jesus said unto him, \redlet{``Why callest thou me good? none is good, save one, \supptext{even} God.}}
\bv{20}{\redlet{Thou knowest the commandments, `Do not commit adultery, Do not kill, Do not steal, Do not bear false witness, Honor thy father and mother.'{''}}}
\bv{21}{And he said, ``All these things have I observed from my youth up.''}
\bv{22}{And when Jesus heard it, he said unto him, \redlet{``One thing thou lackest yet: sell all that thou hast, and distribute unto the poor, and thou shalt have treasure in heaven: and come, follow me.''}}
\bv{23}{But when he heard these things, he became exceeding sorrowful; for he was very rich.}
\bv{24}{And Jesus seeing him said, \redlet{``How hardly shall they that have riches enter into the kingdom of God!}}
\bv{25}{\redlet{For it is easier for a camel to enter in through a needle's eye, than for a rich man to enter into the kingdom of God.''}}
\bv{26}{And they that heard it said, ``Then who can be saved?''}
\bv{27}{But he said, \redlet{``The things which are impossible with men are possible with God.''}}
\bv{28}{And Peter said, ``Lo, we have left our own, and followed thee.''}
\bv{29}{And he said unto them, \redlet{``Verily I say unto you, There is no man that hath left house, or wife, or brethren, or parents, or children, for the kingdom of God's sake,}}
\bv{30}{\redlet{who shall not receive manifold more in this time, and in the world to come eternal life.''}}
\chapsec{Jesus again Foretells his Death \& Resurrection}
\bv{31}{And he took unto him the twelve, and said unto them, \redlet{``Behold, we go up to Jerusalem, and all the things that are written through the prophets shall be accomplished unto the Son of man.}}
\bv{32}{\redlet{For he shall be delivered up unto the Gentiles, and shall be mocked, and shamefully treated, and spit upon:}}
\bv{33}{\redlet{and they shall scourge and kill him: and the third day he shall rise again.''}}
\bv{34}{And they understood none of these things; and this saying was hid from them, and they perceived not the things that were said.}
\chapsec{A Blind Man Healed near Jericho}
\bv{35}{And it came to pass, as he drew nigh unto Jericho, a certain blind man sat by the way side begging:}
\bv{36}{and hearing a multitude going by, he inquired what this meant.}
\bv{37}{And they told him, that Jesus of Nazareth passeth by.}
\bv{38}{And he cried, saying, ``Jesus, thou son of David, have mercy on me.''}
\bv{39}{And they that went before rebuked him, that he should hold his peace: but he cried out the more a great deal, ``Thou son of David, have mercy on me.''}
\bv{40}{And Jesus stood, and commanded him to be brought unto him: and when he was come near, he asked him,}
\bv{41}{\redlet{``What wilt thou that I should do unto thee?''} And he said, ``Lord, that I may receive my sight.''}
\bv{42}{And Jesus said unto him, \redlet{``Receive thy sight: thy faith hath made thee whole.''}}
\bv{43}{And immediately he received his sight, and followed him, glorifying God: and all the people, when they saw it, gave praise unto God.}
\chaphead{Chapter XIX}
\chapdesc{Conversion of Zacchaeus}
\lettrine[image=true, lines=4, findent=3pt, nindent=0pt]{NT/Luke/Lk-A.eps}{nd} he entered and was passing through Jericho.
\bv{2}{And behold, a man called by name Zacchæus; and he was a chief publican, and he was rich.}
\bv{3}{And he sought to see Jesus who he was; and could not for the crowd, because he was little of stature.}
\bv{4}{And he ran on before, and climbed up into a sycomore tree to see him: for he was to pass that way.}
\bv{5}{And when Jesus came to the place, he looked up, and said unto him, \redlet{``Zacchæus, make haste, and come down; for to-day I must abide at thy house.''}}
\bv{6}{And he made haste, and came down, and received him joyfully.}
\bv{7}{And when they saw it, they all murmured, saying, ``He is gone in to lodge with a man that is a sinner.''}
\bv{8}{And Zacchæus stood, and said unto the Lord, ``Behold, Lord, the half of my goods I give to the poor; and if I have wrongfully exacted aught of any man, I restore fourfold.''}
\bv{9}{And Jesus said unto him, \redlet{``To-day is salvation come to this house, forasmuch as he also is a son of Abraham.}}
\bv{10}{\redlet{For the Son of man came to seek and to save that which was lost.}}
\chapsec{Parable of the Ten Minas}
\bv{11}{And as they heard these things, he added and spake a parable, because he was nigh to Jerusalem, and \supptext{because} they supposed that the kingdom of God was immediately to appear.}
\bv{12}{He said therefore, \redlet{``A certain nobleman went into a far country, to receive for himself a kingdom, and to return.}}
\bv{13}{\redlet{And he called ten servants of his, and gave them ten minas,}}\mcomm{Mina: about three-month's wages} \redlet{and said unto them, `Trade ye \supptext{herewith} till I come.'}
\bv{14}{\redlet{But his citizens hated him, and sent an ambassage after him, saying, `We will not that this man reign over us.'}}
\bv{15}{\redlet{And it came to pass, when he was come back again, having received the kingdom, that he commanded these servants, unto whom he had given the money, to be called to him, that he might know what they had gained by trading.}}
\bv{16}{\redlet{And the first came before him, saying, `Lord, thy mina hath made ten minas more.'}}
\bv{17}{\redlet{And he said unto him, `Well done, thou good servant: because thou wast found faithful in a very little, have thou authority over ten cities.'}}
\bv{18}{\redlet{And the second came, saying, `Thy mina, Lord, hath made five minas.'}}
\bv{19}{\redlet{And he said unto him also, `Be thou also over five cities.'}}
\bv{20}{\redlet{And another came, saying, `Lord, behold, \supptext{here is} thy mina, which I kept laid up in a napkin:}}
\bv{21}{\redlet{for I feared thee, because thou art an austere man: thou takest up that which thou layedst not down, and reapest that which thou didst not sow.'}}
\bv{22}{\redlet{He saith unto him, `Out of thine own mouth will I judge thee, thou wicked servant. Thou knewest that I am an austere man, taking up that which I laid not down, and reaping that which I did not sow;}}
\bv{23}{\redlet{then wherefore gavest thou not my money into the bank, and I at my coming should have required it with interest?'}}
\bv{24}{\redlet{And he said unto them that stood by, `Take away from him the mina, and give it unto him that hath the ten minas.'}}
\bv{25}{\redlet{And they said unto him, `Lord, he hath ten minas.'}}
\bv{26}{\redlet{`I say unto you, that unto every one that hath shall be given; but from him that hath not, even that which he hath shall be taken away from him.}}
\bv{27}{\redlet{But these mine enemies, that would not that I should reign over them, bring hither, and slay them before me.'{''}}}
\chapsec{The Triumphal Entry}
\bv{28}{And when he had thus spoken, he went on before, going up to Jerusalem.}
\bv{29}{And it came to pass, when he drew nigh unto Bethphage and Bethany, at the mount that is called Olivet, he sent two of the disciples,}
\bv{30}{saying, \redlet{``Go your way into the village over against \supptext{you}; in which as ye enter ye shall find a colt tied, whereon no man ever yet sat: loose him, and bring him.}}
\bv{31}{\redlet{And if any one ask you, `Why do ye loose him?' thus shall ye say, `The Lord hath need of him.'{''}}}
\bv{32}{And they that were sent went away, and found even as he had said unto them.}
\bv{33}{And as they were loosing the colt, the owners thereof said unto them, ``Why loose ye the colt?''}
\bv{34}{And they said, ``The Lord hath need of him.''}
\bv{35}{And they brought him to Jesus: and they threw their garments upon the colt, and set Jesus thereon.}
\bv{36}{And as he went, they spread their garments in the way.}
\bv{37}{And as he was now drawing nigh, \supptext{even} at the descent of the mount of Olives, the whole multitude of the disciples began to rejoice and praise God with a loud voice for all the mighty works which they had seen;}
\bv{38}{saying, ``Blessed \supptext{is} the King that cometh in the name of the Lord: peace in heaven, and glory in the highest.''}
\bv{39}{And some of the Pharisees from the multitude said unto him, ``Teacher, rebuke thy disciples.''}
\bv{40}{And he answered and said, \redlet{``I tell you that, if these shall hold their peace, the stones will cry out.''}}
\chapsec{Jesus Weeps over Jerusalem}
\bv{41}{And when he drew nigh, he saw the city and wept over it,}
\bv{42}{saying, \redlet{``If thou hadst known in this day, even thou, the things which belong unto peace! but now they are hid from thine eyes.}}
\bv{43}{\redlet{For the days shall come upon thee, when thine enemies shall cast up a bank about thee, and compass thee round, and keep thee in on every side,}}
\bv{44}{\redlet{and shall dash thee to the ground, and thy children within thee; and they shall not leave in thee one stone upon another; because thou knewest not the time of thy visitation.''}}
\par
\bv{45}{And he entered into the temple, and began to cast out them that sold,}
\bv{46}{saying unto them,} \redlet{``It is written,
\otQuote{Is. 56:7}{And my house shall be a house of prayer: but ye have made it a den of robbers.''}}
\bv{47}{And he was teaching daily in the temple. But the chief priests and the scribes and the principal men of the people sought to destroy him:}
\bv{48}{and they could not find what they might do; for the people all hung upon him, listening.}
\chaphead{Chapter XX}
\chapdesc{Jesus' Authority Questioned}
\lettrine[image=true, lines=4, findent=3pt, nindent=0pt]{NT/Luke/Lk-A.eps}{nd} it came to pass, on one of the days, as he was teaching the people in the temple, and preaching the gospel, there came upon him the chief priests and the scribes with the elders;
\bv{2}{and they spake, saying unto him, ``Tell us: By what authority doest thou these things? or who is he that gave thee this authority?''}
\bv{3}{And he answered and said unto them, \redlet{``I also will ask you a question; and tell me:}}
\bv{4}{\redlet{The baptism of John, was it from heaven, or from men?''}}
\bv{5}{And they reasoned with themselves, saying, ``If we shall say, `From heaven;' he will say, `Why did ye not believe him?'}
\bv{6}{But if we shall say, `From men;' all the people will stone us: for they are persuaded that John was a prophet.''}
\bv{7}{And they answered, that they knew not whence \supptext{it was}.}
\bv{8}{And Jesus said unto them, \redlet{``Neither tell I you by what authority I do these things.''}}
\chapsec{Parable of the Vineyard}
\bv{9}{And he began to speak unto the people this parable: \redlet{``A man planted a vineyard, and let it out to husbandmen, and went into another country for a long time.}}
\bv{10}{\redlet{And at the season he sent unto the husbandmen a servant, that they should give him of the fruit of the vineyard: but the husbandmen beat him, and sent him away empty.}}
\bv{11}{\redlet{And he sent yet another servant: and him also they beat, and handled him shamefully, and sent him away empty.}}
\bv{12}{\redlet{And he sent yet a third: and him also they wounded, and cast him forth.}}
\bv{13}{\redlet{And the lord of the vineyard said, `What shall I do? I will send my beloved son; it may be they will reverence him.'}}
\bv{14}{\redlet{But when the husbandmen saw him, they reasoned one with another, saying, `This is the heir; let us kill him, that the inheritance may be ours.'}}
\bv{15}{\redlet{And they cast him forth out of the vineyard, and killed him. What therefore will the lord of the vineyard do unto them?}}
\bv{16}{\redlet{He will come and destroy these husbandmen, and will give the vineyard unto others.''} And when they heard it, they said, ``God forbid.''}
\bv{17}{But he looked upon them, and said,} \redlet{``What then is this that is written,
\otQuote{Ps. 118:22}{The stone which the builders rejected,
The same was made the head of the corner?}}
\bv{18}{\redlet{Every one that falleth on that stone shall be broken to pieces; but on whomsoever it shall fall, it will scatter him as dust.''}}
\chapsec{Question of the Tribute-Money}
\bv{19}{And the scribes and the chief priests sought to lay hands on him in that very hour; and they feared the people: for they perceived that he spake this parable against them.}
\bv{20}{And they watched him, and sent forth spies, who feigned themselves to be righteous, that they might take hold of his speech, so as to deliver him up to the rule and to the authority of the governor.}
\bv{21}{And they asked him, saying, ``Teacher, we know that thou sayest and teachest rightly, and acceptest not the person \supptext{of any}, but of a truth teachest the way of God:}
\bv{22}{Is it lawful for us to give tribute unto Cæsar, or not?''}
\bv{23}{But he perceived their craftiness, and said unto them,}
\bv{24}{\redlet{``Show me a denarius. Whose image and superscription hath it?} And they said, ``Cæsar's.''}
\bv{25}{And he said unto them, \redlet{``Then render unto Cæsar the things that are Cæsar's, and unto God the things that are God's.''}}
\bv{26}{And they were not able to take hold of the saying before the people: and they marvelled at his answer, and held their peace.}
\chapsec{Jesus Answers the Sadducees about the Resurrection}
\bv{27}{And there came to him certain of the Sadducees, they that say that there is no resurrection;}
\bv{28}{and they asked him, saying, ``Teacher, Moses wrote unto us, that if a man's brother die, having a wife, and he be childless, his brother should take the wife, and raise up seed unto his brother.}
\bv{29}{There were therefore seven brethren: and the first took a wife, and died childless;}
\bv{30}{and the second:}
\bv{31}{and the third took her; and likewise the seven also left no children, and died.}
\bv{32}{Afterward the woman also died.}
\bv{33}{In the resurrection therefore whose wife of them shall she be? for the seven had her to wife.''}
\par
\bv{34}{And Jesus said unto them, \redlet{``The sons of this world marry, and are given in marriage:}}
\bv{35}{\redlet{but they that are accounted worthy to attain to that world, and the resurrection from the dead, neither marry, nor are given in marriage:}}
\bv{36}{\redlet{for neither can they die any more: for they are equal unto the angels; and are sons of God, being sons of the resurrection.}}
\bv{37}{\redlet{But that the dead are raised, even Moses showed, in \supptext{the place concerning} the Bush, when he calleth the Lord the God of Abraham, and the God of Isaac, and the God of Jacob.}}
\bv{38}{\redlet{Now he is not the God of the dead, but of the living: for all live unto him.''}}
\chapsec{Jesus Questions the Scribes}
\bv{39}{And certain of the scribes answering said, ``Teacher, thou hast well said.''}
\bv{40}{For they durst not any more ask him any question.}
\bv{41}{And he said unto them, \redlet{``How say they that the Christ is David's son?}}
\bv{42}{\redlet{For David himself saith in the book of Psalms,}}
\otQuote{Ps. 110:1}{\redlet{The Lord said unto my Lord,
Sit thou on my right hand, \bv{43}{Till I make thine enemies the footstool of thy feet.}}}
\bv{44}{\redlet{David therefore calleth him Lord, and how is he his son?''}}
\par
\bv{45}{And in the hearing of all the people he said unto his disciples,}
\bv{46}{\redlet{``Beware of the scribes, who desire to walk in long robes, and love salutations in the marketplaces, and chief seats in the synagogues, and chief places at feasts;}}
\bv{47}{\redlet{who devour widows' houses, and for a pretence make long prayers: these shall receive greater condemnation.''}}
\chaphead{Chapter XXI}
\chapdesc{The Widow's Lepta}
\lettrine[image=true, lines=4, findent=3pt, nindent=0pt]{NT/Luke/Lk-A.eps}{nd} he looked up, and saw the rich men that were casting their gifts into the treasury.
\bv{2}{And he saw a certain poor widow casting in thither two lepta.\mcomm{see 12:59}}
\bv{3}{And he said, \redlet{``Of a truth I say unto you, This poor widow cast in more than they all:}}
\bv{4}{\redlet{for all these did of their superfluity cast in unto the gifts; but she of her want did cast in all the living that she had.''}}
\chapsec{The Olivet Discourse}
\bv{5}{And as some spake of the temple, how it was adorned with goodly stones and offerings, he said,}
\bv{6}{\redlet{``As for these things which ye behold, the days will come, in which there shall not be left here one stone upon another, that shall not be thrown down.''}}
\chapsec{The Disciples' Question}
\bv{7}{And they asked him, saying, ``Teacher, when therefore shall these things be? and what \supptext{shall be} the sign when these things are about to come to pass?''}
\chapsec{The Course of this Age}
\bv{8}{And he said, \redlet{``Take heed that ye be not led astray: for many shall come in my name, saying, `I am \supptext{he};' and, `The time is at hand:' go ye not after them.}}
\bv{9}{\redlet{And when ye shall hear of wars and tumults, be not terrified: for these things must needs come to pass first; but the end is not immediately.''}}
\bv{10}{Then said he unto them, \redlet{``Nation shall rise against nation, and kingdom against kingdom;}}
\bv{11}{\redlet{and there shall be great earthquakes, and in divers places famines and pestilences; and there shall be terrors and great signs from heaven.}}
\bv{12}{\redlet{But before all these things, they shall lay their hands on you, and shall persecute you, delivering you up to the synagogues and prisons, bringing you before kings and governors for my name's sake.}}
\bv{13}{\redlet{It shall turn out unto you for a testimony.}}
\bv{14}{\redlet{Settle it therefore in your hearts, not to meditate beforehand how to answer:}}
\bv{15}{\redlet{for I will give you a mouth and wisdom, which all your adversaries shall not be able to withstand or to gainsay.}}
\bv{16}{\redlet{But ye shall be delivered up even by parents, and brethren, and kinsfolk, and friends; and \supptext{some} of you shall they cause to be put to death.}}
\bv{17}{\redlet{And ye shall be hated of all men for my name's sake.}}
\bv{18}{\redlet{And not a hair of your head shall perish.}}
\bv{19}{\redlet{In your patience ye shall win your souls.}}
\chapsec{The Destruction of Jerusalem Foretold}
\bv{20}{\redlet{But when ye see Jerusalem compassed with armies, then know that her desolation is at hand.}}
\bv{21}{\redlet{Then let them that are in Judæa flee unto the mountains; and let them that are in the midst of her depart out; and let not them that are in the country enter therein.}}
\bv{22}{\redlet{For these are days of vengeance, that all things which are written may be fulfilled.}}
\bv{23}{\redlet{Woe unto them that are with child and to them that give suck in those days! for there shall be great distress upon the land, and wrath unto this people.}}
\bv{24}{\redlet{And they shall fall by the edge of the sword, and shall be led captive into all the nations: and Jerusalem shall be trodden down of the Gentiles, until the times of the Gentiles be fulfilled.}}
\chapsec{The Return of the Lord in Glory}
\bv{25}{\redlet{And there shall be signs in sun and moon and stars; and upon the earth distress of nations, in perplexity for the roaring of the sea and the billows;}}
\bv{26}{\redlet{men fainting for fear, and for expectation of the things which are coming on the world: for the powers of the heavens shall be shaken.}}
\bv{27}{\redlet{And then shall they see the Son of man coming in a cloud with power and great glory.}}
\bv{28}{\redlet{But when these things begin to come to pass, look up, and lift up your heads; because your redemption draweth nigh.''}}
\chapsec{Parable of the Fig Tree}
\bv{29}{And he spake to them a parable: \redlet{``Behold the fig tree, and all the trees:}}
\bv{30}{\redlet{when they now shoot forth, ye see it and know of your own selves that the summer is now nigh.}}
\bv{31}{\redlet{Even so ye also, when ye see these things coming to pass, know ye that the kingdom of God is nigh.}}
\bv{32}{\redlet{Verily I say unto you, This generation shall not pass away, till all things be accomplished.}}
\bv{33}{\redlet{Heaven and earth shall pass away: but my words shall not pass away.}}
\chapsec{Warnings in View of the Lord's Return}
\bv{34}{\redlet{But take heed to yourselves, lest haply your hearts be overcharged with surfeiting, and drunkenness, and cares of this life, and that day come on you suddenly as a snare:}}
\bv{35}{\redlet{for \supptext{so} shall it come upon all them that dwell on the face of all the earth.}}
\bv{36}{\redlet{But watch ye at every season, making supplication, that ye may prevail to escape all these things that shall come to pass, and to stand before the Son of man.''}}
\par
\bv{37}{And every day he was teaching in the temple; and every night he went out, and lodged in the mount that is called Olivet.}
\bv{38}{And all the people came early in the morning to him in the temple, to hear him.}
\chaphead{Chapter XXII}
\chapdesc{Judas Covenants to Betray Jesus}
\lettrine[image=true, lines=4, findent=3pt, nindent=0pt]{NT/Luke/Lk-N.eps}{ow} the feast of unleavened bread drew nigh, which is called the Passover.
\bv{2}{And the chief priests and the scribes sought how they might put him to death; for they feared the people.}
\bv{3}{And Satan entered into Judas who was called Iscariot, being of the number of the twelve.}
\bv{4}{And he went away, and communed with the chief priests and captains, how he might deliver him unto them.}
\bv{5}{And they were glad, and covenanted to give him money.}
\bv{6}{And he consented, and sought opportunity to deliver him unto them in the absence of the multitude.}
\chapsec{Preparation of the Passover}
\bv{7}{And the day of unleavened bread came, on which the passover must be sacrificed.}
\bv{8}{And he sent Peter and John, saying, \redlet{``Go and make ready for us the passover, that we may eat.''}}
\bv{9}{And they said unto him, ``Where wilt thou that we make ready?''}
\bv{10}{And he said unto them, \redlet{``Behold, when ye are entered into the city, there shall meet you a man bearing a pitcher of water; follow him into the house whereinto he goeth.}}
\bv{11}{\redlet{And ye shall say unto the master of the house, `The Teacher saith unto thee, `Where is the guest-chamber, where I shall eat the passover with my disciples?'{'}}}
\bv{12}{\redlet{And he will show you a large upper room furnished: there make ready.''}}
\bv{13}{And they went, and found as he had said unto them: and they made ready the passover.}
\chapsec{The Last Passover}
\bv{14}{And when the hour was come, he sat down, and the apostles with him.}
\bv{15}{And he said unto them, \redlet{``With desire I have desired to eat this passover with you before I suffer:}}
\bv{16}{\redlet{for I say unto you, I shall not eat it, until it be fulfilled in the kingdom of God.''}}
\bv{17}{And he received a cup, and when he had given thanks, he said, \redlet{``Take this, and divide it among yourselves:}}
\bv{18}{\redlet{for I say unto you, I shall not drink from henceforth of the fruit of the vine, until the kingdom of God shall come.''}}
\chapsec{The Institution of the Eucharist}
\bv{19}{And he took bread, and when he had given thanks, he brake it, and gave to them, saying, \redlet{``This is my body which is given for you: this do in remembrance of me.''}}
\bv{20}{And the cup in like manner after supper, saying, \redlet{``This cup is the new covenant in my blood, \supptext{even} that which is poured out for you.}}
\chapsec{Jesus Announces his Betrayal}
\bv{21}{\redlet{But behold, the hand of him that betrayeth me is with me on the table.}}
\bv{22}{\redlet{For the Son of man indeed goeth, as it hath been determined: but woe unto that man through whom he is betrayed!''}}
\bv{23}{And they began to question among themselves, which of them it was that should do this thing.}
\chapsec{The Strife over Greatness}
\bv{24}{And there arose also a contention among them, which of them was accounted to be greatest.}
\bv{25}{And he said unto them, \redlet{``The kings of the Gentiles have lordship over them; and they that have authority over them are called Benefactors.}}
\bv{26}{\redlet{But ye \supptext{shall} not \supptext{be} so: but he that is the greater among you, let him become as the younger; and he that is chief, as he that doth serve.}}
\bv{27}{\redlet{For which is greater, he that sitteth at meat, or he that serveth? is not he that sitteth at meat? but I am in the midst of you as he that serveth.}}
\chapsec{The Apostles' Place in the Kingdom}
\bv{28}{\redlet{But ye are they that have continued with me in my temptations;}}
\bv{29}{\redlet{and I appoint unto you a kingdom, even as my Father appointed unto me,}}
\bv{30}{\redlet{that ye may eat and drink at my table in my kingdom; and ye shall sit on thrones judging the twelve tribes of Israel.}}
\chapsec{Jesus Predicts St. Peter's Denial}
\bv{31}{\redlet{Simon, Simon, behold, Satan asked to have you, that he might sift you as wheat:}}
\bv{32}{\redlet{but I made supplication for thee, that thy faith fail not; and do thou, when once thou hast turned again, establish thy brethren.''}}
\bv{33}{And he said unto him, ``Lord, with thee I am ready to go both to prison and to death.''}
\bv{34}{And he said, \redlet{``I tell thee, Peter, the cock shall not crow this day, until thou shalt thrice deny that thou knowest me.''}}
\chapsec{The Disciples Warned of Coming Conflicts}
\bv{35}{And he said unto them, \redlet{``When I sent you forth without purse, and wallet, and shoes, lacked ye anything?''} And they said, ``Nothing.''}
\bv{36}{And he said unto them, \redlet{``But now, he that hath a purse, let him take it, and likewise a wallet; and he that hath none, let him sell his cloak, and buy a sword.}}
\bv{37}{\redlet{For I say unto you, that this which is written must be fulfilled in me,}}
\otQuote{Is. 53:12}{\redlet{And he was reckoned with transgressors:}}
\redlet{for that which concerneth me hath fulfilment.}
\bv{38}{And they said, ``Lord, behold, here are two swords.'' And he said unto them, \redlet{``It is sufficient.''}}
\chapsec{Jesus in the Mount of Olives}
\bv{39}{And he came out, and went, as his custom was, unto the mount of Olives; and the disciples also followed him.}
\bv{40}{And when he was at the place, he said unto them, \redlet{``Pray that ye enter not into temptation.''}}
\bv{41}{And he was parted from them about a stone's cast; and he kneeled down and prayed,}
\bv{42}{saying, \redlet{``Father, if thou be willing, remove this cup from me: nevertheless not my will, but thine, be done.''}}
\bv{43}{And there appeared unto him an angel from heaven, strengthening him.}
\bv{44}{And being in an agony he prayed more earnestly; and his sweat became as it were great drops of blood falling down upon the ground.}
\bv{45}{And when he rose up from his prayer, he came unto the disciples, and found them sleeping for sorrow,}
\bv{46}{and said unto them, \redlet{``Why sleep ye? rise and pray, that ye enter not into temptation.''}}
\chapsec{Jesus Betrayed by Judas}
\bv{47}{While he yet spake, behold, a multitude, and he that was called Judas, one of the twelve, went before them; and he drew near unto Jesus to kiss him.}
\bv{48}{But Jesus said unto him, \redlet{``Judas, betrayest thou the Son of man with a kiss?''}}
\bv{49}{And when they that were about him saw what would follow, they said, ``Lord, shall we smite with the sword?''}
\bv{50}{And a certain one of them smote the servant of the high priest, and struck off his right ear.}
\bv{51}{But Jesus answered and said, \redlet{``Suffer ye \supptext{them} thus far.''} And he touched his ear, and healed him.}
\bv{52}{And Jesus said unto the chief priests, and captains of the temple, and elders, that were come against him, \redlet{``Are ye come out, as against a robber, with swords and staves?}}
\bv{53}{\redlet{When I was daily with you in the temple, ye stretched not forth your hands against me: but this is your hour, and the power of darkness.''}}
\chapsec{Jesus Arrested \& St. Peter's Denial}
\bv{54}{And they seized him, and led him \supptext{away}, and brought him into the high priest's house. But Peter followed afar off.}
\bv{55}{And when they had kindled a fire in the midst of the court, and had sat down together, Peter sat in the midst of them.}
\bv{56}{And a certain maid seeing him as he sat in the light \supptext{of the fire}, and looking stedfastly upon him, said, ``This man also was with him.''}
\bv{57}{But he denied, saying, ``Woman, I know him not.''}
\bv{58}{And after a little while another saw him, and said, ``Thou also art \supptext{one} of them.'' But Peter said, ``Man, I am not.''}
\bv{59}{And after the space of about one hour another confidently affirmed, saying, ``Of a truth this man also was with him; for he is a Galilæan.''}
\bv{60}{But Peter said, ``Man, I know not what thou sayest.'' And immediately, while he yet spake, the cock crew.}
\bv{61}{And the Lord turned, and looked upon Peter. And Peter remembered the word of the Lord, how that he said unto him, ``Before the cock crow this day thou shalt deny me thrice.''}
\bv{62}{And he went out, and wept bitterly.}
\chapsec{Jesus Buffeted}
\bv{63}{And the men that held \supptext{Jesus} mocked him, and beat him.}
\bv{64}{And they blindfolded him, and asked him, saying, ``Prophesy: who is he that struck thee?''}
\bv{65}{And many other things spake they against him, reviling him.}
\chapsec{Jesus before the Sanhedrin}
\bv{66}{And as soon as it was day, the assembly of the elders of the people was gathered together, both chief priests and scribes; and they led him away into their council, saying,}
\bv{67}{``If thou art the Christ, tell us.'' But he said unto them, \redlet{``If I tell you, ye will not believe:}}
\bv{68}{\redlet{and if I ask \supptext{you}, ye will not answer.}}
\bv{69}{\redlet{But from henceforth shall the Son of man be seated at the right hand of the power of God.''}}
\bv{70}{And they all said, ``Art thou then the Son of God?'' And he said unto them, \redlet{``Ye say that I am.''}}
\bv{71}{And they said, ``What further need have we of witness? for we ourselves have heard from his own mouth.''}
\chaphead{Chapter XXIII}
\chapdesc{Jesus before Pilate}
\lettrine[image=true, lines=4, findent=3pt, nindent=0pt]{NT/Luke/Lk-A.eps}{nd} the whole company of them rose up, and brought him before Pilate.
\bv{2}{And they began to accuse him, saying, ``We found this man perverting our nation, and forbidding to give tribute to Cæsar, and saying that he himself is Christ a king.''}
\bv{3}{And Pilate asked him, saying, ``Art thou the King of the Jews?'' And he answered him and said, \redlet{``Thou sayest.''}}
\bv{4}{And Pilate said unto the chief priests and the multitudes, ``I find no fault in this man.''}
\bv{5}{But they were the more urgent, saying, ``He stirreth up the people, teaching throughout all Judæa, and beginning from Galilee even unto this place.''}
\chapsec{Jesus before Herod}
\bv{6}{But when Pilate heard it, he asked whether the man were a Galilæan.}
\bv{7}{And when he knew that he was of Herod's jurisdiction, he sent him unto Herod, who himself also was at Jerusalem in these days.}
\bv{8}{Now when Herod saw Jesus, he was exceeding glad: for he was of a long time desirous to see him, because he had heard concerning him; and he hoped to see some miracle done by him.}
\bv{9}{And he questioned him in many words; but he answered him nothing.}
\bv{10}{And the chief priests and the scribes stood, vehemently accusing him.}
\bv{11}{And Herod with his soldiers set him at nought, and mocked him, and arraying him in gorgeous apparel sent him back to Pilate.}
\bv{12}{And Herod and Pilate became friends with each other that very day: for before they were at enmity between themselves.}
\chapsec{Jesus again before Pilate}
\bv{13}{And Pilate called together the chief priests and the rulers and the people,}
\bv{14}{and said unto them, ``Ye brought unto me this man, as one that perverteth the people: and behold, I, having examined him before you, found no fault in this man touching those things whereof ye accuse him:}
\bv{15}{no, nor yet Herod: for he sent him back unto us; and behold, nothing worthy of death hath been done by him.}
\bv{16}{I will therefore chastise him, and release him.''\mcomm{Now he must needs release unto them at the feast one prisoner.}}
\par
\bv{18}{But they cried out all together, saying, ``Away with this man, and release unto us Barabbas:''}
\bv{19}{one who for a certain insurrection made in the city, and for murder, was cast into prison.}
\bv{20}{And Pilate spake unto them again, desiring to release Jesus;}
\bv{21}{but they shouted, saying, ``Crucify, crucify him.''}
\bv{22}{And he said unto them the third time, ``Why, what evil hath this man done? I have found no cause of death in him: I will therefore chastise him and release him.''}
\bv{23}{But they were urgent with loud voices, asking that he might be crucified. And their voices prevailed.}
\bv{24}{And Pilate gave sentence that what they asked for should be done.}
\bv{25}{And he released him that for insurrection and murder had been cast into prison, whom they asked for; but Jesus he delivered up to their will.}
\bv{26}{And when they led him away, they laid hold upon one Simon of Cyrene, coming from the country, and laid on him the cross, to bear it after Jesus.}
\chapsec{The Crucifixion}
\bv{27}{And there followed him a great multitude of the people, and of women who bewailed and lamented him.}
\bv{28}{But Jesus turning unto them said, \redlet{``Daughters of Jerusalem, weep not for me, but weep for yourselves, and for your children.}}
\bv{29}{\redlet{For behold, the days are coming, in which they shall say, `Blessed are the barren, and the wombs that never bare, and the breasts that never gave suck.'}}
\bv{30}{\redlet{Then shall they begin to say to the mountains, `Fall on us;' and to the hills, `Cover us.'}}
\bv{31}{\redlet{For if they do these things in the green tree, what shall be done in the dry?''}}
\par
\bv{32}{And there were also two others, malefactors, led with him to be put to death.}
\bv{33}{And when they came unto the place which is called The skull, there they crucified him, and the malefactors, one on the right hand and the other on the left.\mcomm{And Jesus said, ``Father, forgive them; for they know not what they do.''}}
\bv{34}{And parting his garments among them, they cast lots.}
\par
\bv{35}{And the people stood beholding. And the rulers also scoffed at him, saying, ``He saved others; let him save himself, if this is the Christ of God, his chosen.''}
\bv{36}{And the soldiers also mocked him, coming to him, offering him vinegar,}
\bv{37}{and saying, ``If thou art the King of the Jews, save thyself.''}
\bv{38}{And there was also a superscription over him,\mcomm{Written in Hebrew, Greek, and Latin} ``THIS IS THE KING OF THE JEWS.''}
\chapsec{The Repentant Thief}
\bv{39}{And one of the malefactors that were hanged railed on him, saying, ``Art not thou the Christ? save thyself and us.''}
\bv{40}{But the other answered, and rebuking him said, ``Dost thou not even fear God, seeing thou art in the same condemnation?}
\bv{41}{And we indeed justly; for we receive the due reward of our deeds: but this man hath done nothing amiss.'}
\bv{42}{And he said, ``Jesus, remember me when thou comest in thy kingdom.''}
\bv{43}{And he said unto him, \redlet{``Verily I say unto thee, To-day shalt thou be with me in Paradise.''}}
\par
\bv{44}{And it was now about the sixth hour, and a darkness came over the whole land until the ninth hour,}
\bv{45}{the sun's light failing: and the veil of the temple was rent in the midst.}
\chapsec{Jesus Gives up the Ghost}
\bv{46}{And Jesus, crying with a loud voice, said, \redlet{``Father, into thy hands I commend my spirit:''} and having said this, he gave up the ghost.}
\bv{47}{And when the centurion saw what was done, he glorified God, saying, ``Certainly this was a righteous man.''}
\bv{48}{And all the multitudes that came together to this sight, when they beheld the things that were done, returned smiting their breasts.}
\bv{49}{And all his acquaintance, and the women that followed with him from Galilee, stood afar off, seeing these things.}
\chapsec{The Entombment}
\bv{50}{And behold, a man named Joseph, who was a councillor, a good and righteous man}
\bv{51}{(he had not consented to their counsel and deed), \supptext{a man} of Arimathæa, a city of the Jews, who was looking for the kingdom of God:}
\bv{52}{this man went to Pilate, and asked for the body of Jesus.}
\bv{53}{And he took it down, and wrapped it in a linen cloth, and laid him in a tomb that was hewn in stone, where never man had yet lain.}
\bv{54}{And it was the day of the Preparation, and the sabbath drew on.}
\bv{55}{And the women, who had come with him out of Galilee, followed after, and beheld the tomb, and how his body was laid.}
\bv{56}{And they returned, and prepared spices and ointments. And on the sabbath they rested according to the commandment.}
\chaphead{Chapter XXIV}
\chapdesc{The Resurrection of Jesus Christ}
\lettrine[image=true, lines=4, findent=3pt, nindent=0pt]{NT/Luke/B-Mal.eps}{ut} on the first day of the week, at early dawn, they came unto the tomb, bringing the spices which they had prepared.
\bv{2}{And they found the stone rolled away from the tomb.}
\bv{3}{And they entered in, and found not the body of the Lord Jesus.}
\bv{4}{And it came to pass, while they were perplexed thereabout, behold, two men stood by them in dazzling apparel:}
\bv{5}{and as they were affrighted and bowed down their faces to the earth, they said unto them, ``Why seek ye the living among the dead?}
\bv{6}{He is not here, but is risen: remember how he spake unto you when he was yet in Galilee,}
\bv{7}{saying that the Son of man must be delivered up into the hands of sinful men, and be crucified, and the third day rise again.''}
\bv{8}{And they remembered his words,}
\bv{9}{and returned from the tomb, and told all these things to the eleven, and to all the rest.}
\bv{10}{Now they were Mary Magdalene, and Joanna, and Mary the \supptext{mother} of James: and the other women with them told these things unto the apostles.}
\bv{11}{And these words appeared in their sight as idle talk; and they disbelieved them.}
\par
\bv{12}{But Peter arose, and ran unto the tomb; and stooping and looking in, he seeth the linen cloths by themselves; and he departed to his home, wondering at that which was come to pass.}
\chapsec{The Ministry of the Risen Christ}
\bv{13}{And behold, two of them were going that very day to a village named Emmaus, which was threescore furlongs from Jerusalem.}
\bv{14}{And they communed with each other of all these things which had happened.}
\bv{15}{And it came to pass, while they communed and questioned together, that Jesus himself drew near, and went with them.}
\bv{16}{But their eyes were holden that they should not know him.}
\bv{17}{And he said unto them, \redlet{``What communications are these that ye have one with another, as ye walk?''} And they stood still, looking sad.}
\bv{18}{And one of them, named Cleopas, answering said unto him, ``Dost thou alone sojourn in Jerusalem and not know the things which are come to pass there in these days?''}
\par
\bv{19}{And he said unto them, \redlet{``What things?''} And they said unto him, ``The things concerning Jesus the Nazarene, who was a prophet mighty in deed and word before God and all the people:}
\bv{20}{and how the chief priests and our rulers delivered him up to be condemned to death, and crucified him.}
\bv{21}{But we hoped that it was he who should redeem Israel. Yea and besides all this, it is now the third day since these things came to pass.}
\bv{22}{Moreover certain women of our company amazed us, having been early at the tomb;}
\bv{23}{and when they found not his body, they came, saying, that they had also seen a vision of angels, who said that he was alive.}
\bv{24}{And certain of them that were with us went to the tomb, and found it even so as the women had said: but him they saw not.''}
\par
\bv{25}{And he said unto them, \redlet{``O foolish men, and slow of heart to believe in all that the prophets have spoken!}}
\bv{26}{\redlet{Behooved it not the Christ to suffer these things, and to enter into his glory?''}}
\bv{27}{And beginning from Moses and from all the prophets, he interpreted to them in all the scriptures the things concerning himself.}
\bv{28}{And they drew nigh unto the village, whither they were going: and he made as though he would go further.}
\bv{29}{And they constrained him, saying, ``Abide with us; for it is toward evening, and the day is now far spent.'' And he went in to abide with them.}
\bv{30}{And it came to pass, when he had sat down with them to meat, he took the bread and blessed; and breaking \supptext{it} he gave to them.}
\bv{31}{And their eyes were opened, and they knew him; and he vanished out of their sight.}
\par
\bv{32}{And they said one to another, ``Was not our heart burning within us, while he spake to us in the way, while he opened to us the scriptures?''}
\bv{33}{And they rose up that very hour, and returned to Jerusalem, and found the eleven gathered together, and them that were with them,}
\bv{34}{saying, ``The Lord is risen indeed, and hath appeared to Simon.''}
\bv{35}{And they rehearsed the things \supptext{that happened} in the way, and how he was known of them in the breaking of the bread.}
\par
\bv{36}{And as they spake these things, he himself stood in the midst of them, and saith unto them, \redlet{``Peace \supptext{be} unto you.''}}
\bv{37}{But they were terrified and affrighted, and supposed that they beheld a spirit.}
\bv{38}{And he said unto them, \redlet{``Why are ye troubled? and wherefore do questionings arise in your heart?}}
\bv{39}{\redlet{See my hands and my feet, that it is I myself: handle me, and see; for a spirit hath not flesh and bones, as ye behold me having.''}}
\bv{40}{And when he had said this, he showed them his hands and his feet.}
\bv{41}{And while they still disbelieved for joy, and wondered, he said unto them, \redlet{``Have ye here anything to eat?''}}
\bv{42}{And they gave him a piece of a broiled fish.}
\bv{43}{And he took it, and ate before them.}
\bv{44}{And he said unto them, \redlet{``These are my words which I spake unto you, while I was yet with you, that all things must needs be fulfilled, which are written in the law of Moses, and the prophets, and the psalms, concerning me.''}}
\bv{45}{Then opened he their mind, that they might understand the scriptures;}
\chapsec{The Commission to Evangelise}
\bv{46}{and he said unto them, \redlet{``Thus it is written, that the Christ should suffer, and rise again from the dead the third day;}}
\bv{47}{\redlet{and that repentance and remission of sins should be preached in his name unto all the nations, beginning from Jerusalem.}}
\bv{48}{\redlet{Ye are witnesses of these things.}}
\chapsec{The Ascension of Jesus Christ}
\bv{49}{\redlet{And behold, I send forth the promise of my Father upon you: but tarry ye in the city, until ye be clothed with power from on high.''}}
\bv{50}{And he led them out until \supptext{they were} over against Bethany: and he lifted up his hands, and blessed them.}
\bv{51}{And it came to pass, while he blessed them, he parted from them, and was carried up into heaven.}
\bv{52}{And they worshipped him, and returned to Jerusalem with great joy:}
\bv{53}{and were continually in the temple, blessing God.}
\begin{center}
	{\scshape [Here Endeth the Gospel of Luke]}
\end{center}
	\clearpage
	\chapter{The Holy Gospel of Jesus Christ according to Saint John}
\fancyhead[RE,LO]{The Gospel according to John}
\chaphead{Chapter I}
\chapdesc{The Deity of Jesus Christ}
\lettrine[image=true, lines=4, findent=3pt, nindent=0pt]{NT/John/Jn1-I.eps}{n} the beginning was the Word, and the Word was with God, and the Word was God.
\bv{2}{The same was in the beginning with God.}
\chapsec{His Pre-Incarnation Work}
\bv{3}{All things were made through him; and without him was not anything made that hath been made.}
\bv{4}{In him was life; and the life was the light of men.}
\bv{5}{And the light shineth in the darkness; and the darkness apprehended it not.}
\chapsec{Ministry of St. John the Baptist}
\bv{6}{There came a man, sent from God, whose name was John.}
\bv{7}{The same came for witness, that he might bear witness of the light, that all might believe through him.}
\bv{8}{He was not the light, but \supptext{came} that he might bear witness of the light.}
\chapsec{Jesus Christ the True Light}
\bv{9}{There was the true light, \supptext{even the light} which lighteth every man, coming into the world.}
\bv{10}{He was in the world, and the world was made through him, and the world knew him not.}
\chapsec{Sons \& Unbelievers}
\bv{11}{He came unto his own, and they that were his own received him not.}
\bv{12}{But as many as received him, to them gave he the right to become children of God, \supptext{even} to them that believe on his name:}
\bv{13}{who were born, not of blood, nor of the will of the flesh, nor of the will of man, but of God.}
\chapsec{The Incarnation}
\bv{14}{And the Word became flesh, and dwelt among us (and we beheld his glory, glory as of the only begotten from the Father), full of grace and truth.}
\chapsec{The Witness of St. John the Baptist}
\bv{15}{John beareth witness of him, and crieth, saying, ``This was he of whom I said, `He that cometh after me is become before me: for he was before me.' ''}
\bv{16}{For of his fulness we all received, and grace for grace.}
\bv{17}{For the law was given through Moses; grace and truth came through Jesus Christ.}
\bv{18}{No man hath seen God at any time; the only begotten God, who is in the bosom of the Father, he hath declared \supptext{him}.}
\par
\bv{19}{And this is the witness of John, when the Jews sent unto him from Jerusalem priests and Levites to ask him, ``Who art thou?''}
\bv{20}{And he confessed, and denied not; and he confessed, ``I am not the Christ.''}
\bv{21}{And they asked him, ``What then? Art thou Elijah?'' And he saith, ``I am not.'' ``Art thou the prophet?'' And he answered, ``No.''}
\bv{22}{They said therefore unto him, ``Who art thou? that we may give an answer to them that sent us. What sayest thou of thyself?''}
\bv{23}{He said, ``I am the voice of one crying in the wilderness, `Make straight the way of the Lord,' as said Isaiah the prophet.''\mref{Is. 40:3}}
\bv{24}{And they had been sent from the Pharisees.}
\bv{25}{And they asked him, and said unto him, ``Why then baptisest thou, if thou art not the Christ, neither Elijah, neither the prophet?''}
\bv{26}{John answered them, saying, ``I baptise in water: in the midst of you standeth one whom ye know not,}
\bv{27}{\supptext{even} he that cometh after me, the latchet of whose shoe I am not worthy to unloose.''}
\bv{28}{These things were done in Bethany beyond the Jordan, where John was baptising.}
\par
\bv{29}{On the morrow he seeth Jesus coming unto him, and saith, ``Behold, the Lamb of God, that taketh away the sin of the world!}
\bv{30}{This is he of whom I said, `After me cometh a man who is become before me: for he was before me.'}
\bv{31}{And I knew him not; but that he should be made manifest to Israel, for this cause came I baptising in water.''}
\bv{32}{And John bare witness, saying, ``I have beheld the Spirit descending as a dove out of heaven; and it abode upon him.}
\bv{33}{And I knew him not: but he that sent me to baptise in water, he said unto me, `Upon whomsoever thou shalt see the Spirit descending, and abiding upon him, the same is he that baptiseth in the Holy Ghost.'}
\bv{34}{And I have seen, and have borne witness that this is the Son of God.''}
\par
\bv{35}{Again on the morrow John was standing, and two of his disciples;}
\bv{36}{and he looked upon Jesus as he walked, and saith, ``Behold, the Lamb of God!''}
\bv{37}{And the two disciples heard him speak, and they followed Jesus.}
\bv{38}{And Jesus turned, and beheld them following, and saith unto them, \redlet{``What seek ye?''} And they said unto him, ``Rabbi'' (which is to say, being interpreted, ``Teacher''), ``where abidest thou?''}
\bv{39}{He saith unto them, \redlet{	``Come, and ye shall see.''} They came therefore and saw where he abode; and they abode with him that day: it was about the tenth hour.}
\bv{40}{One of the two that heard John \supptext{speak}, and followed him, was Andrew, Simon Peter's brother.}
\bv{41}{He findeth first his own brother Simon, and saith unto him, ``We have found the Messiah'' (which is, being interpreted, ``Christ'').}
\bv{42}{He brought him unto Jesus. Jesus looked upon him, and said, \redlet{``Thou art Simon the son of John: thou shalt be called Cephas''} (which is by interpretation, ``Peter'').}
\bv{43}{On the morrow he was minded to go forth into Galilee, and he findeth Philip: and Jesus saith unto him, \redlet{``Follow me.''}}
\bv{44}{Now Philip was from Bethsaida, of the city of Andrew and Peter.}
\bv{45}{Philip findeth Nathanael, and saith unto him, ``We have found him, of whom Moses in the law, and the prophets, wrote, Jesus of Nazareth, the son of Joseph.''}
\bv{46}{And Nathanael said unto him, ``Can any good thing come out of Nazareth?'' Philip saith unto him, ``Come and see.''}
\bv{47}{Jesus saw Nathanael coming to him, and saith of him, \redlet{``Behold, an Israelite indeed, in whom is no guile!''}}
\bv{48}{Nathanael saith unto him, ``Whence knowest thou me?'' Jesus answered and said unto him, \redlet{``Before Philip called thee, when thou wast under the fig tree, I saw thee.''}}
\bv{49}{Nathanael answered him, ``Rabbi, thou art the Son of God; thou art King of Israel.''}
\bv{50}{Jesus answered and said unto him, \redlet{``Because I said unto thee, I saw thee underneath the fig tree, believest thou? thou shalt see greater things than these.''}}
\bv{51}{And he saith unto him, \redlet{``Verily, verily, I say unto you, Ye shall see the heaven opened, and the angels of God ascending and descending upon the Son of man.''}}
\chaphead{Chapter II}
\chapdesc{The Marriage at Cana}
\lettrine[image=true, lines=4, findent=3pt, nindent=0pt]{NT/John/Jn-And.eps}{nd} the third day there was a marriage in Cana of Galilee; and the mother of Jesus was there:
\bv{2}{and Jesus also was bidden, and his disciples, to the marriage.}
\bv{3}{And when the wine failed, the mother of Jesus saith unto him, ``They have no wine.''}
\bv{4}{And Jesus saith unto her, \redlet{``Woman, what have I to do with thee? mine hour is not yet come.''}}
\bv{5}{His mother saith unto the servants, ``Whatsoever he saith unto you, do it.''}
\bv{6}{Now there were six waterpots of stone set there after the Jews' manner of purifying, containing two or three firkins apiece.}
\bv{7}{Jesus saith unto them, \redlet{``Fill the waterpots with water.''} And they filled them up to the brim.}
\bv{8}{And he saith unto them, \redlet{``Draw out now, and bear unto the ruler of the feast.''} And they bare it.}
\bv{9}{And when the ruler of the feast tasted the water now become wine, and knew not whence it was (but the servants that had drawn the water knew), the ruler of the feast calleth the bridegroom,}
\bv{10}{and saith unto him, ``Every man setteth on first the good wine; and when \supptext{men} have drunk freely, \supptext{then} that which is worse: thou hast kept the good wine until now.''}
\bv{11}{This beginning of his signs did Jesus in Cana of Galilee, and manifested his glory; and his disciples believed on him.}
\par
\bv{12}{After this he went down to Capernaum, he, and his mother, and \supptext{his} brethren, and his disciples; and there they abode not many days.}
\bv{13}{And the passover of the Jews was at hand, and Jesus went up to Jerusalem.}
\bv{14}{And he found in the temple those that sold oxen and sheep and doves, and the changers of money sitting:}
\bv{15}{and he made a scourge of cords, and cast all out of the temple, both the sheep and the oxen; and he poured out the changers' money, and overthrew their tables;}
\bv{16}{and to them that sold the doves he said, \redlet{``Take these things hence; make not my Father's house a house of merchandise.''}}
\bv{17}{His disciples remembered that it was written, \otQuote{Ps.69:9}{Zeal for thy house shall eat me up.}}
\bv{18}{The Jews therefore answered and said unto him, ``What sign showest thou unto us, seeing that thou doest these things?''}
\bv{19}{Jesus answered and said unto them, \redlet{``Destroy this temple, and in three days I will raise it up.''}}
\bv{20}{The Jews therefore said, ``Forty and six years was this temple in building, and wilt thou raise it up in three days?''}
\bv{21}{But he spake of the temple of his body.}
\bv{22}{When therefore he was raised from the dead, his disciples remembered that he spake this; and they believed the scripture, and the word which Jesus had said.}
\bv{23}{Now when he was in Jerusalem at the passover, during the feast, many believed on his name, beholding his signs which he did.}
\bv{24}{But Jesus did not trust himself unto them, for that he knew all men,}
\bv{25}{and because he needed not that any one should bear witness concerning man; for he himself knew what was in man.}
\chaphead{Chapter III}
\chapdesc{Jesus \& Nicodemus}
\lettrine[image=true, lines=4, findent=3pt, nindent=0pt]{NT/John/Jn-Now.eps}{ow} there was a man of the Pharisees, named Nicodemus, a ruler of the Jews:
\bv{2}{the same came unto him by night, and said to him, ``Rabbi, we know that thou art a teacher come from God; for no one can do these signs that thou doest, except God be with him.''}
\bv{3}{Jesus answered and said unto him, \redlet{``Verily, verily, I say unto thee, Except one be born anew, he cannot see the kingdom of God.''}}
\bv{4}{Nicodemus saith unto him, ``How can a man be born when he is old? can he enter a second time into his mother's womb, and be born?''}
\bv{5}{Jesus answered, \redlet{``Verily, verily, I say unto thee, Except one be born of water and the Spirit, he cannot enter into the kingdom of God.}}
\bv{6}{\redlet{That which is born of the flesh is flesh; and that which is born of the Spirit is spirit.}}
\bv{7}{\redlet{Marvel not that I said unto thee, `Ye must be born anew.'}}
\bv{8}{\redlet{The wind bloweth where it will, and thou hearest the voice thereof, but knowest not whence it cometh, and whither it goeth: so is every one that is born of the Spirit.''}}
\par
\bv{9}{Nicodemus answered and said unto him, ``How can these things be?''}
\bv{10}{Jesus answered and said unto him, \redlet{``Art thou the teacher of Israel, and understandest not these things?}}
\bv{11}{\redlet{Verily, verily, I say unto thee, We speak that which we know, and bear witness of that which we have seen; and ye receive not our witness.}}
\bv{12}{\redlet{If I told you earthly things and ye believe not, how shall ye believe if I tell you heavenly things?}}
\bv{13}{\redlet{And no one hath ascended into heaven, but he that descended out of heaven, \supptext{even} the Son of man, who is in heaven.}}
\bv{14}{\redlet{And as Moses lifted up the serpent in the wilderness, even so must the Son of man be lifted up;}}
\bv{15}{\redlet{that whosoever believeth may in him have eternal life.}}
\par
\bv{16}{\redlet{For God so loved the world, that he gave his only begotten Son, that whosoever believeth on him should not perish, but have eternal life.}}
\bv{17}{\redlet{For God sent not the Son into the world to judge the world; but that the world should be saved through him.}}
\bv{18}{\redlet{He that believeth on him is not judged: he that believeth not hath been judged already, because he hath not believed on the name of the only begotten Son of God.}}
\bv{19}{\redlet{And this is the judgement, that the light is come into the world, and men loved the darkness rather than the light; for their works were evil.}}
\bv{20}{\redlet{For every one that doeth evil hateth the light, and cometh not to the light, lest his works should be reproved.}}
\bv{21}{\redlet{But he that doeth the truth cometh to the light, that his works may be made manifest, that they have been wrought in God.''}}
\par
\bv{22}{After these things came Jesus and his disciples into the land of Judæa; and there he tarried with them, and baptised.}
\bv{23}{And John also was baptising in Ænon near to Salim, because there was much water there: and they came, and were baptised.}
\bv{24}{For John was not yet cast into prison.}
\bv{25}{There arose therefore a questioning on the part of John's disciples with a Jew about purifying.}
\bv{26}{And they came unto John, and said to him, ``Rabbi, he that was with thee beyond the Jordan, to whom thou hast borne witness, behold, the same baptiseth, and all men come to him.''}
\bv{27}{John answered and said, ``A man can receive nothing, except it have been given him from heaven.}
\bv{28}{Ye yourselves bear me witness, that I said, `I am not the Christ,' but, that I am sent before him.}
\bv{29}{He that hath the bride is the bridegroom: but the friend of the bridegroom, that standeth and heareth him, rejoiceth greatly because of the bridegroom's voice: this my joy therefore is made full.}
\bv{30}{He must increase, but I must decrease.}
\chapsec{Declaration about Jesus Christ}
\bv{31}{He that cometh from above is above all: he that is of the earth is of the earth, and of the earth he speaketh: he that cometh from heaven is above all.}
\bv{32}{What he hath seen and heard, of that he beareth witness; and no man receiveth his witness.}
\bv{33}{He that hath received his witness hath set his seal to \supptext{this}, that God is true.}
\bv{34}{For he whom God hath sent speaketh the words of God: for he giveth not the Spirit by measure.}
\bv{35}{The Father loveth the Son, and hath given all things into his hand.}
\bv{36}{He that believeth on the Son hath eternal life; but he that obeyeth not the Son shall not see life, but the wrath of God abideth on him.''}
\chaphead{Chapter IV}
\chapdesc{Jesus Departs into Galilee}
\lettrine[image=true, lines=4, findent=3pt, nindent=0pt]{NT/John/Jn4-W.eps}{hen} therefore the Lord knew that the Pharisees had heard that Jesus was making and baptising more disciples than John
\bv{2}{(although Jesus himself baptised not, but his disciples),}
\bv{3}{he left Judæa, and departed again into Galilee.}
\bv{4}{And he must needs pass through Samaria.}
\bv{5}{So he cometh to a city of Samaria, called Sychar, near to the parcel of ground that Jacob gave to his son Joseph:}
\chapsec{Jesus \& the Samaritan Woman}
\bv{6}{and Jacob's well was there. Jesus therefore, being wearied with his journey, sat thus by the well. It was about the sixth hour.}
\bv{7}{There cometh a woman of Samaria to draw water: Jesus saith unto her, \redlet{``Give me to drink.''}}
\bv{8}{For his disciples were gone away into the city to buy food.}
\bv{9}{The Samaritan woman therefore saith unto him, ``How is it that thou, being a Jew, askest drink of me, who am a Samaritan woman?'' (For Jews have no dealings with Samaritans.)}
\bv{10}{Jesus answered and said unto her, \redlet{``If thou knewest the gift of God, and who it is that saith to thee, `Give me to drink;' thou wouldest have asked of him, and he would have given thee living water.''}}
\bv{11}{The woman saith unto him, ``Sir, thou hast nothing to draw with, and the well is deep: whence then hast thou that living water?}
\bv{12}{Art thou greater than our father Jacob, who gave us the well, and drank thereof himself, and his sons, and his cattle?''}
\bv{13}{Jesus answered and said unto her, \redlet{``Every one that drinketh of this water shall thirst again:}}
\chapsec{The Indwelling Spirit}
\bv{14}{\redlet{but whosoever drinketh of the water that I shall give him shall never thirst; but the water that I shall give him shall become in him a well of water springing up unto eternal life.''}}
\bv{15}{The woman saith unto him, ``Sir, give me this water, that I thirst not, neither come all the way hither to draw.''}
\par
\bv{16}{Jesus saith unto her, \redlet{``Go, call thy husband, and come hither.''}}
\bv{17}{The woman answered and said unto him, ``I have no husband.'' Jesus saith unto her, \redlet{``Thou saidst well, I have no husband:}}
\bv{18}{\redlet{for thou hast had five husbands; and he whom thou now hast is not thy husband: this hast thou said truly.''}}
\par
\bv{19}{The woman saith unto him, ``Sir, I perceive that thou art a prophet.}
\bv{20}{Our fathers worshipped in this mountain; and ye say, that in Jerusalem is the place where men ought to worship.''}
\bv{21}{Jesus saith unto her, \redlet{``Woman, believe me, the hour cometh, when neither in this mountain, nor in Jerusalem, shall ye worship the Father.}}
\bv{22}{\redlet{Ye worship that which ye know not: we worship that which we know; for salvation is from the Jews.}}
\bv{23}{\redlet{But the hour cometh, and now is, when the true worshippers shall worship the Father in spirit and truth: for such doth the Father seek to be his worshippers.}}
\bv{24}{\redlet{God is a Spirit: and they that worship him must worship in spirit and truth.''}}
\bv{25}{The woman saith unto him, ``I know that Messiah cometh (he that is called Christ): when he is come, he will declare unto us all things.''}
\bv{26}{Jesus saith unto her, \redlet{``I that speak unto thee am \supptext{he}.''}}
\par
\bv{27}{And upon this came his disciples; and they marvelled that he was speaking with a woman; yet no man said, ``What seekest thou?'' or, ``Why speakest thou with her?''}
\par
\bv{28}{So the woman left her waterpot, and went away into the city, and saith to the people,}
\bv{29}{``Come, see a man, who told me all things that \supptext{ever} I did: can this be the Christ?''}
\bv{30}{They went out of the city, and were coming to him.}
\par
\bv{31}{In the mean while the disciples prayed him, saying, ``Rabbi, eat.''}
\bv{32}{But he said unto them, \redlet{``I have meat to eat that ye know not.''}}
\bv{33}{The disciples therefore said one to another, ``Hath any man brought him \supptext{aught} to eat?''}
\bv{34}{Jesus saith unto them, \redlet{``My meat is to do the will of him that sent me, and to accomplish his work.}}
\bv{35}{\redlet{Say not ye, `There are yet four months, and \supptext{then} cometh the harvest?' behold, I say unto you, Lift up your eyes, and look on the fields, that they are white already unto harvest.}}
\bv{36}{\redlet{He that reapeth receiveth wages, and gathereth fruit unto life eternal; that he that soweth and he that reapeth may rejoice together.}}
\bv{37}{\redlet{For herein is the saying true, `One soweth, and another reapeth.'}}
\bv{38}{\redlet{I sent you to reap that whereon ye have not labored: others have labored, and ye are entered into their labor.''}}
\par
\bv{39}{And from that city many of the Samaritans believed on him because of the word of the woman, who testified, ``He told me all things that \supptext{ever} I did.''}
\chapsec{Jesus \& the Samaritans}
\bv{40}{So when the Samaritans came unto him, they besought him to abide with them: and he abode there two days.}
\bv{41}{And many more believed because of his word;}
\bv{42}{and they said to the woman, ``Now we believe, not because of thy speaking: for we have heard for ourselves, and know that this is indeed the Saviour of the world.''}
\bv{43}{And after the two days he went forth from thence into Galilee.}
\bv{44}{For Jesus himself testified, that a prophet hath no honor in his own country.}
\bv{45}{So when he came into Galilee, the Galilæans received him, having seen all the things that he did in Jerusalem at the feast: for they also went unto the feast.}
\chapsec{The Nobleman's Son Healed}
\bv{46}{He came therefore again unto Cana of Galilee, where he made the water wine. And there was a certain nobleman, whose son was sick at Capernaum.}
\bv{47}{When he heard that Jesus was come out of Judæa into Galilee, he went unto him, and besought \supptext{him} that he would come down, and heal his son; for he was at the point of death.}
\bv{48}{Jesus therefore said unto him, \redlet{``Except ye see signs and wonders, ye will in no wise believe.''}}
\bv{49}{The nobleman saith unto him, ``Sir, come down ere my child die.''}
\bv{50}{Jesus saith unto him, \redlet{``Go thy way; thy son liveth.''} The man believed the word that Jesus spake unto him, and he went his way.}
\bv{51}{And as he was now going down, his servants met him, saying, that his son lived.}
\bv{52}{So he inquired of them the hour when he began to amend. They said therefore unto him, ``Yesterday at the seventh hour the fever left him.''}
\bv{53}{So the father knew that \supptext{it was} at that hour in which Jesus said unto him, ``Thy son liveth:'' and himself believed, and his whole house.}
\bv{54}{This is again the second sign that Jesus did, having come out of Judæa into Galilee.}
\chaphead{Chapter V}
\chapdesc{The Pool of Bethesda \& Healing}
\lettrine[image=true, lines=4, findent=3pt, nindent=0pt]{NT/John/Jn-After.eps}{fter} these things there was a feast of the Jews; and Jesus went up to Jerusalem.
\bv{2}{Now there is in Jerusalem by the sheep \supptext{gate} a pool, which is called in Hebrew Bethesda, having five porches.}
\bv{3}{In these lay a multitude of them that were sick, blind, halt, withered.\mcomm{waiting for the moving of the water: for an angel of the Lord went down at certain seasons into the pool, and troubled the water: whosoever then first after the troubling of the water stepped in was made whole, with whatsoever disease he was holden.}}
\bv{5}{And a certain man was there, who had been thirty and eight years in his infirmity.}
\bv{6}{When Jesus saw him lying, and knew that he had been now a long time \supptext{in that case}, he saith unto him, \redlet{``Wouldest thou be made whole?''}}
\bv{7}{The sick man answered him, ``Sir, I have no man, when the water is troubled, to put me into the pool: but while I am coming, another steppeth down before me.''}
\bv{8}{Jesus saith unto him, \redlet{``Arise, take up thy bed, and walk.''}}
\bv{9}{And straightway the man was made whole, and took up his bed and walked. Now it was the sabbath on that day.}
\bv{10}{So the Jews said unto him that was cured, ``It is the sabbath, and it is not lawful for thee to take up thy bed.''}
\bv{11}{But he answered them, ``He that made me whole, the same said unto me, `Take up thy bed, and walk.' ''}
\bv{12}{They asked him, ``Who is the man that said unto thee, `Take up \supptext{thy bed}, and walk?' ''}
\bv{13}{But he that was healed knew not who it was; for Jesus had conveyed himself away, a multitude being in the place.}
\bv{14}{Afterward Jesus findeth him in the temple, and said unto him, \redlet{``Behold, thou art made whole: sin no more, lest a worse thing befall thee.''}}
\bv{15}{The man went away, and told the Jews that it was Jesus who had made him whole.}
\bv{16}{And for this cause the Jews persecuted Jesus, because he did these things on the sabbath.}
\par
\bv{17}{But Jesus answered them, \redlet{``My Father worketh even until now, and I work.''}}
\bv{18}{For this cause therefore the Jews sought the more to kill him, because he not only brake the sabbath, but also called God his own Father, making himself equal with God.}
\bv{19}{Jesus therefore answered and said unto them, \redlet{``Verily, verily, I say unto you, The Son can do nothing of himself, but what he seeth the Father doing: for what things soever he doeth, these the Son also doeth in like manner.}}
\bv{20}{\redlet{For the Father loveth the Son, and showeth him all things that himself doeth: and greater works than these will he show him, that ye may marvel.}}
\bv{21}{\redlet{For as the Father raiseth the dead and giveth them life, even so the Son also giveth life to whom he will.}}
\bv{22}{\redlet{For neither doth the Father judge any man, but he hath given all judgement unto the Son;}}
\bv{23}{\redlet{that all may honor the Son, even as they honor the Father. He that honoreth not the Son honoreth not the Father that sent him.}}
\bv{24}{\redlet{Verily, verily, I say unto you, He that heareth my word, and believeth him that sent me, hath eternal life, and cometh not into judgement, but hath passed out of death into life.}}
\par
\bv{25}{\redlet{Verily, verily, I say unto you, The hour cometh, and now is, when the dead shall hear the voice of the Son of God; and they that hear shall live.}}
\bv{26}{\redlet{For as the Father hath life in himself, even so gave he to the Son also to have life in himself:}}
\bv{27}{\redlet{and he gave him authority to execute judgement, because he is a son of man.}}
\chapsec{The Two Resurrections}
\bv{28}{\redlet{Marvel not at this: for the hour cometh, in which all that are in the tombs shall hear his voice,}}
\bv{29}{\redlet{and shall come forth; they that have done good, unto the resurrection of life; and they that have done evil, unto the resurrection of judgement.}}
\bv{30}{\redlet{I can of myself do nothing: as I hear, I judge: and my judgement is righteous; because I seek not mine own will, but the will of him that sent me.}}
\bv{31}{\redlet{If I bear witness of myself, my witness is not true.}}
\bv{32}{\redlet{It is another that beareth witness of me; and I know that the witness which he witnesseth of me is true.}}
\chapsec{The Witness of St. John the Baptist}
\bv{33}{\redlet{Ye have sent unto John, and he hath borne witness unto the truth.}}
\bv{34}{\redlet{But the witness which I receive is not from man: howbeit I say these things, that ye may be saved.}}
\bv{35}{\redlet{He was the lamp that burneth and shineth; and ye were willing to rejoice for a season in his light.}}
\chapsec{The Witness of Works}
\bv{36}{\redlet{But the witness which I have is greater than \supptext{that of} John; for the works which the Father hath given me to accomplish, the very works that I do, bear witness of me, that the Father hath sent me.}}
\chapsec{The Witness of the Father}
\bv{37}{\redlet{And the Father that sent me, he hath borne witness of me. Ye have neither heard his voice at any time, nor seen his form.}}
\bv{38}{\redlet{And ye have not his word abiding in you: for whom he sent, him ye believe not.}}
\chapsec{The Witness of the Scriptures}
\bv{39}{\redlet{Ye search the scriptures, because ye think that in them ye have eternal life; and these are they which bear witness of me;}}
\bv{40}{\redlet{and ye will not come to me, that ye may have life.}}
\bv{41}{\redlet{I receive not glory from men.}}
\bv{42}{\redlet{But I know you, that ye have not the love of God in yourselves.}}
\bv{43}{\redlet{I am come in my Father's name, and ye receive me not: if another shall come in his own name, him ye will receive.}}
\bv{44}{\redlet{How can ye believe, who receive glory one of another, and the glory that \supptext{cometh} from the only God ye seek not?}}
\bv{45}{\redlet{Think not that I will accuse you to the Father: there is one that accuseth you, \supptext{even} Moses, on whom ye have set your hope.}}
\bv{46}{\redlet{For if ye believed Moses, ye would believe me; for he wrote of me.}}
\bv{47}{\redlet{But if ye believe not his writings, how shall ye believe my words?''}}
\chaphead{Chapter VI}
\chapdesc{Feeding the Five Thousand}
\lettrine[image=true, lines=4, findent=3pt, nindent=0pt]{NT/John/Jn-After.eps}{fter} these things Jesus went away to the other side of the sea of Galilee, which is \supptext{the sea} of Tiberias.
\bv{2}{And a great multitude followed him, because they beheld the signs which he did on them that were sick.}
\bv{3}{And Jesus went up into the mountain, and there he sat with his disciples.}
\bv{4}{Now the passover, the feast of the Jews, was at hand.}
\bv{5}{Jesus therefore lifting up his eyes, and seeing that a great multitude cometh unto him, saith unto Philip, \redlet{``Whence are we to buy bread, that these may eat?''}}
\bv{6}{And this he said to prove him: for he himself knew what he would do.}
\bv{7}{Philip answered him, ``Two hundred denarii's\mcomm{One denarius was given for a day's wages.} worth of bread is not sufficient for them, that every one may take a little.''}
\bv{8}{One of his disciples, Andrew, Simon Peter's brother, saith unto him,}
\bv{9}{``There is a lad here, who hath five barley loaves, and two fishes: but what are these among so many?''}
\bv{10}{Jesus said, \redlet{``Make the people sit down.''} Now there was much grass in the place. So the men sat down, in number about five thousand.}
\bv{11}{Jesus therefore took the loaves; and having given thanks, he distributed to them that were set down; likewise also of the fishes as much as they would.}
\bv{12}{And when they were filled, he saith unto his disciples, \redlet{``Gather up the broken pieces which remain over, that nothing be lost.''}}
\bv{13}{So they gathered them up, and filled twelve baskets with broken pieces from the five barley loaves, which remained over unto them that had eaten.}
\bv{14}{When therefore the people saw the sign which he did, they said, ``This is of a truth the prophet that cometh into the world.''}
\chapsec{Jesus Walks upon the Sea}
\bv{15}{Jesus therefore perceiving that they were about to come and take him by force, to make him king, withdrew again into the mountain himself alone.}
\bv{16}{And when evening came, his disciples went down unto the sea;}
\bv{17}{and they entered into a boat, and were going over the sea unto Capernaum. And it was now dark, and Jesus had not yet come to them.}
\bv{18}{And the sea was rising by reason of a great wind that blew.}
\bv{19}{When therefore they had rowed about five and twenty or thirty furlongs, they behold Jesus walking on the sea, and drawing nigh unto the boat: and they were afraid.}
\bv{20}{But he saith unto them, \redlet{``It is I; be not afraid.''}}
\bv{21}{They were willing therefore to receive him into the boat: and straightway the boat was at the land whither they were going.}
\chapsec{The Bread of Life Discourse}
\bv{22}{On the morrow the multitude that stood on the other side of the sea saw that there was no other boat there, save one, and that Jesus entered not with his disciples into the boat, but \supptext{that} his disciples went away alone}
\bv{23}{(howbeit there came boats from Tiberias nigh unto the place where they ate the bread after the Lord had given thanks):}
\bv{24}{when the multitude therefore saw that Jesus was not there, neither his disciples, they themselves got into the boats, and came to Capernaum, seeking Jesus.}
\bv{25}{And when they found him on the other side of the sea, they said unto him, ``Rabbi, when camest thou hither?''}
\bv{26}{Jesus answered them and said, \redlet{``Verily, verily, I say unto you, Ye seek me, not because ye saw signs, but because ye ate of the loaves, and were filled.}}
\bv{27}{\redlet{Work not for the food which perisheth, but for the food which abideth unto eternal life, which the Son of man shall give unto you: for him the Father, \supptext{even} God, hath sealed.}}
\bv{28}{They said therefore unto him, ``What must we do, that we may work the works of God?''}
\bv{29}{Jesus answered and said unto them, \redlet{``This is the work of God, that ye believe on him whom he hath sent.''}}
\bv{30}{They said therefore unto him, ``What then doest thou for a sign, that we may see, and believe thee? what workest thou?}
\par
\bv{31}{Our fathers ate the manna in the wilderness; as it is written, `He gave them bread out of heaven to eat.'\mref{Neh. 9:15}''}
\bv{32}{Jesus therefore said unto them, \redlet{``Verily, verily, I say unto you, It was not Moses that gave you the bread out of heaven; but my Father giveth you the true bread out of heaven.}}
\bv{33}{\redlet{For the bread of God is that which cometh down out of heaven, and giveth life unto the world.''}}
\bv{34}{They said therefore unto him, ``Lord, evermore give us this bread.''}
\bv{35}{Jesus said unto them, \redlet{``I am the bread of life: he that cometh to me shall not hunger, and he that believeth on me shall never thirst.}}
\bv{36}{\redlet{But I said unto you, that ye have seen me, and yet believe not.}}
\bv{37}{\redlet{All that which the Father giveth me shall come unto me; and him that cometh to me I will in no wise cast out.}}
\bv{38}{\redlet{For I am come down from heaven, not to do mine own will, but the will of him that sent me.}}
\bv{39}{\redlet{And this is the will of him that sent me, that of all that which he hath given me I should lose nothing, but should raise it up at the last day.}}
\bv{40}{\redlet{For this is the will of my Father, that every one that beholdeth the Son, and believeth on him, should have eternal life; and I will raise him up at the last day.''}}
\par
\bv{41}{The Jews therefore murmured concerning him, because he said, ``I am the bread which came down out of heaven.''}
\bv{42}{And they said, ``Is not this Jesus, the son of Joseph, whose father and mother we know? how doth he now say, I am come down out of heaven?''}
\bv{43}{Jesus answered and said unto them, \redlet{``Murmur not among yourselves.}}
\bv{44}{\redlet{No man can come to me, except the Father that sent me draw him: and I will raise him up in the last day.}}
\bv{45}{\redlet{It is written in the prophets, \otQuote{Is. 54:13}{And they shall all be taught of God.} Every one that hath heard from the Father, and hath learned, cometh unto me.}}
\bv{46}{\redlet{Not that any man hath seen the Father, save he that is from God, he hath seen the Father.}}
\bv{47}{\redlet{Verily, verily, I say unto you, He that believeth hath eternal life.}}
\bv{48}{\redlet{I am the bread of life.}}
\bv{49}{\redlet{Your fathers ate the manna in the wilderness, and they died.}}
\bv{50}{\redlet{This is the bread which cometh down out of heaven, that a man may eat thereof, and not die.}}
\bv{51}{\redlet{I am the living bread which came down out of heaven: if any man eat of this bread, he shall live for ever: yea and the bread which I will give is my flesh, for the life of the world.''}}
\par
\bv{52}{The Jews therefore strove one with another, saying, ``How can this man give us his flesh to eat?''}
\bv{53}{Jesus therefore said unto them, \redlet{``Verily, verily, I say unto you, Except ye eat the flesh of the Son of man and drink his blood, ye have not life in yourselves.}}
\bv{54}{\redlet{He that feedeth on my flesh and drinketh my blood hath eternal life; and I will raise him up at the last day.}}
\bv{55}{\redlet{For my flesh is meat indeed, and my blood is drink indeed.}}
\bv{56}{\redlet{He that feedeth on my flesh and drinketh my blood abideth in me, and I in him.}}
\bv{57}{\redlet{As the living Father sent me, and I live because of the Father; so he that feedeth on me, he also shall live because of me.}}
\bv{58}{\redlet{This is the bread which came down out of heaven: not as the fathers ate, and died; he that feedeth on this bread shall live for ever.''}}
\bv{59}{These things said he in the synagogue, as he taught in Capernaum.}
\chapsec{Discipleship Tested by Doctrine}
\bv{60}{Many therefore of his disciples, when they heard \supptext{this}, said, ``This is a hard saying; who can hear it?''}
\bv{61}{But Jesus knowing in himself that his disciples murmured at this, said unto them, \redlet{``Doth this cause you to stumble?}}
\bv{62}{\redlet{\supptext{What} then if ye should behold the Son of man ascending where he was before?}}
\bv{63}{\redlet{It is the spirit that giveth life; the flesh profiteth nothing: the words that I have spoken unto you are spirit, and are life.}}
\bv{64}{\redlet{But there are some of you that believe not.''} For Jesus knew from the beginning who they were that believed not, and who it was that should betray him.}
\bv{65}{And he said, \redlet{``For this cause have I said unto you, that no man can come unto me, except it be given unto him of the Father.''}}
\bv{66}{Upon this many of his disciples went back, and walked no more with him.}
\chapsec{Peter's Confession of Faith}
\bv{67}{Jesus said therefore unto the twelve, \redlet{``Would ye also go away?''}}
\bv{68}{Simon Peter answered him, ``Lord, to whom shall we go? thou hast the words of eternal life.}
\bv{69}{And we have believed and know that thou art the Holy One of God.''}
\par
\bv{70}{Jesus answered them, \redlet{``Did not I choose you the twelve, and one of you is a devil?''}}
\bv{71}{Now he spake of Judas \supptext{the son} of Simon Iscariot, for he it was that should betray him, \supptext{being} one of the twelve.}
\chaphead{Chapter VII}
\chapdesc{Jesus Urged to Go to Judæa}
\lettrine[image=true, lines=4, findent=3pt, nindent=0pt]{NT/John/Jn-And.eps}{nd} after these things Jesus walked in Galilee: for he would not walk in Judæa, because the Jews sought to kill him.
\bv{2}{Now the feast of the Jews, the feast of tabernacles, was at hand.}
\bv{3}{His brethren therefore said unto him, ``Depart hence, and go into Judæa, that thy disciples also may behold thy works which thou doest.}
\bv{4}{For no man doeth anything in secret, and himself seeketh to be known openly. If thou doest these things, manifest thyself to the world.''}
\bv{5}{For even his brethren did not believe on him.}
\bv{6}{Jesus therefore saith unto them, \redlet{``My time is not yet come; but your time is always ready.}}
\bv{7}{\redlet{The world cannot hate you; but me it hateth, because I testify of it, that its works are evil.}}
\bv{8}{\redlet{Go ye up unto the feast: I go not up unto this feast; because my time is not yet fulfilled.''}}
\bv{9}{And having said these things unto them, he abode \supptext{still} in Galilee.}
\chapsec{Final Departure from Galilee}
\bv{10}{But when his brethren were gone up unto the feast, then went he also up, not publicly, but as it were in secret.}
\bv{11}{The Jews therefore sought him at the feast, and said, ``Where is he?''}
\bv{12}{And there was much murmuring among the multitudes concerning him: some said, ``He is a good man;'' others said, ``Not so, but he leadeth the multitude astray.''}
\bv{13}{Yet no man spake openly of him for fear of the Jews.}
\chapsec{Jesus at the Feast of Tabernacles}
\bv{14}{But when it was now the midst of the feast Jesus went up into the temple, and taught.}
\bv{15}{The Jews therefore marvelled, saying, ``How knoweth this man letters, having never learned?''}
\bv{16}{Jesus therefore answered them, and said, \redlet{``My teaching is not mine, but his that sent me.}}
\bv{17}{\redlet{If any man willeth to do his will, he shall know of the teaching, whether it is of God, or \supptext{whether} I speak from myself.}}
\bv{18}{\redlet{He that speaketh from himself seeketh his own glory: but he that seeketh the glory of him that sent him, the same is true, and no unrighteousness is in him.}}
\bv{19}{\redlet{Did not Moses give you the law, and \supptext{yet} none of you doeth the law? Why seek ye to kill me?''}}
\bv{20}{The multitude answered, ``Thou hast a demon: who seeketh to kill thee?''}
\bv{21}{Jesus answered and said unto them, \redlet{``I did one work, and ye all marvel because thereof.}}
\bv{22}{\redlet{Moses hath given you circumcision (not that it is of Moses, but of the fathers); and on the sabbath ye circumcise a man.}}
\bv{23}{\redlet{If a man receiveth circumcision on the sabbath, that the law of Moses may not be broken; are ye wroth with me, because I made a man every whit whole on the sabbath?}}
\bv{24}{\redlet{Judge not according to appearance, but judge righteous judgement.''}}
\par
\bv{25}{Some therefore of them of Jerusalem said, ``Is not this he whom they seek to kill?}
\bv{26}{And lo, he speaketh openly, and they say nothing unto him. Can it be that the rulers indeed know that this is the Christ?}
\bv{27}{Howbeit we know this man whence he is: but when the Christ cometh, no one knoweth whence he is.''}
\bv{28}{Jesus therefore cried in the temple, teaching and saying, \redlet{``Ye both know me, and know whence I am; and I am not come of myself, but he that sent me is true, whom ye know not.}}
\bv{29}{\redlet{I know him; because I am from him, and he sent me.''}}
\bv{30}{They sought therefore to take him: and no man laid his hand on him, because his hour was not yet come.}
\bv{31}{But of the multitude many believed on him; and they said, ``When the Christ shall come, will he do more signs than those which this man hath done?''}
\bv{32}{The Pharisees heard the multitude murmuring these things concerning him; and the chief priests and the Pharisees sent officers to take him.}
\bv{33}{Jesus therefore said, \redlet{``Yet a little while am I with you, and I go unto him that sent me.}}
\bv{34}{\redlet{Ye shall seek me, and shall not find me: and where I am, ye cannot come.''}}
\bv{35}{The Jews therefore said among themselves, ``Whither will this man go that we shall not find him? will he go unto the Dispersion among the Greeks, and teach the Greeks?}
\bv{36}{What is this word that he said, `Ye shall seek me, and shall not find me; and where I am, ye cannot come?' ''}
\chapsec{The Great Prophecy of the Holy Ghost}
\bv{37}{Now on the last day, the great \supptext{day} of the feast, Jesus stood and cried, saying, \redlet{``If any man thirst, let him come unto me and drink.}}
\bv{38}{\redlet{He that believeth on me, as the scripture hath said, \otQuote{Prov. 18:4}{`from within him shall flow rivers of living water.' ''}}}
\bv{39}{But this spake he of the Spirit, which they that believed on him were to receive: for the Spirit was not yet \supptext{given}; because Jesus was not yet glorified.}
\chapsec{The People Divided in Opinion}
\bv{40}{\supptext{Some} of the multitude therefore, when they heard these words, said, ``This is of a truth the prophet.''}
\bv{41}{Others said, ``This is the Christ.'' But some said, ``What, doth the Christ come out of Galilee?}
\bv{42}{Hath not the scripture said that the Christ cometh of the seed of David, and from Bethlehem, the village where David was?''}
\bv{43}{So there arose a division in the multitude because of him.}
\bv{44}{And some of them would have taken him; but no man laid hands on him.}
\bv{45}{The officers therefore came to the chief priests and Pharisees; and they said unto them, ``Why did ye not bring him?''}
\bv{46}{The officers answered, ``Never man so spake.''}
\bv{47}{The Pharisees therefore answered them, ``Are ye also led astray?}
\bv{48}{Hath any of the rulers believed on him, or of the Pharisees?}
\bv{49}{But this multitude that knoweth not the law are accursed.''}
\bv{50}{Nicodemus saith unto them (he that came to him before, being one of them),}
\bv{51}{``Doth our law judge a man, except it first hear from himself and know what he doeth?''}
\bv{52}{They answered and said unto him, ``Art thou also of Galilee? Search, and see that out of Galilee ariseth no prophet.''}
\chaphead{Chapter VIII}
\chapdesc{Jesus is the Light of the World}
\lettrine[image=true, lines=4, findent=3pt, nindent=0pt]{NT/John/Jn-Again.eps}{gain} therefore Jesus spake unto them, saying, \redlet{``I am the light of the world: he that followeth me shall not walk in the darkness, but shall have the light of life.''}
\bv{13}{The Pharisees therefore said unto him, ``Thou bearest witness of thyself; thy witness is not true.''}
\bv{14}{Jesus answered and said unto them, \redlet{``Even if I bear witness of myself, my witness is true; for I know whence I came, and whither I go; but ye know not whence I come, or whither I go.}}
\bv{15}{\redlet{Ye judge after the flesh; I judge no man.}}
\bv{16}{\redlet{Yea and if I judge, my judgement is true; for I am not alone, but I and the Father that sent me.}}
\bv{17}{\redlet{Yea and in your law it is written, that the witness of two men is true.}}
\bv{18}{\redlet{I am he that beareth witness of myself, and the Father that sent me beareth witness of me.''}}
\bv{19}{They said therefore unto him, ``Where is thy Father?'' Jesus answered, \redlet{``Ye know neither me, nor my Father: if ye knew me, ye would know my Father also.''}}
\bv{20}{These words spake he in the treasury, as he taught in the temple: and no man took him; because his hour was not yet come.}
\par
\bv{21}{He said therefore again unto them, \redlet{``I go away, and ye shall seek me, and shall die in your sin: whither I go, ye cannot come.''}}
\bv{22}{The Jews therefore said, ``Will he kill himself, that he saith, `Whither I go, ye cannot come?' ''}
\par
\bv{23}{And he said unto them, \redlet{``Ye are from beneath; I am from above: ye are of this world; I am not of this world.}}
\bv{24}{\redlet{I said therefore unto you, that ye shall die in your sins: for except ye believe that I am \supptext{he}, ye shall die in your sins.}}
\bv{25}{They said therefore unto him, ``Who art thou?'' Jesus said unto them, \redlet{``Even that which I have also spoken unto you from the beginning.}}
\bv{26}{\redlet{I have many things to speak and to judge concerning you: howbeit he that sent me is true; and the things which I heard from him, these speak I unto the world.''}}
\bv{27}{They perceived not that he spake to them of the Father.}
\bv{28}{Jesus therefore said, \redlet{``When ye have lifted up the Son of man, then shall ye know that I am \supptext{he}, and \supptext{that} I do nothing of myself, but as the Father taught me, I speak these things.}}
\bv{29}{\redlet{And he that sent me is with me; he hath not left me alone; for I do always the things that are pleasing to him.''}}
\bv{30}{As he spake these things, many believed on him.}
\chapsec{The True Seed of Abraham}
\bv{31}{Jesus therefore said to those Jews that had believed him, \redlet{``If ye abide in my word, \supptext{then} are ye truly my disciples;}}
\bv{32}{\redlet{and ye shall know the truth, and the truth shall make you free.''}}
\bv{33}{They answered unto him, ``We are Abraham's seed, and have never yet been in bondage to any man: how sayest thou, `Ye shall be made free?' ''}
\bv{34}{Jesus answered them, \redlet{``Verily, verily, I say unto you, Every one that committeth sin is the bondservant of sin.}}
\bv{35}{\redlet{And the bondservant abideth not in the house for ever: the son abideth for ever.}}
\bv{36}{\redlet{If therefore the Son shall make you free, ye shall be free indeed.}}
\bv{37}{\redlet{I know that ye are Abraham's seed; yet ye seek to kill me, because my word hath not free course in you.}}
\bv{38}{\redlet{I speak the things which I have seen with \supptext{my} Father: and ye also do the things which ye heard from \supptext{your} father.''}\mref{cf. Wis. 2:13-24}}
\bv{39}{They answered and said unto him, ``Our father is Abraham.'' Jesus saith unto them, \redlet{``If ye were Abraham's children, ye would do the works of Abraham.}}
\bv{40}{\redlet{But now ye seek to kill me, a man that hath told you the truth, which I heard from God: this did not Abraham.}}
\bv{41}{\redlet{Ye do the works of your father.''} They said unto him, ``We were not born of fornication; we have one Father, \supptext{even} God.''}
\bv{42}{Jesus said unto them, \redlet{``If God were your Father, ye would love me: for I came forth and am come from God; for neither have I come of myself, but he sent me.}}
\bv{43}{\redlet{Why do ye not understand my speech? \supptext{Even} because ye cannot hear my word.}}
\bv{44}{\redlet{Ye are of \supptext{your} father the devil, and the lusts of your father it is your will to do. He was a murderer from the beginning, and standeth not in the truth, because there is no truth in him. When he speaketh a lie, he speaketh of his own: for he is a liar, and the father thereof.\mref{cf. Wis. 2:24}}}
\bv{45}{\redlet{But because I say the truth, ye believe me not.}}
\bv{46}{\redlet{Which of you convicteth me of sin? If I say truth, why do ye not believe me?}}
\bv{47}{\redlet{He that is of God heareth the words of God: for this cause ye hear \supptext{them} not, because ye are not of God.''}}
\bv{48}{The Jews answered and said unto him, ``Say we not well that thou art a Samaritan, and hast a demon?''}
\bv{49}{Jesus answered, \redlet{``I have not a demon; but I honor my Father, and ye dishonor me.}}
\bv{50}{\redlet{But I seek not mine own glory: there is one that seeketh and judgeth.}}
\chapsec{The Pre-Existence of Jesus Christ}
\bv{51}{\redlet{Verily, verily, I say unto you, If a man keep my word, he shall never see death.''}}
\bv{52}{The Jews said unto him, ``Now we know that thou hast a demon. Abraham died, and the prophets; and thou sayest, `If a man keep my word, he shall never taste of death.'}
\bv{53}{Art thou greater than our father Abraham, who died? and the prophets died: whom makest thou thyself?''}
\bv{54}{Jesus answered, \redlet{``If I glorify myself, my glory is nothing: it is my Father that glorifieth me; of whom ye say, that he is your God;}}
\bv{55}{\redlet{and ye have not known him: but I know him; and if I should say, I know him not, I shall be like unto you, a liar: but I know him, and keep his word.}}
\bv{56}{\redlet{Your father Abraham rejoiced to see my day; and he saw it, and was glad.''}}
\bv{57}{The Jews therefore said unto him, ``Thou art not yet fifty years old, and hast thou seen Abraham?''}
\bv{58}{Jesus said unto them, \redlet{``Verily, verily, I say unto you, Before Abraham was, {\scshape I am}.''}}
\bv{59}{They took up stones therefore to cast at him: but Jesus hid himself, and went out of the temple.}
\chaphead{Chapter IX}
\chapdesc{The Man Born Blind is Healed}
\lettrine[image=true, lines=4, findent=3pt, nindent=0pt]{NT/John/Jn-And.eps}{nd} as he passed by, he saw a man blind from his birth.
\bv{2}{And his disciples asked him, saying, ``Rabbi, who sinned, this man, or his parents, that he should be born blind?''}
\bv{3}{Jesus answered, \redlet{``Neither did this man sin, nor his parents: but that the works of God should be made manifest in him.}}
\bv{4}{\redlet{We must work the works of him that sent me, while it is day: the night cometh, when no man can work.}}
\bv{5}{\redlet{When I am in the world, I am the light of the world.''}}
\bv{6}{When he had thus spoken, he spat on the ground, and made clay of the spittle, and anointed his eyes with the clay,}
\bv{7}{and said unto him, \redlet{``Go, wash in the pool of Siloam''} (which is by interpretation, ``Sent''). He went away therefore, and washed, and came seeing.}
\bv{8}{The neighbors therefore, and they that saw him aforetime, that he was a beggar, said, ``Is not this he that sat and begged?''}
\bv{9}{Others said, ``It is he:'' others said, ``No, but he is like him.'' He said, ``I am \supptext{he}.''}
\par
\bv{10}{They said therefore unto him, ``How then were thine eyes opened?''}
\bv{11}{He answered, ``The man that is called Jesus made clay, and anointed mine eyes, and said unto me, `Go to Siloam, and wash:' so I went away and washed, and I received sight.''}
\bv{12}{And they said unto him, ``Where is he?'' He saith, ``I know not.''}
\bv{13}{They bring to the Pharisees him that aforetime was blind.}
\bv{14}{Now it was the sabbath on the day when Jesus made the clay, and opened his eyes.}
\par
\bv{15}{Again therefore the Pharisees also asked him how he received his sight. And he said unto them, ``He put clay upon mine eyes, and I washed, and I see.''}
\bv{16}{Some therefore of the Pharisees said, ``This man is not from God, because he keepeth not the sabbath.'' But others said, ``How can a man that is a sinner do such signs?'' And there was a division among them.}
\bv{17}{They say therefore unto the blind man again, ``What sayest thou of him, in that he opened thine eyes?'' And he said, ``He is a prophet.''}
\bv{18}{The Jews therefore did not believe concerning him, that he had been blind, and had received his sight, until they called the parents of him that had received his sight,}
\bv{19}{and asked them, saying, ``Is this your son, who ye say was born blind? how then doth he now see?''}
\bv{20}{His parents answered and said, ``We know that this is our son, and that he was born blind:}
\bv{21}{but how he now seeth, we know not; or who opened his eyes, we know not: ask him; he is of age; he shall speak for himself.''}
\bv{22}{These things said his parents, because they feared the Jews: for the Jews had agreed already, that if any man should confess him \supptext{to be} Christ, he should be put out of the synagogue.}
\bv{23}{Therefore said his parents, ``He is of age; ask him.''}
\par
\bv{24}{So they called a second time the man that was blind, and said unto him, ``Give glory to God: we know that this man is a sinner.''}
\bv{25}{He therefore answered, ``Whether he is a sinner, I know not: one thing I know, that, whereas I was blind, now I see.''}
\bv{26}{They said therefore unto him, ``What did he to thee? how opened he thine eyes?''}
\bv{27}{He answered them, ``I told you even now, and ye did not hear; wherefore would ye hear it again? would ye also become his disciples?''}
\bv{28}{And they reviled him, and said, ``Thou art his disciple; but we are disciples of Moses.}
\bv{29}{We know that God hath spoken unto Moses: but as for this man, we know not whence he is.''}
\bv{30}{The man answered and said unto them, ``Why, herein is the marvel, that ye know not whence he is, and \supptext{yet} he opened mine eyes.}
\bv{31}{We know that God heareth not sinners: but if any man be a worshipper of God, and do his will, him he heareth.}
\bv{32}{Since the world began it was never heard that any one opened the eyes of a man born blind.}
\bv{33}{If this man were not from God, he could do nothing.''}
\bv{34}{They answered and said unto him, ``Thou wast altogether born in sins, and dost thou teach us?'' And they cast him out.}
\par
\bv{35}{Jesus heard that they had cast him out; and finding him, he said, \redlet{``Dost thou believe on the Son of God?''}}
\bv{36}{He answered and said, ``And who is he, Lord, that I may believe on him?''}
\bv{37}{Jesus said unto him, \redlet{``Thou hast both seen him, and he it is that speaketh with thee.''}}
\bv{38}{And he said, ``Lord, I believe.'' And he worshipped him.}
\bv{39}{And Jesus said, \redlet{``For judgement came I into this world, that they that see not may see; and that they that see may become blind.''}}
\bv{40}{Those of the Pharisees who were with him heard these things, and said unto him, ``Are we also blind?''}
\bv{41}{Jesus said unto them, \redlet{``If ye were blind, ye would have no sin: but now ye say, `We see:' your sin remaineth.}}
\chaphead{Chapter X}
\chapdesc{The Good Shepherd Discourse}
\lettrine[image=true, lines=4, findent=3pt, nindent=0pt]{NT/John/Jn10-V.eps}{\redlet{erily}}, \redlet{verily, I say unto you, He that entereth not by the door into the fold of the sheep, but climbeth up some other way, the same is a thief and a robber.}
\bv{2}{\redlet{But he that entereth in by the door is the shepherd of the sheep.}}
\bv{3}{\redlet{To him the porter openeth; and the sheep hear his voice: and he calleth his own sheep by name, and leadeth them out.}}
\bv{4}{\redlet{When he hath put forth all his own, he goeth before them, and the sheep follow him: for they know his voice.}}
\bv{5}{\redlet{And a stranger will they not follow, but will flee from him: for they know not the voice of strangers.''}}
\bv{6}{This parable spake Jesus unto them: but they understood not what things they were which he spake unto them.}
\par
\bv{7}{Jesus therefore said unto them again, \redlet{``Verily, verily, I say unto you, I am the door of the sheep.}}
\bv{8}{\redlet{All that came before me are thieves and robbers: but the sheep did not hear them.}}
\bv{9}{\redlet{I am the door; by me if any man enter in, he shall be saved, and shall go in and go out, and shall find pasture.}}
\bv{10}{\redlet{The thief cometh not, but that he may steal, and kill, and destroy: I came that they may have life, and may have \supptext{it} abundantly.}}
\bv{11}{\redlet{I am the good shepherd: the good shepherd layeth down his life for the sheep.}}
\bv{12}{\redlet{He that is a hireling, and not a shepherd, whose own the sheep are not, beholdeth the wolf coming, and leaveth the sheep, and fleeth, and the wolf snatcheth them, and scattereth \supptext{them}:}}
\bv{13}{\redlet{\supptext{he fleeth} because he is a hireling, and careth not for the sheep.}}
\bv{14}{\redlet{I am the good shepherd; and I know mine own, and mine own know me,}}
\bv{15}{\redlet{even as the Father knoweth me, and I know the Father; and I lay down my life for the sheep.}}
\bv{16}{\redlet{And other sheep I have, which are not of this fold: them also I must bring, and they shall hear my voice; and they shall become one flock, one shepherd.}}
\bv{17}{\redlet{Therefore doth the Father love me, because I lay down my life, that I may take it again.}}
\bv{18}{\redlet{No one taketh it away from me, but I lay it down of myself. I have power to lay it down, and I have power to take it again. This commandment received I from my Father.''}}
\par
\bv{19}{There arose a division again among the Jews because of these words.}
\bv{20}{And many of them said, ``He hath a demon, and is mad; why hear ye him?''}
\bv{21}{Others said, ``These are not the sayings of one possessed with a demon. Can a demon open the eyes of the blind?''}
\chapsec{Jesus Asserts His Deity}
\bv{22}{And it was the feast of the dedication at Jerusalem:}
\bv{23}{it was winter; and Jesus was walking in the temple in Solomon's porch.}
\bv{24}{The Jews therefore came round about him, and said unto him, ``How long dost thou hold us in suspense? If thou art the Christ, tell us plainly.''}
\bv{25}{Jesus answered them, \redlet{``I told you, and ye believe not: the works that I do in my Father's name, these bear witness of me.}}
\bv{26}{\redlet{But ye believe not, because ye are not of my sheep.}}
\bv{27}{\redlet{My sheep hear my voice, and I know them, and they follow me:}}
\bv{28}{\redlet{and I give unto them eternal life; and they shall never perish, and no one shall snatch them out of my hand.}}
\bv{29}{\redlet{My Father, who hath given \supptext{them} unto me, is greater than all; and no one is able to snatch \supptext{them} out of the Father's hand.}}
\bv{30}{\redlet{I and the Father are one.''}}
\par
\bv{31}{The Jews took up stones again to stone him.}
\bv{32}{Jesus answered them, \redlet{``Many good works have I showed you from the Father; for which of those works do ye stone me?''}}
\bv{33}{The Jews answered him, ``For a good work we stone thee not, but for blasphemy; and because that thou, being a man, makest thyself God.''}
\bv{34}{Jesus answered them, \redlet{``Is it not written in your law, \otQuote{Ps. 82:6}{`I said, Ye are gods?'}}}
\bv{35}{\redlet{If he called them gods, unto whom the word of God came (and the scripture cannot be broken),}}
\bv{36}{\redlet{say ye of him, whom the Father sanctified and sent into the world, Thou blasphemest; because I said, `I am \supptext{the} Son of God?'}}
\bv{37}{\redlet{If I do not the works of my Father, believe me not.}}
\bv{38}{\redlet{But if I do them, though ye believe not me, believe the works: that ye may know and understand that the Father is in me, and I in the Father.''}}
\bv{39}{They sought again to take him: and he went forth out of their hand.}
\chapsec{Jesus Goes to Where He Was Baptised}
\bv{40}{And he went away again beyond the Jordan into the place where John was at the first baptising; and there he abode.}
\bv{41}{And many came unto him; and they said, ``John indeed did no sign: but all things whatsoever John spake of this man were true.''}
\bv{42}{And many believed on him there.}
\chaphead{Chapter XI}
\chapdesc{The Raising of Lazarus}
\lettrine[image=true, lines=4, findent=3pt, nindent=0pt]{NT/John/Jn-Now.eps}{ow} a certain man was sick, Lazarus of Bethany, of the village of Mary and her sister Martha.
\bv{2}{And it was that Mary who anointed the Lord with ointment, and wiped his feet with her hair, whose brother Lazarus was sick.}
\bv{3}{The sisters therefore sent unto him, saying, ``Lord, behold, he whom thou lovest is sick.''}
\bv{4}{But when Jesus heard it, he said, \redlet{``This sickness is not unto death, but for the glory of God, that the Son of God may be glorified thereby.''}}
\bv{5}{Now Jesus loved Martha, and her sister, and Lazarus.}
\bv{6}{When therefore he heard that he was sick, he abode at that time two days in the place where he was.}
\par
\bv{7}{Then after this he saith to the disciples, \redlet{``Let us go into Judæa again.''}}
\bv{8}{The disciples say unto him, ``Rabbi, the Jews were but now seeking to stone thee; and goest thou thither again?''}
\bv{9}{Jesus answered, \redlet{``Are there not twelve hours in the day? If a man walk in the day, he stumbleth not, because he seeth the light of this world.}}
\bv{10}{\redlet{But if a man walk in the night, he stumbleth, because the light is not in him.''}}
\bv{11}{These things spake he: and after this he saith unto them, \redlet{``Our friend Lazarus is fallen asleep; but I go, that I may awake him out of sleep.''}}
\bv{12}{The disciples therefore said unto him, ``Lord, if he is fallen asleep, he will recover.''}
\bv{13}{Now Jesus had spoken of his death: but they thought that he spake of taking rest in sleep.}
\bv{14}{Then Jesus therefore said unto them plainly, \redlet{``Lazarus is dead.}}
\bv{15}{\redlet{And I am glad for your sakes that I was not there, to the intent ye may believe; nevertheless let us go unto him.''}}
\bv{16}{Thomas therefore, who is called Didymus, said unto his fellow-disciples, ``Let us also go, that we may die with him.''}
\par
\bv{17}{So when Jesus came, he found that he had been in the tomb four days already.}
\bv{18}{Now Bethany was nigh unto Jerusalem, about fifteen furlongs off;}
\bv{19}{and many of the Jews had come to Martha and Mary, to console them concerning their brother.}
\bv{20}{Martha therefore, when she heard that Jesus was coming, went and met him: but Mary still sat in the house.}
\bv{21}{Martha therefore said unto Jesus, ``Lord, if thou hadst been here, my brother had not died.}
\bv{22}{And even now I know that, whatsoever thou shalt ask of God, God will give thee.''}
\bv{23}{Jesus saith unto her, \redlet{``Thy brother shall rise again.''}}
\bv{24}{Martha saith unto him, ``I know that he shall rise again in the resurrection at the last day.''}
\bv{25}{Jesus said unto her, \redlet{``I am the resurrection, and the life: he that believeth on me, though he die, yet shall he live;}}
\bv{26}{\redlet{and whosoever liveth and believeth on me shall never die. Believest thou this?''}}
\bv{27}{She saith unto him, ``Yea, Lord: I have believed that thou art the Christ, the Son of God, \supptext{even} he that cometh into the world.''}
\par
\bv{28}{And when she had said this, she went away, and called Mary her sister secretly, saying, ``The Teacher is here, and calleth thee.''}
\bv{29}{And she, when she heard it, arose quickly, and went unto him.}
\bv{30}{(Now Jesus was not yet come into the village, but was still in the place where Martha met him.)}
\bv{31}{The Jews then who were with her in the house, and were consoling her, when they saw Mary, that she rose up quickly and went out, followed her, supposing that she was going unto the tomb to weep there.}
\bv{32}{Mary therefore, when she came where Jesus was, and saw him, fell down at his feet, saying unto him, ``Lord, if thou hadst been here, my brother had not died.''}
\bv{33}{When Jesus therefore saw her weeping, and the Jews \supptext{also} weeping who came with her, he groaned in the spirit, and was troubled,}
\bv{34}{and said, \redlet{``Where have ye laid him?''} They say unto him, ``Lord, come and see.''}
\bv{35}{Jesus wept.}
\par
\bv{36}{The Jews therefore said, ``Behold how he loved him!''}
\bv{37}{But some of them said, ``Could not this man, who opened the eyes of him that was blind, have caused that this man also should not die?''}
\bv{38}{Jesus therefore again groaning in himself cometh to the tomb. Now it was a cave, and a stone lay against it.}
\bv{39}{Jesus saith, \redlet{``Take ye away the stone.''} Martha, the sister of him that was dead, saith unto him, ``Lord, by this time the body decayeth; for he hath been \supptext{dead} four days.''}
\bv{40}{Jesus saith unto her, \redlet{``Said I not unto thee, that, if thou believedst, thou shouldest see the glory of God?''}}
\bv{41}{So they took away the stone. And Jesus lifted up his eyes, and said, \redlet{``Father, I thank thee that thou heardest me.}}
\bv{42}{\redlet{And I knew that thou hearest me always: but because of the multitude that standeth around I said it, that they may believe that thou didst send me.''}}
\bv{43}{And when he had thus spoken, he cried with a loud voice, \redlet{``Lazarus, come forth.''}}
\bv{44}{He that was dead came forth, bound hand and foot with grave-clothes; and his face was bound about with a napkin. Jesus saith unto them, \redlet{``Loose him, and let him go.''}}
\chapsec{The Friends of Mary Converted}
\bv{45}{Many therefore of the Jews, who came to Mary and beheld that which he did, believed on him.}
\bv{46}{But some of them went away to the Pharisees, and told them the things which Jesus had done.}
\chapsec{The Pharisees Plot Jesus' Death}
\bv{47}{The chief priests therefore and the Pharisees gathered a council, and said, ``What do we? for this man doeth many signs.}
\bv{48}{If we let him thus alone, all men will believe on him: and the Romans will come and take away both our place and our nation.''}
\bv{49}{But a certain one of them, Caiaphas, being high priest that year, said unto them, ``Ye know nothing at all,}
\bv{50}{nor do ye take account that it is expedient for you that one man should die for the people, and that the whole nation perish not.''}
\bv{51}{Now this he said not of himself: but being high priest that year, he prophesied that Jesus should die for the nation;}
\bv{52}{and not for the nation only, but that he might also gather together into one the children of God that are scattered abroad.}
\bv{53}{So from that day forth they took counsel that they might put him to death.}
\bv{54}{Jesus therefore walked no more openly among the Jews, but departed thence into the country near to the wilderness, into a city called Ephraim; and there he tarried with the disciples.}
\par
\bv{55}{Now the passover of the Jews was at hand: and many went up to Jerusalem out of the country before the passover, to purify themselves.}
\bv{56}{They sought therefore for Jesus, and spake one with another, as they stood in the temple, ``What think ye? That he will not come to the feast?''}
\bv{57}{Now the chief priests and the Pharisees had given commandment, that, if any man knew where he was, he should show it, that they might take him.}
\chaphead{Chapter XII}
\chapdesc{The Supper at Bethany}
\lettrine[image=true, lines=4, findent=3pt, nindent=0pt]{NT/John/Jn12-J.eps}{esus} therefore six days before the passover came to Bethany, where Lazarus was, whom Jesus raised from the dead.
\bv{2}{So they made him a supper there: and Martha served; but Lazarus was one of them that sat at meat with him.}
\bv{3}{Mary therefore took a pound of ointment of pure nard, very precious, and anointed the feet of Jesus, and wiped his feet with her hair: and the house was filled with the odor of the ointment.}
\bv{4}{But Judas Iscariot, one of his disciples, that should betray him, saith,}
\bv{5}{``Why was not this ointment sold for three hundred shillings, and given to the poor?''}
\bv{6}{Now this he said, not because he cared for the poor; but because he was a thief, and having the bag took away what was put therein.}
\bv{7}{Jesus therefore said, \redlet{``Suffer her to keep it against the day of my burying.}}
\bv{8}{\redlet{For the poor ye have always with you; but me ye have not always.''}}
\bv{9}{The common people therefore of the Jews learned that he was there: and they came, not for Jesus' sake only, but that they might see Lazarus also, whom he had raised from the dead.}
\bv{10}{But the chief priests took counsel that they might put Lazarus also to death;}
\bv{11}{because that by reason of him many of the Jews went away, and believed on Jesus.}
\chapsec{The Triumphal Entry}
\bv{12}{On the morrow a great multitude that had come to the feast, when they heard that Jesus was coming to Jerusalem,}
\bv{13}{took the branches of the palm trees, and went forth to meet him, and cried out, ``Hosanna: Blessed \supptext{is} he that cometh in the name of the Lord, even the King of Israel.''}
\bv{14}{And Jesus, having found a young ass, sat thereon; as it is written,}
\otQuote{Zech. 9:9}{\bv{15}{Fear not, daughter of Zion: behold, thy King cometh, sitting on an ass's colt.}}
\bv{16}{These things understood not his disciples at the first: but when Jesus was glorified, then remembered they that these things were written of him, and that they had done these things unto him.}
\bv{17}{The multitude therefore that was with him when he called Lazarus out of the tomb, and raised him from the dead, bare witness.}
\bv{18}{For this cause also the multitude went and met him, for that they heard that he had done this sign.}
\bv{19}{The Pharisees therefore said among themselves, ``Behold how ye prevail nothing; lo, the world is gone after him.''}
\chapsec{Certain Greeks Would See Jesus}
\bv{20}{Now there were certain Greeks among those that went up to worship at the feast:}
\bv{21}{these therefore came to Philip, who was of Bethsaida of Galilee, and asked him, saying, ``Sir, we would see Jesus.''}
\bv{22}{Philip cometh and telleth Andrew: Andrew cometh, and Philip, and they tell Jesus.}
\chapsec{Jesus' Answer}
\bv{23}{And Jesus answereth them, saying, \redlet{``The hour is come, that the Son of man should be glorified.}}
\bv{24}{\redlet{Verily, verily, I say unto you, Except a grain of wheat fall into the earth and die, it abideth by itself alone; but if it die, it beareth much fruit.}}
\bv{25}{\redlet{He that loveth his life loseth it; and he that hateth his life in this world shall keep it unto life eternal.}}
\bv{26}{\redlet{If any man serve me, let him follow me; and where I am, there shall also my servant be: if any man serve me, him will the Father honor.}}
\bv{27}{\redlet{Now is my soul troubled; and what shall I say? Father, save me from this hour. But for this cause came I unto this hour.}}
\bv{28}{\redlet{Father, glorify thy name.''} There came therefore a voice out of heaven, \supptext{saying}, \god{``I have both glorified it, and will glorify it again.''}}
\bv{29}{The multitude therefore, that stood by, and heard it, said that it had thundered: others said, ``An angel hath spoken to him.''}
\bv{30}{Jesus answered and said, \redlet{``This voice hath not come for my sake, but for your sakes.}}
\bv{31}{\redlet{Now is the judgement of this world: now shall the prince of this world be cast out.}}
\bv{32}{\redlet{And I, if I be lifted up from the earth, will draw all men unto myself.''}}
\bv{33}{But this he said, signifying by what manner of death he should die.}
\bv{34}{The multitude therefore answered him, ``We have heard out of the law that the Christ abideth for ever: and how sayest thou, `The Son of man must be lifted up?' who is this Son of man?''}
\bv{35}{Jesus therefore said unto them, \redlet{``Yet a little while is the light among you. Walk while ye have the light, that darkness overtake you not: and he that walketh in the darkness knoweth not whither he goeth.}}
\bv{36}{\redlet{While ye have the light, believe on the light, that ye may become sons of light.''} These things spake Jesus, and he departed and hid himself from them.}
\par
\bv{37}{But though he had done so many signs before them, yet they believed not on him:}
\bv{38}{that the word of Isaiah the prophet might be fulfilled, which he spake,}
\otQuote{Is. 53:1}{Lord, who hath believed our report? And to whom hath the arm of the Lord been revealed?}
\bv{39}{For this cause they could not believe, for that Isaiah said again,}
\otQuote{Is. 6:10}{\bv{40}{He hath blinded their eyes, and he hardened their heart; Lest they should see with their eyes, and perceive with their heart, And should turn, And I should heal them.}}
\bv{41}{These things said Isaiah, because he saw his glory; and he spake of him.}
\bv{42}{Nevertheless even of the rulers many believed on him; but because of the Pharisees they did not confess \supptext{it}, lest they should be put out of the synagogue:}
\bv{43}{for they loved the glory \supptext{that is} of men more than the glory \supptext{that is} of God.}
\par
\bv{44}{And Jesus cried and said, \redlet{``He that believeth on me, believeth not on me, but on him that sent me.}}
\bv{45}{\redlet{And he that beholdeth me beholdeth him that sent me.}}
\bv{46}{\redlet{I am come a light into the world, that whosoever believeth on me may not abide in the darkness.}}
\bv{47}{\redlet{And if any man hear my sayings, and keep them not, I judge him not: for I came not to judge the world, but to save the world.}}
\bv{48}{\redlet{He that rejecteth me, and receiveth not my sayings, hath one that judgeth him: the word that I spake, the same shall judge him in the last day.}}
\bv{49}{\redlet{For I spake not from myself; but the Father that sent me, he hath given me a commandment, what I should say, and what I should speak.}}
\bv{50}{\redlet{And I know that his commandment is life eternal; the things therefore which I speak, even as the Father hath said unto me, so I speak.''}}
\chaphead{Chapter XIII}
\chapdesc{The Last Passover}
\lettrine[image=true, lines=4, findent=3pt, nindent=0pt]{NT/John/Jn-Now.eps}{ow} before the feast of the passover, Jesus knowing that his hour was come that he should depart out of this world unto the Father, having loved his own that were in the world, he loved them unto the end.
\chapsec{Jesus Washes the Disciples' Feet}
\bv{2}{And during supper, the devil having already put into the heart of Judas Iscariot, Simon's \supptext{son}, to betray him,}
\bv{3}{\supptext{Jesus}, knowing that the Father had given all things into his hands, and that he came forth from God, and goeth unto God,}
\bv{4}{riseth from supper, and layeth aside his garments; and he took a towel, and girded himself.}
\bv{5}{Then he poureth water into the basin, and began to wash the disciples' feet, and to wipe them with the towel wherewith he was girded.}
\bv{6}{So he cometh to Simon Peter. He saith unto him, ``Lord, dost thou wash my feet?''}
\bv{7}{Jesus answered and said unto him, \redlet{``What I do thou knowest not now; but thou shalt understand hereafter.''}}
\bv{8}{Peter saith unto him, ``Thou shalt never wash my feet.'' Jesus answered him, \redlet{``If I wash thee not, thou hast no part with me.''}}
\bv{9}{Simon Peter saith unto him, ``Lord, not my feet only, but also my hands and my head.''}
\bv{10}{Jesus saith to him, \redlet{``He that is bathed needeth not save to wash his feet, but is clean every whit: and ye are clean, but not all.''}}
\bv{11}{For he knew him that should betray him; therefore said he, \redlet{``Ye are not all clean.''}}
\par
\bv{12}{So when he had washed their feet, and taken his garments, and sat down again, he said unto them, \redlet{``Know ye what I have done to you?}}
\bv{13}{\redlet{Ye call me, Teacher, and, Lord: and ye say well; for so I am.}}
\bv{14}{\redlet{If I then, the Lord and the Teacher, have washed your feet, ye also ought to wash one another's feet.}}
\bv{15}{\redlet{For I have given you an example, that ye also should do as I have done to you.}}
\bv{16}{\redlet{Verily, verily, I say unto you, A servant is not greater than his lord; neither one that is sent greater than he that sent him.}}
\bv{17}{\redlet{If ye know these things, blessed are ye if ye do them.}}
\bv{18}{\redlet{I speak not of you all: I know whom I have chosen: but that the scripture may be fulfilled, He that eateth my bread lifted up his heel against me.}}
\bv{19}{\redlet{From henceforth I tell you before it come to pass, that, when it is come to pass, ye may believe that I am \supptext{he}.}}
\bv{20}{\redlet{Verily, verily, I say unto you, He that receiveth whomsoever I send receiveth me; and he that receiveth me receiveth him that sent me.''}}
\chapsec{Jesus Foretells His Betrayal}
\bv{21}{When Jesus had thus said, he was troubled in the spirit, and testified, and said, \redlet{``Verily, verily, I say unto you, that one of you shall betray me.''}}
\bv{22}{The disciples looked one on another, doubting of whom he spake.}
\bv{23}{There was at the table reclining in Jesus' bosom one of his disciples, whom Jesus loved.}
\bv{24}{Simon Peter therefore beckoneth to him, and saith unto him, ``Tell \supptext{us} who it is of whom he speaketh.''}
\bv{25}{He leaning back, as he was, on Jesus' breast saith unto him, ``Lord, who is it?''}
\bv{26}{Jesus therefore answereth, \redlet{``He it is, for whom I shall dip the sop, and give it him.''} So when he had dipped the sop, he taketh and giveth it to Judas, \supptext{the son} of Simon Iscariot.}
\bv{27}{And after the sop, then entered Satan into him. Jesus therefore saith unto him, \redlet{``What thou doest, do quickly.''}}
\bv{28}{Now no man at the table knew for what intent he spake this unto him.}
\bv{29}{For some thought, because Judas had the bag, that Jesus said unto him, \redlet{``Buy what things we have need of for the feast;''} or, that he should give something to the poor.}
\bv{30}{He then having received the sop went out straightway: and it was night.}
\par
\bv{31}{When therefore he was gone out, Jesus saith, \redlet{``Now is the Son of man glorified, and God is glorified in him;}}
\bv{32}{\redlet{and God shall glorify him in himself, and straightway shall he glorify him.}}
\bv{33}{\redlet{Little children, yet a little while I am with you. Ye shall seek me: and as I said unto the Jews, `Whither I go, ye cannot come;' so now I say unto you.}}
\bv{34}{\redlet{A new commandment I give unto you, that ye love one another; even as I have loved you, that ye also love one another.}}
\bv{35}{\redlet{By this shall all men know that ye are my disciples, if ye have love one to another.''}}
\chapsec{Jesus Foretells Peter's Denial}
\bv{36}{Simon Peter saith unto him, ``Lord, whither goest thou?'' Jesus answered, \redlet{``Whither I go, thou canst not follow me now; but thou shalt follow afterwards.''}}
\bv{37}{Peter saith unto him, ``Lord, why cannot I follow thee even now? I will lay down my life for thee.''}
\bv{38}{Jesus answereth, \redlet{``Wilt thou lay down thy life for me? Verily, verily, I say unto thee, The cock shall not crow, till thou hast denied me thrice.}}
\chaphead{Chapter XIV}
\chapdesc{Jesus Foretells His Coming for His Own}
\lettrine[image=true, lines=4, findent=3pt, nindent=0pt]{NT/John/Jn14-L.eps}{\redlet{et}} \redlet{not your heart be troubled: believe in God, believe also in me.}
\bv{2}{\redlet{In my Father's house are many mansions; if it were not so, I would have told you; for I go to prepare a place for you.}}
\bv{3}{\redlet{And if I go and prepare a place for you, I come again, and will receive you unto myself; that where I am, \supptext{there} ye may be also.}}
\bv{4}{\redlet{And whither I go, ye know the way.''}}
\bv{5}{Thomas saith unto him, ``Lord, we know not whither thou goest; how know we the way?''}
\bv{6}{Jesus saith unto him, \redlet{``I am the way, and the truth, and the life: no one cometh unto the Father, but by me.'}}
\chapsec{Jesus \& the Father Are One}
\bv{7}{\redlet{If ye had known me, ye would have known my Father also: from henceforth ye know him, and have seen him.''}}
\bv{8}{Philip saith unto him, ``Lord, show us the Father, and it sufficeth us.''}
\bv{9}{Jesus saith unto him, \redlet{``Have I been so long time with you, and dost thou not know me, Philip? he that hath seen me hath seen the Father; how sayest thou, `Show us the Father?'}}
\bv{10}{\redlet{Believest thou not that I am in the Father, and the Father in me? the words that I say unto you I speak not from myself: but the Father abiding in me doeth his works.}}
\bv{11}{\redlet{Believe me that I am in the Father, and the Father in me: or else believe me for the very works' sake.}}
\bv{12}{\redlet{Verily, verily, I say unto you, He that believeth on me, the works that I do shall he do also; and greater \supptext{works} than these shall he do; because I go unto the Father.}}
\chapsec{The New Promise in Prayer}
\bv{13}{\redlet{And whatsoever ye shall ask in my name, that will I do, that the Father may be glorified in the Son.}}
\bv{14}{\redlet{If ye shall ask anything in my name, that will I do.}}
\bv{15}{\redlet{If ye love me, ye will keep my commandments.}}
\chapsec{The Promise of the Spirit}
\bv{16}{\redlet{And I will pray the Father, and he shall give you another Comforter, that he may be with you for ever,}}
\bv{17}{\redlet{\supptext{even} the Spirit of truth: whom the world cannot receive; for it beholdeth him not, neither knoweth him: ye know him; for he abideth with you, and shall be in you.}}
\bv{18}{\redlet{I will not leave you desolate: I come unto you.}}
\bv{19}{\redlet{Yet a little while, and the world beholdeth me no more; but ye behold me: because I live, ye shall live also.}}
\bv{20}{\redlet{In that day ye shall know that I am in my Father, and ye in me, and I in you.}}
\bv{21}{\redlet{He that hath my commandments, and keepeth them, he it is that loveth me: and he that loveth me shall be loved of my Father, and I will love him, and will manifest myself unto him.''}}
\par
\bv{22}{Judas (not Iscariot) saith unto him, ``Lord, what is come to pass that thou wilt manifest thyself unto us, and not unto the world?''}
\bv{23}{Jesus answered and said unto him, \redlet{``If a man love me, he will keep my word: and my Father will love him, and we will come unto him, and make our abode with him.}}
\bv{24}{\redlet{He that loveth me not keepeth not my words: and the word which ye hear is not mine, but the Father's who sent me.}}
\bv{25}{\redlet{These things have I spoken unto you, while \supptext{yet} abiding with you.}}
\bv{26}{\redlet{But the Comforter, \supptext{even} the Holy Ghost, whom the Father will send in my name, he shall teach you all things, and bring to your remembrance all that I said unto you.}}
\chapsec{The Bequest of Peace}
\bv{27}{\redlet{Peace I leave with you; my peace I give unto you: not as the world giveth, give I unto you. Let not your heart be troubled, neither let it be fearful.}}
\bv{28}{\redlet{Ye heard how I said to you, I go away, and I come unto you. If ye loved me, ye would have rejoiced, because I go unto the Father: for the Father is greater than I.}}
\bv{29}{\redlet{And now I have told you before it come to pass, that, when it is come to pass, ye may believe.}}
\bv{30}{\redlet{I will no more speak much with you, for the prince of the world cometh: and he hath nothing in me;}}
\bv{31}{\redlet{but that the world may know that I love the Father, and as the Father gave me commandment, even so I do. Arise, let us go hence.}}
\chaphead{Chapter XV}
\chapdesc{The Vine \& Branches}
\lettrine[image=true, lines=4, findent=3pt, nindent=0pt]{NT/John/Jn-I.eps}{} \redlet{am the true vine, and my Father is the husbandman.}
\bv{2}{\redlet{Every branch in me that beareth not fruit, he taketh it away: and every \supptext{branch} that beareth fruit, he cleanseth it, that it may bear more fruit.}}
\bv{3}{\redlet{Already ye are clean because of the word which I have spoken unto you.}}
\bv{4}{\redlet{Abide in me, and I in you. As the branch cannot bear fruit of itself, except it abide in the vine; so neither can ye, except ye abide in me.}}
\bv{5}{\redlet{I am the vine, ye are the branches: He that abideth in me, and I in him, the same beareth much fruit: for apart from me ye can do nothing.}}
\par
\bv{6}{\redlet{If a man abide not in me, he is cast forth as a branch, and is withered; and they gather them, and cast them into the fire, and they are burned.}}
\bv{7}{\redlet{If ye abide in me, and my words abide in you, ask whatsoever ye will, and it shall be done unto you.}}
\bv{8}{\redlet{Herein is my Father glorified, that ye bear much fruit; and \supptext{so} shall ye be my disciples.}}
\bv{9}{\redlet{Even as the Father hath loved me, I also have loved you: abide ye in my love.}}
\bv{10}{\redlet{If ye keep my commandments, ye shall abide in my love; even as I have kept my Father's commandments, and abide in his love.}}
\bv{11}{\redlet{These things have I spoken unto you, that my joy may be in you, and \supptext{that} your joy may be made full.}}
\par
\bv{12}{\redlet{This is my commandment, that ye love one another, even as I have loved you.}}
\bv{13}{\redlet{Greater love hath no man than this, that a man lay down his life for his friends.}}
\bv{14}{\redlet{Ye are my friends, if ye do the things which I command you.}}
\chapsec{New Intimacy}
\bv{15}{\redlet{No longer do I call you servants; for the servant knoweth not what his lord doeth: but I have called you friends; for all things that I heard from my Father I have made known unto you.}}
\bv{16}{\redlet{Ye did not choose me, but I chose you, and appointed you, that ye should go and bear fruit, and \supptext{that} your fruit should abide: that whatsoever ye shall ask of the Father in my name, he may give it you.}}
\bv{17}{\redlet{These things I command you, that ye may love one another.}}
\chapsec{The Believer \& the World}
\bv{18}{\redlet{If the world hateth you, ye know that it hath hated me before \supptext{it hated} you.}}
\bv{19}{\redlet{If ye were of the world, the world would love its own: but because ye are not of the world, but I chose you out of the world, therefore the world hateth you.}}
\bv{20}{\redlet{Remember the word that I said unto you, A servant is not greater than his lord. If they persecuted me, they will also persecute you; if they kept my word, they will keep yours also.}}
\bv{21}{\redlet{But all these things will they do unto you for my name's sake, because they know not him that sent me.}}
\bv{22}{\redlet{If I had not come and spoken unto them, they had not had sin: but now they have no excuse for their sin.}}
\bv{23}{\redlet{He that hateth me hateth my Father also.}}
\bv{24}{\redlet{If I had not done among them the works which none other did, they had not had sin: but now have they both seen and hated both me and my Father.}}
\bv{25}{\redlet{But \supptext{this cometh to pass}, that the word may be fulfilled that is written in their law,}}
\otQuote{Ps. 35:19}{They hated me without a cause.}
\chapsec{The Believer \& the Spirit}
\bv{26}{\redlet{But when the Comforter is come, whom I will send unto you from the Father, \supptext{even} the Spirit of truth, which proceedeth from the Father, he shall bear witness of me:}}
\bv{27}{\redlet{and ye also bear witness, because ye have been with me from the beginning.}}
\chaphead{Chapter XVI}
\chapdesc{The Disciples Warned of Persecutions}
\lettrine[image=true, lines=4, findent=3pt, nindent=0pt]{NT/John/Jn-These.eps}{\redlet{hese}} \redlet{things have I spoken unto you, that ye should not be caused to stumble.}
\bv{2}{\redlet{They shall put you out of the synagogues: yea, the hour cometh, that whosoever killeth you shall think that he offereth service unto God.}}
\bv{3}{\redlet{And these things will they do, because they have not known the Father, nor me.}}
\bv{4}{\redlet{But these things have I spoken unto you, that when their hour is come, ye may remember them, how that I told you. And these things I said not unto you from the beginning, because I was with you.}}
\bv{5}{\redlet{But now I go unto him that sent me; and none of you asketh me, `Whither goest thou?'}}
\bv{6}{\redlet{But because I have spoken these things unto you, sorrow hath filled your heart.}}
\chapsec{Threefold Work of the Spirit}
\bv{7}{\redlet{Nevertheless I tell you the truth: It is expedient for you that I go away; for if I go not away, the Comforter will not come unto you; but if I go, I will send him unto you.}}
\bv{8}{\redlet{And he, when he is come, will convict the world in respect of sin, and of righteousness, and of judgement:}}
\bv{9}{\redlet{of sin, because they believe not on me;}}
\bv{10}{\redlet{of righteousness, because I go to the Father, and ye behold me no more;}}
\bv{11}{\redlet{of judgement, because the prince of this world hath been judged.}}
\chapsec{New Truth to be Revealed by the Spirit}
\bv{12}{\redlet{I have yet many things to say unto you, but ye cannot bear them now.}}
\bv{13}{\redlet{Howbeit when he, the Spirit of truth, is come, he shall guide you into all the truth: for he shall not speak from himself; but what things soever he shall hear, \supptext{these} shall he speak: and he shall declare unto you the things that are to come.}}
\bv{14}{\redlet{He shall glorify me: for he shall take of mine, and shall declare \supptext{it} unto you.}}
\bv{15}{\redlet{All things whatsoever the Father hath are mine: therefore said I, that he taketh of mine, and shall declare \supptext{it} unto you.}}
\bv{16}{\redlet{A little while, and ye behold me no more; and again a little while, and ye shall see me.''}}
\par
\bv{17}{\supptext{Some} of his disciples therefore said one to another, ``What is this that he saith unto us, `A little while, and ye behold me not; and again `a little while, and ye shall see me:' and, `Because I go to the Father?' ''}
\bv{18}{They said therefore, ``What is this that he saith, `A little while?' We know not what he saith.''}
\bv{19}{Jesus perceived that they were desirous to ask him, and he said unto them, \redlet{Do ye inquire among yourselves concerning this, that I said, `A little while, and ye behold me not, and again a little while, and ye shall see me?'}}
\bv{20}{\redlet{Verily, verily, I say unto you, that ye shall weep and lament, but the world shall rejoice: ye shall be sorrowful, but your sorrow shall be turned into joy.}}
\bv{21}{\redlet{A woman when she is in travail hath sorrow, because her hour is come: but when she is delivered of the child, she remembereth no more the anguish, for the joy that a man is born into the world.}}
\bv{22}{\redlet{And ye therefore now have sorrow: but I will see you again, and your heart shall rejoice, and your joy no one taketh away from you.}}
\bv{23}{\redlet{And in that day ye shall ask me no question. Verily, verily, I say unto you, If ye shall ask anything of the Father, he will give it you in my name.}}
\bv{24}{\redlet{Hitherto have ye asked nothing in my name: ask, and ye shall receive, that your joy may be made full.}}
\bv{25}{\redlet{These things have I spoken unto you in dark sayings: the hour cometh, when I shall no more speak unto you in dark sayings, but shall tell you plainly of the Father.}}
\bv{26}{\redlet{In that day ye shall ask in my name: and I say not unto you, that I will pray the Father for you;}}
\bv{27}{\redlet{for the Father himself loveth you, because ye have loved me, and have believed that I came forth from the Father.}}
\bv{28}{\redlet{I came out from the Father, and am come into the world: again, I leave the world, and go unto the Father.''}}
\par
\bv{29}{His disciples say, ``Lo, now speakest thou plainly, and speakest no dark saying.}
\bv{30}{Now know we that thou knowest all things, and needest not that any man should ask thee: by this we believe that thou camest forth from God.''}
\bv{31}{Jesus answered them, \redlet{``Do ye now believe?}}
\bv{32}{\redlet{Behold, the hour cometh, yea, is come, that ye shall be scattered, every man to his own, and shall leave me alone: and \supptext{yet} I am not alone, because the Father is with me.}}
\bv{33}{\redlet{These things have I spoken unto you, that in me ye may have peace. In the world ye have tribulation: but be of good cheer; I have overcome the world.''}}
\chaphead{Chapter XVII}
\chapdesc{The High Priestly Prayer}
\lettrine[image=true, lines=4, findent=3pt, nindent=0pt]{NT/John/Jn-These.eps}{hese} things spake Jesus; and lifting up his eyes to heaven, he said, \redlet{``Father, the hour is come; glorify thy Son, that the Son may glorify thee:}
\bv{2}{\redlet{even as thou gavest him authority over all flesh, that to all whom thou hast given him, he should give eternal life.}}
\bv{3}{\redlet{And this is life eternal, that they should know thee the only true God, and him whom thou didst send, \supptext{even} Jesus Christ.}}
\bv{4}{\redlet{I glorified thee on the earth, having accomplished the work which thou hast given me to do.}}
\bv{5}{\redlet{And now, Father, glorify thou me with thine own self with the glory which I had with thee before the world was.}}
\par
\bv{6}{\redlet{I manifested thy name unto the men whom thou gavest me out of the world: thine they were, and thou gavest them to me; and they have kept thy word.}}
\bv{7}{\redlet{Now they know that all things whatsoever thou hast given me are from thee:}}
\bv{8}{\redlet{for the words which thou gavest me I have given unto them; and they received \supptext{them}, and knew of a truth that I came forth from thee, and they believed that thou didst send me.}}
\bv{9}{\redlet{I pray for them: I pray not for the world, but for those whom thou hast given me; for they are thine:}}
\bv{10}{\redlet{and all things that are mine are thine, and thine are mine: and I am glorified in them.}}
\bv{11}{\redlet{And I am no more in the world, and these are in the world, and I come to thee. Holy Father, keep them in thy name which thou hast given me, that they may be one, even as we \supptext{are}.}}
\par
\bv{12}{\redlet{While I was with them, I kept them in thy name which thou hast given me: and I guarded them, and not one of them perished, but the son of perdition; that the scripture might be fulfilled.}}
\bv{13}{\redlet{But now I come to thee; and these things I speak in the world, that they may have my joy made full in themselves.}}
\bv{14}{\redlet{I have given them thy word; and the world hated them, because they are not of the world, even as I am not of the world.}}
\bv{15}{\redlet{I pray not that thou shouldest take them from the world, but that thou shouldest keep them from the evil \supptext{one}.}}
\bv{16}{\redlet{They are not of the world, even as I am not of the world.}}
\par
\bv{17}{\redlet{Sanctify them in the truth: thy word is truth.}}
\bv{18}{\redlet{As thou didst send me into the world, even so sent I them into the world.}}
\bv{19}{\redlet{And for their sakes I sanctify myself, that they themselves also may be sanctified in truth.}}
\bv{20}{\redlet{Neither for these only do I pray, but for them also that believe on me through their word;}}
\bv{21}{\redlet{that they may all be one; even as thou, Father, \supptext{art} in me, and I in thee, that they also may be in us: that the world may believe that thou didst send me.}}
\bv{22}{\redlet{And the glory which thou hast given me I have given unto them; that they may be one, even as we \supptext{are} one;}}
\par
\bv{23}{\redlet{I in them, and thou in me, that they may be perfected into one; that the world may know that thou didst send me, and lovedst them, even as thou lovedst me.}}
\bv{24}{\redlet{Father, I desire that they also whom thou hast given me be with me where I am, that they may behold my glory, which thou hast given me: for thou lovedst me before the foundation of the world.}}
\bv{25}{\redlet{O righteous Father, the world knew thee not, but I knew thee; and these knew that thou didst send me;}}
\bv{26}{\redlet{and I made known unto them thy name, and will make it known; that the love wherewith thou lovedst me may be in them, and I in them.''}}
\chaphead{Chapter XVIII}
\chapdesc{Jesus at Gethsemane}
\lettrine[image=true, lines=4, findent=3pt, nindent=0pt]{NT/John/Jn-When.eps}{hen} Jesus had spoken these words, he went forth with his disciples over the brook Kidron, where was a garden, into which he entered, himself and his disciples.
\chapsec{Jesus' Betrayal \& Arrest}
\bv{2}{Now Judas also, who betrayed him, knew the place: for Jesus oft-times resorted thither with his disciples.}
\bv{3}{Judas then, having received the band \supptext{of soldiers}, and officers from the chief priests and the Pharisees, cometh thither with lanterns and torches and weapons.}
\bv{4}{Jesus therefore, knowing all the things that were coming upon him, went forth, and saith unto them, \redlet{``Whom seek ye?''}}
\bv{5}{They answered him, ``Jesus of Nazareth.'' Jesus saith unto them, \redlet{\scshape I am}. And Judas also, who betrayed him, was standing with them.}
\bv{6}{When therefore he said unto them, \redlet{\scshape I am}, they went backward, and fell to the ground.}
\bv{7}{Again therefore he asked them, \redlet{``Whom seek ye?''} And they said, ``Jesus of Nazareth.''}
\bv{8}{Jesus answered, \redlet{``I told you that {\scshape I am}; if therefore ye seek me, let these go their way:''}}
\bv{9}{that the word might be fulfilled which he spake, \redlet{``Of those whom thou hast given me I lost not one.''}}
\par
\bv{10}{Simon Peter therefore having a sword drew it, and struck the high priest's servant, and cut off his right ear. Now the servant's name was Malchus.}
\bv{11}{Jesus therefore said unto Peter, \redlet{``Put up the sword into the sheath: the cup which the Father hath given me, shall I not drink it?''}}
\chapsec{Jesus Brought before the High Priest}
\bv{12}{So the band and the chief captain, and the officers of the Jews, seized Jesus and bound him,}
\bv{13}{and led him to Annas first; for he was father in law to Caiaphas, who was high priest that year.}
\par
\bv{14}{Now Caiaphas was he that gave counsel to the Jews, that it was expedient that one man should die for the people.}
\bv{15}{And Simon Peter followed Jesus, and \supptext{so did} another disciple. Now that disciple was known unto the high priest, and entered in with Jesus into the court of the high priest;}
\bv{16}{but Peter was standing at the door without. So the other disciple, who was known unto the high priest, went out and spake unto her that kept the door, and brought in Peter.}
\chapsec{St. Peter's First Denial}
\bv{17}{The maid therefore that kept the door saith unto Peter, ``Art thou also \supptext{one} of this man's disciples?'' He saith, ``I am not.''}
\bv{18}{Now the servants and the officers were standing \supptext{there}, having made a fire of coals; for it was cold; and they were warming themselves: and Peter also was with them, standing and warming himself.}
\chapsec{Interrogation by the High Priest}
\bv{19}{The high priest therefore asked Jesus of his disciples, and of his teaching.}
\bv{20}{Jesus answered him, \redlet{``I have spoken openly to the world; I ever taught in synagogues, and in the temple, where all the Jews come together; and in secret spake I nothing.}}
\bv{21}{\redlet{Why askest thou me? ask them that have heard \supptext{me}, what I spake unto them: behold, these know the things which I said.''}}
\bv{22}{And when he had said this, one of the officers standing by struck Jesus with his hand, saying, ``Answerest thou the high priest so?''}
\bv{23}{Jesus answered him, \redlet{``If I have spoken evil, bear witness of the evil: but if well, why smitest thou me?''}}
\bv{24}{Annas therefore sent him bound unto Caiaphas the high priest.}
\chapsec{St. Peter's Final Denials}
\bv{25}{Now Simon Peter was standing and warming himself. They said therefore unto him, ``Art thou also \supptext{one} of his disciples?'' He denied, and said, ``I am not.''}
\bv{26}{One of the servants of the high priest, being a kinsman of him whose ear Peter cut off, saith, ``Did not I see thee in the garden with him?''}
\bv{27}{Peter therefore denied again: and straightway the cock crew.}
\chapsec{Jesus Brought before Pilate}
\bv{28}{They lead Jesus therefore from Caiaphas into the Prætorium: and it was early; and they themselves entered not into the Prætorium, that they might not be defiled, but might eat the passover.}
\bv{29}{Pilate therefore went out unto them, and saith, ``What accusation bring ye against this man?''}
\bv{30}{They answered and said unto him, ``If this man were not an evil-doer, we should not have delivered him up unto thee.''}
\bv{31}{Pilate therefore said unto them, ``Take him yourselves, and judge him according to your law.'' The Jews said unto him, ``It is not lawful for us to put any man to death:''}
\bv{32}{that the word of Jesus might be fulfilled, which he spake, signifying by what manner of death he should die.}
\bv{33}{Pilate therefore entered again into the Prætorium, and called Jesus, and said unto him, ``Art thou the King of the Jews?''}
\bv{34}{Jesus answered, \redlet{``Sayest thou this of thyself, or did others tell it thee concerning me?''}}
\bv{35}{Pilate answered, ``Am I a Jew? Thine own nation and the chief priests delivered thee unto me: what hast thou done?''}
\bv{36}{Jesus answered, \redlet{``My kingdom is not of this world: if my kingdom were of this world, then would my servants fight, that I should not be delivered to the Jews: but now is my kingdom not from hence.''}}
\bv{37}{Pilate therefore said unto him, ``Art thou a king then?'' Jesus answered, \redlet{``Thou sayest that I am a king. To this end have I been born, and to this end am I come into the world, that I should bear witness unto the truth. Every one that is of the truth heareth my voice.''}}
\bv{38}{Pilate saith unto him, ``What is truth?'' And when he had said this, he went out again unto the Jews, and saith unto them, ``I find no crime in him.}
\chapsec{Jesus Condemned \& Barabbas Released}
\bv{39}{But ye have a custom, that I should release unto you one at the passover: will ye therefore that I release unto you the King of the Jews?''}
\bv{40}{They cried out therefore again, saying, ``Not this man, but Barabbas.'' Now Barabbas was a robber.}
\chaphead{Chapter XIX}
\chapdesc{Jesus Crowned with Thorns}
\lettrine[image=true, lines=4, findent=3pt, nindent=0pt]{NT/John/Jn-The.eps}{hen} Pilate therefore took Jesus, and scourged him.
\bv{2}{And the soldiers platted a crown of thorns, and put it on his head, and arrayed him in a purple garment;}
\bv{3}{and they came unto him, and said, ``Hail, King of the Jews!'' and they struck him with their hands.}
\chapsec{Pilate Brings Jesus before the Multitude}
\bv{4}{And Pilate went out again, and saith unto them, ``Behold, I bring him out to you, that ye may know that I find no crime in him.''}
\bv{5}{Jesus therefore came out, wearing the crown of thorns and the purple garment. And \supptext{Pilate} saith unto them, ``Behold, the man!''}
\bv{6}{When therefore the chief priests and the officers saw him, they cried out, saying, ``Crucify \supptext{him}, crucify \supptext{him}!'' Pilate saith unto them, ``Take him yourselves, and crucify him: for I find no crime in him.''}
\bv{7}{The Jews answered him, ``We have a law, and by that law he ought to die, because he made himself the Son of God.''}
\bv{8}{When Pilate therefore heard this saying, he was the more afraid;}
\bv{9}{and he entered into the Prætorium again, and saith unto Jesus, ``Whence art thou?'' But Jesus gave him no answer.}
\bv{10}{Pilate therefore saith unto him, ``Speakest thou not unto me? knowest thou not that I have power to release thee, and have power to crucify thee?''}
\bv{11}{Jesus answered him, \redlet{``Thou wouldest have no power against me, except it were given thee from above: therefore he that delivered me unto thee hath greater sin.''}}
\bv{12}{Upon this Pilate sought to release him: but the Jews cried out, saying, ``If thou release this man, thou art not Cæsar's friend: every one that maketh himself a king speaketh against Cæsar.''}
\bv{13}{When Pilate therefore heard these words, he brought Jesus out, and sat down on the judgement-seat at a place called The Pavement, but in Hebrew, Gabbatha.}
\chapsec{Final Rejection of Jesus by the Jews}
\bv{14}{Now it was the Preparation of the passover: it was about the sixth hour. And he saith unto the Jews, ``Behold, your King!''}
\bv{15}{They therefore cried out, ``Away with \supptext{him}, away with \supptext{him}, crucify him!'' Pilate saith unto them, ``Shall I crucify your King?'' The chief priests answered, ``We have no king but Cæsar.''}
\chapsec{The Crucifixion of Jesus Christ}
\bv{16}{Then therefore he delivered him unto them to be crucified.}
\bv{17}{They took Jesus therefore: and he went out, bearing the cross for himself, unto the place called The place of a skull, which is called in Hebrew Golgotha:}
\bv{18}{where they crucified him, and with him two others, on either side one, and Jesus in the midst.}
\bv{19}{And Pilate wrote a title also, and put it on the cross. And there was written, ``JESUS OF NAZARETH, THE KING OF THE JEWS.''}
\bv{20}{This title therefore read many of the Jews, for the place where Jesus was crucified was nigh to the city; and it was written in Hebrew, \supptext{and} in Latin, \supptext{and} in Greek.}
\bv{21}{The chief priests of the Jews therefore said to Pilate, ``Write not, `The King of the Jews;' but, that he said, `I am King of the Jews.' ''}
\bv{22}{Pilate answered, ``What I have written I have written.''}
\par
\bv{23}{The soldiers therefore, when they had crucified Jesus, took his garments and made four parts, to every soldier a part; and also the coat: now the coat was without seam, woven from the top throughout.}
\bv{24}{They said therefore one to another, ``Let us not rend it, but cast lots for it, whose it shall be:'' that the scripture might be fulfilled, which saith,}
\otQuote{Ps. 22:18}{They parted my garments among them, And upon my vesture did they cast lots.}
\par
\bv{25}{These things therefore the soldiers did. But there were standing by the cross of Jesus his mother, and his mother's sister, Mary the \supptext{wife} of Clopas, and Mary Magdalene.}
\chapsec{Mary given to the Beloved Disciple}
\bv{26}{When Jesus therefore saw his mother, and the disciple standing by whom he loved, he saith unto his mother, \redlet{``Woman, behold, thy son!''}}
\bv{27}{Then saith he to the disciple, \redlet{``Behold, thy mother!''} And from that hour the disciple took her unto his own \supptext{home}.}
\bv{28}{After this Jesus, knowing that all things are now finished, that the scripture might be accomplished, saith, \redlet{``I thirst.''}}
\chapsec{Consummation of the New Covenant}
\bv{29}{There was set there a vessel full of vinegar: so they put a sponge full of the vinegar upon hyssop, and brought it to his mouth.}
\bv{30}{When Jesus therefore had received the vinegar, he said, \redlet{``It is finished:''} and he bowed his head, and gave up his spirit.}
\chapsec{Old Testament Fulfilment}
\bv{31}{The Jews therefore, because it was the Preparation, that the bodies should not remain on the cross upon the sabbath (for the day of that sabbath was a high \supptext{day}), asked of Pilate that their legs might be broken, and \supptext{that} they might be taken away.}
\bv{32}{The soldiers therefore came, and brake the legs of the first, and of the other that was crucified with him:}
\bv{33}{but when they came to Jesus, and saw that he was dead already, they brake not his legs:}
\bv{34}{howbeit one of the soldiers with a spear pierced his side, and straightway there came out blood and water.}
\bv{35}{And he that hath seen hath borne witness, and his witness is true: and he knoweth that he saith true, that ye also may believe.}
\bv{36}{For these things came to pass, that the scripture might be fulfilled,}
\otQuote{Ex. 12:46}{A bone of him shall not be broken.}
\bv{37}{And again another scripture saith,}
\otQuote{Zech. 12:10}{They shall look on him whom they pierced.}
\chapsec{The Entombment}
\bv{38}{And after these things Joseph of Arimathæa, being a disciple of Jesus, but secretly for fear of the Jews, asked of Pilate that he might take away the body of Jesus: and Pilate gave \supptext{him} leave. He came therefore, and took away his body.}
\bv{39}{And there came also Nicodemus, he who at the first came to him by night, bringing a mixture of myrrh and aloes, about a hundred pounds.}
\bv{40}{So they took the body of Jesus, and bound it in linen cloths with the spices, as the custom of the Jews is to bury.}
\bv{41}{Now in the place where he was crucified there was a garden; and in the garden a new tomb wherein was never man yet laid.}
\bv{42}{There then because of the Jews' Preparation (for the tomb was nigh at hand) they laid Jesus.}
\chaphead{Chapter XX}
\chapdesc{The Resurrection of Jesus Christ}
\lettrine[image=true, lines=4, findent=3pt, nindent=0pt]{NT/John/Jn-Now.eps}{ow} on the first \supptext{day} of the week cometh Mary Magdalene early, while it was yet dark, unto the tomb, and seeth the stone taken away from the tomb.
\bv{2}{She runneth therefore, and cometh to Simon Peter, and to the other disciple whom Jesus loved, and saith unto them, ``They have taken away the Lord out of the tomb, and we know not where they have laid him.''}
\bv{3}{Peter therefore went forth, and the other disciple, and they went toward the tomb.}
\bv{4}{And they ran both together: and the other disciple outran Peter, and came first to the tomb;}
\bv{5}{and stooping and looking in, he seeth the linen cloths lying; yet entered he not in.}
\bv{6}{Simon Peter therefore also cometh, following him, and entered into the tomb; and he beholdeth the linen cloths lying,}
\bv{7}{and the napkin, that was upon his head, not lying with the linen cloths, but rolled up in a place by itself.}
\bv{8}{Then entered in therefore the other disciple also, who came first to the tomb, and he saw, and believed.}
\bv{9}{For as yet they knew not the scripture, that he must rise again from the dead.}
\bv{10}{So the disciples went away again unto their own home.}
\chapsec{Jesus Appears to Mary magdalene}
\bv{11}{But Mary was standing without at the tomb weeping: so, as she wept, she stooped and looked into the tomb;}
\bv{12}{and she beholdeth two angels in white sitting, one at the head, and one at the feet, where the body of Jesus had lain.}
\bv{13}{And they say unto her, ``Woman, why weepest thou?'' She saith unto them, ``Because they have taken away my Lord, and I know not where they have laid him.''}
\bv{14}{When she had thus said, she turned herself back, and beholdeth Jesus standing, and knew not that it was Jesus.}
\bv{15}{Jesus saith unto her, \redlet{``Woman, why weepest thou? whom seekest thou?''} She, supposing him to be the gardener, saith unto him, ``Sir, if thou hast borne him hence, tell me where thou hast laid him, and I will take him away.''}
\bv{16}{Jesus saith unto her, \redlet{``Mary.''} She turneth herself, and saith unto him in Hebrew, ``Rabboni;'' which is to say, ```Teacher.''}
\bv{17}{Jesus saith to her, Touch me not; for I am not yet ascended unto the Father: but go unto my brethren, and say to them, I ascend unto my Father and your Father, and my God and your God.}
\bv{18}{Mary Magdalene cometh and telleth the disciples, ``I have seen the Lord;'' and \supptext{that} he had said these things unto her.}
\chapsec{Jesus Appears to the Disciples}
\bv{19}{When therefore it was evening, on that day, the first \supptext{day} of the week, and when the doors were shut where the disciples were, for fear of the Jews, Jesus came and stood in the midst, and saith unto them, \redlet{``Peace \supptext{be} unto you.''}}
\bv{20}{And when he had said this, he showed unto them his hands and his side. The disciples therefore were glad, when they saw the Lord.}
\bv{21}{Jesus therefore said to them again, \redlet{``Peace \supptext{be} unto you: as the Father hath sent me, even so send I you.''}}
\bv{22}{And when he had said this, he breathed on them, and saith unto them, \redlet{``Receive ye the Holy Ghost:}}
\bv{23}{\redlet{whose soever sins ye forgive, they are forgiven unto them; whose soever \supptext{sins} ye retain, they are retained.''}}
\chapsec{St. Thomas' Faith}
\bv{24}{But Thomas, one of the twelve, called Didymus, was not with them when Jesus came.}
\bv{25}{The other disciples therefore said unto him, ``We have seen the Lord.'' But he said unto them, ``Except I shall see in his hands the print of the nails, and put my finger into the print of the nails, and put my hand into his side, I will not believe.''}
\bv{26}{And after eight days again his disciples were within, and Thomas with them. Jesus cometh, the doors being shut, and stood in the midst, and said, \redlet{``Peace \supptext{be} unto you.''}}
\bv{27}{Then saith he to Thomas, \redlet{``Reach hither thy finger, and see my hands; and reach \supptext{hither} thy hand, and put it into my side: and be not faithless, but believing.''}}
\bv{28}{Thomas answered and said unto him, ``My Lord and my God.''}
\bv{29}{Jesus saith unto him, \redlet{``Because thou hast seen me, thou hast believed: blessed \supptext{are} they that have not seen, and \supptext{yet} have believed.''}}
\chapsec{Purpose of St. John's Gospel}
\bv{30}{Many other signs therefore did Jesus in the presence of the disciples, which are not written in this book:}
\bv{31}{but these are written, that ye may believe that Jesus is the Christ, the Son of God; and that believing ye may have life in his name.}
\chaphead{Chapter XXI}
\chapdesc{Epilogue}
\lettrine[image=true, lines=4, findent=3pt, nindent=0pt]{NT/John/Jn-After.eps}{fter} these things Jesus manifested himself again to the disciples at the sea of Tiberias; and he manifested \supptext{himself} on this wise.
\bv{2}{There were together Simon Peter, and Thomas called Didymus, and Nathanael of Cana in Galilee, and the \supptext{sons} of Zebedee, and two other of his disciples.}
\par
\bv{3}{Simon Peter saith unto them, ``I go a fishing.'' They say unto him, ``We also come with thee.'' They went forth, and entered into the boat; and that night they took nothing.}
\bv{4}{But when day was now breaking, Jesus stood on the beach: yet the disciples knew not that it was Jesus.}
\par
\bv{5}{Jesus therefore saith unto them, \redlet{``Children, have ye aught to eat?''} They answered him, ``No.''}
\par
\bv{6}{And he said unto them, \redlet{``Cast the net on the right side of the boat, and ye shall find.''} They cast therefore, and now they were not able to draw it for the multitude of fishes.}
\bv{7}{That disciple therefore whom Jesus loved saith unto Peter, ``It is the Lord.'' So when Simon Peter heard that it was the Lord, he girt his coat about him (for he was naked), and cast himself into the sea.}
\bv{8}{But the other disciples came in the little boat (for they were not far from the land, but about two hundred cubits off), dragging the net \supptext{full} of fishes.}
\bv{9}{So when they got out upon the land, they see a fire of coals there, and fish laid thereon, and bread.}
\bv{10}{Jesus saith unto them, \redlet{``Bring of the fish which ye have now taken.''}}
\bv{11}{Simon Peter therefore went up, and drew the net to land, full of great fishes, a hundred and fifty and three: and for all there were so many, the net was not rent.}
\par
\bv{12}{Jesus saith unto them, \redlet{``Come \supptext{and} break your fast.''} And none of the disciples durst inquire of him, ``Who art thou?'' knowing that it was the Lord.}
\bv{13}{Jesus cometh, and taketh the bread, and giveth them, and the fish likewise.}
\bv{14}{This is now the third time that Jesus was manifested to the disciples, after that he was risen from the dead.}
\chapsec{Jesus Restores St. Peter's Faith}
\bv{15}{So when they had broken their fast, Jesus saith to Simon Peter, \redlet{``Simon, \supptext{son} of John, lovest thou me more than these?''} He saith unto him, ``Yea, Lord; thou knowest that I love thee.'' He saith unto him, \redlet{``Feed my lambs.''}}
\bv{16}{He saith to him again a second time, \redlet{``Simon, \supptext{son} of John, lovest thou me?''} He saith unto him, ``Yea, Lord; thou knowest that I love thee.'' He saith unto him, \redlet{``Tend my sheep.''}}
\bv{17}{He saith unto him the third time, \redlet{``Simon, \supptext{son} of John, lovest thou me?''} Peter was grieved because he said unto him the third time, \redlet{``Lovest thou me?''} And he said unto him, ``Lord, thou knowest all things; thou knowest that I love thee.'' Jesus saith unto him, \redlet{``Feed my sheep.}}
\par
\bv{18}{\redlet{Verily, verily, I say unto thee, When thou wast young, thou girdedst thyself, and walkedst whither thou wouldest: but when thou shalt be old, thou shalt stretch forth thy hands, and another shall gird thee, and carry thee whither thou wouldest not.''}}
\bv{19}{Now this he spake, signifying by what manner of death he should glorify God. And when he had spoken this, he saith unto him, \redlet{``Follow me.''}}
\par
\bv{20}{Peter, turning about, seeth the disciple whom Jesus loved following; who also leaned back on his breast at the supper, and said, ``Lord, who is he that betrayeth thee?''}
\bv{21}{Peter therefore seeing him saith to Jesus, ``Lord, and what shall this man do?''}
\bv{22}{Jesus saith unto him, \redlet{``If I will that he tarry till I come, what \supptext{is that} to thee? follow thou me.''}}
\bv{23}{This saying therefore went forth among the brethren, that that disciple should not die: yet Jesus said not unto him, that he should not die; but, \redlet{``If I will that he tarry till I come, what \supptext{is that} to thee?''}}
\bv{24}{This is the disciple that beareth witness of these things, and wrote these things: and we know that his witness is true.}
\bv{25}{And there are also many other things which Jesus did, the which if they should be written every one, I suppose that even the world itself would not contain the books that should be written.}
\begin{center}
	{\scshape [Here Endeth the Gospel of John]}
\end{center}
\clearpage
\chapter{The Woman Caught in Adultery}
\chapdesc{The story of the woman caught in adultery \textit{(pericope adulterae)} traditionally has been inserted into either St. John's or St. Luke's Gospel. It is likely a common historical story shared by early Christians about Our Lord.}
And they went every man unto his own house: but Jesus went unto the mount of Olives. And early in the morning he came again into the temple, and all the people came unto him; and he sat down, and taught them. And the scribes and the Pharisees bring a woman taken in adultery; and having set her in the midst, they say unto him, ``Teacher, this woman hath been taken in adultery, in the very act. Now in the law Moses commanded us to stone such: what then sayest thou of her?''
\par
And this they said, trying him, that they might have \supptext{whereof} to accuse him. But Jesus stooped down, and with his finger wrote on the ground. But when they continued asking him, he lifted up himself, and said unto them, ``He that is without sin among you, let him first cast a stone at her.'' And again he stooped down, and with his finger wrote on the ground. And they, when they heard it, went out one by one, beginning from the eldest, \supptext{even} unto the last: and Jesus was left alone, and the woman, where she was, in the midst.
\par
And Jesus lifted up himself, and said unto her, ``Woman, where are they? did no man condemn thee?'' And she said, ``No man, Lord.'' And Jesus said, ``Neither do I condemn thee: go thy way; from henceforth sin no more.''
	\clearpage
	\chapter{The Acts of the Apostles}
\fancyhead[RE,LO]{The Book of Acts}
\chaphead{Chapter I}
\chapdesc{Introduction}
\lettrine[image=true, lines=4, findent=3pt, nindent=0pt]{NT/Acts/Acts1-T.eps}{he} former treatise I made, O Theophilus, concerning all that Jesus began both to do and to teach,
\bv{2}{until the day in which he was received up, after that he had given commandment through the Holy Ghost unto the apostles whom he had chosen:}
\chapsec{The Resurrection \& Ministry}
\bv{3}{to whom he also showed himself alive after his passion by many proofs, appearing unto them by the space of forty days, and speaking the things concerning the kingdom of God:}
\bv{4}{and, being assembled together with them, he charged them not to depart from Jerusalem, but to wait for the promise of the Father, which, \supptext{said he}, \redlet{``ye heard from me:}}
\bv{5}{\redlet{for John indeed baptised with water; but ye shall be baptised in the Holy Ghost not many days hence.''}}
\bv{6}{They therefore, when they were come together, asked him, saying, ``Lord, dost thou at this time restore the kingdom to Israel?''}
\bv{7}{And he said unto them, \redlet{``It is not for you to know times or seasons, which the Father hath set within his own authority.}}
\chapsec{The Apostolic Commission}
\bv{8}{\redlet{But ye shall receive power, when the Holy Ghost is come upon you: and ye shall be my witnesses both in Jerusalem, and in all Jud{\ae}a and Samaria, and unto the uttermost part of the earth.''}}
\bv{9}{And when he had said these things, as they were looking, he was taken up; and a cloud received him out of their sight.}
\chapsec{The Promise of the Return of Christ}
\bv{10}{And while they were looking stedfastly into heaven as he went, behold two men stood by them in white apparel;}
\bv{11}{who also said, ``Ye men of Galilee, why stand ye looking into heaven? this Jesus, who was received up from you into heaven, shall so come in like manner as ye beheld him going into heaven.''}
\chapsec{Waiting for the Holy Ghost}
\bv{12}{Then returned they unto Jerusalem from the mount called Olivet, which is nigh unto Jerusalem, a sabbath day's journey off.}
\bv{13}{And when they were come in, they went up into the upper chamber, where they were abiding; both Peter and John and James and Andrew, Philip and Thomas, Bartholomew and Matthew, James \supptext{the son} of Alph{\ae}us, and Simon the Zealot, and Judas \supptext{the son} of James.}
\bv{14}{These all with one accord continued stedfastly in prayer, with the women, and Mary the mother of Jesus, and with his brethren.}
\chapsec{The Election of Matthias}
\bv{15}{And in these days Peter stood up in the midst of the brethren, and said (and there was a multitude of persons \supptext{gathered} together, about a hundred and twenty),}
\bv{16}{``Brethren, it was needful that the scripture should be fulfilled, which the Holy Ghost spake before by the mouth of David concerning Judas, who was guide to them that took Jesus.}
\bv{17}{For he was numbered among us, and received his portion in this ministry.}
\bv{18}{(Now this man obtained a field with the reward of his iniquity; and falling headlong, he burst asunder in the midst, and all his bowels gushed out.}
\bv{19}{And it became known to all the dwellers at Jerusalem; insomuch that in their language that field was called Akeldama, that is, The field of blood.)}
\bv{20}{For it is written in the book of Psalms,}
\otQuote{Ps. 69:25}{Let his habitation be made desolate, And let no man dwell therein:}
and,
\otQuote{Ps. 109:8}{His office let another take.}
\bv{21}{Of the men therefore that have companied with us all the time that the Lord Jesus went in and went out among us,}
\bv{22}{beginning from the baptism of John, unto the day that he was received up from us, of these must one become a witness with us of his resurrection.''}
\bv{23}{And they put forward two, Joseph called Barsabbas, who was surnamed Justus, and Matthias.}
\bv{24}{And they prayed, and said, ``Thou, Lord, who knowest the hearts of all men, show of these two the one whom thou hast chosen,}
\bv{25}{to take the place in this ministry and apostleship from which Judas fell away, that he might go to his own place.''}
\bv{26}{And they gave lots for them; and the lot fell upon Matthias; and he was numbered with the eleven apostles.}
\chaphead{Chapter II}
\chapdesc{Pentecost}
\lettrine[image=true, lines=4, findent=3pt, nindent=0pt]{NT/Acts/Acts-And.eps}{nd} when the day of Pentecost was now come, they were all together in one place.
\bv{2}{And suddenly there came from heaven a sound as of the rushing of a mighty wind, and it filled all the house where they were sitting.}
\bv{3}{And there appeared unto them tongues parting asunder, like as of fire; and it sat upon each one of them.}
\bv{4}{And they were all filled with the Holy Ghost, and began to speak with other tongues, as the Spirit gave them utterance.}
\bv{5}{Now there were dwelling at Jerusalem Jews, devout men, from every nation under heaven.}
\chapsec{Speaking Tongues}
\bv{6}{And when this sound was heard, the multitude came together, and were confounded, because that every man heard them speaking in his own language.}
\bv{7}{And they were all amazed and marvelled, saying, ``Behold, are not all these that speak Galil{\ae}ans?}
\bv{8}{And how hear we, every man in our own language wherein we were born?}
\bv{9}{Parthians and Medes and Elamites, and the dwellers in Mesopotamia, in Jud{\ae}a and Cappadocia, in Pontus and Asia,}
\bv{10}{in Phrygia and Pamphylia, in Egypt and the parts of Libya about Cyrene, and sojourners from Rome, both Jews and proselytes,}
\bv{11}{Cretans and Arabians, we hear them speaking in our tongues the mighty works of God.''}
\bv{12}{And they were all amazed, and were perplexed, saying one to another, ``What meaneth this?''}
\bv{13}{But others mocking said, ``They are filled with new wine.''}
\chapsec{Peter's Sermon}
\bv{14}{But Peter, standing up with the eleven, lifted up his voice, and spake forth unto them, \supptext{saying}, ``Ye men of Jud{\ae}a, and all ye that dwell at Jerusalem, be this known unto you, and give ear unto my words.}
\chapsec{Fulfilment of Prophecy}
\bv{15}{For these are not drunken, as ye suppose; seeing it is \supptext{but} the third hour of the day;}
\bv{16}{but this is that which hath been spoken through the prophet Joel:}
\otQuote{Joel 2:28-32}{\bv{17}{And it shall be in the last days, saith God,
I will pour forth of my Spirit upon all flesh:
And your sons and your daughters shall prophesy,
And your young men shall see visions,
And your old men shall dream dreams:}
\bv{18}{Yea and on my servants and on my handmaidens in those days
Will I pour forth of my Spirit; and they shall prophesy.}
\bv{19}{And I will show wonders in the heaven above,
And signs on the earth beneath;
Blood, and fire, and vapour of smoke:}
\bv{20}{The sun shall be turned into darkness,
And the moon into blood,
Before the day of the Lord come,
That great and notable \supptext{day}:}
\bv{21}{And it shall be, that whosoever shall call on the name of the Lord shall be saved.}}
\chapsec{The Works of Jesus Prove his Lordship}
\bv{22}{Ye men of Israel, hear these words: Jesus of Nazareth, a man approved of God unto you by mighty works and wonders and signs which God did by him in the midst of you, even as ye yourselves know;}
\bv{23}{him, being delivered up by the determinate counsel and foreknowledge of God, ye by the hand of lawless men did crucify and slay:}
\bv{24}{whom God raised up, having loosed the pangs of death: because it was not possible that he should be holden of it.}
\chapsec{David's Testimony of Jesus}
\bv{25}{For David saith concerning him,}
\otQuote{Ps. 16:8-11}{I beheld the Lord always before my face;
For he is on my right hand, that I should not be moved:
\bv{26}{Therefore my heart was glad, and my tongue rejoiced;
Moreover my flesh also shall dwell in hope:}
\bv{27}{Because thou wilt not leave my soul unto Hades,
Neither wilt thou give thy Holy One to see corruption.}
\bv{28}{Thou madest known unto me the ways of life;
Thou shalt make me full of gladness with thy countenance.}}
\bv{29}{Brethren, I may say unto you freely of the patriarch David, that he both died and was buried, and his tomb is with us unto this day.}
\bv{30}{Being therefore a prophet, and knowing that God had sworn with an oath to him, that of the fruit of his loins he would set \supptext{one} upon his throne;}
\bv{31}{he foreseeing \supptext{this} spake of the resurrection of the Christ, that neither was he left unto Hades, nor did his flesh see corruption.}
\chapsec{Resurrection Proves Christ}
\bv{32}{This Jesus did God raise up, whereof we all are witnesses.}
\bv{33}{Being therefore by the right hand of God exalted, and having received of the Father the promise of the Holy Ghost, he hath poured forth this, which ye see and hear.}
\bv{34}{For David ascended not into the heavens: but he saith himself,}
\otQuote{Ps. 110:1}{The Lord said unto my Lord, Sit thou on my right hand,
\bv{35}{Till I make thine enemies the footstool of thy feet.}}
\bv{36}{Let all the house of Israel therefore know assuredly, that God hath made him both Lord and Christ, this Jesus whom ye crucified.''}
\chapsec{What Israel Must Do}
\bv{37}{Now when they heard \supptext{this}, they were pricked in their heart, and said unto Peter and the rest of the apostles, ``Brethren, what shall we do?''}
\bv{38}{And Peter \supptext{said} unto them, ``Repent ye, and be baptised every one of you in the name of Jesus Christ unto the remission of your sins; and ye shall receive the gift of the Holy Ghost.\mref{cf. John 3:5}}
\bv{39}{For to you is the promise, and to your children, and to all that are afar off, \supptext{even} as many as the Lord our God shall call unto him.''}
\bv{40}{And with many other words he testified, and exhorted them, saying, ``Save yourselves from this crooked generation.''}
\bv{41}{They then that received his word were baptised: and there were added \supptext{unto them} in that day about three thousand souls.}
\chapsec{The First Local Church}
\bv{42}{And they continued stedfastly in the apostles' teaching and fellowship, in the breaking of bread and the prayers.}
\bv{43}{And fear came upon every soul: and many wonders and signs were done through the apostles.}
\bv{44}{And all that believed were together, and had all things common;}
\bv{45}{and they sold their possessions and goods, and parted them to all, according as any man had need.}
\bv{46}{And day by day, continuing stedfastly with one accord in the temple, and breaking bread at home, they took their food with gladness and singleness of heart,}
\bv{47}{praising God, and having favour with all the people. And the Lord added to them day by day those that were saved.}
\chaphead{Chapter III}
\chapdesc{The First Apostolic Miracle}
\lettrine[image=true, lines=4, findent=3pt, nindent=0pt]{NT/Acts/Acts-Now.eps}{ow} Peter and John were going up into the temple at the hour of prayer, \supptext{being} the ninth \supptext{hour}.
\bv{2}{And a certain man that was lame from his mother's womb was carried, whom they laid daily at the door of the temple which is called Beautiful, to ask alms of them that entered into the temple;}
\bv{3}{who seeing Peter and John about to go into the temple, asked to receive an alms.}
\bv{4}{And Peter, fastening his eyes upon him, with John, said, ``Look on us.''}
\bv{5}{And he gave heed unto them, expecting to receive something from them.}
\bv{6}{But Peter said, ``Silver and gold have I none; but what I have, that give I thee. In the name of Jesus Christ of Nazareth, walk.''}
\par
\bv{7}{And he took him by the right hand, and raised him up: and immediately his feet and his ankle-bones received strength.}
\bv{8}{And leaping up, he stood, and began to walk; and he entered with them into the temple, walking, and leaping, and praising God.}
\bv{9}{And all the people saw him walking and praising God:}
\bv{10}{and they took knowledge of him, that it was he that sat for alms at the Beautiful Gate of the temple; and they were filled with wonder and amazement at that which had happened unto him.}
\bv{11}{And as he held Peter and John, all the people ran together unto them in the porch that is called Solomon's, greatly wondering.}
\chapsec{Peter's Second Sermon}
\bv{12}{And when Peter saw it, he answered unto the people, ``Ye men of Israel, why marvel ye at this man? or why fasten ye your eyes on us, as though by our own power or godliness we had made him to walk?}
\bv{13}{The God of Abraham, and of Isaac, and of Jacob, the God of our fathers, hath glorified his Servant Jesus; whom ye delivered up, and denied before the face of Pilate, when he had determined to release him.}
\bv{14}{But ye denied the Holy and Righteous One, and asked for a murderer to be granted unto you,}
\bv{15}{and killed the Prince of life; whom God raised from the dead; whereof we are witnesses.}
\bv{16}{And by faith in his name hath his name made this man strong, whom ye behold and know: yea, the faith which is through him hath given him this perfect soundness in the presence of you all.}
\bv{17}{And now, brethren, I know that in ignorance ye did it, as did also your rulers.}
\bv{18}{But the things which God foreshowed by the mouth of all the prophets, that his Christ should suffer, he thus fulfilled.}
\par
\bv{19}{Repent ye therefore, and turn again, that your sins may be blotted out, that so there may come seasons of refreshing from the presence of the Lord;}
\bv{20}{and that he may send the Christ who hath been appointed for you, \supptext{even} Jesus:}
\bv{21}{whom the heaven must receive until the times of restoration of all things, whereof God spake by the mouth of his holy prophets that have been from of old.}
\bv{22}{Moses indeed said,}
\otQuote{Deut. 18:15,18,19}{A prophet shall the Lord God raise up unto you from among your brethren, like unto me; to him shall ye hearken in all things whatsoever he shall speak unto you.
\bv{23}{And it shall be, that every soul that shall not hearken to that prophet, shall be utterly destroyed from among the people.}}
\bv{24}{Yea and all the prophets from Samuel and them that followed after, as many as have spoken, they also told of these days.}
\bv{25}{Ye are the sons of the prophets, and of the covenant which God made with your fathers, saying unto Abraham,}
\otQuote{Gen. 22:18}{And in thy seed shall all the families of the earth be blessed.}
\bv{26}{Unto you first God, having raised up his Servant, sent him to bless you, in turning away every one of you from your iniquities.''}
\chaphead{Chapter IV}
\chapdesc{The First Persecution}
\lettrine[image=true, lines=4, findent=3pt, nindent=0pt]{NT/Acts/Acts-And.eps}{nd} as they spake unto the people, the priests and the captain of the temple and the Sadducees came upon them,
\bv{2}{being sore troubled because they taught the people, and proclaimed in Jesus the resurrection from the dead.}
\bv{3}{And they laid hands on them, and put them in ward unto the morrow: for it was now eventide.}
\bv{4}{But many of them that heard the word believed; and the number of the men came to be about five thousand.}
\chapsec{Peter Adresses the Sanhedrin}
\bv{5}{And it came to pass on the morrow, that their rulers and elders and scribes were gathered together in Jerusalem;}
\bv{6}{and Annas the high priest \supptext{was there}, and Caiaphas, and John, and Alexander, and as many as were of the kindred of the high priest.}
\bv{7}{And when they had set them in the midst, they inquired, ``By what power, or in what name, have ye done this?''}
\bv{8}{Then Peter, filled with the Holy Ghost, said unto them, ``Ye rulers of the people, and elders,}
\bv{9}{if we this day are examined concerning a good deed done to an impotent man, by what means this man is made whole;}
\bv{10}{be it known unto you all, and to all the people of Israel, that in the name of Jesus Christ of Nazareth, whom ye crucified, whom God raised from the dead, \supptext{even} in him doth this man stand here before you whole.}
\bv{11}{He is the stone which was set at nought of you the builders, which was made the head of the corner.}
\bv{12}{And in none other is there salvation: for neither is there any other name under heaven, that is given among men, wherein we must be saved.''}
\chapsec{Gospel Forbidden}
\bv{13}{Now when they beheld the boldness of Peter and John, and had perceived that they were unlearned and ignorant men, they marvelled; and they took knowledge of them, that they had been with Jesus.}
\bv{14}{And seeing the man that was healed standing with them, they could say nothing against it.}
\bv{15}{But when they had commanded them to go aside out of the council, they conferred among themselves,}
\bv{16}{saying, ``What shall we do to these men? for that indeed a notable miracle hath been wrought through them, is manifest to all that dwell in Jerusalem; and we cannot deny it.}
\bv{17}{But that it spread no further among the people, let us threaten them, that they speak henceforth to no man in this name.''}
\par
\bv{18}{And they called them, and charged them not to speak at all nor teach in the name of Jesus.}
\bv{19}{But Peter and John answered and said unto them, ``Whether it is right in the sight of God to hearken unto you rather than unto God, judge ye:}
\bv{20}{for we cannot but speak the things which we saw and heard.''}
\bv{21}{And they, when they had further threatened them, let them go, finding nothing how they might punish them, because of the people; for all men glorified God for that which was done.}
\bv{22}{For the man was more than forty years old, on whom this miracle of healing was wrought.}
\chapsec{Christians Again Filled with the Spirit}
\bv{23}{And being let go, they came to their own company, and reported all that the chief priests and the elders had said unto them.}
\bv{24}{And they, when they heard it, lifted up their voice to God with one accord, and said, ``O Lord, thou that didst make the heaven and the earth and the sea, and all that in them is:}
\bv{25}{who by the Holy Ghost, \supptext{by} the mouth of our father David thy servant, didst say,}
\otQuote{Ps. 2:1-2}{Why did the Gentiles rage,
And the peoples imagine vain things?
\bv{26}{The kings of the earth set themselves in array,
And the rulers were gathered together,
Against the Lord, and against his Anointed:}}
\bv{27}{for of a truth in this city against thy holy Servant Jesus, whom thou didst anoint, both Herod and Pontius Pilate, with the Gentiles and the peoples of Israel, were gathered together,}
\bv{28}{to do whatsoever thy hand and thy council foreordained to come to pass.}
\bv{29}{And now, Lord, look upon their threatenings: and grant unto thy servants to speak thy word with all boldness,}
\bv{30}{while thou stretchest forth thy hand to heal; and that signs and wonders may be done through the name of thy holy Servant Jesus.''}
\bv{31}{And when they had prayed, the place was shaken wherein they were gathered together; and they were all filled with the Holy Ghost, and they spake the word of God with boldness.}
\chapsec{State of the Jerusalem Church}
\bv{32}{And the multitude of them that believed were of one heart and soul: and not one \supptext{of them} said that aught of the things which he possessed was his own; but they had all things common.}
\bv{33}{And with great power gave the apostles their witness of the resurrection of the Lord Jesus: and great grace was upon them all.}
\bv{34}{For neither was there among them any that lacked: for as many as were possessors of lands or houses sold them, and brought the prices of the things that were sold,}
\bv{35}{and laid them at the apostles' feet: and distribution was made unto each, according as any one had need.}
\bv{36}{And Joseph, who by the apostles was surnamed Barnabas (which is, being interpreted, Son of exhortation), a Levite, a man of Cyprus by race,}
\bv{37}{having a field, sold it, and brought the money and laid it at the apostles' feet.}
\chaphead{Chapter V}
\chapdesc{Sin \& Death of Ananias}
\lettrine[image=true, lines=4, findent=3pt, nindent=0pt]{NT/Acts/Acts-But.eps}{ut} a certain man named Ananias, with Sapphira his wife, sold a possession,
\bv{2}{and kept back \supptext{part} of the price, his wife also being privy to it, and brought a certain part, and laid it at the apostles' feet.}
\bv{3}{But Peter said, Ananias, why hath Satan filled thy heart to lie to the Holy Ghost, and to keep back \supptext{part} of the price of the land?}
\bv{4}{While it remained, did it not remain thine own? and after it was sold, was it not in thy power? How is it that thou hast conceived this thing in thy heart? thou hast not lied unto men, but unto God.}
\bv{5}{And Ananias hearing these words fell down and gave up the ghost: and great fear came upon all that heard it.}
\bv{6}{And the young men arose and wrapped him round, and they carried him out and buried him.}
\chapsec{Sin \& Death of Sapphira}
\bv{7}{And it was about the space of three hours after, when his wife, not knowing what was done, came in.}
\bv{8}{And Peter answered unto her, Tell me whether ye sold the land for so much. And she said, Yea, for so much.}
\bv{9}{But Peter \supptext{said} unto her, How is it that ye have agreed together to try the Spirit of the Lord? behold, the feet of them that have buried thy husband are at the door, and they shall carry thee out.}
\bv{10}{And she fell down immediately at his feet, and gave up the ghost: and the young men came in and found her dead, and they carried her out and buried her by her husband.}
\bv{11}{And great fear came upon the whole church, and upon all that heard these things.}
\chapsec{Work of the Church}
\bv{12}{And by the hands of the apostles were many signs and wonders wrought among the people: and they were all with one accord in Solomon's porch.}
\bv{13}{But of the rest durst no man join himself to them: howbeit the people magnified them;}
\bv{14}{and believers were the more added to the Lord, multitudes both of men and women:}
\bv{15}{insomuch that they even carried out the sick into the streets, and laid them on beds and couches, that, as Peter came by, at the least his shadow might overshadow some one of them.}
\bv{16}{And there also came together the multitude from the cities round about Jerusalem, bringing sick folk, and them that were vexed with unclean spirits: and they were healed every one.}
\chapsec{The Second Persecution}
\bv{17}{But the high priest rose up, and all they that were with him (which is the sect of the Sadducees), and they were filled with jealousy,}
\bv{18}{and laid hands on the apostles, and put them in public ward.}
\bv{19}{But an angel of the Lord by night opened the prison doors, and brought them out, and said,}
\bv{20}{``Go ye, and stand and speak in the temple to the people all the words of this Life.''}
\bv{21}{And when they heard \supptext{this}, they entered into the temple about daybreak, and taught. But the high priest came, and they that were with him, and called the council together, and all the senate of the children of Israel, and sent to the prison-house to have them brought.}
\par
\bv{22}{But the officers that came found them not in the prison; and they returned, and told,}
\bv{23}{saying, ``The prison-house we found shut in all safety, and the keepers standing at the doors: but when we had opened, we found no man within.''}
\bv{24}{Now when the captain of the temple and the chief priests heard these words, they were much perplexed concerning them whereunto this would grow.}
\bv{25}{And there came one and told them, ``Behold, the men whom ye put in the prison are in the temple standing and teaching the people.''}
\bv{26}{Then went the captain with the officers, and brought them, \supptext{but} without violence; for they feared the people, lest they should be stoned.}
\chapsec{Interrogation of the Apostles}
\bv{27}{And when they had brought them, they set them before the council. And the high priest asked them,}
\bv{28}{saying, ``We strictly charged you not to teach in this name: and behold, ye have filled Jerusalem with your teaching, and intend to bring this man's blood upon us.''}
\bv{29}{But Peter and the apostles answered and said, ``We must obey God rather than men.}
\bv{30}{The God of our fathers raised up Jesus, whom ye slew, hanging him on a tree.}
\bv{31}{Him did God exalt with his right hand \supptext{to be} a Prince and a Saviour, to give repentance to Israel, and remission of sins.}
\bv{32}{And we are witnesses of these things; and \supptext{so is} the Holy Ghost, whom God hath given to them that obey him.''}
\bv{33}{But they, when they heard this, were cut to the heart, and were minded to slay them.}
\chapsec{The Warning of Gamaliel}
\bv{34}{But there stood up one in the council, a Pharisee, named Gamaliel, a doctor of the law, had in honour of all the people, and commanded to put the men forth a little while.}
\bv{35}{And he said unto them, ``Ye men of Israel, take heed to yourselves as touching these men, what ye are about to do.}
\bv{36}{For before these days rose up Theudas, giving himself out to be somebody; to whom a number of men, about four hundred, joined themselves: who was slain; and all, as many as obeyed him, were dispersed, and came to nought.}
\bv{37}{After this man rose up Judas of Galilee in the days of the enrolment, and drew away \supptext{some of the} people after him: he also perished; and all, as many as obeyed him, were scattered abroad.}
\bv{38}{And now I say unto you, Refrain from these men, and let them alone: for if this counsel or this work be of men, it will be overthrown:}
\bv{39}{but if it is of God, ye will not be able to overthrow them; lest haply ye be found even to be fighting against God.''}
\chapsec{The Apostles Beaten}
\bv{40}{And to him they agreed: and when they had called the apostles unto them, they beat them and charged them not to speak in the name of Jesus, and let them go.}
\bv{41}{They therefore departed from the presence of the council, rejoicing that they were counted worthy to suffer dishonour for the Name.}
\bv{42}{And every day, in the temple and at home, they ceased not to teach and to preach Jesus \supptext{as} the Christ.}
\chaphead{Chapter VI}
\chapdesc{The First Deacons}
\lettrine[image=true, lines=4, findent=3pt, nindent=0pt]{NT/Acts/Acts-Now.eps}{ow} in these days, when the number of the disciples was multiplying, there arose a murmuring of the Grecian Jews against the Hebrews, because their widows were neglected in the daily ministration.
\bv{2}{And the twelve called the multitude of the disciples unto them, and said, ``It is not fit that we should forsake the word of God, and serve tables.}
\bv{3}{Look ye out therefore, brethren, from among you seven men of good report, full of the Spirit and of wisdom, whom we may appoint over this business.}
\bv{4}{But we will continue stedfastly in prayer, and in the ministry of the word.''}
\bv{5}{And the saying pleased the whole multitude: and they chose Stephen, a man full of faith and of the Holy Ghost, and Philip, and Prochorus, and Nicanor, and Timon, and Parmenas, and Nicolaüs a proselyte of Antioch;}
\bv{6}{whom they set before the apostles: and when they had prayed, they laid their hands upon them.}
\bv{7}{And the word of God increased; and the number of the disciples multiplied in Jerusalem exceedingly; and a great company of the priests were obedient to the faith.}
\chapsec{The Third Persecution}
\bv{8}{And Stephen, full of grace and power, wrought great wonders and signs among the people.}
\bv{9}{But there arose certain of them that were of the synagogue called \supptext{the synagogue} of the Libertines, and of the Cyrenians, and of the Alexandrians, and of them of Cilicia and Asia, disputing with Stephen.}
\bv{10}{And they were not able to withstand the wisdom and the Spirit by which he spake.}
\bv{11}{Then they suborned men, who said, ``We have heard him speak blasphemous words against Moses, and \supptext{against} God.''}
\bv{12}{And they stirred up the people, and the elders, and the scribes, and came upon him, and seized him, and brought him into the council,}
\bv{13}{and set up false witnesses, who said, ``This man ceaseth not to speak words against this holy place, and the law:}
\bv{14}{for we have heard him say, that this Jesus of Nazareth shall destroy this place, and shall change the customs which Moses delivered unto us.''}
\bv{15}{And all that sat in the council, fastening their eyes on him, saw his face as it had been the face of an angel.}
\chaphead{Chapter VII}
\lettrine[image=true, lines=4, findent=3pt, nindent=0pt]{NT/Acts/Acts-And.eps}{nd} the high priest said, ``Are these things so?''
\chapdesc{The Response of Stephen}
\bv{2}{And he said,
``Brethren and fathers, hearken: The God of glory appeared unto our father Abraham, when he was in Mesopotamia, before he dwelt in Haran,}
\bv{3}{and said unto him, `Get thee out of thy land, and from thy kindred, and come into the land which I shall show thee.'}
\bv{4}{Then came he out of the land of the Chald{\ae}ans, and dwelt in Haran: and from thence, when his father was dead, \supptext{God} removed him into this land, wherein ye now dwell:}
\bv{5}{and he gave him none inheritance in it, no, not so much as to set his foot on: and he promised that he would give it to him in possession, and to his seed after him, when \supptext{as yet} he had no child.}
\bv{6}{And God spake on this wise, that his seed should sojourn in a strange land, and that they should bring them into bondage, and treat them ill, four hundred years.}
\bv{7}{And the nation to which they shall be in bondage will I judge, said God: and after that shall they come forth, and serve me in this place.}
\bv{8}{And he gave him the covenant of circumcision: and so \supptext{Abraham} begat Isaac, and circumcised him the eighth day; and Isaac \supptext{begat} Jacob, and Jacob the twelve patriarchs.}
\par
\bv{9}{And the patriarchs, moved with jealousy against Joseph, sold him into Egypt: and God was with him,}
\bv{10}{and delivered him out of all his afflictions, and gave him favour and wisdom before Pharaoh king of Egypt; and he made him governor over Egypt and all his house.}
\bv{11}{Now there came a famine over all Egypt and Canaan, and great affliction: and our fathers found no sustenance.}
\bv{12}{But when Jacob heard that there was grain in Egypt, he sent forth our fathers the first time.}
\bv{13}{And at the second time Joseph was made known to his brethren; and Joseph's race became manifest unto Pharaoh.}
\bv{14}{And Joseph sent, and called to him Jacob his father, and all his kindred, threescore and fifteen souls.}
\bv{15}{And Jacob went down into Egypt; and he died, himself and our fathers;}
\bv{16}{and they were carried over unto Shechem, and laid in the tomb that Abraham bought for a price in silver of the sons of Hamor in Shechem.}
\par
\bv{17}{But as the time of the promise drew nigh which God vouchsafed unto Abraham, the people grew and multiplied in Egypt,}
\bv{18}{till there arose another king over Egypt, who knew not Joseph.}
\bv{19}{The same dealt craftily with our race, and ill-treated our fathers, that they should cast out their babes to the end they might not live.}
\bv{20}{At which season Moses was born, and was exceeding fair; and he was nourished three months in his father's house:}
\bv{21}{and when he was cast out, Pharaoh's daughter took him up, and nourished him for her own son.}
\bv{22}{And Moses was instructed in all the wisdom of the Egyptians; and he was mighty in his words and works.}
\bv{23}{But when he was well-nigh forty years old, it came into his heart to visit his brethren the children of Israel.}
\bv{24}{And seeing one \supptext{of them} suffer wrong, he defended him, and avenged him that was oppressed, smiting the Egyptian:}
\bv{25}{and he supposed that his brethren understood that God by his hand was giving them deliverance; but they understood not.}
\par
\bv{26}{And the day following he appeared unto them as they strove, and would have set them at one again, saying, ``Sirs, ye are brethren; why do ye wrong one to another?''}
\bv{27}{But he that did his neighbor wrong thrust him away, saying, ``Who made thee a ruler and a judge over us?}
\bv{28}{Wouldest thou kill me, as thou killedst the Egyptian yesterday?''}
\bv{29}{And Moses fled at this saying, and became a sojourner in the land of Midian, where he begat two sons.}
\bv{30}{And when forty years were fulfilled, an angel appeared to him in the wilderness of mount Sinai, in a flame of fire in a bush.}
\bv{31}{And when Moses saw it, he wondered at the sight: and as he drew near to behold, there came a voice of the Lord,}
\bv{32}{``I am the God of thy fathers, the God of Abraham, and of Isaac, and of Jacob.'' And Moses trembled, and durst not behold.}
\bv{33}{And the Lord said unto him, ``Loose the shoes from thy feet: for the place whereon thou standest is holy ground.}
\bv{34}{I have surely seen the affliction of my people that is in Egypt, and have heard their groaning, and I am come down to deliver them: and now come, I will send thee into Egypt.''}
\par
\bv{35}{This Moses whom they refused, saying, ``Who made thee a ruler and a judge?'' him hath God sent \supptext{to be} both a ruler and a deliverer with the hand of the angel that appeared to him in the bush.}
\bv{36}{This man led them forth, having wrought wonders and signs in Egypt, and in the Red sea, and in the wilderness forty years.}
\bv{37}{This is that Moses, who said unto the children of Israel, ``A prophet shall God raise up unto you from among your brethren, like unto me.''}
\bv{38}{This is he that was in the church in the wilderness with the angel that spake to him in the mount Sinai, and with our fathers: who received living oracles to give unto us:}
\bv{39}{to whom our fathers would not be obedient, but thrust him from them, and turned back in their hearts unto Egypt,}
\bv{40}{saying unto Aaron, ``Make us gods that shall go before us: for as for this Moses, who led us forth out of the land of Egypt, we know not what is become of him.''}
\bv{41}{And they made a calf in those days, and brought a sacrifice unto the idol, and rejoiced in the works of their hands.}
\bv{42}{But God turned, and gave them up to serve the host of heaven; as it is written in the book of the prophets,}
\otQuote{Amos 5:25-7}{Did ye offer unto me slain beasts and sacrifices
Forty years in the wilderness, O house of Israel?
\bv{43}{And ye took up the tabernacle of Moloch,
And the star of the god Rephan,
The figures which ye made to worship them:
And I will carry you away beyond Babylon.}}
\bv{44}{Our fathers had the tabernacle of the testimony in the wilderness, even as he appointed who spake unto Moses, that he should make it according to the figure that he had seen.}
\bv{45}{Which also our fathers, in their turn, brought in with Joshua when they entered on the possession of the nations, that God thrust out before the face of our fathers, unto the days of David;}
\bv{46}{who found favour in the sight of God, and asked to find a habitation for the God of Jacob.}
\bv{47}{But Solomon built him a house.}
\bv{48}{Howbeit the Most High dwelleth not in \supptext{houses} made with hands; as saith the prophet,}
\otQuote{Ps. 11:4}{\bv{49}{The heaven is my throne,
And the earth the footstool of my feet:
What manner of house will ye build me? saith the Lord:
Or what is the place of my rest?}
\bv{50}{Did not my hand make all these things?}}
\bv{51}{Ye stiffnecked and uncircumcised in heart and ears, ye do always resist the Holy Ghost: as your fathers did, so do ye.}
\bv{52}{Which of the prophets did not your fathers persecute? and they killed them that showed before of the coming of the Righteous One; of whom ye have now become betrayers and murderers;}
\bv{53}{ye who received the law as it was ordained by angels, and kept it not.}
\chapsec{The First Martyr}
\bv{54}{Now when they heard these things, they were cut to the heart, and they gnashed on him with their teeth.}
\bv{55}{But he, being full of the Holy Ghost, looked up stedfastly into heaven, and saw the glory of God, and Jesus standing on the right hand of God,}
\bv{56}{and said, ``Behold, I see the heavens opened, and the Son of man standing on the right hand of God.''}
\bv{57}{But they cried out with a loud voice, and stopped their ears, and rushed upon him with one accord;}
\bv{58}{and they cast him out of the city, and stoned him: and the witnesses laid down their garments at the feet of a young man named Saul.}
\bv{59}{And they stoned Stephen, calling upon \supptext{the Lord}, and saying, ``Lord Jesus, receive my spirit.''}
\bv{60}{And he kneeled down, and cried with a loud voice, ``Lord, lay not this sin to their charge.'' And when he had said this, he fell asleep.}
\chaphead{Chapter VIII}
\chapdesc{The Fourth Persecution}
\lettrine[image=true, lines=4, findent=3pt, nindent=0pt]{NT/Acts/Acts-And.eps}{nd} Saul was consenting unto his death. And there arose on that day a great persecution against the church which was in Jerusalem; and they were all scattered abroad throughout the regions of Jud{\ae}a and Samaria, except the apostles.
\bv{2}{And devout men buried Stephen, and made great lamentation over him.}
\bv{3}{But Saul laid waste the church, entering into every house, and dragging men and women committed them to prison.}
\chapsec{The First Missionaries}
\bv{4}{They therefore that were scattered abroad went about preaching the word.}
\chapsec{Ministry of Philip}
\bv{5}{And Philip went down to the city of Samaria, and proclaimed unto them the Christ.}
\bv{6}{And the multitudes gave heed with one accord unto the things that were spoken by Philip, when they heard, and saw the signs which he did.}
\bv{7}{For \supptext{from} many of those that had unclean spirits, they came out, crying with a loud voice: and many that were palsied, and that were lame, were healed.}
\bv{8}{And there was much joy in that city.}
\chapsec{Simon Magus}
\bv{9}{But there was a certain man, Simon by name, who beforetime in the city used sorcery, and amazed the people of Samaria, giving out that himself was some great one:}
\bv{10}{to whom they all gave heed, from the least to the greatest, saying, ``This man is that power of God which is called Great.''}
\bv{11}{And they gave heed to him, because that of long time he had amazed them with his sorceries.}
\bv{12}{But when they believed Philip preaching good tidings concerning the kingdom of God and the name of Jesus Christ, they were baptised, both men and women.}
\bv{13}{And Simon also himself believed: and being baptised, he continued with Philip; and beholding signs and great miracles wrought, he was amazed.}
\chapsec{Sacrament of Confirmation}
\bv{14}{Now when the apostles that were at Jerusalem heard that Samaria had received the word of God, they sent unto them Peter and John:}
\bv{15}{who, when they were come down, prayed for them, that they might receive the Holy Ghost:}
\bv{16}{for as yet it was fallen upon none of them: only they had been baptised into the name of the Lord Jesus.}
\bv{17}{Then laid they their hands on them, and they received the Holy Ghost.}
\bv{18}{Now when Simon saw that through the laying on of the apostles' hands the Holy Ghost was given, he offered them money,}
\bv{19}{saying, ``Give me also this power, that on whomsoever I lay my hands, he may receive the Holy Ghost.''}
\bv{20}{But Peter said unto him, ``Thy silver perish with thee, because thou hast thought to obtain the gift of God with money.}
\bv{21}{Thou hast neither part nor lot in this matter: for thy heart is not right before God.}
\bv{22}{Repent therefore of this thy wickedness, and pray the Lord, if perhaps the thought of thy heart shall be forgiven thee.}
\bv{23}{For I see that thou art in the gall of bitterness and in the bond of iniquity.''}
\bv{24}{And Simon answered and said, ``Pray ye for me to the Lord, that none of the things which ye have spoken come upon me.''}
\bv{25}{They therefore, when they had testified and spoken the word of the Lord, returned to Jerusalem, and preached the gospel to many villages of the Samaritans.}
\chapsec{Philip and the Ethiopian Eunuch}
\bv{26}{But an angel of the Lord spake unto Philip, saying, ``Arise, and go toward the south unto the way that goeth down from Jerusalem unto Gaza: the same is desert.''}
\bv{27}{And he arose and went: and behold, a man of Ethiopia, a eunuch of great authority under Candace, queen of the Ethiopians, who was over all her treasure, who had come to Jerusalem to worship;}
\bv{28}{and he was returning and sitting in his chariot, and was reading the prophet Isaiah.}
\bv{29}{And the Spirit said unto Philip, ``Go near, and join thyself to this chariot.''}
\bv{30}{And Philip ran to him, and heard him reading Isaiah the prophet, and said, ``Understandest thou what thou readest?''}
\bv{31}{And he said, ``How can I, except some one shall guide me?'' And he besought Philip to come up and sit with him.}
\bv{32}{Now the passage of the scripture which he was reading was this,}
\otQuote{Is. 53:7-8}{He was led as a sheep to the slaughter;
And as a lamb before his shearer is dumb,
So he openeth not his mouth:
\bv{33}{In his humiliation his judgement was taken away:
His generation who shall declare?
For his life is taken from the earth.}}
\bv{34}{And the eunuch answered Philip, and said, ``I pray thee, of whom speaketh the prophet this? of himself, or of some other?''}
\bv{35}{And Philip opened his mouth, and beginning from this scripture, preached unto him Jesus.}
\bv{36}{And as they went on the way, they came unto a certain water; and the eunuch saith, ``Behold, \supptext{here is} water; what doth hinder me to be baptised?''\mcomm{And Philip said, ``If thou believest with all thy heart, thou mayest.'' And he answered and said, ``I believe that Jesus Christ is the Son of God.''}}
\bv{38}{And he commanded the chariot to stand still: and they both went down into the water, both Philip and the eunuch; and he baptised him.}
\bv{39}{And when they came up out of the water, the Spirit of the Lord caught away Philip; and the eunuch saw him no more, for he went on his way rejoicing.}
\bv{40}{But Philip was found at Azotus: and passing through he preached the gospel to all the cities, till he came to C{\ae}sarea.}
\chaphead{Chapter IX}
\chapdesc{Conversion of Saul}
\lettrine[image=true, lines=4, findent=3pt, nindent=0pt]{NT/Acts/Acts-But.eps}{ut} Saul, yet breathing threatening and slaughter against the disciples of the Lord, went unto the high priest,
\bv{2}{and asked of him letters to Damascus unto the synagogues, that if he found any that were of the Way, whether men or women, he might bring them bound to Jerusalem.}
\bv{3}{And as he journeyed, it came to pass that he drew nigh unto Damascus: and suddenly there shone round about him a light out of heaven:}
\bv{4}{and he fell upon the earth, and heard a voice saying unto him, ``Saul, Saul, why persecutest thou me?''}
\bv{5}{And he said, ``Who art thou, Lord?'' And he \supptext{said}, ``I am Jesus whom thou persecutest:}
\bv{6}{but rise, and enter into the city, and it shall be told thee what thou must do.''}
\bv{7}{And the men that journeyed with him stood speechless, hearing the voice, but beholding no man.}
\bv{8}{And Saul arose from the earth; and when his eyes were opened, he saw nothing; and they led him by the hand, and brought him into Damascus.}
\bv{9}{And he was three days without sight, and did neither eat nor drink.}
\par
\bv{10}{Now there was a certain disciple at Damascus, named Ananias; and the Lord said unto him in a vision, ``Ananias.'' And he said, ``Behold, I \supptext{am here}, Lord.''}
\bv{11}{And the Lord \supptext{said} unto him, ``Arise, and go to the street which is called Straight, and inquire in the house of Judas for one named Saul, a man of Tarsus: for behold, he prayeth;}
\bv{12}{and he hath seen a man named Ananias coming in, and laying his hands on him, that he might receive his sight.''}
\bv{13}{But Ananias answered, ``Lord, I have heard from many of this man, how much evil he did to thy saints at Jerusalem:}
\bv{14}{and here he hath authority from the chief priests to bind all that call upon thy name.''}
\bv{15}{But the Lord said unto him, ``Go thy way: for he is a chosen vessel unto me, to bear my name before the Gentiles and kings, and the children of Israel:}
\bv{16}{for I will show him how many things he must suffer for my name's sake.''}
\chapsec{Healing of Saul}
\bv{17}{And Ananias departed, and entered into the house; and laying his hands on him said, ``Brother Saul, the Lord, \supptext{even} Jesus, who appeared unto thee in the way which thou camest, hath sent me, that thou mayest receive thy sight, and be filled with the Holy Ghost.''}
\bv{18}{And straightway there fell from his eyes as it were scales, and he received his sight; and he arose and was baptised;}
\bv{19}{and he took food and was strengthened. And he was certain days with the disciples that were at Damascus.}
\chapsec{Teaching of Saul}
\bv{20}{And straightway in the synagogues he proclaimed Jesus, that he is the Son of God.}
\bv{21}{And all that heard him were amazed, and said, ``Is not this he that in Jerusalem made havoc of them that called on this name? and he had come hither for this intent, that he might bring them bound before the chief priests.''}
\bv{22}{But Saul increased the more in strength, and confounded the Jews that dwelt at Damascus, proving that this is the Christ.}
\chapsec{Plot against Saul}
\bv{23}{And when many days were fulfilled, the Jews took counsel together to kill him:}
\bv{24}{but their plot became known to Saul. And they watched the gates also day and night that they might kill him:}
\bv{25}{but his disciples took him by night, and let him down through the wall, lowering him in a basket.}
\bv{26}{And when he was come to Jerusalem, he assayed to join himself to the disciples: and they were all afraid of him, not believing that he was a disciple.}
\bv{27}{But Barnabas took him, and brought him to the apostles, and declared unto them how he had seen the Lord in the way, and that he had spoken to him, and how at Damascus he had preached boldly in the name of Jesus.}
\par
\bv{28}{And he was with them going in and going out at Jerusalem,}
\bv{29}{preaching boldly in the name of the Lord: and he spake and disputed against the Grecian Jews; but they were seeking to kill him.}
\bv{30}{And when the brethren knew it, they brought him down to C{\ae}sarea, and sent him forth to Tarsus.}
\bv{31}{So the church throughout all Jud{\ae}a and Galilee and Samaria had peace, being edified; and, walking in the fear of the Lord and in the comfort of the Holy Ghost, was multiplied.}
\chapsec{Peter Heals Aeneas}
\bv{32}{And it came to pass, as Peter went throughout all parts, he came down also to the saints that dwelt at Lydda.}
\bv{33}{And there he found a certain man named Aeneas, who had kept his bed eight years; for he was palsied.}
\bv{34}{And Peter said unto him, ``Aeneas, Jesus Christ healeth thee: arise, and make thy bed.'' And straightway he arose.}
\bv{35}{And all that dwelt at Lydda and in Sharon saw him, and they turned to the Lord.}
\chapsec{Raising of Tabitha}
\bv{36}{Now there was at Joppa a certain disciple named Tabitha, which by interpretation is called Dorcas: this woman was full of good works and almsdeeds which she did.}
\bv{37}{And it came to pass in those days, that she fell sick, and died: and when they had washed her, they laid her in an upper chamber.}
\bv{38}{And as Lydda was nigh unto Joppa, the disciples, hearing that Peter was there, sent two men unto him, entreating him, ``Delay not to come on unto us.''}
\par
\bv{39}{And Peter arose and went with them. And when he was come, they brought him into the upper chamber: and all the widows stood by him weeping, and showing the coats and garments which Dorcas made, while she was with them.}
\bv{40}{But Peter put them all forth, and kneeled down, and prayed; and turning to the body, he said, ``Tabitha, arise.'' And she opened her eyes; and when she saw Peter, she sat up.}
\bv{41}{And he gave her his hand, and raised her up; and calling the saints and widows, he presented her alive.}
\bv{42}{And it became known throughout all Joppa: and many believed on the Lord.}
\bv{43}{And it came to pass, that he abode many days in Joppa with one Simon a tanner.}
\chaphead{Chapter X}
\chapdesc{Mission of Cornelius}
\lettrine[image=true, lines=4, findent=3pt, nindent=0pt]{NT/Acts/Acts-Now.eps}{ow} \supptext{there was} a certain man in C{\ae}sarea, Cornelius by name, a centurion of the band called the Italian \supptext{band},
\bv{2}{a devout man, and one that feared God with all his house, who gave much alms to the people, and prayed to God always.}
\bv{3}{He saw in a vision openly, as it were about the ninth hour of the day, an angel of God coming in unto him, and saying to him, ``Cornelius.''}
\bv{4}{And he, fastening his eyes upon him, and being affrighted, said, ``What is it, Lord?'' And he said unto him, ``Thy prayers and thine alms are gone up for a memorial before God.}
\bv{5}{And now send men to Joppa, and fetch one Simon, who is surnamed Peter:}
\bv{6}{he lodgeth with one Simon a tanner, whose house is by the sea side.''}
\par
\bv{7}{And when the angel that spake unto him was departed, he called two of his household-servants, and a devout soldier of them that waited on him continually;}
\bv{8}{and having rehearsed all things unto them, he sent them to Joppa.}
\chapsec{Vision to Peter}
\bv{9}{Now on the morrow, as they were on their journey, and drew nigh unto the city, Peter went up upon the housetop to pray, about the sixth hour:}
\bv{10}{and he became hungry, and desired to eat: but while they made ready, he fell into a trance;}
\bv{11}{and he beholdeth the heaven opened, and a certain vessel descending, as it were a great sheet, let down by four corners upon the earth:}
\bv{12}{wherein were all manner of fourfooted beasts and creeping things of the earth and birds of the heaven.}
\par
\bv{13}{And there came a voice to him, ``Rise, Peter; kill and eat.''}
\bv{14}{But Peter said, ``Not so, Lord; for I have never eaten anything that is common and unclean.''}
\bv{15}{And a voice \supptext{came} unto him again the second time, ``What God hath cleansed, make not thou common.''}
\bv{16}{And this was done thrice: and straightway the vessel was received up into heaven.}
\chapsec{Cornelius' Servants Find Peter}
\bv{17}{Now while Peter was much perplexed in himself what the vision which he had seen might mean, behold, the men that were sent by Cornelius, having made inquiry for Simon's house, stood before the gate,}
\bv{18}{and called and asked whether Simon, who was surnamed Peter, were lodging there.}
\bv{19}{And while Peter thought on the vision, the Spirit said unto him, ``Behold, three men seek thee.}
\bv{20}{But arise, and get thee down, and go with them, nothing doubting: for I have sent them.''}
\par
\bv{21}{And Peter went down to the men, and said, ``Behold, I am he whom ye seek: what is the cause wherefore ye are come?''}
\bv{22}{And they said, ``Cornelius a centurion, a righteous man and one that feareth God, and well reported of by all the nation of the Jews, was warned \supptext{of God} by a holy angel to send for thee into his house, and to hear words from thee.''}
\bv{23}{So he called them in and lodged them. And on the morrow he arose and went forth with them, and certain of the brethren from Joppa accompanied him.}
\bv{24}{And on the morrow they entered into C{\ae}sarea. And Cornelius was waiting for them, having called together his kinsmen and his near friends.}
\chapsec{Peter Corrects Cornelius}
\bv{25}{And when it came to pass that Peter entered, Cornelius met him, and fell down at his feet, and worshipped him.}
\bv{26}{But Peter raised him up, saying, ``Stand up; I myself also am a man.''}
\bv{27}{And as he talked with him, he went in, and findeth many come together:}
\bv{28}{and he said unto them, ``Ye yourselves know how it is an unlawful thing for a man that is a Jew to join himself or come unto one of another nation; and \supptext{yet} unto me hath God showed that I should not call any man common or unclean:}
\bv{29}{wherefore also I came without gainsaying, when I was sent for. I ask therefore with what intent ye sent for me.''}
\bv{30}{And Cornelius said, ``Four days ago, until this hour, I was keeping the ninth hour of prayer in my house; and behold, a man stood before me in bright apparel,}
\bv{31}{and saith, `Cornelius, thy prayer is heard, and thine alms are had in remembrance in the sight of God.}
\bv{32}{Send therefore to Joppa, and call unto thee Simon, who is surnamed Peter; he lodgeth in the house of Simon a tanner, by the sea side.'}
\bv{33}{Forthwith therefore I sent to thee; and thou hast well done that thou art come. Now therefore we are all here present in the sight of God, to hear all things that have been commanded thee of the Lord.''}
\chapsec{Peter Gives the Gospel}
\bv{34}{And Peter opened his mouth, and said, ``Of a truth I perceive that God is no respecter of persons:}
\bv{35}{but in every nation he that feareth him, and worketh righteousness, is acceptable to him.}
\bv{36}{The word which he sent unto the children of Israel, preaching good tidings of peace by Jesus Christ (he is Lord of all)---}
\bv{37}{that saying ye yourselves know, which was published throughout all Jud{\ae}a, beginning from Galilee, after the baptism which John preached;}
\bv{38}{\supptext{even} Jesus of Nazareth, how God anointed him with the Holy Ghost and with power: who went about doing good, and healing all that were oppressed of the devil; for God was with him.}
\par
\bv{39}{And we are witnesses of all things which he did both in the country of the Jews, and in Jerusalem; whom also they slew, hanging him on a tree.}
\bv{40}{Him God raised up the third day, and gave him to be made manifest,}
\bv{41}{not to all the people, but unto witnesses that were chosen before of God, \supptext{even} to us, who ate and drank with him after he rose from the dead.}
\par
\bv{42}{And he charged us to preach unto the people, and to testify that this is he who is ordained of God \supptext{to be} the Judge of the living and the dead.}
\bv{43}{To him bear all the prophets witness, that through his name every one that believeth on him shall receive remission of sins.''}
\chapsec{Gentiles are Given the Holy Ghost}
\bv{44}{While Peter yet spake these words, the Holy Ghost fell on all them that heard the word.}
\bv{45}{And they of the circumcision that believed were amazed, as many as came with Peter, because that on the Gentiles also was poured out the gift of the Holy Ghost.}
\bv{46}{For they heard them speak with tongues, and magnify God. Then answered Peter,}
\bv{47}{``Can any man forbid the water, that these should not be baptised, who have received the Holy Ghost as well as we?''}
\bv{48}{And he commanded them to be baptised in the name of Jesus Christ. Then prayed they him to tarry certain days.}
\chaphead{Chapter XI}
\chapdesc{Peter Declares Gentiles Clean}
\lettrine[image=true, lines=4, findent=3pt, nindent=0pt]{NT/Acts/Acts-Now.eps}{ow} the apostles and the brethren that were in Jud{\ae}a heard that the Gentiles also had received the word of God.
\bv{2}{And when Peter was come up to Jerusalem, they that were of the circumcision contended with him,}
\bv{3}{saying, ``Thou wentest in to men uncircumcised, and didst eat with them.''}
\par
\bv{4}{But Peter began, and expounded \supptext{the matter} unto them in order, saying,}
\bv{5}{``I was in the city of Joppa praying: and in a trance I saw a vision, a certain vessel descending, as it were a great sheet let down from heaven by four corners; and it came even unto me:}
\bv{6}{upon which when I had fastened mine eyes, I considered, and saw the fourfooted beasts of the earth and wild beasts and creeping things and birds of the heaven.}
\bv{7}{And I heard also a voice saying unto me, `Rise, Peter; kill and eat.'}
\bv{8}{But I said, `Not so, Lord: for nothing common or unclean hath ever entered into my mouth.'}
\bv{9}{But a voice answered the second time out of heaven, `What God hath cleansed, make not thou common.'}
\bv{10}{And this was done thrice: and all were drawn up again into heaven.}
\par
\bv{11}{And behold, forthwith three men stood before the house in which we were, having been sent from C{\ae}sarea unto me.}
\bv{12}{And the Spirit bade me go with them, making no distinction. And these six brethren also accompanied me; and we entered into the man's house:}
\bv{13}{and he told us how he had seen the angel standing in his house, and saying, `Send to Joppa, and fetch Simon, whose surname is Peter;}
\bv{14}{who shall speak unto thee words, whereby thou shalt be saved, thou and all thy house.'}
\par
\bv{15}{And as I began to speak, the Holy Ghost fell on them, even as on us at the beginning.}
\bv{16}{And I remembered the word of the Lord, how he said, `John indeed baptised with water; but ye shall be baptised in the Holy Ghost.'}
\bv{17}{If then God gave unto them the like gift as \supptext{he did} also unto us, when we believed on the Lord Jesus Christ, who was I, that I could withstand God?''}
\chapsec{Judaisers Repent}
\bv{18}{And when they heard these things, they held their peace, and glorified God, saying, ``Then to the Gentiles also hath God granted repentance unto life.''}
\bv{19}{They therefore that were scattered abroad upon the tribulation that arose about Stephen travelled as far as Phoenicia, and Cyprus, and Antioch, speaking the word to none save only to Jews.}
\bv{20}{But there were some of them, men of Cyprus and Cyrene, who, when they were come to Antioch, spake unto the Greeks also, preaching the Lord Jesus.}
\bv{21}{And the hand of the Lord was with them: and a great number that believed turned unto the Lord.}
\par
\bv{22}{And the report concerning them came to the ears of the church which was in Jerusalem: and they sent forth Barnabas as far as Antioch:}
\bv{23}{who, when he was come, and had seen the grace of God, was glad; and he exhorted them all, that with purpose of heart they would cleave unto the Lord:}
\bv{24}{for he was a good man, and full of the Holy Ghost and of faith: and much people was added unto the Lord.}
\par
\bv{25}{And he went forth to Tarsus to seek for Saul;}
\bv{26}{and when he had found him, he brought him unto Antioch. And it came to pass, that even for a whole year they were gathered together with the church, and taught much people; and that the disciples were called Christians first in Antioch.}
\chapsec{Prophecy of Famine}
\bv{27}{Now in these days there came down prophets from Jerusalem unto Antioch.}
\bv{28}{And there stood up one of them named Agabus, and signified by the Spirit that there should be a great famine over all the world: which came to pass in the days of Claudius.}
\bv{29}{And the disciples, every man according to his ability, determined to send relief unto the brethren that dwelt in Jud{\ae}a:}
\bv{30}{which also they did, sending it to the elders by the hand of Barnabas and Saul.}
\chaphead{Chapter XII}
\chapdesc{Fifth Persecution}
\lettrine[image=true, lines=4, findent=3pt, nindent=0pt]{NT/Acts/Acts-Now.eps}{ow} about that time Herod the king put forth his hands to afflict certain of the church.
\bv{2}{And he killed James the brother of John with the sword.}
\bv{3}{And when he saw that it pleased the Jews, he proceeded to seize Peter also. And \supptext{those} were the days of unleavened bread.}
\bv{4}{And when he had taken him, he put him in prison, and delivered him to four quaternions of soldiers to guard him; intending after the Passover to bring him forth to the people.}
\bv{5}{Peter therefore was kept in the prison: but prayer was made earnestly of the church unto God for him.}
\chapsec{Angel Frees Peter from Prison}
\bv{6}{And when Herod was about to bring him forth, the same night Peter was sleeping between two soldiers, bound with two chains: and guards before the door kept the prison.}
\bv{7}{And behold, an angel of the Lord stood by him, and a light shined in the cell: and he smote Peter on the side, and awoke him, saying, ``Rise up quickly.'' And his chains fell off from his hands.}
\bv{8}{And the angel said unto him, ``Gird thyself, and bind on thy sandals.'' And he did so. And he saith unto him, ``Cast thy garment about thee, and follow me.''}
\bv{9}{And he went out, and followed; and he knew not that it was true which was done by the angel, but thought he saw a vision.}
\par
\bv{10}{And when they were past the first and the second guard, they came unto the iron gate that leadeth into the city; which opened to them of its own accord: and they went out, and passed on through one street; and straightway the angel departed from him.}
\bv{11}{And when Peter was come to himself, he said, ``Now I know of a truth, that the Lord hath sent forth his angel and delivered me out of the hand of Herod, and from all the expectation of the people of the Jews.''}
\chapsec{Peter's Freedom Disbelieved}
\bv{12}{And when he had considered \supptext{the thing}, he came to the house of Mary the mother of John whose surname was Mark; where many were gathered together and were praying.}
\bv{13}{And when he knocked at the door of the gate, a maid came to answer, named Rhoda.}
\bv{14}{And when she knew Peter's voice, she opened not the gate for joy, but ran in, and told that Peter stood before the gate.}
\bv{15}{And they said unto her, ``Thou art mad.'' But she confidently affirmed that it was even so. And they said, ``It is his angel.''}
\par
\bv{16}{But Peter continued knocking: and when they had opened, they saw him, and were amazed.}
\bv{17}{But he, beckoning unto them with the hand to hold their peace, declared unto them how the Lord had brought him forth out of the prison. And he said, ``Tell these things unto James, and to the brethren.''' And he departed, and went to another place.}
\bv{18}{Now as soon as it was day, there was no small stir among the soldiers, what was become of Peter.}
\bv{19}{And when Herod had sought for him, and found him not, he examined the guards, and commanded that they should be put to death. And he went down from Jud{\ae}a to C{\ae}sarea, and tarried there.}
\chapsec{Death of Herod}
\bv{20}{Now he was highly displeased with them of Tyre and Sidon: and they came with one accord to him, and, having made Blastus the king's chamberlain their friend, they asked for peace, because their country was fed from the king's country.}
\bv{21}{And upon a set day Herod arrayed himself in royal apparel, and sat on the throne, and made an oration unto them.}
\bv{22}{And the people shouted, \supptext{saying}, ``The voice of a god, and not of a man.''}
\bv{23}{And immediately an angel of the Lord smote him, because he gave not God the glory: and he was eaten of worms, and gave up the ghost.}
\bv{24}{But the word of God grew and multiplied.}
\bv{25}{And Barnabas and Saul returned from Jerusalem, when they had fulfilled their ministration, taking with them John whose surname was Mark.}
\chaphead{Chapter XIII}
\chapdesc{}
\lettrine[image=true, lines=4, findent=3pt, nindent=0pt]{NT/Acts/Acts-Now.eps}{ow} there were at Antioch, in the church that was \supptext{there}, prophets and teachers, Barnabas, and Symeon that was called Niger, and Lucius of Cyrene, and Manaen the foster-brother of Herod the tetrarch, and Saul.
\bv{2}{And as they ministered to the Lord, and fasted, the Holy Ghost said, Separate me Barnabas and Saul for the work whereunto I have called them.}
\bv{3}{Then, when they had fasted and prayed and laid their hands on them, they sent them away.}
\bv{4}{So they, being sent forth by the Holy Ghost, went down to Seleucia; and from thence they sailed to Cyprus.}
\bv{5}{And when they were at Salamis, they proclaimed the word of God in the synagogues of the Jews: and they had also John as their attendant.}
\bv{6}{And when they had gone through the whole island unto Paphos, they found a certain sorcerer, a false prophet, a Jew, whose name was Bar-Jesus;}
\bv{7}{who was with the proconsul, Sergius Paulus, a man of understanding. The same called unto him Barnabas and Saul, and sought to hear the word of God.}
\bv{8}{But Elymas the sorcerer (for so is his name by interpretation) withstood them, seeking to turn aside the proconsul from the faith.}
\bv{9}{But Saul, who is also \supptext{called} Paul, filled with the Holy Ghost, fastened his eyes on him,}
\bv{10}{and said, O full of all guile and all villany, thou son of the devil, thou enemy of all righteousness, wilt thou not cease to pervert the right ways of the Lord?}
\bv{11}{And now, behold, the hand of the Lord is upon thee, and thou shalt be blind, not seeing the sun for a season. And immediately there fell on him a mist and a darkness; and he went about seeking some to lead him by the hand.}
\bv{12}{Then the proconsul, when he saw what was done, believed, being astonished at the teaching of the Lord.}
\bv{13}{Now Paul and his company set sail from Paphos, and came to Perga in Pamphylia: and John departed from them and returned to Jerusalem.}
\bv{14}{But they, passing through from Perga, came to Antioch of Pisidia; and they went into the synagogue on the sabbath day, and sat down.}
\bv{15}{And after the reading of the law and the prophets the rulers of the synagogue sent unto them, saying, Brethren, if ye have any word of exhortation for the people, say on.}
\bv{16}{And Paul stood up, and beckoning with the hand said,
Men of Israel, and ye that fear God, hearken:}
\bv{17}{The God of this people Israel chose our fathers, and exalted the people when they sojourned in the land of Egypt, and with a high arm led he them forth out of it.}
\bv{18}{And for about the time of forty years as a nursing-father bare he them in the wilderness.}
\bv{19}{And when he had destroyed seven nations in the land of Canaan, he gave \supptext{them} their land for an inheritance, for about four hundred and fifty years:}
\bv{20}{and after these things he gave \supptext{them} judges until Samuel the prophet.}
\bv{21}{And afterward they asked for a king: and God gave unto them Saul the son of Kish, a man of the tribe of Benjamin, for the space of forty years.}
\bv{22}{And when he had removed him, he raised up David to be their king; to whom also he bare witness and said, I have found David the son of Jesse, a man after my heart, who shall do all my will.}
\bv{23}{Of this man's seed hath God according to promise brought unto Israel a Saviour, Jesus;}
\bv{24}{when John had first preached before his coming the baptism of repentance to all the people of Israel.}
\bv{25}{And as John was fulfilling his course, he said, What suppose ye that I am? I am not \supptext{he}. But behold, there cometh one after me the shoes of whose feet I am not worthy to unloose.}
\bv{26}{Brethren, children of the stock of Abraham, and those among you that fear God, to us is the word of this salvation sent forth.}
\bv{27}{For they that dwell in Jerusalem, and their rulers, because they knew him not, nor the voices of the prophets which are read every sabbath, fulfilled \supptext{them} by condemning \supptext{him}.}
\bv{28}{And though they found no cause of death \supptext{in him}, yet asked they of Pilate that he should be slain.}
\bv{29}{And when they had fulfilled all things that were written of him, they took him down from the tree, and laid him in a tomb.}
\bv{30}{But God raised him from the dead:}
\bv{31}{and he was seen for many days of them that came up with him from Galilee to Jerusalem, who are now his witnesses unto the people.}
\bv{32}{And we bring you good tidings of the promise made unto the fathers,}
\bv{33}{that God hath fulfilled the same unto our children, in that he raised up Jesus; as also it is written in the second psalm, Thou art my Son, this day have I begotten thee.}
\bv{34}{And as concerning that he raised him up from the dead, now no more to return to corruption, he hath spoken on this wise, I will give you the holy and sure \supptext{blessings} of David.}
\bv{35}{Because he saith also in another \supptext{psalm}, Thou wilt not give thy Holy One to see corruption.}
\bv{36}{For David, after he had in his own generation served the counsel of God, fell asleep, and was laid unto his fathers, and saw corruption:}
\bv{37}{but he whom God raised up saw no corruption.}
\bv{38}{Be it known unto you therefore, brethren, that through this man is proclaimed unto you remission of sins:}
\bv{39}{and by him every one that believeth is justified from all things, from which ye could not be justified by the law of Moses.}
\bv{40}{Beware therefore, lest that come upon \supptext{you} which is spoken in the prophets:}
\otQuote{Hab. 1:5}{\bv{41}{Behold, ye despisers, and wonder, and perish;
For I work a work in your days,
A work which ye shall in no wise believe, if one declare it unto you.}}
\bv{42}{And as they went out, they besought that these words might be spoken to them the next sabbath.}
\bv{43}{Now when the synagogue broke up, many of the Jews and of the devout proselytes followed Paul and Barnabas; who, speaking to them, urged them to continue in the grace of God.}
\bv{44}{And the next sabbath almost the whole city was gathered together to hear the word of God.}
\bv{45}{But when the Jews saw the multitudes, they were filled with jealousy, and contradicted the things which were spoken by Paul, and blasphemed.}
\bv{46}{And Paul and Barnabas spake out boldly, and said, It was necessary that the word of God should first be spoken to you. Seeing ye thrust it from you, and judge yourselves unworthy of eternal life, lo, we turn to the Gentiles.}
\bv{47}{For so hath the Lord commanded us, \supptext{saying},}
\otQuote{Is. 49:6}{I have set thee for a light of the Gentiles,
That thou shouldest be for salvation unto the uttermost part of the earth.}
\bv{48}{And as the Gentiles heard this, they were glad, and glorified the word of God: and as many as were ordained to eternal life believed.}
\bv{49}{And the word of the Lord was spread abroad throughout all the region.}
\bv{50}{But the Jews urged on the devout women of honourable estate, and the chief men of the city, and stirred up a persecution against Paul and Barnabas, and cast them out of their borders.}
\bv{51}{But they shook off the dust of their feet against them, and came unto Iconium.}
\bv{52}{And the disciples were filled with joy and with the Holy Ghost.}
\chaphead{Chapter XIV}
\chapdesc{}
\lettrine[image=true, lines=4, findent=3pt, nindent=0pt]{NT/Acts/Acts-And.eps}{nd} it came to pass in Iconium that they entered together into the synagogue of the Jews, and so spake that a great multitude both of Jews and of Greeks believed.
\bv{2}{But the Jews that were disobedient stirred up the souls of the Gentiles, and made them evil affected against the brethren.}
\bv{3}{Long time therefore they tarried \supptext{there} speaking boldly in the Lord, who bare witness unto the word of his grace, granting signs and wonders to be done by their hands.}
\bv{4}{But the multitude of the city was divided; and part held with the Jews, and part with the apostles.}
\bv{5}{And when there was made an onset both of the Gentiles and of the Jews with their rulers, to treat them shamefully and to stone them,}
\bv{6}{they became aware of it, and fled unto the cities of Lycaonia, Lystra and Derbe, and the region round about:}
\bv{7}{and there they preached the gospel.}
\bv{8}{And at Lystra there sat a certain man, impotent in his feet, a cripple from his mother's womb, who never had walked.}
\bv{9}{The same heard Paul speaking: who, fastening his eyes upon him, and seeing that he had faith to be made whole,}
\bv{10}{said with a loud voice, Stand upright on thy feet. And he leaped up and walked.}
\bv{11}{And when the multitude saw what Paul had done, they lifted up their voice, saying in the speech of Lycaonia, The gods are come down to us in the likeness of men.}
\bv{12}{And they called Barnabas, Jupiter; and Paul, Mercury, because he was the chief speaker.}
\bv{13}{And the priest of Jupiter whose \supptext{temple} was before the city, brought oxen and garlands unto the gates, and would have done sacrifice with the multitudes.}
\bv{14}{But when the apostles, Barnabas and Paul, heard of it, they rent their garments, and sprang forth among the multitude, crying out}
\bv{15}{and saying, Sirs, why do ye these things? We also are men of like passions with you, and bring you good tidings, that ye should turn from these vain things unto a living God, who made the heaven and the earth and the sea, and all that in them is:}
\bv{16}{who in the generations gone by suffered all the nations to walk in their own ways.}
\bv{17}{And yet he left not himself without witness, in that he did good and gave you from heaven rains and fruitful seasons, filling your hearts with food and gladness.}
\bv{18}{And with these sayings scarce restrained they the multitudes from doing sacrifice unto them.}
\bv{19}{But there came Jews thither from Antioch and Iconium: and having persuaded the multitudes, they stoned Paul, and dragged him out of the city, supposing that he was dead.}
\bv{20}{But as the disciples stood round about him, he rose up, and entered into the city: and on the morrow he went forth with Barnabas to Derbe.}
\bv{21}{And when they had preached the gospel to that city, and had made many disciples, they returned to Lystra, and to Iconium, and to Antioch,}
\bv{22}{confirming the souls of the disciples, exhorting them to continue in the faith, and that through many tribulations we must enter into the kingdom of God.}
\bv{23}{And when they had appointed for them elders in every church, and had prayed with fasting, they commended them to the Lord, on whom they had believed.}
\bv{24}{And they passed through Pisidia, and came to Pamphylia.}
\bv{25}{And when they had spoken the word in Perga, they went down to Attalia;}
\bv{26}{and thence they sailed to Antioch, from whence they had been committed to the grace of God for the work which they had fulfilled.}
\bv{27}{And when they were come, and had gathered the church together, they rehearsed all things that God had done with them, and that he had opened a door of faith unto the Gentiles.}
\bv{28}{And they tarried no little time with the disciples.}
\chaphead{Chapter XV}
\chapdesc{}
\lettrine[image=true, lines=4, findent=3pt, nindent=0pt]{NT/Acts/Acts-And.eps}{nd} certain men came down from Jud{\ae}a and taught the brethren, \supptext{saying}, Except ye be circumcised after the custom of Moses, ye cannot be saved.
\bv{2}{And when Paul and Barnabas had no small dissension and questioning with them, \supptext{the brethren} appointed that Paul and Barnabas, and certain other of them, should go up to Jerusalem unto the apostles and elders about this question.}
\bv{3}{They therefore, being brought on their way by the church, passed through both Phoenicia and Samaria, declaring the conversion of the Gentiles: and they caused great joy unto all the brethren.}
\bv{4}{And when they were come to Jerusalem, they were received of the church and the apostles and the elders, and they rehearsed all things that God had done with them.}
\bv{5}{But there rose up certain of the sect of the Pharisees who believed, saying, It is needful to circumcise them, and to charge them to keep the law of Moses.}
\bv{6}{And the apostles and the elders were gathered together to consider of this matter.}
\bv{7}{And when there had been much questioning, Peter rose up, and said unto them,
Brethren, ye know that a good while ago God made choice among you, that by my mouth the Gentiles should hear the word of the gospel, and believe.}
\bv{8}{And God, who knoweth the heart, bare them witness, giving them the Holy Ghost, even as he did unto us;}
\bv{9}{and he made no distinction between us and them, cleansing their hearts by faith.}
\bv{10}{Now therefore why make ye trial of God, that ye should put a yoke upon the neck of the disciples which neither our fathers nor we were able to bear?}
\bv{11}{But we believe that we shall be saved through the grace of the Lord Jesus, in like manner as they.}
\bv{12}{And all the multitude kept silence; and they hearkened unto Barnabas and Paul rehearsing what signs and wonders God had wrought among the Gentiles through them.}
\bv{13}{And after they had held their peace, James answered, saying,
Brethren, hearken unto me:}
\bv{14}{Symeon hath rehearsed how first God visited the Gentiles, to take out of them a people for his name.}
\bv{15}{And to this agree the words of the prophets; as it is written,}
\otQuote{Amos 9:11-2}{\bv{16}{After these things I will return,
And I will set it up:}
\bv{17}{That the residue of men may seek after the Lord,
And all the Gentiles, upon whom my name is called,}
\bv{18}{Saith the Lord, who maketh these things known from of old.}}
\bv{19}{Wherefore my judgement is, that we trouble not them that from among the Gentiles turn to God;}
\bv{20}{but that we write unto them, that they abstain from the pollutions of idols, and from fornication, and from what is strangled, and from blood.}
\bv{21}{For Moses from generations of old hath in every city them that preach him, being read in the synagogues every sabbath.}
\bv{22}{Then it seemed good to the apostles and the elders, with the whole church, to choose men out of their company, and send them to Antioch with Paul and Barnabas; \supptext{namely}, Judas called Barsabbas, and Silas, chief men among the brethren:}
\bv{23}{and they wrote \supptext{thus} by them, The apostles and the elders, brethren, unto the brethren who are of the Gentiles in Antioch and Syria and Cilicia, greeting:}
\bv{24}{Forasmuch as we have heard that certain who went out from us have troubled you with words, subverting your souls; to whom we gave no commandment;}
\bv{25}{it seemed good unto us, having come to one accord, to choose out men and send them unto you with our beloved Barnabas and Paul,}
\bv{26}{men that have hazarded their lives for the name of our Lord Jesus Christ.}
\bv{27}{We have sent therefore Judas and Silas, who themselves also shall tell you the same things by word of mouth.}
\bv{28}{For it seemed good to the Holy Ghost, and to us, to lay upon you no greater burden than these necessary things:}
\bv{29}{that ye abstain from things sacrificed to idols, and from blood, and from things strangled, and from fornication; from which if ye keep yourselves, it shall be well with you. Fare ye well.}
\bv{30}{So they, when they were dismissed, came down to Antioch; and having gathered the multitude together, they delivered the epistle.}
\bv{31}{And when they had read it, they rejoiced for the consolation.}
\bv{32}{And Judas and Silas, being themselves also prophets, exhorted the brethren with many words, and confirmed them.}
\bv{33}{And after they had spent some time \supptext{there}, they were dismissed in peace from the brethren unto those that had sent them forth.\mcomm{But it seemed good unto Silas to abide there.}}
\bv{35}{But Paul and Barnabas tarried in Antioch, teaching and preaching the word of the Lord, with many others also.}
\bv{36}{And after some days Paul said unto Barnabas, Let us return now and visit the brethren in every city wherein we proclaimed the word of the Lord, \supptext{and see} how they fare.}
\bv{37}{And Barnabas was minded to take with them John also, who was called Mark.}
\bv{38}{But Paul thought not good to take with them him who withdrew from them from Pamphylia, and went not with them to the work.}
\bv{39}{And there arose a sharp contention, so that they parted asunder one from the other, and Barnabas took Mark with him, and sailed away unto Cyprus:}
\bv{40}{but Paul chose Silas, and went forth, being commended by the brethren to the grace of the Lord.}
\bv{41}{And he went through Syria and Cilicia, confirming the churches.}
\chaphead{Chapter XVI}
\chapdesc{}
\lettrine[image=true, lines=4, findent=3pt, nindent=0pt]{NT/Acts/Acts-And.eps}{nd} he came also to Derbe and to Lystra: and behold, a certain disciple was there, named Timothy, the son of a Jewess that believed; but his father was a Greek.
\bv{2}{The same was well reported of by the brethren that were at Lystra and Iconium.}
\bv{3}{Him would Paul have to go forth with him; and he took and circumcised him because of the Jews that were in those parts: for they all knew that his father was a Greek.}
\bv{4}{And as they went on their way through the cities, they delivered them the decrees to keep which had been ordained of the apostles and elders that were at Jerusalem.}
\bv{5}{So the churches were strengthened in the faith, and increased in number daily.}
\bv{6}{And they went through the region of Phrygia and Galatia, having been forbidden of the Holy Ghost to speak the word in Asia;}
\bv{7}{and when they were come over against Mysia, they assayed to go into Bithynia; and the Spirit of Jesus suffered them not;}
\bv{8}{and passing by Mysia, they came down to Troas.}
\bv{9}{And a vision appeared to Paul in the night: There was a man of Macedonia standing, beseeching him, and saying, Come over into Macedonia, and help us.}
\bv{10}{And when he had seen the vision, straightway we sought to go forth into Macedonia, concluding that God had called us to preach the gospel unto them.}
\bv{11}{Setting sail therefore from Troas, we made a straight course to Samothrace, and the day following to Neapolis;}
\bv{12}{and from thence to Philippi, which is a city of Macedonia, the first of the district, a \supptext{Roman} colony: and we were in this city tarrying certain days.}
\bv{13}{And on the sabbath day we went forth without the gate by a river side, where we supposed there was a place of prayer; and we sat down, and spake unto the women that were come together.}
\bv{14}{And a certain woman named Lydia, a seller of purple, of the city of Thyatira, one that worshipped God, heard us: whose heart the Lord opened to give heed unto the things which were spoken by Paul.}
\bv{15}{And when she was baptised, and her household, she besought us, saying, If ye have judged me to be faithful to the Lord, come into my house, and abide \supptext{there}. And she constrained us.}
\bv{16}{And it came to pass, as we were going to the place of prayer, that a certain maid having a spirit of divination met us, who brought her masters much gain by soothsaying.}
\bv{17}{The same following after Paul and us cried out, saying, These men are servants of the Most High God, who proclaim unto you the way of salvation.}
\bv{18}{And this she did for many days. But Paul, being sore troubled, turned and said to the spirit, I charge thee in the name of Jesus Christ to come out of her. And it came out that very hour.}
\bv{19}{But when her masters saw that the hope of their gain was gone, they laid hold on Paul and Silas, and dragged them into the marketplace before the rulers,}
\bv{20}{and when they had brought them unto the magistrates, they said, These men, being Jews, do exceedingly trouble our city,}
\bv{21}{and set forth customs which it is not lawful for us to receive, or to observe, being Romans.}
\bv{22}{And the multitude rose up together against them: and the magistrates rent their garments off them, and commanded to beat them with rods.}
\bv{23}{And when they had laid many stripes upon them, they cast them into prison, charging the jailor to keep them safely:}
\bv{24}{who, having received such a charge, cast them into the inner prison, and made their feet fast in the stocks.}
\bv{25}{But about midnight Paul and Silas were praying and singing hymns unto God, and the prisoners were listening to them;}
\bv{26}{and suddenly there was a great earthquake, so that the foundations of the prison-house were shaken: and immediately all the doors were opened; and every one's bands were loosed.}
\bv{27}{And the jailor, being roused out of sleep and seeing the prison doors open, drew his sword and was about to kill himself, supposing that the prisoners had escaped.}
\bv{28}{But Paul cried with a loud voice, saying, Do thyself no harm: for we are all here.}
\bv{29}{And he called for lights and sprang in, and, trembling for fear, fell down before Paul and Silas,}
\bv{30}{and brought them out and said, Sirs, what must I do to be saved?}
\bv{31}{And they said, Believe on the Lord Jesus, and thou shalt be saved, thou and thy house.}
\bv{32}{And they spake the word of the Lord unto him, with all that were in his house.}
\bv{33}{And he took them the same hour of the night, and washed their stripes; and was baptised, he and all his, immediately.}
\bv{34}{And he brought them up into his house, and set food before them, and rejoiced greatly, with all his house, having believed in God.}
\bv{35}{But when it was day, the magistrates sent the serjeants, saying, Let those men go.}
\bv{36}{And the jailor reported the words to Paul, \supptext{saying}, The magistrates have sent to let you go: now therefore come forth, and go in peace.}
\bv{37}{But Paul said unto them, They have beaten us publicly, uncondemned, men that are Romans, and have cast us into prison; and do they now cast us out privily? nay verily; but let them come themselves and bring us out.}
\bv{38}{And the serjeants reported these words unto the magistrates: and they feared when they heard that they were Romans;}
\bv{39}{and they came and besought them; and when they had brought them out, they asked them to go away from the city.}
\bv{40}{And they went out of the prison, and entered into \supptext{the house of} Lydia: and when they had seen the brethren, they comforted them, and departed.}
\chaphead{Chapter XVII}
\chapdesc{}
\lettrine[image=true, lines=4, findent=3pt, nindent=0pt]{NT/Acts/Acts-Now.eps}{ow} when they had passed through Amphipolis and Apollonia, they came to Thessalonica, where was a synagogue of the Jews:
\bv{2}{and Paul, as his custom was, went in unto them, and for three sabbath days reasoned with them from the scriptures,}
\bv{3}{opening and alleging that it behooved the Christ to suffer, and to rise again from the dead; and that this Jesus, whom, \supptext{said he}, I proclaim unto you, is the Christ.}
\bv{4}{And some of them were persuaded, and consorted with Paul and Silas; and of the devout Greeks a great multitude, and of the chief women not a few.}
\bv{5}{But the Jews, being moved with jealousy, took unto them certain vile fellows of the rabble, and gathering a crowd, set the city on an uproar; and assaulting the house of Jason, they sought to bring them forth to the people.}
\bv{6}{And when they found them not, they dragged Jason and certain brethren before the rulers of the city, crying, These that have turned the world upside down are come hither also;}
\bv{7}{whom Jason hath received: and these all act contrary to the decrees of C{\ae}sar, saying that there is another king, \supptext{one} Jesus.}
\bv{8}{And they troubled the multitude and the rulers of the city, when they heard these things.}
\bv{9}{And when they had taken security from Jason and the rest, they let them go.}
\bv{10}{And the brethren immediately sent away Paul and Silas by night unto Beroea: who when they were come thither went into the synagogue of the Jews.}
\bv{11}{Now these were more noble than those in Thessalonica, in that they received the word with all readiness of mind, examining the scriptures daily, whether these things were so.}
\bv{12}{Many of them therefore believed; also of the Greek women of honourable estate, and of men, not a few.}
\bv{13}{But when the Jews of Thessalonica had knowledge that the word of God was proclaimed of Paul at Beroea also, they came thither likewise, stirring up and troubling the multitudes.}
\bv{14}{And then immediately the brethren sent forth Paul to go as far as to the sea: and Silas and Timothy abode there still.}
\bv{15}{But they that conducted Paul brought him as far as Athens: and receiving a commandment unto Silas and Timothy that they should come to him with all speed, they departed.}
\bv{16}{Now while Paul waited for them at Athens, his spirit was provoked within him as he beheld the city full of idols.}
\bv{17}{So he reasoned in the synagogue with the Jews and the devout persons, and in the marketplace every day with them that met him.}
\bv{18}{And certain also of the Epicurean and Stoic philosophers encountered him. And some said, What would this babbler say? others, He seemeth to be a setter forth of strange gods: because he preached Jesus and the resurrection.}
\bv{19}{And they took hold of him, and brought him unto the Areopagus, saying, May we know what this new teaching is, which is spoken by thee?}
\bv{20}{For thou bringest certain strange things to our ears: we would know therefore what these things mean.}
\bv{21}{(Now all the Athenians and the strangers sojourning there spent their time in nothing else, but either to tell or to hear some new thing.)}
\bv{22}{And Paul stood in the midst of the Areopagus, and said,
Ye men of Athens, in all things I perceive that ye are very religious.}
\bv{23}{For as I passed along, and observed the objects of your worship, I found also an altar with this inscription, TO AN UNKNOWN GOD. What therefore ye worship in ignorance, this I set forth unto you.}
\bv{24}{The God that made the world and all things therein, he, being Lord of heaven and earth, dwelleth not in temples made with hands;}
\bv{25}{neither is he served by men's hands, as though he needed anything, seeing he himself giveth to all life, and breath, and all things;}
\bv{26}{and he made of one every nation of men to dwell on all the face of the earth, having determined \supptext{their} appointed seasons, and the bounds of their habitation;}
\bv{27}{that they should seek God, if haply they might feel after him and find him, though he is not far from each one of us:}
\bv{28}{for in him we live, and move, and have our being; as certain even of your own poets have said,}
\canticle{For we are also his offspring.}
\bv{29}{Being then the offspring of God, we ought not to think that the Godhead is like unto gold, or silver, or stone, graven by art and device of man.}
\bv{30}{The times of ignorance therefore God overlooked; but now he commandeth men that they should all everywhere repent:}
\bv{31}{inasmuch as he hath appointed a day in which he will judge the world in righteousness by the man whom he hath ordained; whereof he hath given assurance unto all men, in that he hath raised him from the dead.}
\bv{32}{Now when they heard of the resurrection of the dead, some mocked; but others said, We will hear thee concerning this yet again.}
\bv{33}{Thus Paul went out from among them.}
\bv{34}{But certain men clave unto him, and believed: among whom also was Dionysius the Areopagite, and a woman named Damaris, and others with them.}
\chaphead{Chapter XVIII}
\chapdesc{}
\lettrine[image=true, lines=4, findent=3pt, nindent=0pt]{NT/Acts/Acts-After.eps}{fter} these things he departed from Athens, and came to Corinth.
\bv{2}{And he found a certain Jew named Aquila, a man of Pontus by race, lately come from Italy, with his wife Priscilla, because Claudius had commanded all the Jews to depart from Rome: and he came unto them;}
\bv{3}{and because he was of the same trade, he abode with them, and they wrought; for by their trade they were tentmakers.}
\bv{4}{And he reasoned in the synagogue every sabbath, and persuaded Jews and Greeks.}
\bv{5}{But when Silas and Timothy came down from Macedonia, Paul was constrained by the word, testifying to the Jews that Jesus was the Christ.}
\bv{6}{And when they opposed themselves and blasphemed, he shook out his raiment and said unto them, Your blood \supptext{be} upon your own heads; I am clean: from henceforth I will go unto the Gentiles.}
\bv{7}{And he departed thence, and went into the house of a certain man named Titus Justus, one that worshipped God, whose house joined hard to the synagogue.}
\bv{8}{And Crispus, the ruler of the synagogue, believed in the Lord with all his house; and many of the Corinthians hearing believed, and were baptised.}
\bv{9}{And the Lord said unto Paul in the night by a vision, Be not afraid, but speak and hold not thy peace:}
\bv{10}{for I am with thee, and no man shall set on thee to harm thee: for I have much people in this city.}
\bv{11}{And he dwelt \supptext{there} a year and six months, teaching the word of God among them.}
\bv{12}{But when Gallio was proconsul of Achaia, the Jews with one accord rose up against Paul and brought him before the judgement-seat,}
\bv{13}{saying, This man persuadeth men to worship God contrary to the law.}
\bv{14}{But when Paul was about to open his mouth, Gallio said unto the Jews, If indeed it were a matter of wrong or of wicked villany, O ye Jews, reason would that I should bear with you:}
\bv{15}{but if they are questions about words and names and your own law, look to it yourselves; I am not minded to be a judge of these matters.}
\bv{16}{And he drove them from the judgement-seat.}
\bv{17}{And they all laid hold on Sosthenes, the ruler of the synagogue, and beat him before the judgement-seat. And Gallio cared for none of these things.}
\bv{18}{And Paul, having tarried after this yet many days, took his leave of the brethren, and sailed thence for Syria, and with him Priscilla and Aquila: having shorn his head in Cenchre{\ae}; for he had a vow.}
\bv{19}{And they came to Ephesus, and he left them there: but he himself entered into the synagogue, and reasoned with the Jews.}
\bv{20}{And when they asked him to abide a longer time, he consented not;}
\bv{21}{but taking his leave of them, and saying, I will return again unto you if God will, he set sail from Ephesus.}
\bv{22}{And when he had landed at C{\ae}sarea, he went up and saluted the church, and went down to Antioch.}
\bv{23}{And having spent some time \supptext{there}, he departed, and went through the region of Galatia, and Phrygia, in order, establishing all the disciples.}
\bv{24}{Now a certain Jew named Apollos, an Alexandrian by race, an eloquent man, came to Ephesus; and he was mighty in the scriptures.}
\bv{25}{This man had been instructed in the way of the Lord; and being fervent in spirit, he spake and taught accurately the things concerning Jesus, knowing only the baptism of John:}
\bv{26}{and he began to speak boldly in the synagogue. But when Priscilla and Aquila heard him, they took him unto them, and expounded unto him the way of God more accurately.}
\bv{27}{And when he was minded to pass over into Achaia, the brethren encouraged him, and wrote to the disciples to receive him: and when he was come, he helped them much that had believed through grace;}
\bv{28}{for he powerfully confuted the Jews, \supptext{and that} publicly, showing by the scriptures that Jesus was the Christ.}
\chaphead{Chapter XIX}
\chapdesc{}
\lettrine[image=true, lines=4, findent=3pt, nindent=0pt]{NT/Acts/Acts-And.eps}{nd} it came to pass, that, while Apollos was at Corinth, Paul having passed through the upper country came to Ephesus, and found certain disciples:
\bv{2}{and he said unto them, Did ye receive the Holy Ghost when ye believed? And they \supptext{said} unto him, Nay, we did not so much as hear whether the Holy Ghost was \supptext{given}.}
\bv{3}{And he said, Into what then were ye baptised? And they said, Into John's baptism.}
\bv{4}{And Paul said, John baptised with the baptism of repentance, saying unto the people that they should believe on him that should come after him, that is, on Jesus.}
\bv{5}{And when they heard this, they were baptised into the name of the Lord Jesus.}
\bv{6}{And when Paul had laid his hands upon them, the Holy Ghost came on them; and they spake with tongues, and prophesied.}
\bv{7}{And they were in all about twelve men.}
\bv{8}{And he entered into the synagogue, and spake boldly for the space of three months, reasoning and persuading \supptext{as to} the things concerning the kingdom of God.}
\bv{9}{But when some were hardened and disobedient, speaking evil of the Way before the multitude, he departed from them, and separated the disciples, reasoning daily in the school of Tyrannus.}
\bv{10}{And this continued for the space of two years; so that all they that dwelt in Asia heard the word of the Lord, both Jews and Greeks.}
\bv{11}{And God wrought special miracles by the hands of Paul:}
\bv{12}{insomuch that unto the sick were carried away from his body handkerchiefs or aprons, and the diseases departed from them, and the evil spirits went out.}
\bv{13}{But certain also of the strolling Jews, exorcists, took upon them to name over them that had the evil spirits the name of the Lord Jesus, saying, I adjure you by Jesus whom Paul preacheth.}
\bv{14}{And there were seven sons of one Sceva, a Jew, a chief priest, who did this.}
\bv{15}{And the evil spirit answered and said unto them, Jesus I know, and Paul I know; but who are ye?}
\bv{16}{And the man in whom the evil spirit was leaped on them, and mastered both of them, and prevailed against them, so that they fled out of that house naked and wounded.}
\bv{17}{And this became known to all, both Jews and Greeks, that dwelt at Ephesus; and fear fell upon them all, and the name of the Lord Jesus was magnified.}
\bv{18}{Many also of them that had believed came, confessing, and declaring their deeds.}
\bv{19}{And not a few of them that practised magical arts brought their books together and burned them in the sight of all; and they counted the price of them, and found it fifty thousand pieces of silver.}
\bv{20}{So mightily grew the word of the Lord and prevailed.}
\bv{21}{Now after these things were ended, Paul purposed in the spirit, when he had passed through Macedonia and Achaia, to go to Jerusalem, saying, After I have been there, I must also see Rome.}
\bv{22}{And having sent into Macedonia two of them that ministered unto him, Timothy and Erastus, he himself stayed in Asia for a while.}
\bv{23}{And about that time there arose no small stir concerning the Way.}
\bv{24}{For a certain man named Demetrius, a silversmith, who made silver shrines of Diana, brought no little business unto the craftsmen;}
\bv{25}{whom he gathered together, with the workmen of like occupation, and said, Sirs, ye know that by this business we have our wealth.}
\bv{26}{And ye see and hear, that not alone at Ephesus, but almost throughout all Asia, this Paul hath persuaded and turned away much people, saying that they are no gods, that are made with hands:}
\bv{27}{and not only is there danger that this our trade come into disrepute; but also that the temple of the great goddess Diana be made of no account, and that she should even be deposed from her magnificence whom all Asia and the world worshippeth.}
\bv{28}{And when they heard this they were filled with wrath, and cried out, saying, Great \supptext{is} Diana of the Ephesians.}
\bv{29}{And the city was filled with the confusion: and they rushed with one accord into the theatre, having seized Gaius and Aristarchus, men of Macedonia, Paul's companions in travel.}
\bv{30}{And when Paul was minded to enter in unto the people, the disciples suffered him not.}
\bv{31}{And certain also of the Asiarchs, being his friends, sent unto him and besought him not to adventure himself into the theatre.}
\bv{32}{Some therefore cried one thing, and some another: for the assembly was in confusion; and the more part knew not wherefore they were come together.}
\bv{33}{And they brought Alexander out of the multitude, the Jews putting him forward. And Alexander beckoned with the hand, and would have made a defence unto the people.}
\bv{34}{But when they perceived that he was a Jew, all with one voice about the space of two hours cried out, Great \supptext{is} Diana of the Ephesians.}
\bv{35}{And when the townclerk had quieted the multitude, he saith, Ye men of Ephesus, what man is there who knoweth not that the city of the Ephesians is temple-keeper of the great Diana, and of the \supptext{image} which fell down from Jupiter?}
\bv{36}{Seeing then that these things cannot be gainsaid, ye ought to be quiet, and to do nothing rash.}
\bv{37}{For ye have brought \supptext{hither} these men, who are neither robbers of temples nor blasphemers of our goddess.}
\bv{38}{If therefore Demetrius, and the craftsmen that are with him, have a matter against any man, the courts are open, and there are proconsuls: let them accuse one another.}
\bv{39}{But if ye seek anything about other matters, it shall be settled in the regular assembly.}
\bv{40}{For indeed we are in danger to be accused concerning this day's riot, there being no cause \supptext{for it}: and as touching it we shall not be able to give account of this concourse.}
\bv{41}{And when he had thus spoken, he dismissed the assembly.}
\chaphead{Chapter XX}
\chapdesc{}
\lettrine[image=true, lines=4, findent=3pt, nindent=0pt]{NT/Acts/Acts-And.eps}{nd} after the uproar ceased, Paul having sent for the disciples and exhorted them, took leave of them, and departed to go into Macedonia.
\bv{2}{And when he had gone through those parts, and had given them much exhortation, he came into Greece.}
\bv{3}{And when he had spent three months \supptext{there}, and a plot was laid against him by the Jews as he was about to set sail for Syria, he determined to return through Macedonia.}
\bv{4}{And there accompanied him as far as Asia, Sopater of Beroea, \supptext{the son} of Pyrrhus; and of the Thessalonians, Aristarchus and Secundus; and Gaius of Derbe, and Timothy; and of Asia, Tychicus and Trophimus.}
\bv{5}{But these had gone before, and were waiting for us at Troas.}
\bv{6}{And we sailed away from Philippi after the days of unleavened bread, and came unto them to Troas in five days; where we tarried seven days.}
\bv{7}{And upon the first day of the week, when we were gathered together to break bread, Paul discoursed with them, intending to depart on the morrow; and prolonged his speech until midnight.}
\bv{8}{And there were many lights in the upper chamber where we were gathered together.}
\bv{9}{And there sat in the window a certain young man named Eutychus, borne down with deep sleep; and as Paul discoursed yet longer, being borne down by his sleep he fell down from the third story, and was taken up dead.}
\bv{10}{And Paul went down, and fell on him, and embracing him said, Make ye no ado; for his life is in him.}
\bv{11}{And when he was gone up, and had broken the bread, and eaten, and had talked with them a long while, even till break of day, so he departed.}
\bv{12}{And they brought the lad alive, and were not a little comforted.}
\bv{13}{But we, going before to the ship, set sail for Assos, there intending to take in Paul: for so had he appointed, intending himself to go by land.}
\bv{14}{And when he met us at Assos, we took him in, and came to Mitylene.}
\bv{15}{And sailing from thence, we came the following day over against Chios; and the next day we touched at Samos; and the day after we came to Miletus.}
\bv{16}{For Paul had determined to sail past Ephesus, that he might not have to spend time in Asia; for he was hastening, if it were possible for him, to be at Jerusalem the day of Pentecost.}
\bv{17}{And from Miletus he sent to Ephesus, and called to him the elders of the church.}
\bv{18}{And when they were come to him, he said unto them,
Ye yourselves know, from the first day that I set foot in Asia, after what manner I was with you all the time,}
\bv{19}{serving the Lord with all lowliness of mind, and with tears, and with trials which befell me by the plots of the Jews;}
\bv{20}{how I shrank not from declaring unto you anything that was profitable, and teaching you publicly, and from house to house,}
\bv{21}{testifying both to Jews and to Greeks repentance toward God, and faith toward our Lord Jesus Christ.}
\bv{22}{And now, behold, I go bound in the spirit unto Jerusalem, not knowing the things that shall befall me there:}
\bv{23}{save that the Holy Ghost testifieth unto me in every city, saying that bonds and afflictions abide me.}
\bv{24}{But I hold not my life of any account as dear unto myself, so that I may accomplish my course, and the ministry which I received from the Lord Jesus, to testify the gospel of the grace of God.}
\bv{25}{And now, behold, I know that ye all, among whom I went about preaching the kingdom, shall see my face no more.}
\bv{26}{Wherefore I testify unto you this day, that I am pure from the blood of all men.}
\bv{27}{For I shrank not from declaring unto you the whole counsel of God.}
\bv{28}{Take heed unto yourselves, and to all the flock, in which the Holy Ghost hath made you bishops, to feed the church of the Lord which he purchased with his own blood.}
\bv{29}{I know that after my departing grievous wolves shall enter in among you, not sparing the flock;}
\bv{30}{and from among your own selves shall men arise, speaking perverse things, to draw away the disciples after them.}
\bv{31}{Wherefore watch ye, remembering that by the space of three years I ceased not to admonish every one night and day with tears.}
\bv{32}{And now I commend you to God, and to the word of his grace, which is able to build \supptext{you} up, and to give \supptext{you} the inheritance among all them that are sanctified.}
\bv{33}{I coveted no man's silver, or gold, or apparel.}
\bv{34}{Ye yourselves know that these hands ministered unto my necessities, and to them that were with me.}
\bv{35}{In all things I gave you an example, that so laboring ye ought to help the weak, and to remember the words of the Lord Jesus, that he himself said, It is more blessed to give than to receive.}
\bv{36}{And when he had thus spoken, he kneeled down and prayed with them all.}
\bv{37}{And they all wept sore, and fell on Paul's neck and kissed him,}
\bv{38}{sorrowing most of all for the word which he had spoken, that they should behold his face no more. And they brought him on his way unto the ship.}
\chaphead{Chapter XXI}
\chapdesc{}
\lettrine[image=true, lines=4, findent=3pt, nindent=0pt]{NT/Acts/Acts-And.eps}{nd} when it came to pass that we were parted from them and had set sail, we came with a straight course unto Cos, and the next day unto Rhodes, and from thence unto Patara:
\bv{2}{and having found a ship crossing over unto Phoenicia, we went aboard, and set sail.}
\bv{3}{And when we had come in sight of Cyprus, leaving it on the left hand, we sailed unto Syria, and landed at Tyre; for there the ship was to unlade her burden.}
\bv{4}{And having found the disciples, we tarried there seven days: and these said to Paul through the Spirit, that he should not set foot in Jerusalem.}
\bv{5}{And when it came to pass that we had accomplished the days, we departed and went on our journey; and they all, with wives and children, brought us on our way till we were out of the city: and kneeling down on the beach, we prayed, and bade each other farewell;}
\bv{6}{and we went on board the ship, but they returned home again.}
\bv{7}{And when we had finished the voyage from Tyre, we arrived at Ptolemais; and we saluted the brethren, and abode with them one day.}
\bv{8}{And on the morrow we departed, and came unto C{\ae}sarea: and entering into the house of Philip the evangelist, who was one of the seven, we abode with him.}
\bv{9}{Now this man had four virgin daughters, who prophesied.}
\bv{10}{And as we tarried there some days, there came down from Jud{\ae}a a certain prophet, named Agabus.}
\bv{11}{And coming to us, and taking Paul's girdle, he bound his own feet and hands, and said, Thus saith the Holy Ghost, So shall the Jews at Jerusalem bind the man that owneth this girdle, and shall deliver him into the hands of the Gentiles.}
\bv{12}{And when we heard these things, both we and they of that place besought him not to go up to Jerusalem.}
\bv{13}{Then Paul answered, What do ye, weeping and breaking my heart? for I am ready not to be bound only, but also to die at Jerusalem for the name of the Lord Jesus.}
\bv{14}{And when he would not be persuaded, we ceased, saying, The will of the Lord be done.}
\bv{15}{And after these days we took up our baggage and went up to Jerusalem.}
\bv{16}{And there went with us also \supptext{certain} of the disciples from C{\ae}sarea, bringing \supptext{with them} one Mnason of Cyprus, an early disciple, with whom we should lodge.}
\bv{17}{And when we were come to Jerusalem, the brethren received us gladly.}
\bv{18}{And the day following Paul went in with us unto James; and all the elders were present.}
\bv{19}{And when he had saluted them, he rehearsed one by one the things which God had wrought among the Gentiles through his ministry.}
\bv{20}{And they, when they heard it, glorified God; and they said unto him, Thou seest, brother, how many thousands there are among the Jews of them that have believed; and they are all zealous for the law:}
\bv{21}{and they have been informed concerning thee, that thou teachest all the Jews who are among the Gentiles to forsake Moses, telling them not to circumcise their children, neither to walk after the customs.}
\bv{22}{What is it therefore? they will certainly hear that thou art come.}
\bv{23}{Do therefore this that we say to thee: We have four men that have a vow on them;}
\bv{24}{these take, and purify thyself with them, and be at charges for them, that they may shave their heads: and all shall know that there is no truth in the things whereof they have been informed concerning thee; but that thou thyself also walkest orderly, keeping the law.}
\bv{25}{But as touching the Gentiles that have believed, we wrote, giving judgement that they should keep themselves from things sacrificed to idols, and from blood, and from what is strangled, and from fornication.}
\bv{26}{Then Paul took the men, and the next day purifying himself with them went into the temple, declaring the fulfilment of the days of purification, until the offering was offered for every one of them.}
\bv{27}{And when the seven days were almost completed, the Jews from Asia, when they saw him in the temple, stirred up all the multitude and laid hands on him,}
\bv{28}{crying out, Men of Israel, help: This is the man that teacheth all men everywhere against the people, and the law, and this place; and moreover he brought Greeks also into the temple, and hath defiled this holy place.}
\bv{29}{For they had before seen with him in the city Trophimus the Ephesian, whom they supposed that Paul had brought into the temple.}
\bv{30}{And all the city was moved, and the people ran together; and they laid hold on Paul, and dragged him out of the temple: and straightway the doors were shut.}
\bv{31}{And as they were seeking to kill him, tidings came up to the chief captain of the band, that all Jerusalem was in confusion.}
\bv{32}{And forthwith he took soldiers and centurions, and ran down upon them: and they, when they saw the chief captain and the soldiers, left off beating Paul.}
\bv{33}{Then the chief captain came near, and laid hold on him, and commanded him to be bound with two chains; and inquired who he was, and what he had done.}
\bv{34}{And some shouted one thing, some another, among the crowd: and when he could not know the certainty for the uproar, he commanded him to be brought into the castle.}
\bv{35}{And when he came upon the stairs, so it was that he was borne of the soldiers for the violence of the crowd;}
\bv{36}{for the multitude of the people followed after, crying out, Away with him.}
\bv{37}{And as Paul was about to be brought into the castle, he saith unto the chief captain, May I say something unto thee? And he said, Dost thou know Greek?}
\bv{38}{Art thou not then the Egyptian, who before these days stirred up to sedition and led out into the wilderness the four thousand men of the Assassins?}
\bv{39}{But Paul said, I am a Jew, of Tarsus in Cilicia, a citizen of no mean city: and I beseech thee, give me leave to speak unto the people.}
\bv{40}{And when he had given him leave, Paul, standing on the stairs, beckoned with the hand unto the people; and when there was made a great silence, he spake unto them in the Hebrew language, saying,}
\chaphead{Chapter XXII}
\chapdesc{}
\lettrine[image=true, lines=4, findent=3pt, nindent=0pt]{NT/Acts/Acts-Before.eps}{rethren} and fathers, hear ye the defence which I now make unto you.
\bv{2}{And when they heard that he spake unto them in the Hebrew language, they were the more quiet: and he saith,}
\bv{3}{I am a Jew, born in Tarsus of Cilicia, but brought up in this city, at the feet of Gamaliel, instructed according to the strict manner of the law of our fathers, being zealous for God, even as ye all are this day:}
\bv{4}{and I persecuted this Way unto the death, binding and delivering into prisons both men and women.}
\bv{5}{As also the high priest doth bear me witness, and all the estate of the elders: from whom also I received letters unto the brethren, and journeyed to Damascus to bring them also that were there unto Jerusalem in bonds to be punished.}
\bv{6}{And it came to pass, that, as I made my journey, and drew nigh unto Damascus, about noon, suddenly there shone from heaven a great light round about me.}
\bv{7}{And I fell unto the ground, and heard a voice saying unto me, Saul, Saul, why persecutest thou me?}
\bv{8}{And I answered, Who art thou, Lord? And he said unto me, I am Jesus of Nazareth, whom thou persecutest.}
\bv{9}{And they that were with me beheld indeed the light, but they heard not the voice of him that spake to me.}
\bv{10}{And I said, What shall I do, Lord? And the Lord said unto me, Arise, and go into Damascus; and there it shall be told thee of all things which are appointed for thee to do.}
\bv{11}{And when I could not see for the glory of that light, being led by the hand of them that were with me I came into Damascus.}
\bv{12}{And one Ananias, a devout man according to the law, well reported of by all the Jews that dwelt there,}
\bv{13}{came unto me, and standing by me said unto me, Brother Saul, receive thy sight. And in that very hour I looked up on him.}
\bv{14}{And he said, The God of our fathers hath appointed thee to know his will, and to see the Righteous One, and to hear a voice from his mouth.}
\bv{15}{For thou shalt be a witness for him unto all men of what thou hast seen and heard.}
\bv{16}{And now why tarriest thou? arise, and be baptised, and wash away thy sins, calling on his name.}
\bv{17}{And it came to pass, that, when I had returned to Jerusalem, and while I prayed in the temple, I fell into a trance,}
\bv{18}{and saw him saying unto me, Make haste, and get thee quickly out of Jerusalem; because they will not receive of thee testimony concerning me.}
\bv{19}{And I said, Lord, they themselves know that I imprisoned and beat in every synagogue them that believed on thee:}
\bv{20}{and when the blood of Stephen thy witness was shed, I also was standing by, and consenting, and keeping the garments of them that slew him.}
\bv{21}{And he said unto me, Depart: for I will send thee forth far hence unto the Gentiles.}
\bv{22}{And they gave him audience unto this word; and they lifted up their voice, and said, Away with such a fellow from the earth: for it is not fit that he should live.}
\bv{23}{And as they cried out, and threw off their garments, and cast dust into the air,}
\bv{24}{the chief captain commanded him to be brought into the castle, bidding that he should be examined by scourging, that he might know for what cause they so shouted against him.}
\bv{25}{And when they had tied him up with the thongs, Paul said unto the centurion that stood by, Is it lawful for you to scourge a man that is a Roman, and uncondemned?}
\bv{26}{And when the centurion heard it, he went to the chief captain and told him, saying, What art thou about to do? for this man is a Roman.}
\bv{27}{And the chief captain came and said unto him, Tell me, art thou a Roman? And he said, Yea.}
\bv{28}{And the chief captain answered, With a great sum obtained I this citizenship. And Paul said, But I am \supptext{a Roman} born.}
\bv{29}{They then that were about to examine him straightway departed from him: and the chief captain also was afraid when he knew that he was a Roman, and because he had bound him.}
\bv{30}{But on the morrow, desiring to know the certainty wherefore he was accused of the Jews, he loosed him, and commanded the chief priests and all the council to come together, and brought Paul down and set him before them.}
\chaphead{Chapter XXIII}
\chapdesc{}
\lettrine[image=true, lines=4, findent=3pt, nindent=0pt]{NT/Acts/Acts-And.eps}{nd} Paul, looking stedfastly on the council, said, Brethren, I have lived before God in all good conscience until this day.
\bv{2}{And the high priest Ananias commanded them that stood by him to smite him on the mouth.}
\bv{3}{Then said Paul unto him, God shall smite thee, thou whited wall: and sittest thou to judge me according to the law, and commandest me to be smitten contrary to the law?}
\bv{4}{And they that stood by said, Revilest thou God's high priest?}
\bv{5}{And Paul said, I knew not, brethren, that he was high priest: for it is written, Thou shalt not speak evil of a ruler of thy people.}
\bv{6}{But when Paul perceived that the one part were Sadducees and the other Pharisees, he cried out in the council, Brethren, I am a Pharisee, a son of Pharisees: touching the hope and resurrection of the dead I am called in question.}
\bv{7}{And when he had so said, there arose a dissension between the Pharisees and Sadducees; and the assembly was divided.}
\bv{8}{For the Sadducees say that there is no resurrection, neither angel, nor spirit; but the Pharisees confess both.}
\bv{9}{And there arose a great clamor: and some of the scribes of the Pharisees' part stood up, and strove, saying, We find no evil in this man: and what if a spirit hath spoken to him, or an angel?}
\bv{10}{And when there arose a great dissension, the chief captain, fearing lest Paul should be torn in pieces by them, commanded the soldiers to go down and take him by force from among them, and bring him into the castle.}
\bv{11}{And the night following the Lord stood by him, and said, Be of good cheer: for as thou hast testified concerning me at Jerusalem, so must thou bear witness also at Rome.}
\bv{12}{And when it was day, the Jews banded together, and bound themselves under a curse, saying that they would neither eat nor drink till they had killed Paul.}
\bv{13}{And they were more than forty that made this conspiracy.}
\bv{14}{And they came to the chief priests and the elders, and said, We have bound ourselves under a great curse, to taste nothing until we have killed Paul.}
\bv{15}{Now therefore do ye with the council signify to the chief captain that he bring him down unto you, as though ye would judge of his case more exactly: and we, before he comes near, are ready to slay him.}
\bv{16}{But Paul's sister's son heard of their lying in wait, and he came and entered into the castle and told Paul.}
\bv{17}{And Paul called unto him one of the centurions, and said, Bring this young man unto the chief captain; for he hath something to tell him.}
\bv{18}{So he took him, and brought him to the chief captain, and saith, Paul the prisoner called me unto him, and asked me to bring this young man unto thee, who hath something to say to thee.}
\bv{19}{And the chief captain took him by the hand, and going aside asked him privately, What is it that thou hast to tell me?}
\bv{20}{And he said, The Jews have agreed to ask thee to bring down Paul to-morrow unto the council, as though thou wouldest inquire somewhat more exactly concerning him.}
\bv{21}{Do not thou therefore yield unto them: for there lie in wait for him of them more than forty men, who have bound themselves under a curse, neither to eat nor to drink till they have slain him: and now are they ready, looking for the promise from thee.}
\bv{22}{So the chief captain let the young man go, charging him, Tell no man that thou hast signified these things to me.}
\bv{23}{And he called unto him two of the centurions, and said, Make ready two hundred soldiers to go as far as C{\ae}sarea, and horsemen threescore and ten, and spearmen two hundred, at the third hour of the night:}
\bv{24}{and \supptext{he bade them} provide beasts, that they might set Paul thereon, and bring him safe unto Felix the governor.}
\bv{25}{And he wrote a letter after this form:}
\bv{26}{Claudius Lysias unto the most excellent governor Felix, greeting.}
\bv{27}{This man was seized by the Jews, and was about to be slain of them, when I came upon them with the soldiers and rescued him, having learned that he was a Roman.}
\bv{28}{And desiring to know the cause wherefore they accused him, I brought him down unto their council:}
\bv{29}{whom I found to be accused about questions of their law, but to have nothing laid to his charge worthy of death or of bonds.}
\bv{30}{And when it was shown to me that there would be a plot against the man, I sent him to thee forthwith, charging his accusers also to speak against him before thee.}
\bv{31}{So the soldiers, as it was commanded them, took Paul and brought him by night to Antipatris.}
\bv{32}{But on the morrow they left the horsemen to go with him, and returned to the castle:}
\bv{33}{and they, when they came to C{\ae}sarea and delivered the letter to the governor, presented Paul also before him.}
\bv{34}{And when he had read it, he asked of what province he was; and when he understood that he was of Cilicia,}
\bv{35}{I will hear thee fully, said he, when thine accusers also are come: and he commanded him to be kept in Herod's palace.}
\chaphead{Chapter XXIV}
\chapdesc{}
\lettrine[image=true, lines=4, findent=3pt, nindent=0pt]{NT/Acts/Acts-And.eps}{nd} after five days the high priest Ananias came down with certain elders, and \supptext{with} an orator, one Tertullus; and they informed the governor against Paul.
\bv{2}{And when he was called, Tertullus began to accuse him, saying,
Seeing that by thee we enjoy much peace, and that by thy providence evils are corrected for this nation,}
\bv{3}{we accept it in all ways and in all places, most excellent Felix, with all thankfulness.}
\bv{4}{But, that I be not further tedious unto thee, I entreat thee to hear us of thy clemency a few words.}
\bv{5}{For we have found this man a pestilent fellow, and a mover of insurrections among all the Jews throughout the world, and a ringleader of the sect of the Nazarenes:}
\bv{6}{who moreover assayed to profane the temple: on whom also we laid hold:\mcomm{and we would have judged him according to our law. But the chief captain Lysias came, and with great violence took him away out of our hands, commanding his accusers to come before thee.}}
\bv{8}{from whom thou wilt be able, by examining him thyself, to take knowledge of all these things whereof we accuse him.}
\bv{9}{And the Jews also joined in the charge, affirming that these things were so.}
\bv{10}{And when the governor had beckoned unto him to speak, Paul answered,
Forasmuch as I know that thou hast been of many years a judge unto this nation, I cheerfully make my defence:}
\bv{11}{seeing that thou canst take knowledge that it is not more than twelve days since I went up to worship at Jerusalem:}
\bv{12}{and neither in the temple did they find me disputing with any man or stirring up a crowd, nor in the synagogues, nor in the city.}
\bv{13}{Neither can they prove to thee the things whereof they now accuse me.}
\bv{14}{But this I confess unto thee, that after the Way which they call a sect, so serve I the God of our fathers, believing all things which are according to the law, and which are written in the prophets;}
\bv{15}{having hope toward God, which these also themselves look for, that there shall be a resurrection both of the just and unjust.}
\bv{16}{Herein I also exercise myself to have a conscience void of offence toward God and men always.}
\bv{17}{Now after some years I came to bring alms to my nation, and offerings:}
\bv{18}{amidst which they found me purified in the temple, with no crowd, nor yet with tumult: but \supptext{there were} certain Jews from Asia---}
\bv{19}{who ought to have been here before thee, and to make accusation, if they had aught against me.}
\bv{20}{Or else let these men themselves say what wrong-doing they found when I stood before the council,}
\bv{21}{except it be for this one voice, that I cried standing among them, Touching the resurrection of the dead I am called in question before you this day.}
\bv{22}{But Felix, having more exact knowledge concerning the Way, deferred them, saying, When Lysias the chief captain shall come down, I will determine your matter.}
\bv{23}{And he gave order to the centurion that he should be kept in charge, and should have indulgence; and not to forbid any of his friends to minister unto him.}
\bv{24}{But after certain days, Felix came with Drusilla, his wife, who was a Jewess, and sent for Paul, and heard him concerning the faith in Christ Jesus.}
\bv{25}{And as he reasoned of righteousness, and self-control, and the judgement to come, Felix was terrified, and answered, Go thy way for this time; and when I have a convenient season, I will call thee unto me.}
\bv{26}{He hoped withal that money would be given him of Paul: wherefore also he sent for him the oftener, and communed with him.}
\bv{27}{But when two years were fulfilled, Felix was succeeded by Porcius Festus; and desiring to gain favour with the Jews, Felix left Paul in bonds.}
\chaphead{Chapter XXV}
\chapdesc{}
\lettrine[image=true, lines=4, findent=3pt, nindent=0pt]{NT/Acts/Acts-Festus.eps}{estus} therefore, having come into the province, after three days went up to Jerusalem from C{\ae}sarea.
\bv{2}{And the chief priests and the principal men of the Jews informed him against Paul; and they besought him,}
\bv{3}{asking a favour against him, that he would send for him to Jerusalem; laying a plot to kill him on the way.}
\bv{4}{Howbeit Festus answered, that Paul was kept in charge at C{\ae}sarea, and that he himself was about to depart \supptext{thither} shortly.}
\bv{5}{Let them therefore, saith he, that are of power among you go down with me, and if there is anything amiss in the man, let them accuse him.}
\bv{6}{And when he had tarried among them not more than eight or ten days, he went down unto C{\ae}sarea; and on the morrow he sat on the judgement-seat, and commanded Paul to be brought.}
\bv{7}{And when he was come, the Jews that had come down from Jerusalem stood round about him, bringing against him many and grievous charges which they could not prove;}
\bv{8}{while Paul said in his defence, Neither against the law of the Jews, nor against the temple, nor against C{\ae}sar, have I sinned at all.}
\bv{9}{But Festus, desiring to gain favour with the Jews, answered Paul and said, Wilt thou go up to Jerusalem, and there be judged of these things before me?}
\bv{10}{But Paul said, I am standing before C{\ae}sar's judgement-seat, where I ought to be judged: to the Jews have I done no wrong, as thou also very well knowest.}
\bv{11}{If then I am a wrong-doer, and have committed anything worthy of death, I refuse not to die; but if none of those things is \supptext{true} whereof these accuse me, no man can give me up unto them. I appeal unto C{\ae}sar.}
\bv{12}{Then Festus, when he had conferred with the council, answered, Thou hast appealed unto C{\ae}sar: unto C{\ae}sar shalt thou go.}
\bv{13}{Now when certain days were passed, Agrippa the king and Bernice arrived at C{\ae}sarea, and saluted Festus.}
\bv{14}{And as they tarried there many days, Festus laid Paul's case before the king, saying, There is a certain man left a prisoner by Felix;}
\bv{15}{about whom, when I was at Jerusalem, the chief priests and the elders of the Jews informed \supptext{me}, asking for sentence against him.}
\bv{16}{To whom I answered, that it is not the custom of the Romans to give up any man, before that the accused have the accusers face to face, and have had opportunity to make his defence concerning the matter laid against him.}
\bv{17}{When therefore they were come together here, I made no delay, but on the next day sat on the judgement-seat, and commanded the man to be brought.}
\bv{18}{Concerning whom, when the accusers stood up, they brought no charge of such evil things as I supposed;}
\bv{19}{but had certain questions against him of their own religion, and of one Jesus, who was dead, whom Paul affirmed to be alive.}
\bv{20}{And I, being perplexed how to inquire concerning these things, asked whether he would go to Jerusalem and there be judged of these matters.}
\bv{21}{But when Paul had appealed to be kept for the decision of the emperor, I commanded him to be kept till I should send him to C{\ae}sar.}
\bv{22}{And Agrippa \supptext{said} unto Festus, I also could wish to hear the man myself. To-morrow, saith he, thou shalt hear him.}
\bv{23}{So on the morrow, when Agrippa was come, and Bernice, with great pomp, and they were entered into the place of hearing with the chief captains and the principal men of the city, at the command of Festus Paul was brought in.}
\bv{24}{And Festus saith, King Agrippa, and all men who are here present with us, ye behold this man, about whom all the multitude of the Jews made suit to me, both at Jerusalem and here, crying that he ought not to live any longer.}
\bv{25}{But I found that he had committed nothing worthy of death: and as he himself appealed to the emperor I determined to send him.}
\bv{26}{Of whom I have no certain thing to write unto my lord. Wherefore I have brought him forth before you, and specially before thee, king Agrippa, that, after examination had, I may have somewhat to write.}
\bv{27}{For it seemeth to me unreasonable, in sending a prisoner, not withal to signify the charges against him.}
\chaphead{Chapter XXVI}
\chapdesc{}
\lettrine[image=true, lines=4, findent=3pt, nindent=0pt]{NT/Acts/Acts-And.eps}{nd} Agrippa said unto Paul, Thou art permitted to speak for thyself. Then Paul stretched forth his hand, and made his defence:
\bv{2}{I think myself happy, king Agrippa, that I am to make my defence before thee this day touching all the things whereof I am accused by the Jews:}
\bv{3}{especially because thou art expert in all customs and questions which are among the Jews: wherefore I beseech thee to hear me patiently.}
\bv{4}{My manner of life then from my youth up, which was from the beginning among mine own nation and at Jerusalem, know all the Jews;}
\bv{5}{having knowledge of me from the first, if they be willing to testify, that after the straitest sect of our religion I lived a Pharisee.}
\bv{6}{And now I stand \supptext{here} to be judged for the hope of the promise made of God unto our fathers;}
\bv{7}{unto which \supptext{promise} our twelve tribes, earnestly serving \supptext{God} night and day, hope to attain. And concerning this hope I am accused by the Jews, O king!}
\bv{8}{Why is it judged incredible with you, if God doth raise the dead?}
\bv{9}{I verily thought with myself that I ought to do many things contrary to the name of Jesus of Nazareth.}
\bv{10}{And this I also did in Jerusalem: and I both shut up many of the saints in prisons, having received authority from the chief priests, and when they were put to death I gave my vote against them.}
\bv{11}{And punishing them oftentimes in all the synagogues, I strove to make them blaspheme; and being exceedingly mad against them, I persecuted them even unto foreign cities.}
\bv{12}{Whereupon as I journeyed to Damascus with the authority and commission of the chief priests,}
\bv{13}{at midday, O king, I saw on the way a light from heaven, above the brightness of the sun, shining round about me and them that journeyed with me.}
\bv{14}{And when we were all fallen to the earth, I heard a voice saying unto me in the Hebrew language, Saul, Saul, why persecutest thou me? it is hard for thee to kick against the goad.}
\bv{15}{And I said, Who art thou, Lord? And the Lord said, I am Jesus whom thou persecutest.}
\bv{16}{But arise, and stand upon thy feet: for to this end have I appeared unto thee, to appoint thee a minister and a witness both of the things wherein thou hast seen me, and of the things wherein I will appear unto thee;}
\bv{17}{delivering thee from the people, and from the Gentiles, unto whom I send thee,}
\bv{18}{to open their eyes, that they may turn from darkness to light and from the power of Satan unto God, that they may receive remission of sins and an inheritance among them that are sanctified by faith in me.}
\bv{19}{Wherefore, O king Agrippa, I was not disobedient unto the heavenly vision:}
\bv{20}{but declared both to them of Damascus first, and at Jerusalem, and throughout all the country of Jud{\ae}a, and also to the Gentiles, that they should repent and turn to God, doing works worthy of repentance.}
\bv{21}{For this cause the Jews seized me in the temple, and assayed to kill me.}
\bv{22}{Having therefore obtained the help that is from God, I stand unto this day testifying both to small and great, saying nothing but what the prophets and Moses did say should come;}
\bv{23}{how that the Christ must suffer, \supptext{and} how that he first by the resurrection of the dead should proclaim light both to the people and to the Gentiles.}
\bv{24}{And as he thus made his defence, Festus saith with a loud voice, Paul, thou art mad; thy much learning is turning thee mad.}
\bv{25}{But Paul saith, I am not mad, most excellent Festus; but speak forth words of truth and soberness.}
\bv{26}{For the king knoweth of these things, unto whom also I speak freely: for I am persuaded that none of these things is hidden from him; for this hath not been done in a corner.}
\bv{27}{King Agrippa, believest thou the prophets? I know that thou believest.}
\bv{28}{And Agrippa \supptext{said} unto Paul, With but little persuasion thou wouldest fain make me a Christian.}
\bv{29}{And Paul \supptext{said}, I would to God, that whether with little or with much, not thou only, but also all that hear me this day, might become such as I am, except these bonds.}
\bv{30}{And the king rose up, and the governor, and Bernice, and they that sat with them:}
\bv{31}{and when they had withdrawn, they spake one to another, saying, This man doeth nothing worthy of death or of bonds.}
\bv{32}{And Agrippa said unto Festus, This man might have been set at liberty, if he had not appealed unto C{\ae}sar.}
\chaphead{Chapter XXVII}
\chapdesc{}
\lettrine[image=true, lines=4, findent=3pt, nindent=0pt]{NT/Acts/Acts-And.eps}{nd} when it was determined that we should sail for Italy, they delivered Paul and certain other prisoners to a centurion named Julius, of the Augustan band.
\bv{2}{And embarking in a ship of Adramyttium, which was about to sail unto the places on the coast of Asia, we put to sea, Aristarchus, a Macedonian of Thessalonica, being with us.}
\bv{3}{And the next day we touched at Sidon: and Julius treated Paul kindly, and gave him leave to go unto his friends and refresh himself.}
\bv{4}{And putting to sea from thence, we sailed under the lee of Cyprus, because the winds were contrary.}
\bv{5}{And when we had sailed across the sea which is off Cilicia and Pamphylia, we came to Myra, \supptext{a city} of Lycia.}
\bv{6}{And there the centurion found a ship of Alexandria sailing for Italy; and he put us therein.}
\bv{7}{And when we had sailed slowly many days, and were come with difficulty over against Cnidus, the wind not further suffering us, we sailed under the lee of Crete, over against Salmone;}
\bv{8}{and with difficulty coasting along it we came unto a certain place called Fair Havens; nigh whereunto was the city of Lasea.}
\bv{9}{And when much time was spent, and the voyage was now dangerous, because the Fast was now already gone by, Paul admonished them,}
\bv{10}{and said unto them, Sirs, I perceive that the voyage will be with injury and much loss, not only of the lading and the ship, but also of our lives.}
\bv{11}{But the centurion gave more heed to the master and to the owner of the ship, than to those things which were spoken by Paul.}
\bv{12}{And because the haven was not commodious to winter in, the more part advised to put to sea from thence, if by any means they could reach Phoenix, and winter \supptext{there; which is} a haven of Crete, looking north-east and south-east.}
\bv{13}{And when the south wind blew softly, supposing that they had obtained their purpose, they weighed anchor and sailed along Crete, close in shore.}
\bv{14}{But after no long time there beat down from it a tempestuous wind, which is called Euraquilo:}
\bv{15}{and when the ship was caught, and could not face the wind, we gave way \supptext{to it}, and were driven.}
\bv{16}{And running under the lee of a small island called Cauda, we were able, with difficulty, to secure the boat:}
\bv{17}{and when they had hoisted it up, they used helps, under-girding the ship; and, fearing lest they should be cast upon the Syrtis, they lowered the gear, and so were driven.}
\bv{18}{And as we labored exceedingly with the storm, the next day they began to throw \supptext{the freight} overboard;}
\bv{19}{and the third day they cast out with their own hands the tackling of the ship.}
\bv{20}{And when neither sun nor stars shone upon \supptext{us} for many days, and no small tempest lay on \supptext{us}, all hope that we should be saved was now taken away.}
\bv{21}{And when they had been long without food, then Paul stood forth in the midst of them, and said, Sirs, ye should have hearkened unto me, and not have set sail from Crete, and have gotten this injury and loss.}
\bv{22}{And now I exhort you to be of good cheer; for there shall be no loss of life among you, but \supptext{only} of the ship.}
\bv{23}{For there stood by me this night an angel of the God whose I am, whom also I serve,}
\bv{24}{saying, Fear not, Paul; thou must stand before C{\ae}sar: and lo, God hath granted thee all them that sail with thee.}
\bv{25}{Wherefore, sirs, be of good cheer: for I believe God, that it shall be even so as it hath been spoken unto me.}
\bv{26}{But we must be cast upon a certain island.}
\bv{27}{But when the fourteenth night was come, as we were driven to and fro in the \supptext{sea of} Adria, about midnight the sailors surmised that they were drawing near to some country:}
\bv{28}{and they sounded, and found twenty fathoms; and after a little space, they sounded again, and found fifteen fathoms.}
\bv{29}{And fearing lest haply we should be cast ashore on rocky ground, they let go four anchors from the stern, and wished for the day.}
\bv{30}{And as the sailors were seeking to flee out of the ship, and had lowered the boat into the sea, under color as though they would lay out anchors from the foreship,}
\bv{31}{Paul said to the centurion and to the soldiers, Except these abide in the ship, ye cannot be saved.}
\bv{32}{Then the soldiers cut away the ropes of the boat, and let her fall off.}
\bv{33}{And while the day was coming on, Paul besought them all to take some food, saying, This day is the fourteenth day that ye wait and continue fasting, having taken nothing.}
\bv{34}{Wherefore I beseech you to take some food: for this is for your safety: for there shall not a hair perish from the head of any of you.}
\bv{35}{And when he had said this, and had taken bread, he gave thanks to God in the presence of all; and he brake it, and began to eat.}
\bv{36}{Then were they all of good cheer, and themselves also took food.}
\bv{37}{And we were in all in the ship two hundred threescore and sixteen souls.}
\bv{38}{And when they had eaten enough, they lightened the ship, throwing out the wheat into the sea.}
\bv{39}{And when it was day, they knew not the land: but they perceived a certain bay with a beach, and they took counsel whether they could drive the ship upon it.}
\bv{40}{And casting off the anchors, they left them in the sea, at the same time loosing the bands of the rudders; and hoisting up the foresail to the wind, they made for the beach.}
\bv{41}{But lighting upon a place where two seas met, they ran the vessel aground; and the foreship struck and remained unmoveable, but the stern began to break up by the violence \supptext{of the waves}.}
\bv{42}{And the soldiers' counsel was to kill the prisoners, lest any \supptext{of them} should swim out, and escape.}
\bv{43}{But the centurion, desiring to save Paul, stayed them from their purpose; and commanded that they who could swim should cast themselves overboard, and get first to the land;}
\bv{44}{and the rest, some on planks, and some on \supptext{other} things from the ship. And so it came to pass, that they all escaped safe to the land.}
\chaphead{Chapter XXVIII}
\chapdesc{}
\lettrine[image=true, lines=4, findent=3pt, nindent=0pt]{NT/Acts/Acts-And.eps}{nd} when we were escaped, then we knew that the island was called Melita.
\bv{2}{And the barbarians showed us no common kindness: for they kindled a fire, and received us all, because of the present rain, and because of the cold.}
\bv{3}{But when Paul had gathered a bundle of sticks and laid them on the fire, a viper came out by reason of the heat, and fastened on his hand.}
\bv{4}{And when the barbarians saw the \supptext{venomous} creature hanging from his hand, they said one to another, No doubt this man is a murderer, whom, though he hath escaped from the sea, yet Justice hath not suffered to live.}
\bv{5}{Howbeit he shook off the creature into the fire, and took no harm.}
\bv{6}{But they expected that he would have swollen, or fallen down dead suddenly: but when they were long in expectation and beheld nothing amiss come to him, they changed their minds, and said that he was a god.}
\bv{7}{Now in the neighborhood of that place were lands belonging to the chief man of the island, named Publius; who received us, and entertained us three days courteously.}
\bv{8}{And it was so, that the father of Publius lay sick of fever and dysentery: unto whom Paul entered in, and prayed, and laying his hands on him healed him.}
\bv{9}{And when this was done, the rest also that had diseases in the island came, and were cured:}
\bv{10}{who also honoured us with many honours; and when we sailed, they put on board such things as we needed.}
\bv{11}{And after three months we set sail in a ship of Alexandria which had wintered in the island, whose sign was The Twin Brothers.}
\bv{12}{And touching at Syracuse, we tarried there three days.}
\bv{13}{And from thence we made a circuit, and arrived at Rhegium: and after one day a south wind sprang up, and on the second day we came to Puteoli;}
\bv{14}{where we found brethren, and were entreated to tarry with them seven days: and so we came to Rome.}
\bv{15}{And from thence the brethren, when they heard of us, came to meet us as far as The Market of Appius and The Three Taverns; whom when Paul saw, he thanked God, and took courage.}
\bv{16}{And when we entered into Rome, Paul was suffered to abide by himself with the soldier that guarded him.}
\bv{17}{And it came to pass, that after three days he called together those that were the chief of the Jews: and when they were come together, he said unto them, I, brethren, though I had done nothing against the people, or the customs of our fathers, yet was delivered prisoner from Jerusalem into the hands of the Romans:}
\bv{18}{who, when they had examined me, desired to set me at liberty, because there was no cause of death in me.}
\bv{19}{But when the Jews spake against it, I was constrained to appeal unto C{\ae}sar; not that I had aught whereof to accuse my nation.}
\bv{20}{For this cause therefore did I entreat you to see and to speak with \supptext{me}: for because of the hope of Israel I am bound with this chain.}
\bv{21}{And they said unto him, We neither received letters from Jud{\ae}a concerning thee, nor did any of the brethren come hither and report or speak any harm of thee.}
\bv{22}{But we desire to hear of thee what thou thinkest: for as concerning this sect, it is known to us that everywhere it is spoken against.}
\bv{23}{And when they had appointed him a day, they came to him into his lodging in great number; to whom he expounded \supptext{the matter}, testifying the kingdom of God, and persuading them concerning Jesus, both from the law of Moses and from the prophets, from morning till evening.}
\bv{24}{And some believed the things which were spoken, and some disbelieved.}
\bv{25}{And when they agreed not among themselves, they departed after that Paul had spoken one word, Well spake the Holy Ghost through Isaiah the prophet unto your fathers,}
\bv{26}{saying,}
\otQuote{Is. 6:9-10}{Go thou unto this people, and say,
By hearing ye shall hear, and shall in no wise understand;
And seeing ye shall see, and shall in no wise perceive:
\bv{27}{For this people's heart is waxed gross,
And their eyes they have closed;
Lest haply they should perceive with their eyes,
And hear with their ears,
And understand with their heart,
And should turn again,
And I should heal them.}}
\bv{28}{Be it known therefore unto you, that this salvation of God is sent unto the Gentiles: they will also hear.\mcomm{And when he had said these words, the Jews departed, having much disputing among themselves.}}
\bv{30}{And he abode two whole years in his own hired dwelling, and received all that went in unto him,}
\bv{31}{preaching the kingdom of God, and teaching the things concerning the Lord Jesus Christ with all boldness, none forbidding him.}
	\clearpage
	\chapter{The Epistle of Paul to the Romans}
\fancyhead[RE,LO]{Romans}
\chaphead{Chapter I}
\chapdesc{Introduction}
\lettrine[image=true, lines=4, findent=3pt, nindent=0pt]{NT/Romans/Paul.eps}{aul}, a servant of Jesus Christ, called \supptext{to be} an apostle, separated unto the gospel of God,
\bv{2}{which he promised afore through his prophets in the holy scriptures,}
\bv{3}{concerning his Son, who was born of the seed of David according to the flesh,}
\bv{4}{who was declared \supptext{to be} the Son of God with power, according to the spirit of holiness, by the resurrection from the dead; \supptext{even} Jesus Christ our Lord,}
\bv{5}{through whom we received grace and apostleship, unto obedience of faith among all the nations, for his name's sake;}
\bv{6}{among whom are ye also, called \supptext{to be} Jesus Christ's:}
\bv{7}{to all that are in Rome, beloved of God, called \supptext{to be} saints: Grace to you and peace from God our Father and the Lord Jesus Christ.}
\chapsec{Thanksgiving for Faith}
\bv{8}{First, I thank my God through Jesus Christ for you all, that your faith is proclaimed throughout the whole world.}
\bv{9}{For God is my witness, whom I serve in my spirit in the gospel of his Son, how unceasingly I make mention of you, always in my prayers}
\bv{10}{making request, if by any means now at length I may be prospered by the will of God to come unto you.}
\par
\bv{11}{For I long to see you, that I may impart unto you some spiritual gift, to the end ye may be established;}
\bv{12}{that is, that I with you may be comforted in you, each of us by the other's faith, both yours and mine.}
\bv{13}{And I would not have you ignorant, brethren, that oftentimes I purposed to come unto you (and was hindered hitherto), that I might have some fruit in you also, even as in the rest of the Gentiles.}
\bv{14}{I am debtor both to Greeks and to Barbarians, both to the wise and to the foolish.}
\bv{15}{So, as much as in me is, I am ready to preach the gospel to you also that are in Rome.}
\par
\bv{16}{For I am not ashamed of the gospel: for it is the power of God unto salvation to every one that believeth; to the Jew first, and also to the Greek.}
\bv{17}{For therein is revealed a righteousness of God from faith unto faith: as it is written,}
\otQuote{Hab. 2:4}{But the righteous shall live by faith.}
\chapsec{The Guilt of the World}
\bv{18}{For the wrath of God is revealed from heaven against all ungodliness and unrighteousness of men, who hinder the truth in unrighteousness;}
\chapsec{Natural Law}
\bv{19}{because that which is known of God is manifest in them; for God manifested it unto them.}
\bv{20}{For the invisible things of him since the creation of the world are clearly seen, being perceived through the things that are made, \supptext{even} his everlasting power and divinity; that they may be without excuse:}
\bv{21}{because that, knowing God, they glorified him not as God, neither gave thanks; but became vain in their reasonings, and their senseless heart was darkened.}
\chapsec{Descent into Sin}
\bv{22}{Professing themselves to be wise, they became fools,}
\bv{23}{and changed the glory of the incorruptible God for the likeness of an image of corruptible man, and of birds, and four-footed beasts, and creeping things.}
\chapsec{Consequence of Sin}
\bv{24}{Wherefore God gave them up in the lusts of their hearts unto uncleanness, that their bodies should be dishonoured among themselves:}
\bv{25}{for that they exchanged the truth of God for a lie, and worshipped and served the creature rather than the Creator, who is blessed for ever. Amen.}
\bv{26}{For this cause God gave them up unto vile passions: for their women changed the natural use into that which is against nature:}
\bv{27}{and likewise also the men, leaving the natural use of the woman, burned in their lust one toward another, men with men working unseemliness, and receiving in themselves that recompense of their error which was due.}
\bv{28}{And even as they refused to have God in \supptext{their} knowledge, God gave them up unto a reprobate mind, to do those things which are not fitting;}
\bv{29}{being filled with all unrighteousness, wickedness, covetousness, maliciousness; full of envy, murder, strife, deceit, malignity; whisperers,}
\bv{30}{backbiters, hateful to God, insolent, haughty, boastful, inventors of evil things, disobedient to parents,}
\bv{31}{without understanding, covenant-breakers, without natural affection, unmerciful:}
\bv{32}{who, knowing the ordinance of God, that they that practise such things are worthy of death, not only do the same, but also consent with them that practise them.}
\chaphead{Chapter II}
\chapdesc{Gentiles Can Have No Excuse}
\lettrine[image=true, lines=4, findent=3pt, nindent=0pt]{NT/Romans/Wherefore.eps}{herefore} thou art without excuse, O man, whosoever thou art that judgest: for wherein thou judgest another, thou condemnest thyself; for thou that judgest dost practise the same things.
\bv{2}{And we know that the judgement of God is according to truth against them that practise such things.}
\bv{3}{And reckonest thou this, O man, who judgest them that practise such things, and doest the same, that thou shalt escape the judgement of God?}
\bv{4}{Or despisest thou the riches of his goodness and forbearance and longsuffering, not knowing that the goodness of God leadeth thee to repentance?}
\bv{5}{but after thy hardness and impenitent heart treasurest up for thyself wrath in the day of wrath and revelation of the righteous judgement of God;}
\bv{6}{who will render to every man according to his works:}
\bv{7}{to them that by patience in well-doing seek for glory and honour and incorruption, eternal life:}
\bv{8}{but unto them that are factious, and obey not the truth, but obey unrighteousness, \supptext{shall be} wrath and indignation,}
\bv{9}{tribulation and anguish, upon every soul of man that worketh evil, of the Jew first, and also of the Greek;}
\bv{10}{but glory and honour and peace to every man that worketh good, to the Jew first, and also to the Greek:}
\bv{11}{for there is no respect of persons with God.}
\chapsec{Natural Law Condemns the Gentiles}
\bv{12}{For as many as have sinned without the law shall also perish without the law: and as many as have sinned under the law shall be judged by the law;}
\bv{13}{for not the hearers of the law are just before God, but the doers of the law shall be justified;}
\bv{14}{(for when Gentiles that have not the law do by nature the things of the law, these, not having the law, are the law unto themselves;}
\bv{15}{in that they show the work of the law written in their hearts, their conscience bearing witness therewith, and their thoughts one with another accusing or else excusing \supptext{them});\mref{Wisdom}}
\bv{16}{in the day when God shall judge the secrets of men, according to my gospel, by Jesus Christ.}
\chapsec{Mosaic Law Condemns the Jews}
\bv{17}{But if thou bearest the name of a Jew, and restest upon the law, and gloriest in God,}
\bv{18}{and knowest his will, and approvest the things that are excellent, being instructed out of the law,}
\bv{19}{and art confident that thou thyself art a guide of the blind, a light of them that are in darkness,}
\bv{20}{a corrector of the foolish, a teacher of babes, having in the law the form of knowledge and of the truth;}
\bv{21}{thou therefore that teachest another, teachest thou not thyself? thou that preachest a man should not steal, dost thou steal?}
\bv{22}{thou that sayest a man should not commit adultery, dost thou commit adultery? thou that abhorrest idols, dost thou rob temples?}
\bv{23}{thou who gloriest in the law, through thy transgression of the law dishonourest thou God?}
\chapsec{Guilt of the Jews' Sin}
\bv{24}{For the name of God is blasphemed among the Gentiles because of you, even as it is written.}
\bv{25}{For circumcision indeed profiteth, if thou be a doer of the law: but if thou be a transgressor of the law, thy circumcision is become uncircumcision.}
\bv{26}{If therefore the uncircumcision keep the ordinances of the law, shall not his uncircumcision be reckoned for circumcision?}
\bv{27}{and shall not the uncircumcision which is by nature, if it fulfil the law, judge thee, who with the letter and circumcision art a transgressor of the law?}
\bv{28}{For he is not a Jew who is one outwardly; neither is that circumcision which is outward in the flesh:}
\bv{29}{but he is a Jew who is one inwardly; and circumcision is that of the heart, in the spirit not in the letter; whose praise is not of men, but of God.}
\chaphead{Chapter III}
\chapdesc{The Great Guilt of the Jews}
\lettrine[image=true, lines=4, findent=3pt, nindent=0pt]{NT/Romans/What.eps}{hat} advantage then hath the Jew? or what is the profit of circumcision?
\bv{2}{Much every way: first of all, that they were intrusted with the oracles of God.}
\bv{3}{For what if some were without faith? shall their want of faith make of none effect the faithfulness of God?}
\bv{4}{God forbid: yea, let God be found true, but every man a liar; as it is written,}
\otQuote{Ps. 51:4}{That thou mightest be justified in thy words,
And mightest prevail when thou comest into judgement.}
\bv{5}{But if our unrighteousness commendeth the righteousness of God, what shall we say? Is God unrighteous who visiteth with wrath? (I speak after the manner of men.)}
\bv{6}{God forbid: for then how shall God judge the world?}
\bv{7}{But if the truth of God through my lie abounded unto his glory, why am I also still judged as a sinner?}
\bv{8}{and why not (as we are slanderously reported, and as some affirm that we say), Let us do evil, that good may come? whose condemnation is just.}
\chapsec{Law Condemns Jew \& Gentile}
\bv{9}{What then? are we better than they? No, in no wise: for we before laid to the charge both of Jews and Greeks, that they are all under sin;}
\bv{10}{as it is written,}
\otQuote{Ps. 14:1-3; 53:1-3}{There is none righteous, no, not one;
\bv{11}{There is none that understandeth,
There is none that seeketh after God;}
\bv{12}{They have all turned aside, they are together become unprofitable;
There is none that doeth good, no, not so much as one:}}
\otQuote{Ps. 5:9}{\bv{13}{Their throat is an open sepulchre;
With their tongues they have used deceit:
The poison of asps is under their lips:}}
\otQuote{Ps. 10:7}{\bv{14}{Whose mouth is full of cursing and bitterness:}}
\otQuote{Is. 59:7-8}{\bv{15}{Their feet are swift to shed blood;}
\bv{16}{Destruction and misery are in their ways;}
\bv{17}{And the way of peace have they not known:}}
\otQuote{Ps. 36:1}{\bv{18}{There is no fear of God before their eyes.}}
\bv{19}{Now we know that what things soever the law saith, it speaketh to them that are under the law; that every mouth may be stopped, and all the world may be brought under the judgement of God:}
\bv{20}{because by the works of the law shall no flesh be justified in his sight; for through the law \supptext{cometh} the knowledge of sin.}
\chapsec{Justification in Christ apart from Law}
\bv{21}{But now apart from the law a righteousness of God hath been manifested, being witnessed by the law and the prophets;\mcomm{``\supptext{We} are not justified through ourselves or through our own wisdom or understanding or piety or works which we wrought in holiness of heart, but through faith.'' -St. Clement}}
\bv{22}{even the righteousness of God through faith in Jesus Christ unto all them that believe; for there is no distinction;}
\bv{23}{for all have sinned, and fall short of the glory of God;}
\bv{24}{being justified freely by his grace through the redemption that is in Christ Jesus:}
\bv{25}{whom God set forth \supptext{to be} a propitiation, through faith, in his blood, to show his righteousness because of the passing over of the sins done aforetime, in the forbearance of God;}
\bv{26}{for the showing, \supptext{I say}, of his righteousness at this present season: that he might himself be just, and the justifier of him that hath faith in Jesus.}
\par
\bv{27}{Where then is the glorying? It is excluded. By what manner of law? of works? Nay: but by a law of faith.}
\bv{28}{We reckon therefore that a man is justified by faith apart from the works of the law.}
\chapsec{Justification for Those Condemned by Law}
\bv{29}{Or is God \supptext{the God} of Jews only? is he not \supptext{the God} of Gentiles also? Yea, of Gentiles also:}
\bv{30}{if so be that God is one, and he shall justify the circumcision by faith, and the uncircumcision through faith.}
\chapsec{Justification by Faith Honours Law}
\bv{31}{Do we then make the law of none effect through faith? God forbid: nay, we establish the law.}
\chaphead{Chapter IV}
\chapdesc{Justification by Faith Alone Illustrated}
\lettrine[image=true, lines=4, findent=3pt, nindent=0pt]{NT/Romans/What.eps}{hat} then shall we say that Abraham, our forefather, hath found according to the flesh?
\bv{2}{For if Abraham was justified by works, he hath whereof to glory; but not toward God.}
\bv{3}{For what saith the scripture? And Abraham believed God, and it was reckoned unto him for righteousness.}
\bv{4}{Now to him that worketh, the reward is not reckoned as of grace, but as of debt.\mcomm{``What could man...do of himself to recover the righteousness which he had once lost? Therefore another’s righteousness was ascribed to him who lacked his own.'' -St. Bernard}}
\chapsec{Justifying Faith Defined}
\bv{5}{But to him that worketh not, but believeth on him that justifieth the ungodly, his faith is reckoned for righteousness.}
\bv{6}{Even as David also pronounceth blessing upon the man, unto whom God reckoneth righteousness apart from works,}
\bv{7}{\supptext{saying},}
\otQuote{Ps. 32:1-2}{Blessed are they whose iniquities are forgiven,
And whose sins are covered.
\bv{8}{Blessed is the man to whom the Lord will not reckon sin.}}
\chapsec{Both Jews \& Gentiles are Justified}
\bv{9}{Is this blessing then pronounced upon the circumcision, or upon the uncircumcision also? for we say, To Abraham his faith was reckoned for righteousness.}
\bv{10}{How then was it reckoned? when he was in circumcision, or in uncircumcision? Not in circumcision, but in uncircumcision:}
\bv{11}{and he received the sign of circumcision, a seal of the righteousness of the faith which he had while he was in uncircumcision: that he might be the father of all them that believe, though they be in uncircumcision, that righteousness might be reckoned unto them;}
\bv{12}{and the father of circumcision to them who not only are of the circumcision, but who also walk in the steps of that faith of our father Abraham which he had in uncircumcision.}
\chapsec{Justification apart from Law}
\bv{13}{For not through the law was the promise to Abraham or to his seed that he should be heir of the world, but through the righteousness of faith.}
\bv{14}{For if they that are of the law are heirs, faith is made void, and the promise is made of none effect:}
\bv{15}{for the law worketh wrath; but where there is no law, neither is there transgression.}
\bv{16}{For this cause \supptext{it is} of faith, that \supptext{it may be} according to grace; to the end that the promise may be sure to all the seed; not to that only which is of the law, but to that also which is of the faith of Abraham, who is the father of us all}
\bv{17}{(as it is written, \shortQ{Gen. 17:5}{A father of many nations have I made thee}) before him whom he believed, \supptext{even} God, who giveth life to the dead, and calleth the things that are not, as though they were.}
\bv{18}{Who in hope believed against hope, to the end that he might become a father of many nations, according to that which had been spoken, So shall thy seed be.}
\bv{19}{And without being weakened in faith he considered his own body now as good as dead (he being about a hundred years old), and the deadness of Sarah's womb;}
\bv{20}{yet, looking unto the promise of God, he wavered not through unbelief, but waxed strong through faith, giving glory to God,}
\bv{21}{and being fully assured that what he had promised, he was able also to perform.}
\bv{22}{Wherefore also it was reckoned unto him for righteousness.}
\bv{23}{Now it was not written for his sake alone, that it was reckoned unto him;}
\bv{24}{but for our sake also, unto whom it shall be reckoned, who believe on him that raised Jesus our Lord from the dead,}
\bv{25}{who was delivered up for our trespasses, and was raised for our justification.}
\chaphead{Chapter V}
\chapdesc{Peace with God through Justification}
\lettrine[image=true, lines=4, findent=3pt, nindent=0pt]{NT/Romans/Being.eps}{eing} therefore justified by faith, we have peace with God through our Lord Jesus Christ;
\bv{2}{through whom also we have had our access by faith into this grace wherein we stand; and we rejoice in hope of the glory of God.}
\bv{3}{And not only so, but we also rejoice in our tribulations: knowing that tribulation worketh stedfastness;}
\bv{4}{and stedfastness, approvedness; and approvedness, hope:}
\bv{5}{and hope putteth not to shame; because the love of God hath been shed abroad in our hearts through the Holy Spirit which was given unto us.}
\bv{6}{For while we were yet weak, in due season Christ died for the ungodly.}
\bv{7}{For scarcely for a righteous man will one die: for peradventure for the good man some one would even dare to die.}
\bv{8}{But God commendeth his own love toward us, in that, while we were yet sinners, Christ died for us.}
\bv{9}{Much more then, being now justified by his blood, shall we be saved from the wrath \supptext{of God} through him.}
\bv{10}{For if, while we were enemies, we were reconciled to God through the death of his Son, much more, being reconciled, shall we be saved by his life;}
\bv{11}{and not only so, but we also rejoice in God through our Lord Jesus Christ, through whom we have now received the reconciliation.}
\chapsec{Sin from Adam}
\bv{12}{Therefore, as through one man sin entered into the world, and death through sin; and so death passed unto all men, for that all sinned:—}
\bv{13}{for until the law sin was in the world; but sin is not imputed when there is no law.}
\bv{14}{Nevertheless death reigned from Adam until Moses, even over them that had not sinned after the likeness of Adam's transgression, who is a figure of him that was to come.}
\chapsec{Grace from Jesus Christ}
\bv{15}{But not as the trespass, so also \supptext{is} the free gift. For if by the trespass of the one the many died, much more did the grace of God, and the gift by the grace of the one man, Jesus Christ, abound unto the many.}
\bv{16}{And not as through one that sinned, \supptext{so} is the gift: for the judgement \supptext{came} of many trespasses unto justification.}
\par
\bv{17}{For if, by the trespass of the one, death reigned through the one; much more shall they that receive the abundance of grace and of the gift of righteousness reign in life through the one, \supptext{even} Jesus Christ.}
\bv{18}{So then as through one trespass \supptext{the judgement came} unto all men to condemnation; even so through one act of righteousness \supptext{the free gift came} unto all men to justification of life.}
\bv{19}{For as through the one man's disobedience the many were made sinners, even so through the obedience of the one shall the many be made righteous.}
\par
\bv{20}{And the law came in besides, that the trespass might abound; but where sin abounded, grace did abound more exceedingly:}
\bv{21}{that, as sin reigned in death, even so might grace reign through righteousness unto eternal life through Jesus Christ our Lord.}
\chaphead{Chapter VI}
\chapdesc{Union with Christ Frees from Sin}
\lettrine[image=true, lines=4, findent=3pt, nindent=0pt]{NT/Romans/What.eps}{hat} shall we say then? Shall we continue in sin, that grace may abound?
\bv{2}{God forbid. We who died to sin, how shall we any longer live therein?}
\bv{3}{Or are ye ignorant that all we who were baptised into Christ Jesus were baptised into his death?}
\bv{4}{We were buried therefore with him through baptism into death: that like as Christ was raised from the dead through the glory of the Father, so we also might walk in newness of life.}
\par
\bv{5}{For if we have become united with \supptext{him} in the likeness of his death, we shall be also \supptext{in the likeness} of his resurrection;}
\bv{6}{knowing this, that our old man was crucified with \supptext{him}, that the body of sin might be done away, that so we should no longer be in bondage to sin;}
\bv{7}{for he that hath died is justified from sin.}
\bv{8}{But if we died with Christ, we believe that we shall also live with him;}
\bv{9}{knowing that Christ being raised from the dead dieth no more; death no more hath dominion over him.}
\bv{10}{For the death that he died, he died unto sin once: but the life that he liveth, he liveth unto God.}
\chapsec{Sanctification Follows Justification}
\bv{11}{Even so reckon ye also yourselves to be dead unto sin, but alive unto God in Christ Jesus.}
\bv{12}{Let not sin therefore reign in your mortal body, that ye should obey the lusts thereof:}
\bv{13}{neither present your members unto sin \supptext{as} instruments of unrighteousness; but present yourselves unto God, as alive from the dead, and your members \supptext{as} instruments of righteousness unto God.}
\chapsec{Law Condemns not the Justified}
\bv{14}{For sin shall not have dominion over you: for ye are not under law, but under grace.}
\bv{15}{What then? shall we sin, because we are not under law, but under grace? God forbid.}
\bv{16}{Know ye not, that to whom ye present yourselves \supptext{as} servants unto obedience, his servants ye are whom ye obey; whether of sin unto death, or of obedience unto righteousness?}
\bv{17}{But thanks be to God, that, whereas ye were servants of sin, ye became obedient from the heart to that form of teaching whereunto ye were delivered;}
\bv{18}{and being made free from sin, ye became servants of righteousness.}
\bv{19}{I speak after the manner of men because of the infirmity of your flesh: for as ye presented your members \supptext{as} servants to uncleanness and to iniquity unto iniquity, even so now present your members \supptext{as} servants to righteousness unto sanctification.}
\bv{20}{For when ye were servants of sin, ye were free in regard of righteousness.}
\bv{21}{What fruit then had ye at that time in the things whereof ye are now ashamed? for the end of those things is death.}
\bv{22}{But now being made free from sin and become servants to God, ye have your fruit unto sanctification, and the end eternal life.}
\bv{23}{For the wages of sin is death; but the free gift of God is eternal life in Christ Jesus our Lord.}
\chaphead{Chapter VII}
\chapdesc{Sin Remains in the Justified}
\lettrine[image=true, lines=4, findent=3pt, nindent=0pt]{NT/Romans/Or.eps}{r} are ye ignorant, brethren (for I speak to men who know the law), that the law hath dominion over a man for so long time as he liveth?
\bv{2}{For the woman that hath a husband is bound by law to the husband while he liveth; but if the husband die, she is discharged from the law of the husband.}
\bv{3}{So then if, while the husband liveth, she be joined to another man, she shall be called an adulteress: but if the husband die, she is free from the law, so that she is no adulteress, though she be joined to another man.}
\bv{4}{Wherefore, my brethren, ye also were made dead to the law through the body of Christ; that ye should be joined to another, \supptext{even} to him who was raised from the dead, that we might bring forth fruit unto God.}
\bv{5}{For when we were in the flesh, the sinful passions, which were through the law, wrought in our members to bring forth fruit unto death.}
\bv{6}{But now we have been discharged from the law, having died to that wherein we were held; so that we serve in newness of the spirit, and not in oldness of the letter.}
\chapsec{Law Does Not Sanctify}
\bv{7}{What shall we say then? Is the law sin? God forbid. Howbeit, I had not known sin, except through the law: for I had not known coveting, except the law had said, ``Thou shalt not covet:''}
\bv{8}{but sin, finding occasion, wrought in me through the commandment all manner of coveting: for apart from the law sin \supptext{is} dead.}
\bv{9}{And I was alive apart from the law once: but when the commandment came, sin revived, and I died;}
\bv{10}{and the commandment, which \supptext{was} unto life, this I found \supptext{to be} unto death:}
\bv{11}{for sin, finding occasion, through the commandment beguiled me, and through it slew me.}
\bv{12}{So that the law is holy, and the commandment holy, and righteous, and good.}
\bv{13}{Did then that which is good become death unto me? God forbid. But sin, that it might be shown to be sin, by working death to me through that which is good;---that through the commandment sin might become exceeding sinful.}
\chapsec{Man's Disordered Desires}
\bv{14}{For we know that the law is spiritual: but I am carnal, sold under sin.}
\bv{15}{For that which I do I know not: for not what I would, that do I practise; but what I hate, that I do.}
\bv{16}{But if what I would not, that I do, I consent unto the law that it is good.}
\bv{17}{So now it is no more I that do it, but sin which dwelleth in me.}
\bv{18}{For I know that in me, that is, in my flesh, dwelleth no good thing: for to will is present with me, but to do that which is good \supptext{is} not.}
\bv{19}{For the good which I would I do not: but the evil which I would not, that I practise.}
\chapsec{Concupiscence is Sin}
\bv{20}{But if what I would not, that I do, it is no more I that do it, but sin which dwelleth in me.}
\bv{21}{I find then the law, that, to me who would do good, evil is present.}
\bv{22}{For I delight in the law of God after the inward man:}
\bv{23}{but I see a different law in my members, warring against the law of my mind, and bringing me into captivity under the law of sin which is in my members.}
\bv{24}{Wretched man that I am! who shall deliver me out of the body of this death?}
\chapsec{The Just Man is Acquitted of his Sin}
\bv{25}{I thank God through Jesus Christ our Lord. So then I of myself with the mind, indeed, serve the law of God; but with the flesh the law of sin.}
\chaphead{Chapter VIII}
\lettrine[image=true, lines=4, findent=3pt, nindent=0pt]{NT/Romans/There.eps}{here} is therefore now no condemnation to them that are in Christ Jesus.
\chapsec{Faith Conquers Law}
\bv{2}{For the law of the Spirit of life in Christ Jesus made me free from the law of sin and of death.}
\bv{3}{For what the law could not do, in that it was weak through the flesh, God, sending his own Son in the likeness of sinful flesh and for sin, condemned sin in the flesh:}
\bv{4}{that the ordinance of the law might be fulfilled in us, who walk not after the flesh, but after the Spirit.}
\chapsec{Conflict of Spirit \& Flesh}
\bv{5}{For they that are after the flesh mind the things of the flesh; but they that are after the Spirit the things of the Spirit.}
\bv{6}{For the mind of the flesh is death; but the mind of the Spirit is life and peace:}
\bv{7}{because the mind of the flesh is enmity against God; for it is not subject to the law of God, neither indeed can it be:}
\bv{8}{and they that are in the flesh cannot please God.}
\bv{9}{But ye are not in the flesh but in the Spirit, if so be that the Spirit of God dwelleth in you. But if any man hath not the Spirit of Christ, he is none of his.}
\bv{10}{And if Christ is in you, the body is dead because of sin; but the spirit is life because of righteousness.}
\bv{11}{But if the Spirit of him that raised up Jesus from the dead dwelleth in you, he that raised up Christ Jesus from the dead shall give life also to your mortal bodies through his Spirit that dwelleth in you.}
\bv{12}{So then, brethren, we are debtors, not to the flesh, to live after the flesh:}
\bv{13}{for if ye live after the flesh, ye must die; but if by the Spirit ye put to death the deeds of the body, ye shall live.}
\chapsec{Born of the Spirit}
\bv{14}{For as many as are led by the Spirit of God, these are sons of God.}
\bv{15}{For ye received not the spirit of bondage again unto fear; but ye received the spirit of adoption, whereby we cry, Abba, Father.}
\bv{16}{The Spirit himself beareth witness with our spirit, that we are children of God:}
\bv{17}{and if children, then heirs; heirs of God, and joint-heirs with Christ; if so be that we suffer with \supptext{him}, that we may be also glorified with \supptext{him}.}
\chapsec{Final Deliverance}
\bv{18}{For I reckon that the sufferings of this present time are not worthy to be compared with the glory which shall be revealed to us-ward.}
\bv{19}{For the earnest expectation of the creation waiteth for the revealing of the sons of God.}
\bv{20}{For the creation was subjected to vanity, not of its own will, but by reason of him who subjected it, in hope}
\bv{21}{that the creation itself also shall be delivered from the bondage of corruption into the liberty of the glory of the children of God.}
\bv{22}{For we know that the whole creation groaneth and travaileth in pain together until now.}
\bv{23}{And not only so, but ourselves also, who have the first-fruits of the Spirit, even we ourselves groan within ourselves, waiting for \supptext{our} adoption, \supptext{to wit}, the redemption of our body.}
\bv{24}{For in hope were we saved: but hope that is seen is not hope: for who hopeth for that which he seeth?}
\bv{25}{But if we hope for that which we see not, \supptext{then} do we with patience wait for it.}
\chapsec{The Spirit Intercedes}
\bv{26}{And in like manner the Spirit also helpeth our infirmity: for we know not how to pray as we ought; but the Spirit himself maketh intercession for \supptext{us} with groanings which cannot be uttered;}
\bv{27}{and he that searcheth the hearts knoweth what is the mind of the Spirit, because he maketh intercession for the saints according to \supptext{the will of} God.}
\chapsec{God's Purpose in the Gospel}
\bv{28}{And we know that to them that love God all things work together for good, \supptext{even} to them that are called according to \supptext{his} purpose.}
\bv{29}{For whom he foreknew, he also foreordained \supptext{to be} conformed to the image of his Son, that he might be the firstborn among many brethren:}
\bv{30}{and whom he foreordained, them he also called: and whom he called, them he also justified: and whom he justified, them he also glorified.}
\bv{31}{What then shall we say to these things? If God \supptext{is} for us, who \supptext{is} against us?}
\bv{32}{He that spared not his own Son, but delivered him up for us all, how shall he not also with him freely give us all things?}
\bv{33}{Who shall lay anything to the charge of God's elect? It is God that justifieth;}
\bv{34}{who is he that condemneth? It is Christ Jesus that died, yea rather, that was raised from the dead, who is at the right hand of God, who also maketh intercession for us.}
\chapsec{Blessed Assurance}
\bv{35}{Who shall separate us from the love of Christ? shall tribulation, or anguish, or persecution, or famine, or nakedness, or peril, or sword?}
\bv{36}{Even as it is written,}
\otQuote{Ps. 44:22}{For thy sake we are killed all the day long;
We were accounted as sheep for the slaughter.}
\bv{37}{Nay, in all these things we are more than conquerors through him that loved us.}
\bv{38}{For I am persuaded, that neither death, nor life, nor angels, nor principalities, nor things present, nor things to come, nor powers,}
\bv{39}{nor height, nor depth, nor any other creature, shall be able to separate us from the love of God, which is in Christ Jesus our Lord.}
\chaphead{Chapter IX}
\chapdesc{Apostolic Solicitude for Israel}
\lettrine[image=true, lines=4, findent=3pt, nindent=0pt]{NT/Romans/I.eps}{} say the truth in Christ, I lie not, my conscience bearing witness with me in the Holy Spirit,
\bv{2}{that I have great sorrow and unceasing pain in my heart.}
\bv{3}{For I could wish that I myself were anathema from Christ for my brethren's sake, my kinsmen according to the flesh:}
\par
\bv{4}{who are Israelites; whose is the adoption, and the glory, and the covenants, and the giving of the law, and the service \supptext{of God}, and the promises;}
\bv{5}{whose are the fathers, and of whom is Christ as concerning the flesh, who is over all, God blessed for ever. Amen.}
\chapsec{True vs. False Israel}
\bv{6}{But \supptext{it is} not as though the word of God hath come to nought. For they are not all Israel, that are of Israel:}
\bv{7}{neither, because they are Abraham's seed, are they all children: but, \otQuote{Gen. 21:12}{In Isaac shall thy seed be called.}}
\par
\bv{8}{That is, it is not the children of the flesh that are children of God; but the children of the promise are reckoned for a seed.}
\bv{9}{For this is a word of promise, \otQuote{Gen. 18:10, 14}{According to this season will I come, and Sarah shall have a son.}}
\bv{10}{And not only so; but Rebecca also having conceived by one, \supptext{even} by our father Isaac---}
\bv{11}{for \supptext{the children} being not yet born, neither having done anything good or bad, that the purpose of God according to election might stand, not of works, but of him that calleth,}
\bv{12}{it was said unto her, \otQuote{Gen. 25:23}{The elder shall serve the younger.}}
\bv{13}{Even as it is written, \otQuote{Mal. 1:2-3}{Jacob I loved, but Esau I hated.}}
\chapsec{Righteousness of God}
\bv{14}{What shall we say then? Is there unrighteousness with God? God forbid.}
\bv{15}{For he saith to Moses, \otQuote{Ex. 33:19}{I will have mercy on whom I have mercy, and I will have compassion on whom I have compassion.}}
\bv{16}{So then it is not of him that willeth, nor of him that runneth, but of God that hath mercy.}
\bv{17}{For the scripture saith unto Pharaoh, \otQuote{Ex. 9:16}{For this very purpose did I raise thee up, that I might show in thee my power, and that my name might be published abroad in all the earth.}}
\bv{18}{So then he hath mercy on whom he will, and whom he will he hardeneth.}
\chapsec{God's Sovereign Election}
\bv{19}{Thou wilt say then unto me, ``Why doth he still find fault? For who withstandeth his will?''\mcomm{ENGLISH FATHER}}
\bv{20}{Nay but, O man, who art thou that repliest against God? Shall the thing formed say to him that formed it, ``Why didst thou make me thus?''}
\bv{21}{Or hath not the potter a right over the clay, from the same lump to make one part a vessel unto honour, and another unto dishonour?}
\bv{22}{What if God, willing to show his wrath, and to make his power known, endured with much longsuffering vessels of wrath fitted unto destruction:}
\bv{23}{and that he might make known the riches of his glory upon vessels of mercy, which he afore prepared unto glory,}
\bv{24}{\supptext{even} us, whom he also called, not from the Jews only, but also from the Gentiles?}
\par
\bv{25}{As he saith also in Hosea,}
\otQuote{Hos. 2:23}{I will call that my people, which was not my people;
And her beloved, that was not beloved.}
\otQuote{Hos. 1:10}{\bv{26}{And it shall be,} \supptext{that} in the place where it was said unto them, ``Ye are not my people,'' There shall they be called sons of the living God.}
\bv{27}{And Isaiah crieth concerning Israel, \otQuote{Is. 10:22-23}{If the number of the children of Israel be as the sand of the sea, it is the remnant that shall be saved:}
\bv{28}{for the Lord will execute \supptext{his} word upon the earth, finishing it and cutting it short.}}
\bv{29}{And, as Isaiah hath said before,}
\otQuote{Is. 1:9}{Except the Lord of Sabaoth had left us a seed,
We had become as Sodom, and had been made like unto Gomorrah.}
\par
\bv{30}{What shall we say then? That the Gentiles, who followed not after righteousness, attained to righteousness, even the righteousness which is of faith:}
\bv{31}{but Israel, following after a law of righteousness, did not arrive at \supptext{that} law.}
\bv{32}{Wherefore? Because \supptext{they sought it} not by faith, but as it were by works. They stumbled at the stone of stumbling;}
\bv{33}{even as it is written,}
\otQuote{Is. 28:16}{Behold, I lay in Zion a stone of stumbling and a rock of offence:
And he that believeth on him shall not be put to shame.}
\chaphead{Chapter X}
\chapdesc{Israel's Faithlessness}
\lettrine[image=true, lines=4, findent=3pt, nindent=0pt]{NT/Romans/Brethren.eps}{rethren}, my heart's desire and my supplication to God is for them, that they may be saved.
\bv{2}{For I bear them witness that they have a zeal for God, but not according to knowledge.}
\bv{3}{For being ignorant of God's righteousness, and seeking to establish their own, they did not subject themselves to the righteousness of God.}
\bv{4}{For Christ is the end of the law unto righteousness to every one that believeth.}
\bv{5}{For Moses writeth\mref{cf. Lev. 18:5} that the man that doeth the righteousness which is of the law shall live thereby.}
\bv{6}{But the righteousness which is of faith saith thus, ``Say not in thy heart, `Who shall ascend into heaven?'{''} (that is, to bring Christ down:)}
\bv{7}{``or, `Who shall descend into the abyss?'{''} (that is, to bring Christ up from the dead.)}
\chapsec{The Saving Word}
\bv{8}{But what saith it? ``The word is nigh thee, in thy mouth, and in thy heart:'' that is, the word of faith, which we preach:}
\bv{9}{because if thou shalt confess with thy mouth Jesus \supptext{as} Lord, and shalt believe in thy heart that God raised him from the dead, thou shalt be saved:}
\bv{10}{for with the heart man believeth unto righteousness; and with the mouth confession is made unto salvation.}
\bv{11}{For the scripture saith, \otQuote{Is. 28:16}{Whosoever believeth on him shall not be put to shame.}}
\bv{12}{For there is no distinction between Jew and Greek: for the same \supptext{Lord} is Lord of all, and is rich unto all that call upon him:}
\bv{13}{for, \otQuote{Jl. 2:32}{Whosoever shall call upon the name of the Lord shall be saved.}}
\bv{14}{How then shall they call on him in whom they have not believed? and how shall they believe in him whom they have not heard? and how shall they hear without a preacher?}
\bv{15}{and how shall they preach, except they be sent? even as it is written, How beautiful are the feet of them that bring glad tidings of good things!}
\bv{16}{But they did not all hearken to the glad tidings. For Isaiah saith, \otQuote{Is. 53:1}{Lord, who hath believed our report?}}
\bv{17}{So belief \supptext{cometh} of hearing, and hearing by the word of Christ.}
\bv{18}{But I say, Did they not hear? Yea, verily,}
\otQuote{Ps. 19:4}{Their sound went out into all the earth,
And their words unto the ends of the world.}
\bv{19}{But I say, Did Israel not know? First Moses saith,}
\otQuote{Deut. 32:21}{I will provoke you to jealousy with that which is no nation,
With a nation void of understanding will I anger you.}
\bv{20}{And Isaiah is very bold, and saith,}
\otQuote{Is. 65:1}{I was found of them that sought me not;
I became manifest unto them that asked not of me.}
\bv{21}{But as to Israel he saith, \otQuote{Is. 65:2}{All the day long did I spread out my hands unto a disobedient and gainsaying people.}}
\chaphead{Chapter XI}
\chapdesc{Faithful Israel is Saved}
\lettrine[image=true, lines=4, findent=3pt, nindent=0pt]{NT/Romans/I.eps}{} say then, Did God cast off his people? God forbid. For I also am an Israelite, of the seed of Abraham, of the tribe of Benjamin.
\bv{2}{God did not cast off his people which he foreknew. Or know ye not what the scripture saith of Elijah? how he pleadeth with God against Israel:}
\otQuote{1 Kgs. 19:10,14}{\bv{3}{Lord, they have killed thy prophets, they have digged down thine altars; and I am left alone, and they seek my life.}}
\bv{4}{But what saith the answer of God unto him? \otQuote{1 Kgs. 19:18}{I have left for myself seven thousand men, who have not bowed the knee to Baal.}}
\bv{5}{Even so then at this present time also there is a remnant according to the election of grace.}
\bv{6}{But if it is by grace, it is no more of works: otherwise grace is no more grace.}
\chapsec{Faithless Israel is Blind}
\bv{7}{What then? That which Israel seeketh for, that he obtained not; but the election obtained it, and the rest were hardened:}
\bv{8}{according as it is written,} \otQuote{Is. 29:10}{God gave them a spirit of stupor, eyes that they should not see, and ears that they should not hear, unto this very day.}
\bv{9}{And David saith,}
\otQuote{Ps. 69:22-23}{Let their table be made a snare, and a trap, And a stumblingblock, and a recompense unto them: \bv{10}{Let their eyes be darkened, that they may not see, And bow thou down their back always.}}
\bv{11}{I say then, Did they stumble that they might fall? God forbid: but by their fall salvation \supptext{is come} unto the Gentiles, to provoke them to jealousy.}
\bv{12}{Now if their fall is the riches of the world, and their loss the riches of the Gentiles; how much more their fulness?}
\chapsec{Warning to the Gentiles}
\bv{13}{But I speak to you that are Gentiles. Inasmuch then as I am an apostle of Gentiles, I glorify my ministry;}
\bv{14}{if by any means I may provoke to jealousy \supptext{them that are} my flesh, and may save some of them.}
\bv{15}{For if the casting away of them \supptext{is} the reconciling of the world, what \supptext{shall} the receiving \supptext{of them be}, but life from the dead?}
\par
\bv{16}{And if the firstfruit is holy, so is the lump: and if the root is holy, so are the branches.}
\bv{17}{But if some of the branches were broken off, and thou, being a wild olive, wast grafted in among them, and didst become partaker with them of the root of the fatness of the olive tree;}
\bv{18}{glory not over the branches: but if thou gloriest, it is not thou that bearest the root, but the root thee.}
\bv{19}{Thou wilt say then, ``Branches were broken off, that I might be grafted in.''}
\bv{20}{Well; by their unbelief they were broken off, and thou standest by thy faith. Be not highminded, but fear:}
\bv{21}{for if God spared not the natural branches, neither will he spare thee.}
\chapsec{The Just Can Fall Away}
\bv{22}{Behold then the goodness and severity of God: toward them that fell, severity; but toward thee, God's goodness, if thou continue in his goodness: otherwise thou also shalt be cut off.}
\bv{23}{And they also, if they continue not in their unbelief, shall be grafted in: for God is able to graft them in again.}
\bv{24}{For if thou wast cut out of that which is by nature a wild olive tree, and wast grafted contrary to nature into a good olive tree; how much more shall these, which are the natural \supptext{branches}, be grafted into their own olive tree?}
\chapsec{Israel's Salvation}
\bv{25}{For I would not, brethren, have you ignorant of this mystery, lest ye be wise in your own conceits, that a hardening in part hath befallen Israel, until the fulness of the Gentiles be come in;}
\bv{26}{and so all Israel shall be saved: even as it is written,}
\otQuote{Is. 59:20-21}{There shall come out of Zion the Deliverer; He shall turn away ungodliness from Jacob: \bv{27}{And this is my covenant unto them, When I shall take away their sins.}}
\bv{28}{As touching the gospel, they are enemies for your sake: but as touching the election, they are beloved for the fathers' sake.}
\bv{29}{For the gifts and the calling of God are not repented of.}
\bv{30}{For as ye in time past were disobedient to God, but now have obtained mercy by their disobedience,}
\bv{31}{even so have these also now been disobedient, that by the mercy shown to you they also may now obtain mercy.}
\bv{32}{For God hath shut up all unto disobedience, that he might have mercy upon all.}
\bv{33}{O the depth of the riches both of the wisdom and the knowledge of God! how unsearchable are his judgements, and his ways past tracing out!}
\bv{34}{For who hath known the mind of the Lord? or who hath been his counsellor?}
\bv{35}{or who hath first given to him, and it shall be recompensed unto him again?}
\bv{36}{For of him, and through him, and unto him, are all things. To him \supptext{be} the glory for ever. Amen.}
\chaphead{Chapter XII}
\chapdesc{Unity of the Church}
\lettrine[image=true, lines=4, findent=3pt, nindent=0pt]{NT/Romans/I.eps}{} beseech you therefore, brethren, by the mercies of God, to present your bodies a living sacrifice, holy, acceptable to God, \supptext{which is} your spiritual service.
\bv{2}{And be not fashioned according to this world: but be ye transformed by the renewing of your mind, that ye may prove what is the good and acceptable and perfect will of God.}
\bv{3}{For I say, through the grace that was given me, to every man that is among you, not to think of himself more highly than he ought to think; but so to think as to think soberly, according as God hath dealt to each man a measure of faith.}
\par
\bv{4}{For even as we have many members in one body, and all the members have not the same office:}
\bv{5}{so we, who are many, are one body in Christ, and severally members one of another.}
\bv{6}{And having gifts differing according to the grace that was given to us, whether prophecy, \supptext{let us prophesy} according to the proportion of our faith;}
\bv{7}{or ministry, \supptext{let us give ourselves} to our ministry; or he that teacheth, to his teaching;}
\bv{8}{or he that exhorteth, to his exhorting: he that giveth, \supptext{let him do it} with liberality; he that ruleth, with diligence; he that showeth mercy, with cheerfulness.}
\chapsec{Perfect Love}
\bv{9}{Let love be without hypocrisy. Abhor that which is evil; cleave to that which is good.}
\bv{10}{In love of the brethren be tenderly affectioned one to another; in honour preferring one another;}
\bv{11}{in diligence not slothful; fervent in spirit; serving the Lord;}
\bv{12}{rejoicing in hope; patient in tribulation; continuing stedfastly in prayer;}
\bv{13}{communicating to the necessities of the saints; given to hospitality.}
\bv{14}{Bless them that persecute you; bless, and curse not.}
\bv{15}{Rejoice with them that rejoice; weep with them that weep.}
\bv{16}{Be of the same mind one toward another. Set not your mind on high things, but condescend to things that are lowly. Be not wise in your own conceits.}
\par
\bv{17}{Render to no man evil for evil. Take thought for things honourable in the sight of all men.}
\bv{18}{If it be possible, as much as in you lieth, be at peace with all men.}
\bv{19}{Avenge not yourselves, beloved, but give place unto the wrath \supptext{of God}: for it is written, Vengeance belongeth unto me; I will recompense, saith the Lord.}
\bv{20}{But if thine enemy hunger, feed him; if he thirst, give him to drink: for in so doing thou shalt heap coals of fire upon his head.}
\bv{21}{Be not overcome of evil, but overcome evil with good.}
\chaphead{Chapter XIII}
\chapdesc{Holy Obedience}
\lettrine[image=true, lines=4, findent=3pt, nindent=0pt]{NT/Romans/Let.eps}{et} every soul be in subjection to the higher powers: for there is no power but of God; and the \supptext{powers} that be are ordained of God.
\bv{2}{Therefore he that resisteth the power, withstandeth the ordinance of God: and they that withstand shall receive to themselves judgement.}
\chapsec{Authority of the King}
\bv{3}{For rulers are not a terror to the good work, but to the evil. And wouldest thou have no fear of the power? do that which is good, and thou shalt have praise from the same:}
\bv{4}{for he is a minister of God to thee for good. But if thou do that which is evil, be afraid; for he beareth not the sword in vain: for he is a minister of God, an avenger for wrath to him that doeth evil.}
\bv{5}{Wherefore \supptext{ye} must needs be in subjection, not only because of the wrath, but also for conscience' sake.}
\bv{6}{For for this cause ye pay tribute also; for they are ministers of God's service, attending continually upon this very thing.}
\bv{7}{Render to all their dues: tribute to whom tribute \supptext{is due}; custom to whom custom; fear to whom fear; honour to whom honour.}
\chapsec{Love of Neighbour}
\bv{8}{Owe no man anything, save to love one another: for he that loveth his neighbour hath fulfilled the law.}
\bv{9}{For this, Thou shalt not commit adultery, Thou shalt not kill, Thou shalt not steal, Thou shalt not covet, and if there be any other commandment, it is summed up in this word, namely, Thou shalt love thy neighbour as thyself.}
\bv{10}{Love worketh no ill to his neighbour: love therefore is the fulfilment of the law.}
\bv{11}{And this, knowing the season, that already it is time for you to awake out of sleep: for now is salvation nearer to us than when we \supptext{first} believed.}
\par
\bv{12}{The night is far spent, and the day is at hand: let us therefore cast off the works of darkness, and let us put on the armor of light.}
\bv{13}{Let us walk becomingly, as in the day; not in revelling and drunkenness, not in chambering and wantonness, not in strife and jealousy.}
\bv{14}{But put ye on the Lord Jesus Christ, and make not provision for the flesh, to \supptext{fulfil} the lusts \supptext{thereof}.}
\chaphead{Chapter XIV}
\chapdesc{Bearing with Doubt}
\lettrine[image=true, lines=4, findent=3pt, nindent=0pt]{NT/Romans/But.eps}{ut} him that is weak in faith receive ye, \supptext{yet} not for decision of scruples.
\bv{2}{One man hath faith to eat all things: but he that is weak eateth herbs.}
\bv{3}{Let not him that eateth set at nought him that eateth not; and let not him that eateth not judge him that eateth: for God hath received him.}
\bv{4}{Who art thou that judgest the servant of another? to his own lord he standeth or falleth. Yea, he shall be made to stand; for the Lord hath power to make him stand.}
\chapsec{Christian Liberty}
\bv{5}{One man esteemeth one day above another: another esteemeth every day \supptext{alike}. Let each man be fully assured in his own mind.}
\bv{6}{He that regardeth the day, regardeth it unto the Lord: and he that eateth, eateth unto the Lord, for he giveth God thanks; and he that eateth not, unto the Lord he eateth not, and giveth God thanks.}
\bv{7}{For none of us liveth to himself, and none dieth to himself.}
\bv{8}{For whether we live, we live unto the Lord; or whether we die, we die unto the Lord: whether we live therefore, or die, we are the Lord's.}
\bv{9}{For to this end Christ died and lived \supptext{again}, that he might be Lord of both the dead and the living.}
\bv{10}{But thou, why dost thou judge thy brother? or thou again, why dost thou set at nought thy brother? for we shall all stand before the judgement-seat of God.}
\bv{11}{For it is written,}
\otQuote{Is. 45:23}{As I live, saith the Lord, to me every knee shall bow,
And every tongue shall confess to God.}
\bv{12}{So then each one of us shall give account of himself to God.}
\chapsec{Christian Judgement}
\bv{13}{Let us not therefore judge one another any more: but judge ye this rather, that no man put a stumblingblock in his brother's way, or an occasion of falling.}
\bv{14}{I know, and am persuaded in the Lord Jesus, that nothing is unclean of itself: save that to him who accounteth anything to be unclean, to him it is unclean.}
\bv{15}{For if because of meat thy brother is grieved, thou walkest no longer in love. Destroy not with thy meat him for whom Christ died.}
\bv{16}{Let not then your good be evil spoken of:}
\bv{17}{for the kingdom of God is not eating and drinking, but righteousness and peace and joy in the Holy Spirit.}
\bv{18}{For he that herein serveth Christ is well-pleasing to God, and approved of men.}
\bv{19}{So then let us follow after things which make for peace, and things whereby we may edify one another.}
\chapsec{Importance of Conscience}
\bv{20}{Overthrow not for meat's sake the work of God. All things indeed are clean; howbeit it is evil for that man who eateth with offence.}
\bv{21}{It is good not to eat flesh, nor to drink wine, nor \supptext{to do anything} whereby thy brother stumbleth.}
\bv{22}{The faith which thou hast, have thou to thyself before God. Happy is he that judgeth not himself in that which he approveth.}
\bv{23}{But he that doubteth is condemned if he eat, because \supptext{he eateth} not of faith; and whatsoever is not of faith is sin.}
\chaphead{Chapter XV}
\chapdesc{Burden of the Strong}
\lettrine[image=true, lines=4, findent=3pt, nindent=0pt]{NT/Romans/Now.eps}{ow} we that are strong ought to bear the infirmities of the weak, and not to please ourselves.
\bv{2}{Let each one of us please his neighbour for that which is good, unto edifying.}
\bv{3}{For Christ also pleased not himself; but, as it is written, \shortQ{Ps. 69:9}{The reproaches of them that reproached thee fell upon me.}}
\chapsec{Christian Unity}
\bv{4}{For whatsoever things were written aforetime were written for our learning, that through patience and through comfort of the scriptures we might have hope.}
\bv{5}{Now the God of patience and of comfort grant you to be of the same mind one with another according to Christ Jesus:}
\bv{6}{that with one accord ye may with one mouth glorify the God and Father of our Lord Jesus Christ.}
\bv{7}{Wherefore receive ye one another, even as Christ also received you, to the glory of God.}
\bv{8}{For I say that Christ hath been made a minister of the circumcision for the truth of God, that he might confirm the promises \supptext{given} unto the fathers,}
\bv{9}{and that the Gentiles might glorify God for his mercy; as it is written,}
\otQuote{Ps. 18:49}{Therefore will I give praise unto thee among the Gentiles,
And sing unto thy name.}
\bv{10}{And again he saith,}
\otQuote{Deut. 32:43}{Rejoice, ye Gentiles, with his people.}
\bv{11}{And again,}
\otQuote{Ps. 117:1}{Praise the Lord, all ye Gentiles; And let all the peoples praise him.}
\bv{12}{And again, Isaiah saith,}
\otQuote{Is. 11:10}{There shall be the root of Jesse, And he that ariseth to rule over the Gentiles; On him shall the Gentiles hope.}
\bv{13}{Now the God of hope fill you with all joy and peace in believing, that ye may abound in hope, in the power of the Holy Spirit.}
\chapsec{St. Paul's Ministry}
\bv{14}{And I myself also am persuaded of you, my brethren, that ye yourselves are full of goodness, filled with all knowledge, able also to admonish one another.}
\bv{15}{But I write the more boldly unto you in some measure, as putting you again in remembrance, because of the grace that was given me of God,}
\bv{16}{that I should be a minister of Christ Jesus unto the Gentiles, ministering the gospel of God, that the offering up of the Gentiles might be made acceptable, being sanctified by the Holy Spirit.}
\chapsec{St. Paul's Fruit}
\bv{17}{I have therefore my glorying in Christ Jesus in things pertaining to God.}
\bv{18}{For I will not dare to speak of any things save those which Christ wrought through me, for the obedience of the Gentiles, by word and deed,}
\bv{19}{in the power of signs and wonders, in the power of the Holy Spirit; so that from Jerusalem, and round about even unto Illyricum, I have fully preached the gospel of Christ;}
\bv{20}{yea, making it my aim so to preach the gospel, not where Christ was \supptext{already} named, that I might not build upon another man's foundation;}
\bv{21}{but, as it is written,}
\otQuote{Is. 52:15}{They shall see, to whom no tidings of him came, And they who have not heard shall understand.}
\bv{22}{Wherefore also I was hindered these many times from coming to you:}
\bv{23}{but now, having no more any place in these regions, and having these many years a longing to come unto you,}
\bv{24}{whensoever I go unto Spain (for I hope to see you in my journey, and to be brought on my way thitherward by you, if first in some measure I shall have been satisfied with your company)—}
\bv{25}{but now, \supptext{I say}, I go unto Jerusalem, ministering unto the saints.}
\bv{26}{For it hath been the good pleasure of Macedonia and Achaia to make a certain contribution for the poor among the saints that are at Jerusalem.}
\bv{27}{Yea, it hath been their good pleasure; and their debtors they are. For if the Gentiles have been made partakers of their spiritual things, they owe it \supptext{to them} also to minister unto them in carnal things.}
\bv{28}{When therefore I have accomplished this, and have sealed to them this fruit, I will go on by you unto Spain.}
\bv{29}{And I know that, when I come unto you, I shall come in the fulness of the blessing of Christ.}
\bv{30}{Now I beseech you, brethren, by our Lord Jesus Christ, and by the love of the Spirit, that ye strive together with me in your prayers to God for me;}
\bv{31}{that I may be delivered from them that are disobedient in Jud{\ae}a, and \supptext{that} my ministration which \supptext{I have} for Jerusalem may be acceptable to the saints;}
\bv{32}{that I may come unto you in joy through the will of God, and together with you find rest.}
\bv{33}{Now the God of peace be with you all. Amen.}
\chaphead{Chapter XVI}
\chapdesc{Salutations}
\lettrine[image=true, lines=4, findent=3pt, nindent=0pt]{NT/Romans/I.eps}{} commend unto you Phoebe our sister, who is a servant of the church that is at Cenchre{\ae}:
\bv{2}{that ye receive her in the Lord, worthily of the saints, and that ye assist her in whatsoever matter she may have need of you: for she herself also hath been a helper of many, and of mine own self.}
\bv{3}{Salute Prisca and Aquila my fellow-workers in Christ Jesus,}
\bv{4}{who for my life laid down their own necks; unto whom not only I give thanks, but also all the churches of the Gentiles:}
\bv{5}{and \supptext{salute} the church that is in their house. Salute Ep{\ae}netus my beloved, who is the firstfruits of Asia unto Christ.}
\par
\bv{6}{Salute Mary, who bestowed much labor on you.}
\bv{7}{Salute Andronicus and Junias, my kinsmen, and my fellow-prisoners, who are of note among the apostles, who also have been in Christ before me.}
\bv{8}{Salute Ampliatus my beloved in the Lord.}
\bv{9}{Salute Urbanus our fellow-worker in Christ, and Stachys my beloved.}
\bv{10}{Salute Apelles the approved in Christ. Salute them that are of the \supptext{household} of Aristobulus.}
\bv{11}{Salute Herodion my kinsman. Salute them of the \supptext{household} of Narcissus, that are in the Lord.}
\bv{12}{Salute Tryph{\ae}na and Tryphosa, who labor in the Lord. Salute Persis the beloved, who labored much in the Lord.}
\bv{13}{Salute Rufus the chosen in the Lord, and his mother and mine.}
\bv{14}{Salute Asyncritus, Phlegon, Hermes, Patrobas, Hermas, and the brethren that are with them.}
\bv{15}{Salute Philologus and Julia, Nereus and his sister, and Olympas, and all the saints that are with them.}
\par
\bv{16}{Salute one another with a holy kiss. All the churches of Christ salute you.}
\bv{17}{Now I beseech you, brethren, mark them that are causing the divisions and occasions of stumbling, contrary to the doctrine which ye learned: and turn away from them.}
\bv{18}{For they that are such serve not our Lord Christ, but their own belly; and by their smooth and fair speech they beguile the hearts of the innocent.}
\bv{19}{For your obedience is come abroad unto all men. I rejoice therefore over you: but I would have you wise unto that which is good, and simple unto that which is evil.}
\bv{20}{And the God of peace shall bruise Satan under your feet shortly.
The grace of our Lord Jesus Christ be with you.}
\par
\bv{21}{Timothy my fellow-worker saluteth you; and Lucius and Jason and Sosipater, my kinsmen.}
\bv{22}{I Tertius, who write the epistle, salute you in the Lord.}
\bv{23}{Gaius my host, and of the whole church, saluteth you. Erastus the treasurer of the city saluteth you, and Quartus the brother.\mcomm{The grace of our Lord Jesus Christ be with you all.}}
\bv{25}{Now to him that is able to establish you according to my gospel and the preaching of Jesus Christ, according to the revelation of the mystery which hath been kept in silence through times eternal,}
\bv{26}{but now is manifested, and by the scriptures of the prophets, according to the commandment of the eternal God, is made known unto all the nations unto obedience of faith:}
\bv{27}{to the only wise God, through Jesus Christ, to whom be the glory for ever. Amen.}
	\clearpage
	\chapter{The Epistle of St. Paul to Philemon}
\fancyhead[RE,LO]{Philemon}
\chaphead{Chapter I}
\chapdesc{Apostolic Greeting}
\lettrine[image=true, lines=4, findent=3pt, nindent=0pt]{NT/Philemon/Phil-Paul.eps}{aul}, a prisoner of Christ Jesus, and Timothy our brother, to Philemon our beloved and fellow-worker,
\bv{2}{and to Apphia our sister, and to Archippus our fellow-soldier, and to the church in thy house:}
\bv{3}{Grace to you and peace from God our Father and the Lord Jesus Christ.}
\chapsec{Character of Philemon}
\bv{4}{I thank my God always, making mention of thee in my prayers,}
\bv{5}{hearing of thy love, and of the faith which thou hast toward the Lord Jesus, and toward all the saints;}
\bv{6}{that the fellowship of thy faith may become effectual, in the knowledge of every good thing which is in you, unto Christ.}
\bv{7}{For I had much joy and comfort in thy love, because the hearts of the saints have been refreshed through thee, brother.}
\chapsec{Intercession for Onesimus}
\bv{8}{Wherefore, though I have all boldness in Christ to enjoin thee that which is befitting,}
\bv{9}{yet for love’s sake I rather beseech, being such a one as Paul the aged, and now a prisoner also of Christ Jesus:}
\bv{10}{I beseech thee for my child, whom I have begotten in my bonds, Onesimus,}
\bv{11}{who once was unprofitable to thee, but now is profitable to thee and to me:}
\bv{12}{whom I have sent back to thee in his own person, that is, my very heart:}
\bv{13}{whom I would fain have kept with me, that in thy behalf he might minister unto me in the bonds of the gospel:}
\bv{14}{but without thy mind I would do nothing; that thy goodness should not be as of necessity, but of free will.}
\par
\bv{15}{For perhaps he was therefore parted \supptext{from thee} for a season, that thou shouldest have him for ever;}
\bv{16}{no longer as a servant, but more than a servant, a brother beloved, specially to me, but how much rather to thee, both in the flesh and in the Lord.}
\bv{17}{If then thou countest me a partner, receive him as myself.}
\bv{18}{But if he hath wronged thee at all, or oweth \supptext{thee} aught, put that to mine account;}
\bv{19}{I Paul write it with mine own hand, I will repay it: that I say not unto thee that thou owest to me even thine own self besides.}
\bv{20}{Yea, brother, let me have joy of thee in the Lord: refresh my heart in Christ.}
\bv{21}{Having confidence in thine obedience I write unto thee, knowing that thou wilt do even beyond what I say.}
\chapsec{Salutations \& Conclusion}
\bv{22}{But withal prepare me also a lodging: for I hope that through your prayers I shall be granted unto you.}
\bv{23}{Epaphras, my fellow-prisoner in Christ Jesus, saluteth thee;}
\bv{24}{\supptext{and so do} Mark, Aristarchus, Demas, Luke, my fellow-workers.}
\bv{25}{The grace of our Lord Jesus Christ be with your spirit. Amen.}
	\clearpage
	\chapter{The First Epistle of Saint John}
\fancyhead[RE,LO]{The First Epistle of John}
\chaphead{Chapter I}
\chapdesc{Fellowship in the Incarnation}
\lettrine[image=true, lines=4, findent=3pt, nindent=0pt]{T-1Jn.eps}{hat} which was from the beginning, that which we have heard, that which we have seen with our eyes, that which we beheld, and our hands handled, concerning the Word of life
\bv{2}{(and the life was manifested, and we have seen, and bear witness, and declare unto you the life, the eternal \supptext{life}, which was with the Father, and was manifested unto us);}
\chapsec{Fellowship with the Father \& Son}
\bv{3}{that which we have seen and heard declare we unto you also, that ye also may have fellowship with us: yea, and our fellowship is with the Father, and with his Son Jesus Christ:}
\bv{4}{and these things we write, that our joy may be made full.}
\chapsec{Walk in the Light}
\bv{5}{And this is the message which we have heard from him and announce unto you, that God is light, and in him is no darkness at all.}
\bv{6}{If we say that we have fellowship with him and walk in the darkness, we lie, and do not the truth:}
\bv{7}{but if we walk in the light, as he is in the light, we have fellowship one with another, and the blood of Jesus his Son cleanseth us from all sin.}
\chapsec{Indwelling Sin Remains}
\bv{8}{If we say that we have no sin, we deceive ourselves, and the truth is not in us.}
\chapsec{Sins Confessed are Forgiven}
\bv{9}{If we confess our sins, he is faithful and righteous to forgive us our sins, and to cleanse us from all unrighteousness.}
\bv{10}{If we say that we have not sinned, we make him a liar, and his word is not in us.}
\chaphead{Chapter II}
\chapdesc{Fellowship Maintained by Christ}
\lettrine[image=true, lines=4, findent=3pt, nindent=0pt]{M-1Jn.eps}{y} little children, these things write I unto you that ye may not sin. And if any man sin, we have an Advocate with the Father, Jesus Christ the righteous:
\bv{2}{and he is the propitiation for our sins; and not for ours only, but also for the whole world.}
\chapsec{The Tests of Fellowship}
\bv{3}{And hereby we know that we know him, if we keep his commandments.}
\bv{4}{He that saith, I know him, and keepeth not his commandments, is a liar, and the truth is not in him;}
\bv{5}{but whoso keepeth his word, in him verily hath the love of God been perfected. Hereby we know that we are in him:}
\bv{6}{he that saith he abideth in him ought himself also to walk even as he walked.}
\bv{7}{Beloved, no new commandment write I unto you, but an old commandment which ye had from the beginning: the old commandment is the word which ye heard.}
\bv{8}{Again, a new commandment write I unto you, which thing is true in him and in you; because the darkness is passing away, and the true light already shineth.}
\bv{9}{He that saith he is in the light and hateth his brother, is in the darkness even until now.}
\bv{10}{He that loveth his brother abideth in the light, and there is no occasion of stumbling in him.}
\bv{11}{But he that hateth his brother is in the darkness, and walketh in the darkness, and knoweth not whither he goeth, because the darkness hath blinded his eyes.}
\bv{12}{I write unto you, \supptext{my} little children, because your sins are forgiven you for his name’s sake.}
\bv{13}{I write unto you, fathers, because ye know him who is from the beginning. I write unto you, young men, because ye have overcome the evil one. I have written unto you, little children, because ye know the Father.}
\bv{14}{I have written unto you, fathers, because ye know him who is from the beginning. I have written unto you, young men, because ye are strong, and the word of God abideth in you, and ye have overcome the evil one.}
\chapsec{The Children Must Not Love the Present World}
\bv{15}{Love not the world, neither the things that are in the world. If any man love the world, the love of the Father is not in him.}
\bv{16}{For all that is in the world, the lust of the flesh and the lust of the eyes and the vainglory of life, is not of the Father, but is of the world.}
\bv{17}{And the world passeth away, and the lust thereof: but he that doeth the will of God abideth for ever.}
\chapsec{The Children Warned against Apostates}
\bv{18}{Little children, it is the last hour: and as ye heard that antichrist cometh, even now have there arisen many antichrists; whereby we know that it is the last hour.}
\bv{19}{They went out from us, but they were not of us; for if they had been of us, they would have continued with us: but \supptext{they went out}, that they might be made manifest that they all are not of us.}
\bv{20}{And ye have an anointing from the Holy One, and ye know all things.}
\bv{21}{I have not written unto you because ye know not the truth, but because ye know it, and because no lie is of the truth.}
\bv{22}{Who is the liar but he that denieth that Jesus is the Christ? This is the antichrist, \supptext{even} he that denieth the Father and the Son.}
\bv{23}{Whosoever denieth the Son, the same hath not the Father: he that confesseth the Son hath the Father also.}
\bv{24}{As for you, let that abide in you which ye heard from the beginning. If that which ye heard from the beginning abide in you, ye also shall abide in the Son, and in the Father.}
\bv{25}{And this is the promise which he promised us, \supptext{even} the life eternal.}
\chapsec{The Children's Anointing}
\bv{26}{These things have I written unto you concerning them that would lead you astray.}
\bv{27}{And as for you, the anointing which ye received of him abideth in you, and ye need not that any one teach you; but as his anointing teacheth you concerning all things, and is true, and is no lie, and even as it taught you, ye abide in him.}
\bv{28}{And now, \supptext{my} little children, abide in him; that, if he shall be manifested, we may have boldness, and not be ashamed before him at his coming.}
\bv{29}{If ye know that he is righteous, ye know that every one also that doeth righteousness is begotten of him.}
\chaphead{Chapter III}
\chapdesc{The Love of the Father}
\lettrine[image=true, lines=4, findent=3pt, nindent=0pt]{B-1Jn.eps}{ehold} what manner of love the Father hath bestowed upon us, that we should be called children of God; and \supptext{such} we are. For this cause the world knoweth us not, because it knew him not.
\bv{2}{Beloved, now are we children of God, and it is not yet made manifest what we shall be. We know that, if he shall be manifested, we shall be like him; for we shall see him even as he is.}
\bv{3}{And every one that hath this hope \supptext{set} on him purifieth himself, even as he is pure.}
\par
\bv{4}{Every one that doeth sin doeth also lawlessness; and sin is lawlessness.}
\bv{5}{And ye know that he was manifested to take away sins; and in him is no sin.}
\bv{6}{Whosoever abideth in him sinneth not: whosoever sinneth hath not seen him, neither knoweth him.}
\bv{7}{\supptext{My} little children, let no man lead you astray: he that doeth righteousness is righteous, even as he is righteous:}
\bv{8}{he that doeth sin is of the devil; for the devil sinneth from the beginning. To this end was the Son of God manifested, that he might destroy the works of the devil.}
\bv{9}{Whosoever is begotten of God doeth no sin, because his seed abideth in him: and he cannot sin, because he is begotten of God.}
\bv{10}{In this the children of God are manifest, and the children of the devil: whosoever doeth not righteousness is not of God, neither he that loveth not his brother.}
\chapsec{Little Children Must Live Together}
\bv{11}{For this is the message which ye heard from the beginning, that we should love one another:}
\bv{12}{not as Cain was of the evil one, and slew his brother. And wherefore slew he him? Because his works were evil, and his brother’s righteous.}
\bv{13}{Marvel not, brethren, if the world hateth you.}
\bv{14}{We know that we have passed out of death into life, because we love the brethren. He that loveth not abideth in death.}
\bv{15}{Whosoever hateth his brother is a murderer: and ye know that no murderer hath eternal life abiding in him.}
\bv{16}{Hereby know we love, because he laid down his life for us: and we ought to lay down our lives for the brethren.}
\bv{17}{But whoso hath the world’s goods, and beholdeth his brother in need, and shutteth up his compassion from him, how doth the love of God abide in him?}
\bv{18}{\supptext{My} little children, let us not love in word, neither with the tongue; but in deed and truth.}
\bv{19}{Hereby shall we know that we are of the truth, and shall assure our heart before him:}
\bv{20}{because if our heart condemn us, God is greater than our heart, and knoweth all things.}
\bv{21}{Beloved, if our heart condemn us not, we have boldness toward God;}
\bv{22}{and whatsoever we ask we receive of him, because we keep his commandments and do the things that are pleasing in his sight.}
\bv{23}{And this is his commandment, that we should believe in the name of his Son Jesus Christ, and love one another, even as he gave us commandment.}
\bv{24}{And he that keepeth his commandments abideth in him, and he in him. And hereby we know that he abideth in us, by the Spirit which he gave us.}
\chaphead{Chapter IV}
\chapdesc{The Children Warned against False Teachers}
\lettrine[image=true, lines=4, findent=3pt, nindent=0pt]{B2-1Jn.eps}{eloved}, believe not every spirit, but prove the spirits, whether they are of God; because many false prophets are gone out into the world.
\chapsec{Marks of True Teachers}
\bv{2}{Hereby know ye the Spirit of God: every spirit that confesseth that Jesus Christ is come in the flesh is of God:}
\bv{3}{and every spirit that confesseth not Jesus is not of God: and this is the \supptext{spirit} of the antichrist, whereof ye have heard that it cometh; and now it is in the world already.}
\bv{4}{Ye are of God, \supptext{my} little children, and have overcome them: because greater is he that is in you than he that is in the world.}
\chapsec{Marks of False Teachers}
\bv{5}{They are of the world: therefore speak they \supptext{as} of the world, and the world heareth them.}
\bv{6}{We are of God: he that knoweth God heareth us; he who is not of God heareth us not. By this we know the spirit of truth, and the spirit of error.}
\chapsec{Love of God \& Man}
\bv{7}{Beloved, let us love one another: for love is of God; and every one that loveth is begotten of God, and knoweth God.}
\bv{8}{He that loveth not knoweth not God; for God is love.}
\bv{9}{Herein was the love of God manifested in us, that God hath sent his only begotten Son into the world that we might live through him.}
\bv{10}{Herein is love, not that we loved God, but that he loved us, and sent his Son \supptext{to be} the propitiation for our sins.}
\par
\bv{11}{Beloved, if God so loved us, we also ought to love one another.}
\bv{12}{No man hath beheld God at any time: if we love one another, God abideth in us, and his love is perfected in us:}
\bv{13}{hereby we know that we abide in him and he in us, because he hath given us of his Spirit.}
\bv{14}{And we have beheld and bear witness that the Father hath sent the Son \supptext{to be} the Saviour of the world.}
\bv{15}{Whosoever shall confess that Jesus is the Son of God, God abideth in him, and he in God.}
\bv{16}{And we know and have believed the love which God hath in us. God is love; and he that abideth in love abideth in God, and God abideth in him.}
\bv{17}{Herein is love made perfect with us, that we may have boldness in the day of judgement; because as he is, even so are we in this world.}
\bv{18}{There is no fear in love: but perfect love casteth out fear, because fear hath punishment; and he that feareth is not made perfect in love.}
\bv{19}{We love, because he first loved us.}
\bv{20}{If a man say, I love God, and hateth his brother, he is a liar: for he that loveth not his brother whom he hath seen, cannot love God whom he hath not seen.}
\bv{21}{And this commandment have we from him, that he who loveth God love his brother also.}
\chaphead{Chapter V}
\chapdesc{Faith Overcomes}
\lettrine[image=true, lines=4, findent=3pt, nindent=0pt]{W-1Jn.eps}{hosoever} believeth that Jesus is the Christ is begotten of God: and whosoever loveth him that begat loveth him also that is begotten of him.
\bv{2}{Hereby we know that we love the children of God, when we love God and do his commandments.}
\bv{3}{For this is the love of God, that we keep his commandments: and his commandments are not grievous.}
\bv{4}{For whatsoever is begotten of God overcometh the world: and this is the victory that hath overcome the world, \supptext{even} our faith.}
\bv{5}{And who is he that overcometh the world, but he that believeth that Jesus is the Son of God?}
\bv{6}{This is he that came by water and blood, \supptext{even} Jesus Christ; not with the water only, but with the water and with the blood.}
\bv{7}{And it is the Spirit that beareth witness, because the Spirit is the truth.}
\chapsec{The Three-fold Testimony}
\bv{8}{For there are three who bear witness,\mcomm{There are three who bear witness in heaven: the Father, the Word, and the Holy Ghost; and these three are one.} the Spirit, and the water, and the blood: and the three agree in one.}
\bv{9}{If we receive the witness of men, the witness of God is greater: for the witness of God is this, that he hath borne witness concerning his Son.}
\bv{10}{He that believeth on the Son of God hath the witness in him: he that believeth not God hath made him a liar; because he hath not believed in the witness that God hath borne concerning his Son.}
\bv{11}{And the witness is this, that God gave unto us eternal life, and this life is in his Son.}
\bv{12}{He that hath the Son hath the life; he that hath not the Son of God hath not the life.}
\chapsec{Confidence}
\bv{13}{These things have I written unto you, that ye may know that ye have eternal life, \supptext{even} unto you that believe on the name of the Son of God.}
\bv{14}{And this is the boldness which we have toward him, that, if we ask anything according to his will, he heareth us:}
\bv{15}{and if we know that he heareth us whatsoever we ask, we know that we have the petitions which we have asked of him.}
\bv{16}{If any man see his brother sinning a sin not unto death, he shall ask, and \supptext{God} will give him life for them that sin not unto death. There is a sin unto death: not concerning this do I say that he should make request.}
\bv{17}{All unrighteousness is sin: and there is a sin not unto death.}
\bv{18}{We know that whosoever is begotten of God sinneth not; but he that was begotten of God keepeth himself, and the evil one toucheth him not.}
\bv{19}{We know that we are of God, and the whole world lieth in the evil one.}
\bv{20}{And we know that the Son of God is come, and hath given us an understanding, that we know him that is true, and we are in him that is true, \supptext{even} in his Son Jesus Christ. This is the true God, and eternal life.}
\bv{21}{\supptext{My} little children, guard yourselves from idols.}
	\clearpage
	\chapter{The Second Epistle of Saint John}
\fancyhead[RE,LO]{The Second Epistle of John}
%\chaphead{Chapter I}
\chapdesc{Truth \& Love Inseparable}
\lettrine[image=true, lines=4, findent=3pt, nindent=0pt]{T-2Jn.eps}{he} elder unto the elect lady and her children, whom I love in truth; and not I only, but also all they that know the truth;
\bv{2}{for the truth’s sake which abideth in us, and it shall be with us for ever:}
\bv{3}{Grace, mercy, peace shall be with us, from God the Father, and from Jesus Christ, the Son of the Father, in truth and love.}
\par
\bv{4}{I rejoice greatly that I have found \supptext{certain} of thy children walking in truth, even as we received commandment from the Father.}
\bv{5}{And now I beseech thee, lady, not as though I wrote to thee a new commandment, but that which we had from the beginning, that we love one another.}
\bv{6}{And this is love, that we should walk after his commandments. This is the commandment, even as ye heard from the beginning, that ye should walk in it.}
\chapsec{Doctrine: The Test of Reality}
\bv{7}{For many deceivers are gone forth into the world, \supptext{even} they that confess not that Jesus Christ cometh in the flesh. This is the deceiver and the antichrist.}
\bv{8}{Look to yourselves, that ye lose not the things which we have wrought, but that ye receive a full reward.}
\bv{9}{Whosoever goeth onward and abideth not in the teaching of Christ, hath not God: he that abideth in the teaching, the same hath both the Father and the Son.}
\bv{10}{If any one cometh unto you, and bringeth not this teaching, receive him not into \supptext{your} house, and give him no greeting:}
\bv{11}{for he that giveth him greeting partaketh in his evil works.}
\chapsec{Superscription}
\bv{12}{Having many things to write unto you, I would not \supptext{write them} with paper and ink: but I hope to come unto you, and to speak face to face, that your joy may be made full.}
\bv{13}{The children of thine elect sister salute thee.}
	\clearpage
	\chapter{The Third Epistle of Saint John}
\fancyhead[RE,LO]{The Third Epistle of John}
%\chaphead{Chapter I}
\chapdesc{Personal Greetings}
\lettrine[image=true, lines=4, findent=3pt, nindent=0pt]{T-3Jn.eps}{he} elder unto Gaius the beloved, whom I love in truth.
\bv{2}{Beloved, I pray that in all things thou mayest prosper and be in health, even as thy soul prospereth.}
\bv{3}{For I rejoiced greatly, when brethren came and bare witness unto thy truth, even as thou walkest in truth.}
\bv{4}{Greater joy have I none than this, to hear of my children walking in the truth.}
\chapsec{Concerning Brethren}
\bv{5}{Beloved, thou doest a faithful work in whatsoever thou doest toward them that are brethren and strangers withal;}
\bv{6}{who bare witness to thy love before the church: whom thou wilt do well to set forward on their journey worthily of God:}
\bv{7}{because that for the sake of the Name they went forth, taking nothing of the Gentiles.}
\bv{8}{We therefore ought to welcome such, that we may be fellow-workers for the truth.}
\chapsec{Domineering Diotrephes}
\bv{9}{I wrote somewhat unto the church: but Diotrephes, who loveth to have the preeminence among them, receiveth us not.}
\bv{10}{Therefore, if I come, I will bring to remembrance his works which he doeth, prating against us with wicked words: and not content therewith, neither doth he himself receive the brethren, and them that would he forbiddeth and casteth \supptext{them} out of the church.}
\bv{11}{Beloved, imitate not that which is evil, but that which is good. He that doeth good is of God: he that doeth evil hath not seen God.}
\chapsec{Good Demetrius}
\bv{12}{Demetrius hath the witness of all \supptext{men}, and of the truth itself: yea, we also bear witness; and thou knowest that our witness is true.}
\bv{13}{I had many things to write unto thee, but I am unwilling to write \supptext{them} to thee with ink and pen:}
\bv{14}{but I hope shortly to see thee, and we shall speak face to face. Peace \supptext{be} unto thee. The friends salute thee. Salute the friends by name.}
	\clearpage
	\chapter{The Epistle of Saint Jude}
\fancyhead[RE,LO]{The Epistle of Jude}
%\chaphead{Chapter I}
\chapdesc{Introduction}
\lettrine[image=true, lines=4, findent=3pt, nindent=0pt]{J.eps}{ude}, a servant of Jesus Christ, and brother of James, to them that are called, beloved in God the Father, and kept for Jesus Christ:
\bv{2}{Mercy unto you and peace and love be multiplied.}
\chapsec{Occasion of the Epistle: Apostacy}
\bv{3}{Beloved, while I was giving all diligence to write unto you of our common salvation, I was constrained to write unto you exhorting you to contend earnestly for the faith which was once for all delivered unto the saints.}
\bv{4}{For there are certain men crept in privily, \supptext{even} they who were of old written of beforehand unto this condemnation, ungodly men, turning the grace of our God into lasciviousness, and denying our only Master and Lord, Jesus Christ.}
\chapsec{Instances of the Apostacy}
\bv{5}{Now I desire to put you in remembrance, though ye know all things once for all, that Jesus, having saved a people out of the land of Egypt, afterward destroyed them that believed not.}
\bv{6}{And angels that kept not their own principality, but left their proper habitation, he hath kept in everlasting bonds under darkness unto the judgement of the great day.}
\bv{7}{Even as Sodom and Gomorrah, and the cities about them, having in like manner with these given themselves over to fornication and gone after strange flesh, are set forth as an example, suffering the punishment of eternal fire.}
\chapsec{Apostate Teachers Described}
\bv{8}{Yet in like manner these also in their dreamings defile the flesh, and set at nought dominion, and rail at dignities.}
\bv{9}{But Michael the archangel, when contending with the devil he disputed about the body of Moses, durst not bring against him a railing judgement, but said, The Lord rebuke thee.}
\bv{10}{But these rail at whatsoever things they know not: and what they understand naturally, like the creatures without reason, in these things are they destroyed.}
\bv{11}{Woe unto them! for they went in the way of Cain, and ran riotously in the error of Balaam for hire, and perished in the gainsaying of Korah.}
\bv{12}{These are they who are hidden rocks in your love-feasts when they feast with you, shepherds that without fear feed themselves; clouds without water, carried along by winds; autumn trees without fruit, twice dead, plucked up by the roots;}
\bv{13}{wild waves of the sea, foaming out their own shame; wandering stars, for whom the blackness of darkness hath been reserved for ever.}
\bv{14}{And to these also Enoch, the seventh from Adam, prophesied, saying, Behold, the Lord came with ten thousands of his holy ones,}
\bv{15}{to execute judgement upon all, and to convict all the ungodly of all their works of ungodliness which they have ungodly wrought, and of all the hard things which ungodly sinners have spoken against him.}
\bv{16}{These are murmurers, complainers, walking after their lusts (and their mouth speaketh great swelling \supptext{words}), showing respect of persons for the sake of advantage.}
\chapsec{A Call to Persevere}
\bv{17}{But ye, beloved, remember ye the words which have been spoken before by the apostles of our Lord Jesus Christ;}
\bv{18}{that they said to you, In the last time there shall be mockers, walking after their own ungodly lusts.}
\bv{19}{These are they who make separations, sensual, having not the Spirit.}
\chapsec{True Believers Assured \& Comforted}
\bv{20}{But ye, beloved, building up yourselves on your most holy faith, praying in the Holy Spirit,}
\bv{21}{keep yourselves in the love of God, looking for the mercy of our Lord Jesus Christ unto eternal life.}
\bv{22}{And on some have mercy, who are in doubt;}
\bv{23}{and some save, snatching them out of the fire; and on some have mercy with fear; hating even the garment spotted by the flesh.}
\chapsec{Doxology}
\bv{24}{Now unto him that is able to guard you from stumbling, and to set you before the presence of his glory without blemish in exceeding joy,}
\bv{25}{to the only God our Saviour, through Jesus Christ our Lord, \supptext{be} glory, majesty, dominion and power, before all time, and now, and for evermore. Amen.}
	%\clearpage
	%\chapter{The Revelation of Jesus Christ to St. John the Divine}
\fancyhead[RE,LO]{The Revelation of John}
\chaphead{Chapter I}
\chapdesc{}
\lettrine[image=true, lines=4, findent=3pt, nindent=0pt]{T.ps}{he}The Revelation of Jesus Christ, which God gave him to show unto his servants, \supptext{even} the things which must shortly come to pass: and he sent and signified \supptext{it} by his angel unto his servant John;
\bv{2}{who bare witness of the word of God, and of the testimony of Jesus Christ, \supptext{even} of all things that he saw.}
\bv{3}{Blessed is he that readeth, and they that hear the words of the prophecy, and keep the things that are written therein: for the time is at hand.}
\bv{4}{John to the seven churches that are in Asia: Grace to you and peace, from him who is and who was and who is to come; and from the seven Spirits that are before his throne;}
\bv{5}{and from Jesus Christ, \supptext{who is} the faithful witness, the firstborn of the dead, and the ruler of the kings of the earth. Unto him that loveth us, and loosed us from our sins by his blood;}
\bv{6}{and he made us \supptext{to be} a kingdom, \supptext{to be} priests unto his God and Father; to him \supptext{be} the glory and the dominion for ever and ever. Amen.}
\bv{7}{Behold, he cometh with the clouds; and every eye shall see him, and they that pierced him; and all the tribes of the earth shall mourn over him. Even so, Amen.}
\bv{8}{I am the Alpha and the Omega, saith the Lord God, who is and who was and who is to come, the Almighty.}
\bv{9}{I John, your brother and partaker with you in the tribulation and kingdom and patience \supptext{which are} in Jesus, was in the isle that is called Patmos, for the word of God and the testimony of Jesus.}
\bv{10}{I was in the Spirit on the Lord’s day, and I heard behind me a great voice, as of a trumpet}
\bv{11}{saying, What thou seest, write in a book and send \supptext{it} to the seven churches: unto Ephesus, and unto Smyrna, and unto Pergamum, and unto Thyatira, and unto Sardis, and unto Philadelphia, and unto Laodicea.}
\bv{12}{And I turned to see the voice that spake with me. And having turned I saw seven golden candlesticks;}
\bv{13}{and in the midst of the candlesticks one like unto a son of man, clothed with a garment down to the foot, and girt about at the breasts with a golden girdle.}
\bv{14}{And his head and his hair were white as white wool, \supptext{white} as snow; and his eyes were as a flame of fire;}
\bv{15}{and his feet like unto burnished brass, as if it had been refined in a furnace; and his voice as the voice of many waters.}
\bv{16}{And he had in his right hand seven stars: and out of his mouth proceeded a sharp two-edged sword: and his countenance was as the sun shineth in his strength.}
\bv{17}{And when I saw him, I fell at his feet as one dead. And he laid his right hand upon me, saying, Fear not; I am the first and the last,}
\bv{18}{and the Living one; and I was dead, and behold, I am alive for evermore, and I have the keys of death and of Hades.}
\bv{19}{Write therefore the things which thou sawest, and the things which are, and the things which shall come to pass hereafter;}
\bv{20}{the mystery of the seven stars which thou sawest in my right hand, and the seven golden candlesticks. The seven stars are the angels of the seven churches: and the seven candlesticks are seven churches.}
\chaphead{Chapter II}
\chapdesc{}
\lettrine[image=true, lines=4, findent=3pt, nindent=0pt]{.eps}{}To the angel of the church in Ephesus write:
These things saith he that holdeth the seven stars in his right hand, he that walketh in the midst of the seven golden candlesticks:
\bv{2}{I know thy works, and thy toil and patience, and that thou canst not bear evil men, and didst try them that call themselves apostles, and they are not, and didst find them false;}
\bv{3}{and thou hast patience and didst bear for my name’s sake, and hast not grown weary.}
\bv{4}{But I have \supptext{this} against thee, that thou didst leave thy first love.}
\bv{5}{Remember therefore whence thou art fallen, and repent and do the first works; or else I come to thee, and will move thy candlestick out of its place, except thou repent.}
\bv{6}{But this thou hast, that thou hatest the works of the Nicolaitans, which I also hate.}
\bv{7}{He that hath an ear, let him hear what the Spirit saith to the churches. To him that overcometh, to him will I give to eat of the tree of life, which is in the Paradise of God.}
\bv{8}{And to the angel of the church in Smyrna write:
These things saith the first and the last, who was dead, and lived \supptext{again}:}
\bv{9}{I know thy tribulation, and thy poverty (but thou art rich), and the blasphemy of them that say they are Jews, and they are not, but are a synagogue of Satan.}
\bv{10}{Fear not the things which thou art about to suffer: behold, the devil is about to cast some of you into prison, that ye may be tried; and ye shall have tribulation ten days. Be thou faithful unto death, and I will give thee the crown of life.}
\bv{11}{He that hath an ear, let him hear what the Spirit saith to the churches. He that overcometh shall not be hurt of the second death.}
\bv{12}{And to the angel of the church in Pergamum write:
These things saith he that hath the sharp two-edged sword:}
\bv{13}{I know where thou dwellest, \supptext{even} where Satan’s throne is; and thou holdest fast my name, and didst not deny my faith, even in the days of Antipas my witness, my faithful one, who was killed among you, where Satan dwelleth.}
\bv{14}{But I have a few things against thee, because thou hast there some that hold the teaching of Balaam, who taught Balak to cast a stumblingblock before the children of Israel, to eat things sacrificed to idols, and to commit fornication.}
\bv{15}{So hast thou also some that hold the teaching of the Nicolaitans in like manner.}
\bv{16}{Repent therefore; or else I come to thee quickly, and I will make war against them with the sword of my mouth.}
\bv{17}{He that hath an ear, let him hear what the Spirit saith to the churches. To him that overcometh, to him will I give of the hidden manna, and I will give him a white stone, and upon the stone a new name written, which no one knoweth but he that receiveth it.}
\bv{18}{And to the angel of the church in Thyatira write:
These things saith the Son of God, who hath his eyes like a flame of fire, and his feet are like unto burnished brass:}
\bv{19}{I know thy works, and thy love and faith and ministry and patience, and that thy last works are more than the first.}
\bv{20}{But I have \supptext{this} against thee, that thou sufferest the woman Jezebel, who calleth herself a prophetess; and she teacheth and seduceth my servants to commit fornication, and to eat things sacrificed to idols.}
\bv{21}{And I gave her time that she should repent; and she willeth not to repent of her fornication.}
\bv{22}{Behold, I cast her into a bed, and them that commit adultery with her into great tribulation, except they repent of her works.}
\bv{23}{And I will kill her children with death; and all the churches shall know that I am he that searcheth the reins and hearts: and I will give unto each one of you according to your works.}
\bv{24}{But to you I say, to the rest that are in Thyatira, as many as have not this teaching, who know not the deep things of Satan, as they are wont to say; I cast upon you none other burden.}
\bv{25}{Nevertheless that which ye have, hold fast till I come.}
\bv{26}{And he that overcometh, and he that keepeth my works unto the end, to him will I give authority over the nations:}
\bv{27}{and he shall rule them with a rod of iron, as the vessels of the potter are broken to shivers; as I also have received of my Father:}
\bv{28}{and I will give him the morning star.}
\bv{29}{He that hath an ear, let him hear what the Spirit saith to the churches.}
\chaphead{Chapter III}
\chapdesc{}
\lettrine[image=true, lines=4, findent=3pt, nindent=0pt]{.eps}{}And to the angel of the church in Sardis write:
These things saith he that hath the seven Spirits of God, and the seven stars: I know thy works, that thou hast a name that thou livest, and thou art dead.
\bv{2}{Be thou watchful, and establish the things that remain, which were ready to die: for I have found no works of thine perfected before my God.}
\bv{3}{Remember therefore how thou hast received and didst hear; and keep \supptext{it}, and repent. If therefore thou shalt not watch, I will come as a thief, and thou shalt not know what hour I will come upon thee.}
\bv{4}{But thou hast a few names in Sardis that did not defile their garments: and they shall walk with me in white; for they are worthy.}
\bv{5}{He that overcometh shall thus be arrayed in white garments; and I will in no wise blot his name out of the book of life, and I will confess his name before my Father, and before his angels.}
\bv{6}{He that hath an ear, let him hear what the Spirit saith to the churches.}
\bv{7}{And to the angel of the church in Philadelphia write:
These things saith he that is holy, he that is true, he that hath the key of David, he that openeth and none shall shut, and that shutteth and none openeth:}
\bv{8}{I know thy works (behold, I have set before thee a door opened, which none can shut), that thou hast a little power, and didst keep my word, and didst not deny my name.}
\bv{9}{Behold, I give of the synagogue of Satan, of them that say they are Jews, and they are not, but do lie; behold, I will make them to come and worship before thy feet, and to know that I have loved thee.}
\bv{10}{Because thou didst keep the word of my patience, I also will keep thee from the hour of trial, that \supptext{hour} which is to come upon the whole world, to try them that dwell upon the earth.}
\bv{11}{I come quickly: hold fast that which thou hast, that no one take thy crown.}
\bv{12}{He that overcometh, I will make him a pillar in the temple of my God, and he shall go out thence no more: and I will write upon him the name of my God, and the name of the city of my God, the new Jerusalem, which cometh down out of heaven from my God, and mine own new name.}
\bv{13}{He that hath an ear, let him hear what the Spirit saith to the churches.}
\bv{14}{And to the angel of the church in Laodicea write:
These things saith the Amen, the faithful and true witness, the beginning of the creation of God:}
\bv{15}{I know thy works, that thou art neither cold nor hot: I would thou wert cold or hot.}
\bv{16}{So because thou art lukewarm, and neither hot nor cold, I will spew thee out of my mouth.}
\bv{17}{Because thou sayest, I am rich, and have gotten riches, and have need of nothing; and knowest not that thou art the wretched one and miserable and poor and blind and naked:}
\bv{18}{I counsel thee to buy of me gold refined by fire, that thou mayest become rich; and white garments, that thou mayest clothe thyself, and \supptext{that} the shame of thy nakedness be not made manifest; and eyesalve to anoint thine eyes, that thou mayest see.}
\bv{19}{As many as I love, I reprove and chasten: be zealous therefore, and repent.}
\bv{20}{Behold, I stand at the door and knock: if any man hear my voice and open the door, I will come in to him, and will sup with him, and he with me.}
\bv{21}{He that overcometh, I will give to him to sit down with me in my throne, as I also overcame, and sat down with my Father in his throne.}
\bv{22}{He that hath an ear, let him hear what the Spirit saith to the churches.}
\chaphead{Chapter IV}
\chapdesc{}
\lettrine[image=true, lines=4, findent=3pt, nindent=0pt]{.eps}{}After these things I saw, and behold, a door opened in heaven, and the first voice that I heard, \supptext{a voice} as of a trumpet speaking with me, one saying, Come up hither, and I will show thee the things which must come to pass hereafter.
\bv{2}{Straightway I was in the Spirit: and behold, there was a throne set in heaven, and one sitting upon the throne;}
\bv{3}{and he that sat \supptext{was} to look upon like a jasper stone and a sardius: and \supptext{there was} a rainbow round about the throne, like an emerald to look upon.}
\bv{4}{And round about the throne \supptext{were} four and twenty thrones: and upon the thrones \supptext{I saw} four and twenty elders sitting, arrayed in white garments; and on their heads crowns of gold.}
\bv{5}{And out of the throne proceed lightnings and voices and thunders. And \supptext{there were} seven lamps of fire burning before the throne, which are the seven Spirits of God;}
\bv{6}{and before the throne, as it were a sea of glass like unto crystal; and in the midst of the throne, and round about the throne, four living creatures full of eyes before and behind.}
\bv{7}{And the first creature \supptext{was} like a lion, and the second creature like a calf, and the third creature had a face as of a man, and the fourth creature \supptext{was} like a flying eagle.}
\bv{8}{And the four living creatures, having each one of them six wings, are full of eyes round about and within: and they have no rest day and night, saying,
\canticle{Holy, holy, holy, \supptext{is} the Lord God, the Almighty, who was and who is and who is to come.}
\bv{9}{And when the living creatures shall give glory and honor and thanks to him that sitteth on the throne, to him that liveth for ever and ever,}
\bv{10}{the four and twenty elders shall fall down before him that sitteth on the throne, and shall worship him that liveth for ever and ever, and shall cast their crowns before the throne, saying,
\canticle{\bv{11}{Worthy art thou, our Lord and our God, to receive the glory and the honor and the power: for thou didst create all things, and because of thy will they were, and were created.}
\chaphead{Chapter V}\chapdesc{}
\lettrine[image=true, lines=4, findent=3pt, nindent=0pt]{.eps}{}And I saw in the right hand of him that sat on the throne a book written within and on the back, close sealed with seven seals.
\bv{2}{And I saw a strong angel proclaiming with a great voice, Who is worthy to open the book, and to loose the seals thereof?}
\bv{3}{And no one in the heaven, or on the earth, or under the earth, was able to open the book, or to look thereon.}
\bv{4}{And I wept much, because no one was found worthy to open the book, or to look thereon:}
\bv{5}{and one of the elders saith unto me, Weep not; behold, the Lion that is of the tribe of Judah, the Root of David, hath overcome to open the book and the seven seals thereof.}
\bv{6}{And I saw in the midst of the throne and of the four living creatures, and in the midst of the elders, a Lamb standing, as though it had been slain, having seven horns, and seven eyes, which are the seven Spirits of God, sent forth into all the earth.}
\bv{7}{And he came, and he taketh \supptext{it} out of the right hand of him that sat on the throne.}
\bv{8}{And when he had taken the book, the four living creatures and the four and twenty elders fell down before the Lamb, having each one a harp, and golden bowls full of incense, which are the prayers of the saints.}
\bv{9}{And they sing a new song, saying,
\canticle{Worthy art thou to take the book, and to open the seals thereof: for thou wast slain, and didst purchase unto God with thy blood \supptext{men} of every tribe, and tongue, and people, and nation,}
\bv{10}{and madest them \supptext{to be} unto our God a kingdom and priests; and they reign upon the earth.}
\bv{11}{And I saw, and I heard a voice of many angels round about the throne and the living creatures and the elders; and the number of them was ten thousand times ten thousand, and thousands of thousands;}
\bv{12}{saying with a great voice,
\canticle{Worthy is the Lamb that hath been slain\\
to receive the power, and riches, and wisdom, and might,\\
and honor, and glory, and blessing.}
\bv{13}{And every created thing which is in the heaven, and on the earth, and under the earth, and on the sea, and all things that are in them, heard I saying,
\canticle{Unto him that sitteth on the throne, and unto the Lamb,\\
\supptext{be} the blessing, and the honor, and the glory, and the dominion, for ever and ever.}
\bv{14}{And the four living creatures said, ``Amen.'' And the elders fell down and worshipped.}
\chaphead{Chapter VI}
\chapdesc{}
\lettrine[image=true, lines=4, findent=3pt, nindent=0pt]{.eps}{}And I saw when the Lamb opened one of the seven seals, and I heard one of the four living creatures saying as with a voice of thunder, Come.
\bv{2}{And I saw, and behold, a white horse, and he that sat thereon had a bow; and there was given unto him a crown: and he came forth conquering, and to conquer.}
\bv{3}{And when he opened the second seal, I heard the second living creature saying, Come.}
\bv{4}{And another \supptext{horse} came forth, a red horse: and to him that sat thereon it was given to take peace from the earth, and that they should slay one another: and there was given unto him a great sword.}
\bv{5}{And when he opened the third seal, I heard the third living creature saying, Come. And I saw, and behold, a black horse; and he that sat thereon had a balance in his hand.}
\bv{6}{And I heard as it were a voice in the midst of the four living creatures saying, A measure of wheat for a shilling, and three measures of barley for a shilling; and the oil and the wine hurt thou not.}
\bv{7}{And when he opened the fourth seal, I heard the voice of the fourth living creature saying, Come.}
\bv{8}{And I saw, and behold, a pale horse: and he that sat upon him, his name was Death; and Hades followed with him. And there was given unto them authority over the fourth part of the earth, to kill with sword, and with famine, and with death, and by the wild beasts of the earth.}
\bv{9}{And when he opened the fifth seal, I saw underneath the altar the souls of them that had been slain for the word of God, and for the testimony which they held:}
\bv{10}{and they cried with a great voice, saying, How long, O Master, the holy and true, dost thou not judge and avenge our blood on them that dwell on the earth?}
\bv{11}{And there was given them to each one a white robe; and it was said unto them, that they should rest yet for a little time, until their fellow-servants also and their brethren, who should be killed even as they were, should have fulfilled \supptext{their course}.}
\bv{12}{And I saw when he opened the sixth seal, and there was a great earthquake; and the sun became black as sackcloth of hair, and the whole moon became as blood;}
\bv{13}{and the stars of the heaven fell unto the earth, as a fig tree casteth her unripe figs when she is shaken of a great wind.}
\bv{14}{And the heaven was removed as a scroll when it is rolled up; and every mountain and island were moved out of their places.}
\bv{15}{And the kings of the earth, and the princes, and the chief captains, and the rich, and the strong, and every bondman and freeman, hid themselves in the caves and in the rocks of the mountains;}
\bv{16}{and they say to the mountains and to the rocks, Fall on us, and hide us from the face of him that sitteth on the throne, and from the wrath of the Lamb:}
\bv{17}{for the great day of their wrath is come; and who is able to stand?}
\chaphead{Chapter VII}
\chapdesc{}
\lettrine[image=true, lines=4, findent=3pt, nindent=0pt]{.eps}{}After this I saw four angels standing at the four corners of the earth, holding the four winds of the earth, that no wind should blow on the earth, or on the sea, or upon any tree.
\bv{2}{And I saw another angel ascend from the sunrising, having the seal of the living God: and he cried with a great voice to the four angels to whom it was given to hurt the earth and the sea,}
\bv{3}{saying, Hurt not the earth, neither the sea, nor the trees, till we shall have sealed the servants of our God on their foreheads.}
\bv{4}{And I heard the number of them that were sealed, a hundred and forty and four thousand, sealed out of every tribe of the children of Israel:}
\canticle{Of the tribe of Reuben twelve thousand;
Of the tribe of Gad twelve thousand;}
\bv{6}{Of the tribe of Asher twelve thousand;
Of the tribe of Manasseh twelve thousand;}
\bv{7}{Of the tribe of Simeon twelve thousand;
Of the tribe of Issachar twelve thousand;}
\bv{8}{Of the tribe of Zebulun twelve thousand;
Of the tribe of Benjamin} \supptext{were} sealed twelve thousand.}
\bv{9}{After these things I saw, and behold, a great multitude, which no man could number, out of every nation and of \supptext{all} tribes and peoples and tongues, standing before the throne and before the Lamb, arrayed in white robes, and palms in their hands;}
\bv{10}{and they cry with a great voice, saying,
\canticle{Salvation unto our God who sitteth on the throne, and unto the Lamb.}
\bv{11}{And all the angels were standing round about the throne, and \supptext{about} the elders and the four living creatures; and they fell before the throne on their faces, and worshipped God,}
\bv{12}{saying,
\canticle{Amen: Blessing, and glory, and wisdom, and thanksgiving, and honor, and power, and might, \supptext{be} unto our God for ever and ever. Amen.}
\bv{13}{And one of the elders answered, saying unto me, These that are arrayed in the white robes, who are they, and whence came they?}
\bv{14}{And I say unto him, My lord, thou knowest. And he said to me, These are they that come out of the great tribulation, and they washed their robes, and made them white in the blood of the Lamb.}
\bv{15}{Therefore are they before the throne of God; and they serve him day and night in his temple: and he that sitteth on the throne shall spread his tabernacle over them.}
\bv{16}{They shall hunger no more, neither thirst any more; neither shall the sun strike upon them, nor any heat:}
\bv{17}{for the Lamb that is in the midst of the throne shall be their shepherd, and shall guide them unto fountains of waters of life: and God shall wipe away every tear from their eyes.}
\chaphead{Chapter VIII}
\chapdesc{}
\lettrine[image=true, lines=4, findent=3pt, nindent=0pt]{.eps}{}And when he opened the seventh seal, there followed a silence in heaven about the space of half an hour.
\bv{2}{And I saw the seven angels that stand before God; and there were given unto them seven trumpets.}
\bv{3}{And another angel came and stood over the altar, having a golden censer; and there was given unto him much incense, that he should add it unto the prayers of all the saints upon the golden altar which was before the throne.}
\bv{4}{And the smoke of the incense, with the prayers of the saints, went up before God out of the angel’s hand.}
\bv{5}{And the angel taketh the censer; and he filled it with the fire of the altar, and cast it upon the earth: and there followed thunders, and voices, and lightnings, and an earthquake.}
\bv{6}{And the seven angels that had the seven trumpets prepared themselves to sound.}
\bv{7}{And the first sounded, and there followed hail and fire, mingled with blood, and they were cast upon the earth: and the third part of the earth was burnt up, and the third part of the trees was burnt up, and all green grass was burnt up.}
\bv{8}{And the second angel sounded, and as it were a great mountain burning with fire was cast into the sea: and the third part of the sea became blood;}
\bv{9}{and there died the third part of the creatures which were in the sea, \supptext{even} they that had life; and the third part of the ships was destroyed.}
\bv{10}{And the third angel sounded, and there fell from heaven a great star, burning as a torch, and it fell upon the third part of the rivers, and upon the fountains of the waters;}
\bv{11}{and the name of the star is called Wormwood: and the third part of the waters became wormwood; and many men died of the waters, because they were made bitter.}
\bv{12}{And the fourth angel sounded, and the third part of the sun was smitten, and the third part of the moon, and the third part of the stars; that the third part of them should be darkened, and the day should not shine for the third part of it, and the night in like manner.}
\bv{13}{And I saw, and I heard an eagle, flying in mid heaven, saying with a great voice, Woe, woe, woe, for them that dwell on the earth, by reason of the other voices of the trumpet of the three angels, who are yet to sound.}
\chaphead{Chapter IX}
\chapdesc{}
\lettrine[image=true, lines=4, findent=3pt, nindent=0pt]{.eps}{}And the fifth angel sounded, and I saw a star from heaven fallen unto the earth: and there was given to him the key of the pit of the abyss.
\bv{2}{And he opened the pit of the abyss; and there went up a smoke out of the pit, as the smoke of a great furnace; and the sun and the air were darkened by reason of the smoke of the pit.}
\bv{3}{And out of the smoke came forth locusts upon the earth; and power was given them, as the scorpions of the earth have power.}
\bv{4}{And it was said unto them that they should not hurt the grass of the earth, neither any green thing, neither any tree, but only such men as have not the seal of God on their foreheads.}
\bv{5}{And it was given them that they should not kill them, but that they should be tormented five months: and their torment was as the torment of a scorpion, when it striketh a man.}
\bv{6}{And in those days men shall seek death, and shall in no wise find it; and they shall desire to die, and death fleeth from them.}
\bv{7}{And the shapes of the locusts were like unto horses prepared for war; and upon their heads as it were crowns like unto gold, and their faces were as men’s faces.}
\bv{8}{And they had hair as the hair of women, and their teeth were as \supptext{the teeth} of lions.}
\bv{9}{And they had breastplates, as it were breastplates of iron; and the sound of their wings was as the sound of chariots, of many horses rushing to war.}
\bv{10}{And they have tails like unto scorpions, and stings; and in their tails is their power to hurt men five months.}
\bv{11}{They have over them as king the angel of the abyss: his name in Hebrew is Abaddon, and in the Greek \supptext{tongue} he hath the name Apollyon.}
\bv{12}{The first Woe is past: behold, there come yet two Woes hereafter.}
\bv{13}{And the sixth angel sounded, and I heard a voice from the horns of the golden altar which is before God,}
\bv{14}{one saying to the sixth angel that had the trumpet, Loose the four angels that are bound at the great river Euphrates.}
\bv{15}{And the four angels were loosed, that had been prepared for the hour and day and month and year, that they should kill the third part of men.}
\bv{16}{And the number of the armies of the horsemen was twice ten thousand times ten thousand: I heard the number of them.}
\bv{17}{And thus I saw the horses in the vision, and them that sat on them, having breastplates \supptext{as} of fire and of hyacinth and of brimstone: and the heads of the horses are as the heads of lions; and out of their mouths proceedeth fire and smoke and brimstone.}
\bv{18}{By these three plagues was the third part of men killed, by the fire and the smoke and the brimstone, which proceeded out of their mouths.}
\bv{19}{For the power of the horses is in their mouth, and in their tails: for their tails are like unto serpents, and have heads; and with them they hurt.}
\bv{20}{And the rest of mankind, who were not killed with these plagues, repented not of the works of their hands, that they should not worship demons, and the idols of gold, and of silver, and of brass, and of stone, and of wood; which can neither see, nor hear, nor walk:}
\bv{21}{and they repented not of their murders, nor of their sorceries, nor of their fornication, nor of their thefts.}
\chaphead{Chapter X}
\chapdesc{}
\lettrine[image=true, lines=4, findent=3pt, nindent=0pt]{.eps}{}And I saw another strong angel coming down out of heaven, arrayed with a cloud; and the rainbow was upon his head, and his face was as the sun, and his feet as pillars of fire;
\bv{2}{and he had in his hand a little book open: and he set his right foot upon the sea, and his left upon the earth;}
\bv{3}{and he cried with a great voice, as a lion roareth: and when he cried, the seven thunders uttered their voices.}
\bv{4}{And when the seven thunders uttered \supptext{their voices}, I was about to write: and I heard a voice from heaven saying, Seal up the things which the seven thunders uttered, and write them not.}
\bv{5}{And the angel that I saw standing upon the sea and upon the earth lifted up his right hand to heaven,}
\bv{6}{and sware by him that liveth for ever and ever, who created the heaven and the things that are therein, and the earth and the things that are therein, and the sea and the things that are therein, that there shall be delay no longer:}
\bv{7}{but in the days of the voice of the seventh angel, when he is about to sound, then is finished the mystery of God, according to the good tidings which he declared to his servants the prophets.}
\bv{8}{And the voice which I heard from heaven, \supptext{I heard it} again speaking with me, and saying, Go, take the book which is open in the hand of the angel that standeth upon the sea and upon the earth.}
\bv{9}{And I went unto the angel, saying unto him that he should give me the little book. And he saith unto me, Take it, and eat it up; and it shall make thy belly bitter, but in thy mouth it shall be sweet as honey.}
\bv{10}{And I took the little book out of the angel’s hand, and ate it up; and it was in my mouth sweet as honey: and when I had eaten it, my belly was made bitter.}
\bv{11}{And they say unto me, Thou must prophesy again over many peoples and nations and tongues and kings.}
\chaphead{Chapter XI}
\chapdesc{}
\lettrine[image=true, lines=4, findent=3pt, nindent=0pt]{.eps}{}And there was given me a reed like unto a rod: and one said, Rise, and measure the temple of God, and the altar, and them that worship therein.
\bv{2}{And the court which is without the temple leave without, and measure it not; for it hath been given unto the nations: and the holy city shall they tread under foot forty and two months.}
\bv{3}{And I will give unto my two witnesses, and they shall prophesy a thousand two hundred and threescore days, clothed in sackcloth.}
\bv{4}{These are the two olive trees and the two candlesticks, standing before the Lord of the earth.}
\bv{5}{And if any man desireth to hurt them, fire proceedeth out of their mouth and devoureth their enemies; and if any man shall desire to hurt them, in this manner must he be killed.}
\bv{6}{These have the power to shut the heaven, that it rain not during the days of their prophecy: and they have power over the waters to turn them into blood, and to smite the earth with every plague, as often as they shall desire.}
\bv{7}{And when they shall have finished their testimony, the beast that cometh up out of the abyss shall make war with them, and overcome them, and kill them.}
\bv{8}{And their dead bodies \supptext{lie} in the street of the great city, which spiritually is called Sodom and Egypt, where also their Lord was crucified.}
\bv{9}{And from among the peoples and tribes and tongues and nations do \supptext{men} look upon their dead bodies three days and a half, and suffer not their dead bodies to be laid in a tomb.}
\bv{10}{And they that dwell on the earth rejoice over them, and make merry; and they shall send gifts one to another; because these two prophets tormented them that dwell on the earth.}
\bv{11}{And after the three days and a half the breath of life from God entered into them, and they stood upon their feet; and great fear fell upon them that beheld them.}
\bv{12}{And they heard a great voice from heaven saying unto them, Come up hither. And they went up into heaven in the cloud; and their enemies beheld them.}
\bv{13}{And in that hour there was a great earthquake, and the tenth part of the city fell; and there were killed in the earthquake seven thousand persons: and the rest were affrighted, and gave glory to the God of heaven.}
\bv{14}{The second Woe is past: behold, the third Woe cometh quickly.}
\bv{15}{And the seventh angel sounded; and there followed great voices in heaven, and they said,
\canticle{The kingdom of the world is become \supptext{the kingdom} of our Lord, and of his Christ: and he shall reign for ever and ever.}
\bv{16}{And the four and twenty elders, who sit before God on their thrones, fell upon their faces and worshipped God,}
\bv{17}{saying,
\canticle{We give thee thanks, O Lord God, the Almighty, who art and who wast; because thou hast taken thy great power, and didst reign.}
\bv{18}{And the nations were wroth, and thy wrath came, and the time of the dead to be judged, and \supptext{the time} to give their reward to thy servants the prophets, and to the saints, and to them that fear thy name, the small and the great; and to destroy them that destroy the earth.}
\bv{19}{And there was opened the temple of God that is in heaven; and there was seen in his temple the ark of his covenant; and there followed lightnings, and voices, and thunders, and an earthquake, and great hail.}
\chapter{XII}
\chapdesc{}
\lettrine[image=true, lines=4, findent=3pt, nindent=0pt]{.eps}{}And a great sign was seen in heaven: a woman arrayed with the sun, and the moon under her feet, and upon her head a crown of twelve stars;
\bv{2}{and she was with child; and she crieth out, travailing in birth, and in pain to be delivered.}
\bv{3}{And there was seen another sign in heaven: and behold, a great red dragon, having seven heads and ten horns, and upon his heads seven diadems.}
\bv{4}{And his tail draweth the third part of the stars of heaven, and did cast them to the earth: and the dragon standeth before the woman that is about to be delivered, that when she is delivered he may devour her child.}
\bv{5}{And she was delivered of a son, a man child, who is to rule all the nations with a rod of iron: and her child was caught up unto God, and unto his throne.}
\bv{6}{And the woman fled into the wilderness, where she hath a place prepared of God, that there they may nourish her a thousand two hundred and threescore days.}
\bv{7}{And there was war in heaven: Michael and his angels \supptext{going forth} to war with the dragon; and the dragon warred and his angels;}
\bv{8}{and they prevailed not, neither was their place found any more in heaven.}
\bv{9}{And the great dragon was cast down, the old serpent, he that is called the Devil and Satan, the deceiver of the whole world; he was cast down to the earth, and his angels were cast down with him.}
\bv{10}{And I heard a great voice in heaven, saying,
\canticle{Now is come the salvation, and the power, and the kingdom of our God, and the authority of his Christ: for the accuser of our brethren is cast down, who accuseth them before our God day and night.}
\bv{11}{And they overcame him because of the blood of the Lamb, and because of the word of their testimony; and they loved not their life even unto death.}
\bv{12}{Therefore rejoice, O heavens, and ye that dwell in them. Woe for the earth and for the sea: because the devil is gone down unto you, having great wrath, knowing that he hath but a short time.}
\bv{13}{And when the dragon saw that he was cast down to the earth, he persecuted the woman that brought forth the man \supptext{child}.}
\bv{14}{And there were given to the woman the two wings of the great eagle, that she might fly into the wilderness unto her place, where she is nourished for a time, and times, and half a time, from the face of the serpent.}
\bv{15}{And the serpent cast out of his mouth after the woman water as a river, that he might cause her to be carried away by the stream.}
\bv{16}{And the earth helped the woman, and the earth opened her mouth and swallowed up the river which the dragon cast out of his mouth.}
\bv{17}{And the dragon waxed wroth with the woman, and went away to make war with the rest of her seed, that keep the commandments of God, and hold the testimony of Jesus:
\chaphead{XIII}
\chapdesc{}
\lettrine[image=true, lines=4, findent=3pt, nindent=0pt]{.eps}{}and he stood upon the sand of the sea.
And I saw a beast coming up out of the sea, having ten horns and seven heads, and on his horns ten diadems, and upon his heads names of blasphemy.}
\bv{2}{And the beast which I saw was like unto a leopard, and his feet were as \supptext{the feet} of a bear, and his mouth as the mouth of a lion: and the dragon gave him his power, and his throne, and great authority.}
\bv{3}{And \supptext{I saw} one of his heads as though it had been smitten unto death; and his death-stroke was healed: and the whole earth wondered after the beast;}
\bv{4}{and they worshipped the dragon, because he gave his authority unto the beast; and they worshipped the beast, saying, Who is like unto the beast? and who is able to war with him?}
\bv{5}{and there was given to him a mouth speaking great things and blasphemies; and there was given to him authority to continue forty and two months.}
\bv{6}{And he opened his mouth for blasphemies against God, to blaspheme his name, and his tabernacle, \supptext{even} them that dwell in the heaven.}
\bv{7}{And it was given unto him to make war with the saints, and to overcome them: and there was given to him authority over every tribe and people and tongue and nation.}
\bv{8}{And all that dwell on the earth shall worship him, \supptext{every one} whose name hath not been written from the foundation of the world in the book of life of the Lamb that hath been slain.}
\bv{9}{If any man hath an ear, let him hear.}
\bv{10}{If any man \supptext{is} for captivity, into captivity he goeth: if any man shall kill with the sword, with the sword must he be killed. Here is the patience and the faith of the saints.}
\bv{11}{And I saw another beast coming up out of the earth; and he had two horns like unto a lamb, and he spake as a dragon.}
\bv{12}{And he exerciseth all the authority of the first beast in his sight. And he maketh the earth and them that dwell therein to worship the first beast, whose death-stroke was healed.}
\bv{13}{And he doeth great signs, that he should even make fire to come down out of heaven upon the earth in the sight of men.}
\bv{14}{And he deceiveth them that dwell on the earth by reason of the signs which it was given him to do in the sight of the beast; saying to them that dwell on the earth, that they should make an image to the beast who hath the stroke of the sword and lived.}
\bv{15}{And it was given \supptext{unto him} to give breath to it, \supptext{even} to the image of the beast, that the image of the beast should both speak, and cause that as many as should not worship the image of the beast should be killed.}
\bv{16}{And he causeth all, the small and the great, and the rich and the poor, and the free and the bond, that there be given them a mark on their right hand, or upon their forehead;}
\bv{17}{and that no man should be able to buy or to sell, save he that hath the mark, \supptext{even} the name of the beast or the number of his name.}
\bv{18}{Here is wisdom. He that hath understanding, let him count the number of the beast; for it is the number of a man: and his number is Six hundred and sixty and six.}
\chaphead{Chapter XIV}
\chapdesc{}
\lettrine[image=true, lines=4, findent=3pt, nindent=0pt]{.eps}{}And I saw, and behold, the Lamb standing on the mount Zion, and with him a hundred and forty and four thousand, having his name, and the name of his Father, written on their foreheads.
\bv{2}{And I heard a voice from heaven, as the voice of many waters, and as the voice of a great thunder: and the voice which I heard \supptext{was} as \supptext{the voice} of harpers harping with their harps:}
\bv{3}{and they sing as it were a new song before the throne, and before the four living creatures and the elders: and no man could learn the song save the hundred and forty and four thousand, \supptext{even} they that had been purchased out of the earth.}
\bv{4}{These are they that were not defiled with women; for they are virgins. These \supptext{are} they that follow the Lamb whithersoever he goeth. These were purchased from among men, \supptext{to be} the firstfruits unto God and unto the Lamb.}
\bv{5}{And in their mouth was found no lie: they are without blemish.}
\bv{6}{And I saw another angel flying in mid heaven, having eternal good tidings to proclaim unto them that dwell on the earth, and unto every nation and tribe and tongue and people;}
\bv{7}{and he saith with a great voice, Fear God, and give him glory; for the hour of his judgment is come: and worship him that made the heaven and the earth and sea and fountains of waters.}
\bv{8}{And another, a second angel, followed, saying, Fallen, fallen is Babylon the great, that hath made all the nations to drink of the wine of the wrath of her fornication.}
\bv{9}{And another angel, a third, followed them, saying with a great voice, If any man worshippeth the beast and his image, and receiveth a mark on his forehead, or upon his hand,}
\bv{10}{he also shall drink of the wine of the wrath of God, which is prepared unmixed in the cup of his anger; and he shall be tormented with fire and brimstone in the presence of the holy angels, and in the presence of the Lamb:}
\bv{11}{and the smoke of their torment goeth up for ever and ever; and they have no rest day and night, they that worship the beast and his image, and whoso receiveth the mark of his name.}
\bv{12}{Here is the patience of the saints, they that keep the commandments of God, and the faith of Jesus.}
\bv{13}{And I heard a voice from heaven saying, Write, Blessed are the dead who die in the Lord from henceforth: yea, saith the Spirit, that they may rest from their labors; for their works follow with them.}
\bv{14}{And I saw, and behold, a white cloud; and on the cloud \supptext{I saw} one sitting like unto a son of man, having on his head a golden crown, and in his hand a sharp sickle.}
\bv{15}{And another angel came out from the temple, crying with a great voice to him that sat on the cloud, Send forth thy sickle, and reap: for the hour to reap is come; for the harvest of the earth is ripe.}
\bv{16}{And he that sat on the cloud cast his sickle upon the earth; and the earth was reaped.}
\bv{17}{And another angel came out from the temple which is in heaven, he also having a sharp sickle.}
\bv{18}{And another angel came out from the altar, he that hath power over fire; and he called with a great voice to him that had the sharp sickle, saying, Send forth thy sharp sickle, and gather the clusters of the vine of the earth; for her grapes are fully ripe.}
\bv{19}{And the angel cast his sickle into the earth, and gathered the vintage of the earth, and cast it into the winepress, the great \supptext{winepress}, of the wrath of God.}
\bv{20}{And the winepress was trodden without the city, and there came out blood from the winepress, even unto the bridles of the horses, as far as a thousand and six hundred furlongs.}
\chaphead{Chapter XV}
\chapdesc{}
\lettrine[image=true, lines=4, findent=3pt, nindent=0pt]{.eps}{}And I saw another sign in heaven, great and marvellous, seven angels having seven plagues, \supptext{which are} the last, for in them is finished the wrath of God.
\bv{2}{And I saw as it were a sea of glass mingled with fire; and them that come off victorious from the beast, and from his image, and from the number of his name, standing by the sea of glass, having harps of God.}
\bv{3}{And they sing the song of Moses the servant of God, and the song of the Lamb, saying,
\canticle{Great and marvellous are thy works, O Lord God, the Almighty; righteous and true are thy ways, thou King of the ages.}
\bv{4}{Who shall not fear, O Lord, and glorify thy name? for thou only art holy; for all the nations shall come and worship before thee; for thy righteous acts have been made manifest.}
\bv{5}{And after these things I saw, and the temple of the tabernacle of the testimony in heaven was opened:}
\bv{6}{and there came out from the temple the seven angels that had the seven plagues, arrayed with \supptext{precious} stone, pure \supptext{and} bright, and girt about their breasts with golden girdles.}
\bv{7}{And one of the four living creatures gave unto the seven angels seven golden bowls full of the wrath of God, who liveth for ever and ever.}
\bv{8}{And the temple was filled with smoke from the glory of God, and from his power; and none was able to enter into the temple, till the seven plagues of the seven angels should be finished.}
\chaphead{Chapter XVI}
\chapdesc{}
\lettrine[image=true, lines=4, findent=3pt, nindent=0pt]{.eps}{}And I heard a great voice out of the temple, saying to the seven angels, Go ye, and pour out the seven bowls of the wrath of God into the earth.
\bv{2}{And the first went, and poured out his bowl into the earth; and it became a noisome and grievous sore upon the men that had the mark of the beast, and that worshipped his image.}
\bv{3}{And the second poured out his bowl into the sea; and it became blood as of a dead man; and every living soul died, \supptext{even} the things that were in the sea.}
\bv{4}{And the third poured out his bowl into the rivers and the fountains of the waters; and it became blood.}
\bv{5}{And I heard the angel of the waters saying, Righteous art thou, who art and who wast, thou Holy One, because thou didst thus judge:}
\bv{6}{for they poured out the blood of saints and prophets, and blood hast thou given them to drink: they are worthy.}
\bv{7}{And I heard the altar saying, Yea, O Lord God, the Almighty, true and righteous are thy judgments.}
\bv{8}{And the fourth poured out his bowl upon the sun; and it was given unto it to scorch men with fire.}
\bv{9}{And men were scorched with great heat: and they blasphemed the name of God who hath the power over these plagues; and they repented not to give him glory.}
\bv{10}{And the fifth poured out his bowl upon the throne of the beast; and his kingdom was darkened; and they gnawed their tongues for pain,}
\bv{11}{and they blasphemed the God of heaven because of their pains and their sores; and they repented not of their works.}
\bv{12}{And the sixth poured out his bowl upon the great river, the \supptext{river} Euphrates; and the water thereof was dried up, that the way might be made ready for the kings that \supptext{come} from the sunrising.}
\bv{13}{And I saw \supptext{coming} out of the mouth of the dragon, and out of the mouth of the beast, and out of the mouth of the false prophet, three unclean spirits, as it were frogs:}
\bv{14}{for they are spirits of demons, working signs; which go forth unto the kings of the whole world, to gather them together unto the war of the great day of God, the Almighty.}
\bv{15}{(Behold, I come as a thief. Blessed is he that watcheth, and keepeth his garments, lest he walk naked, and they see his shame.)}
\bv{16}{And they gathered them together into the place which is called in Hebrew Har-Magedon.}
\bv{17}{And the seventh poured out his bowl upon the air; and there came forth a great voice out of the temple, from the throne, saying, It is done:}
\bv{18}{and there were lightnings, and voices, and thunders; and there was a great earthquake, such as was not since there were men upon the earth, so great an earthquake, so mighty.}
\bv{19}{And the great city was divided into three parts, and the cities of the nations fell: and Babylon the great was remembered in the sight of God, to give unto her the cup of the wine of the fierceness of his wrath.}
\bv{20}{And every island fled away, and the mountains were not found.}
\bv{21}{And great hail, \supptext{every stone} about the weight of a talent, cometh down out of heaven upon men: and men blasphemed God because of the plague of the hail; for the plague thereof is exceeding great.}
\chaphead{Chapter XVII}
\chapdesc{}
\lettrine[image=true, lines=4, findent=3pt, nindent=0pt]{.eps}{}And there came one of the seven angels that had the seven bowls, and spake with me, saying, Come hither, I will show thee the judgment of the great harlot that sitteth upon many waters;
\bv{2}{with whom the kings of the earth committed fornication, and they that dwell in the earth were made drunken with the wine of her fornication.}
\bv{3}{And he carried me away in the Spirit into a wilderness: and I saw a woman sitting upon a scarlet-colored beast, full of names of blasphemy, having seven heads and ten horns.}
\bv{4}{And the woman was arrayed in purple and scarlet, and decked with gold and precious stone and pearls, having in her hand a golden cup full of abominations, even the unclean things of her fornication,}
\bv{5}{and upon her forehead a name written, MYSTERY, BABYLON THE GREAT, THE MOTHER OF THE HARLOTS AND OF THE ABOMINATIONS OF THE EARTH.}
\bv{6}{And I saw the woman drunken with the blood of the saints, and with the blood of the martyrs of Jesus. And when I saw her, I wondered with a great wonder.}
\bv{7}{And the angel said unto me, Wherefore didst thou wonder? I will tell thee the mystery of the woman, and of the beast that carrieth her, which hath the seven heads and the ten horns.}
\bv{8}{The beast that thou sawest was, and is not; and is about to come up out of the abyss, and to go into perdition. And they that dwell on the earth shall wonder, \supptext{they} whose name hath not been written in the book of life from the foundation of the world, when they behold the beast, how that he was, and is not, and shall come.}
\bv{9}{Here is the mind that hath wisdom. The seven heads are seven mountains, on which the woman sitteth:}
\bv{10}{and they are seven kings; the five are fallen, the one is, the other is not yet come; and when he cometh, he must continue a little while.}
\bv{11}{And the beast that was, and is not, is himself also an eighth, and is of the seven; and he goeth into perdition.}
\bv{12}{And the ten horns that thou sawest are ten kings, who have received no kingdom as yet; but they receive authority as kings, with the beast, for one hour.}
\bv{13}{These have one mind, and they give their power and authority unto the beast.}
\bv{14}{These shall war against the Lamb, and the Lamb shall overcome them, for he is Lord of lords, and King of kings; and they \supptext{also shall overcome} that are with him, called and chosen and faithful.}
\bv{15}{And he saith unto me, The waters which thou sawest, where the harlot sitteth, are peoples, and multitudes, and nations, and tongues.}
\bv{16}{And the ten horns which thou sawest, and the beast, these shall hate the harlot, and shall make her desolate and naked, and shall eat her flesh, and shall burn her utterly with fire.}
\bv{17}{For God did put in their hearts to do his mind, and to come to one mind, and to give their kingdom unto the beast, until the words of God should be accomplished.}
\bv{18}{And the woman whom thou sawest is the great city, which reigneth over the kings of the earth.}
\chaphead{Chapter XVIII}
\chapdesc{}
\lettrine[image=true, lines=4, findent=3pt, nindent=0pt]{.eps}{}After these things I saw another angel coming down out of heaven, having great authority; and the earth was lightened with his glory.
\bv{2}{And he cried with a mighty voice, saying, Fallen, fallen is Babylon the great, and is become a habitation of demons, and a hold of every unclean spirit, and a hold of every unclean and hateful bird.}
\bv{3}{For by the wine of the wrath of her fornication all the nations are fallen; and the kings of the earth committed fornication with her, and the merchants of the earth waxed rich by the power of her wantonness.}
\bv{4}{And I heard another voice from heaven, saying, Come forth, my people, out of her, that ye have no fellowship with her sins, and that ye receive not of her plagues:}
\bv{5}{for her sins have reached even unto heaven, and God hath remembered her iniquities.}
\bv{6}{Render unto her even as she rendered, and double \supptext{unto her} the double according to her works: in the cup which she mingled, mingle unto her double.}
\bv{7}{How much soever she glorified herself, and waxed wanton, so much give her of torment and mourning: for she saith in her heart, I sit a queen, and am no widow, and shall in no wise see mourning.}
\bv{8}{Therefore in one day shall her plagues come, death, and mourning, and famine; and she shall be utterly burned with fire; for strong is the Lord God who judged her.}
\bv{9}{And the kings of the earth, who committed fornication and lived wantonly with her, shall weep and wail over her, when they look upon the smoke of her burning,}
\bv{10}{standing afar off for the fear of her torment, saying, Woe, woe, the great city, Babylon, the strong city! for in one hour is thy judgment come.}
\bv{11}{And the merchants of the earth weep and mourn over her, for no man buyeth their merchandise any more;}
\bv{12}{merchandise of gold, and silver, and precious stone, and pearls, and fine linen, and purple, and silk, and scarlet; and all thyine wood, and every vessel of ivory, and every vessel made of most precious wood, and of brass, and iron, and marble;}
\bv{13}{and cinnamon, and spice, and incense, and ointment, and frankincense, and wine, and oil, and fine flour, and wheat, and cattle, and sheep; and \supptext{merchandise} of horses and chariots and slaves; and souls of men.}
\bv{14}{And the fruits which thy soul lusted after are gone from thee, and all things that were dainty and sumptuous are perished from thee, and \supptext{men} shall find them no more at all.}
\bv{15}{The merchants of these things, who were made rich by her, shall stand afar off for the fear of her torment, weeping and mourning;}
\bv{16}{saying, Woe, woe, the great city, she that was arrayed in fine linen and purple and scarlet, and decked with gold and precious stone and pearl!}
\bv{17}{for in one hour so great riches is made desolate. And every shipmaster, and every one that saileth any whither, and mariners, and as many as gain their living by sea, stood afar off,}
\bv{18}{and cried out as they looked upon the smoke of her burning, saying, What \supptext{city} is like the great city?}
\bv{19}{And they cast dust on their heads, and cried, weeping and mourning, saying, Woe, woe, the great city, wherein all that had their ships in the sea were made rich by reason of her costliness! for in one hour is she made desolate.}
\bv{20}{Rejoice over her, thou heaven, and ye saints, and ye apostles, and ye prophets; for God hath judged your judgment on her.}
\bv{21}{And a strong angel took up a stone as it were a great millstone and cast it into the sea, saying, Thus with a mighty fall shall Babylon, the great city, be cast down, and shall be found no more at all.}
\bv{22}{And the voice of harpers and minstrels and flute-players and trumpeters shall be heard no more at all in thee; and no craftsman, of whatsoever craft, shall be found any more at all in thee; and the voice of a mill shall be heard no more at all in thee;}
\bv{23}{and the light of a lamp shall shine no more at all in thee; and the voice of the bridegroom and of the bride shall be heard no more at all in thee: for thy merchants were the princes of the earth; for with thy sorcery were all the nations deceived.}
\bv{24}{And in her was found the blood of prophets and of saints, and of all that have been slain upon the earth.}
\chaphead{Chapter XIX}
\chapdesc{}
\lettrine[image=true, lines=4, findent=3pt, nindent=0pt]{.eps}{}After these things I heard as it were a great voice of a great multitude in heaven, saying,
\canticle{Hallelujah; Salvation, and glory, and power, belong to our God:}
\bv{2}{for true and righteous are his judgments; for he hath judged the great harlot, her that corrupted the earth with her fornication, and he hath avenged the blood of his servants at her hand.}
\bv{3}{And a second time they say, Hallelujah. And her smoke goeth up for ever and ever.}
\bv{4}{And the four and twenty elders and the four living creatures fell down and worshipped God that sitteth on the throne, saying, Amen; Hallelujah.}
\bv{5}{And a voice came forth from the throne, saying,
\canticle{Give praise to our God, all ye his servants, ye that fear him, the small and the great.}
\bv{6}{And I heard as it were the voice of a great multitude, and as the voice of many waters, and as the voice of mighty thunders, saying,
\canticle{Hallelujah: for the Lord our God, the Almighty, reigneth.}
\bv{7}{Let us rejoice and be exceeding glad, and let us give the glory unto him: for the marriage of the Lamb is come, and his wife hath made herself ready.}
\bv{8}{And it was given unto her that she should array herself in fine linen, bright \supptext{and} pure: for the fine linen is the righteous acts of the saints.}
\bv{9}{And he saith unto me, Write, Blessed are they that are bidden to the marriage supper of the Lamb. And he saith unto me, These are true words of God.}
\bv{10}{And I fell down before his feet to worship him. And he saith unto me, See thou do it not: I am a fellow-servant with thee and with thy brethren that hold the testimony of Jesus: worship God: for the testimony of Jesus is the spirit of prophecy.}
\bv{11}{And I saw the heaven opened; and behold, a white horse, and he that sat thereon called Faithful and True; and in righteousness he doth judge and make war.}
\bv{12}{And his eyes \supptext{are} a flame of fire, and upon his head \supptext{are} many diadems; and he hath a name written which no one knoweth but he himself.}
\bv{13}{And he \supptext{is} arrayed in a garment sprinkled with blood: and his name is called The Word of God.}
\bv{14}{And the armies which are in heaven followed him upon white horses, clothed in fine linen, white \supptext{and} pure.}
\bv{15}{And out of his mouth proceedeth a sharp sword, that with it he should smite the nations: and he shall rule them with a rod of iron: and he treadeth the winepress of the fierceness of the wrath of God, the Almighty.}
\bv{16}{And he hath on his garment and on his thigh a name written, KING OF KINGS, AND LORD OF LORDS.}
\bv{17}{And I saw an angel standing in the sun; and he cried with a loud voice, saying to all the birds that fly in mid heaven, Come \supptext{and} be gathered together unto the great supper of God;}
\bv{18}{that ye may eat the flesh of kings, and the flesh of captains, and the flesh of mighty men, and the flesh of horses and of them that sit thereon, and the flesh of all men, both free and bond, and small and great.}
\bv{19}{And I saw the beast, and the kings of the earth, and their armies, gathered together to make war against him that sat upon the horse, and against his army.}
\bv{20}{And the beast was taken, and with him the false prophet that wrought the signs in his sight, wherewith he deceived them that had received the mark of the beast and them that worshipped his image: they two were cast alive into the lake of fire that burneth with brimstone:}
\bv{21}{and the rest were killed with the sword of him that sat upon the horse, \supptext{even the sword} which came forth out of his mouth: and all the birds were filled with their flesh.}
\chaphead{Chapter XX}
\chapdesc{}
\lettrine[image=true, lines=4, findent=3pt, nindent=0pt]{.eps}{}And I saw an angel coming down out of heaven, having the key of the abyss and a great chain in his hand.
\bv{2}{And he laid hold on the dragon, the old serpent, which is the Devil and Satan, and bound him for a thousand years,}
\bv{3}{and cast him into the abyss, and shut \supptext{it}, and sealed \supptext{it} over him, that he should deceive the nations no more, until the thousand years should be finished: after this he must be loosed for a little time.}
\bv{4}{And I saw thrones, and they sat upon them, and judgment was given unto them: and \supptext{I saw} the souls of them that had been beheaded for the testimony of Jesus, and for the word of God, and such as worshipped not the beast, neither his image, and received not the mark upon their forehead and upon their hand; and they lived, and reigned with Christ a thousand years.}
\bv{5}{The rest of the dead lived not until the thousand years should be finished. This is the first resurrection.}
\bv{6}{Blessed and holy is he that hath part in the first resurrection: over these the second death hath no power; but they shall be priests of God and of Christ, and shall reign with him a thousand years.}
\bv{7}{And when the thousand years are finished, Satan shall be loosed out of his prison,}
\bv{8}{and shall come forth to deceive the nations which are in the four corners of the earth, Gog and Magog, to gather them together to the war: the number of whom is as the sand of the sea.}
\bv{9}{And they went up over the breadth of the earth, and compassed the camp of the saints about, and the beloved city: and fire came down out of heaven, and devoured them.}
\bv{10}{And the devil that deceived them was cast into the lake of fire and brimstone, where are also the beast and the false prophet; and they shall be tormented day and night for ever and ever.}
\bv{11}{And I saw a great white throne, and him that sat upon it, from whose face the earth and the heaven fled away; and there was found no place for them.}
\bv{12}{And I saw the dead, the great and the small, standing before the throne; and books were opened: and another book was opened, which is \supptext{the book} of life: and the dead were judged out of the things which were written in the books, according to their works.}
\bv{13}{And the sea gave up the dead that were in it; and death and Hades gave up the dead that were in them: and they were judged every man according to their works.}
\bv{14}{And death and Hades were cast into the lake of fire. This is the second death, \supptext{even} the lake of fire.}
\bv{15}{And if any was not found written in the book of life, he was cast into the lake of fire.}
\chaphead{Chapter XXI}
\chapdesc{}
\lettrine[image=true, lines=4, findent=3pt, nindent=0pt]{.eps}{}And I saw a new heaven and a new earth: for the first heaven and the first earth are passed away; and the sea is no more.}
\bv{2}{And I saw the holy city, new Jerusalem, coming down out of heaven from God, made ready as a bride adorned for her husband.}
\bv{3}{And I heard a great voice out of the throne saying, Behold, the tabernacle of God is with men, and he shall dwell with them, and they shall be his peoples, and God himself shall be with them, \supptext{and be} their God:}
\bv{4}{and he shall wipe away every tear from their eyes; and death shall be no more; neither shall there be mourning, nor crying, nor pain, any more: the first things are passed away.}
\bv{5}{And he that sitteth on the throne said, Behold, I make all things new. And he saith, Write: for these words are faithful and true.}
\bv{6}{And he said unto me, They are come to pass. I am the Alpha and the Omega, the beginning and the end. I will give unto him that is athirst of the fountain of the water of life freely.}
\bv{7}{He that overcometh shall inherit these things; and I will be his God, and he shall be my son.}
\bv{8}{But for the fearful, and unbelieving, and abominable, and murderers, and fornicators, and sorcerers, and idolaters, and all liars, their part \supptext{shall be} in the lake that burneth with fire and brimstone; which is the second death.}
\bv{9}{And there came one of the seven angels who had the seven bowls, who were laden with the seven last plagues; and he spake with me, saying, Come hither, I will show thee the bride, the wife of the Lamb.}
\bv{10}{And he carried me away in the Spirit to a mountain great and high, and showed me the holy city Jerusalem, coming down out of heaven from God,}
\bv{11}{having the glory of God: her light was like unto a stone most precious, as it were a jasper stone, clear as crystal:}
\bv{12}{having a wall great and high; having twelve gates, and at the gates twelve angels; and names written thereon, which are \supptext{the names} of the twelve tribes of the children of Israel:}
\bv{13}{on the east were three gates; and on the north three gates; and on the south three gates; and on the west three gates.}
\bv{14}{And the wall of the city had twelve foundations, and on them twelve names of the twelve apostles of the Lamb.}
\bv{15}{And he that spake with me had for a measure a golden reed to measure the city, and the gates thereof, and the wall thereof.}
\bv{16}{And the city lieth foursquare, and the length thereof is as great as the breadth: and he measured the city with the reed, twelve thousand furlongs: the length and the breadth and the height thereof are equal.}
\bv{17}{And he measured the wall thereof, a hundred and forty and four cubits, \supptext{according to} the measure of a man, that is, of an angel.}
\bv{18}{And the building of the wall thereof was jasper: and the city was pure gold, like unto pure glass.}
\bv{19}{The foundations of the wall of the city were adorned with all manner of precious stones. The first foundation was jasper; the second, sapphire; the third, chalcedony; the fourth, emerald;}
\bv{20}{the fifth, sardonyx; the sixth, sardius; the seventh, chrysolite; the eighth, beryl; the ninth, topaz; the tenth, chrysoprase; the eleventh, jacinth; the twelfth, amethyst.}
\bv{21}{And the twelve gates were twelve pearls; each one of the several gates was of one pearl: and the street of the city was pure gold, as it were transparent glass.}
\bv{22}{And I saw no temple therein: for the Lord God the Almighty, and the Lamb, are the temple thereof.}
\bv{23}{And the city hath no need of the sun, neither of the moon, to shine upon it: for the glory of God did lighten it, and the lamp thereof \supptext{is} the Lamb.}
\bv{24}{And the nations shall walk amidst the light thereof: and the kings of the earth bring their glory into it.}
\bv{25}{And the gates thereof shall in no wise be shut by day (for there shall be no night there):}
\bv{26}{and they shall bring the glory and the honor of the nations into it:}
\bv{27}{and there shall in no wise enter into it anything unclean, or he that maketh an abomination and a lie: but only they that are written in the Lamb’s book of life.}
\chaphead{Chapter XXII}
\chapdesc{}
\lettrine[image=true, lines=4, findent=3pt, nindent=0pt]{.eps}{}And he showed me a river of water of life, bright as crystal, proceeding out of the throne of God and of the Lamb,}
\bv{2}{in the midst of the street thereof. And on this side of the river and on that was the tree of life, bearing twelve \supptext{manner of} fruits, yielding its fruit every month: and the leaves of the tree were for the healing of the nations.}
\bv{3}{And there shall be no curse any more: and the throne of God and of the Lamb shall be therein: and his servants shall serve him;}
\bv{4}{and they shall see his face; and his name \supptext{shall be} on their foreheads.}
\bv{5}{And there shall be night no more; and they need no light of lamp, neither light of sun; for the Lord God shall give them light: and they shall reign for ever and ever.}
\bv{6}{And he said unto me, These words are faithful and true: and the Lord, the God of the spirits of the prophets, sent his angel to show unto his servants the things which must shortly come to pass.}
\bv{7}{And behold, I come quickly. Blessed is he that keepeth the words of the prophecy of this book.}
\bv{8}{And I John am he that heard and saw these things. And when I heard and saw, I fell down to worship before the feet of the angel that showed me these things.}
\bv{9}{And he saith unto me, See thou do it not: I am a fellow-servant with thee and with thy brethren the prophets, and with them that keep the words of this book: worship God.}
\bv{10}{And he saith unto me, Seal not up the words of the prophecy of this book; for the time is at hand.}
\bv{11}{He that is unrighteous, let him do unrighteousness still: and he that is filthy, let him be made filthy still: and he that is righteous, let him do righteousness still: and he that is holy, let him be made holy still.}
\bv{12}{Behold, I come quickly; and my reward is with me, to render to each man according as his work is.}
\bv{13}{I am the Alpha and the Omega, the first and the last, the beginning and the end.}
\bv{14}{Blessed are they that wash their robes, that they may have the right \supptext{to come} to the tree of life, and may enter in by the gates into the city.}
\bv{15}{Without are the dogs, and the sorcerers, and the fornicators, and the murderers, and the idolaters, and every one that loveth and maketh a lie.}
\bv{16}{I Jesus have sent mine angel to testify unto you these things for the churches. I am the root and the offspring of David, the bright, the morning star.}
\bv{17}{And the Spirit and the bride say, Come. And he that heareth, let him say, Come. And he that is athirst, let him come: he that will, let him take the water of life freely.}
\bv{18}{I testify unto every man that heareth the words of the prophecy of this book, If any man shall add unto them, God shall add unto him the plagues which are written in this book:}
\bv{19}{and if any man shall take away from the words of the book of this prophecy, God shall take away his part from the tree of life, and out of the holy city, which are written in this book.}
\bv{20}{He who testifieth these things saith, Yea: I come quickly. Amen: come, Lord Jesus.}
\bv{21}{The grace of the Lord Jesus be with the saints. Amen.}
%Translation decisions so far:
%sick of the palsy -> paralytic
%baptize -> baptise
%John 18:5-6: I am he to I AM
%neighbor -> neighbour
%Holy Spirit -> Holy Ghost
%Zachariah -> Zechariah
%Matthew 27:50: yielded up his spirit -> yielded up the ghost
%John 8:58: before Abraham was born -> before Abraham was
%favor -> favour
%vapor -> vapour
%honor -> honour
%judgment -> judgement
\end{document}